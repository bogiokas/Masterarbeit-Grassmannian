\chapter{Bits and Pieces in Topology}
\section{Some Categorical Notions}
The goal of this small section is to motivate the definition of the categorical exponential object. In order to arrive there, we first go through the Yoneda lemma and then through the notion of the adjoint functors. For a more thorough introduction to the topics discussed here, refer to \cite{basic_cat}, or any other introductory category book.

We will use the notation $\mathrm{ob}(\mathcal{C})$ for the objects of $\mathcal{C}$ and $\mathcal{C}(A,B)$ for the morphisms from $A$ to $B$ in $\mathcal{C}$. Moreover, the following few statements will be about locally small categories:
\begin{definition} A category $\mathcal{C}$ is called \emph{locally small}, if for every two objects $A,B$ in $\mathcal{C}$, the morphisms $\mathcal{C}(A,B)$ form a set.
\end{definition}

\subsection{Yoneda Lemma}
\begin{definition} Let $\mathcal{C}$ be any locally small category. Then, we denote by $\mathrm{Set}^{\mathcal{C}}$ the \emph{category of set valued functors of $\mathcal{C}$}, which is defined as follows:
\begin{itemize}
\item $\mathrm{ob}\left(\mathrm{Set}^{\mathcal{C}}\right)$ is the class of all functors $F$ from $\mathcal{C}$ to $\mathrm{Set}$.
\item $\mathrm{Set}^{\mathcal{C}}(F,G)$ is the class of all natural transformations $F\overset{\eta}{\Rightarrow}G$.
\item For $G\overset{\eta}{\Rightarrow}H$ and $F\overset{\theta}{\Rightarrow}G$, the composition $F\overset{\eta\circ\theta}{\Rightarrow}H$ is the usual composition of natural transformations.
\end{itemize}
\end{definition}
\begin{remark} In the above definition, $\mathrm{Set}^{\mathcal{C}}$ need not be locally small. Indeed, let $\mathcal{C}$ be any large discrete category. Let for example, $\mathrm{ob}(\mathcal{C})$ be the same as $\mathrm{ob}(\mathrm{Set})$ and let
\[\mathcal{C}(A,B)=\left\{\begin{array}{ll}\left\{1_A\right\}&,A=B\\\emptyset&,A\neq B\end{array}\right.\]
This is obviously a locally small category. Moreover, define the functors $F,G$:
\begin{center}
\begin{tikzcd}
\mathcal{C}\ar[r,bend left=25, "F"{name=F}]\ar[r,bend right=25, "G"'{name=G}]&[2em]\mathrm{Set}
\end{tikzcd}
\end{center}
both to be the identity on $\mathrm{ob}(\mathcal{C})$. The only morphisms in $\mathcal{C}$ are the identity morphisms, for which there is no choice for $F,G$, since they are functors. They have to satisfy $F1_A=G1_A=1_A$ for every set $A$.

A natural transformation $\eta:F\Rightarrow G$ is a collection of choices $\eta_A:F(A)\to G(A)$, such that $(Ge)\circ\eta_A=\eta_B\circ(Fe)$ for every morphism $e\in\mathcal{C}(A,B)$. The only such morphisms are the identity arrows, so there aren't any restrictions on the choices when constructing $\eta$. This means that $\mathrm{Set}^{\mathcal{C}}(F,G)$ is the class of all $\eta$, each one determined by a collection of choices $\eta_A\in\mathrm{Set}(A,A)$ over all sets $A$. Since the class of all sets is a proper class, so is the class of all such collection of choices, which makes $\mathrm{Set}^{\mathcal{C}}$ not locally small.
\end{remark}

We start with some definitions and results from Chapter~4 in \cite{basic_cat}.
\begin{definition}[4.1.16] Let $\mathcal{C}$ be a locally small category. Moreover, let $X$ be any object of $\mathcal{C}$. We define the functor
\[H_X:\mathcal{C}^{\mathrm{op}}\to\mathrm{Set}\]
as follows:
\begin{itemize}
\item For any object $A$ of $\mathcal{C}^{\mathrm{op}}$, we define $H_X(A):=\mathcal{C}(A,X)$.
\item For any morphism $g^{\mathrm{op}}\in\mathcal{C}^{\mathrm{op}}(A,B)$, i.e. any morphism $g\in\mathcal{C}(B,A)$, we define $H_X(g):=g^*\in\mathrm{Set}(H_X(A),H_X(B))$ taking any $p\in\mathcal{C}(A,X)$ to $p\circ g\in\mathcal{C}(B,X)$, like in the following diagram:
\begin{center}
\begin{tikzcd}
A\ar[d,"g^{\mathrm{op}}"']&&A&&H_X(A)\ar[d,"H_X(g)"']\ar[r,phantom,":="]&[-1.5em]\mathcal{C}(A,X)\ar[r,phantom,"\ni"]&[-1.8em]p\ar[d,mapsto,"g^*"]\\
B&&B\ar[u,"g"]&&H_X(B)\ar[r,phantom,":="]&\mathcal{C}(B,X)\ar[r,phantom,"\ni"]&p\circ g
\end{tikzcd}
\end{center}
\end{itemize}
\end{definition}

\begin{definition}[4.1.21]\label{def:functor} Let $\mathcal{C}$ be a locally small category. We define the functor
\[H_{\bullet}:\mathcal{C}\to\mathrm{Set}^{\mathcal{C}^{\mathrm{op}}}\]
as follows:
\begin{itemize}
\item For any object $X$ of $\mathcal{C}$, we define $H_{\bullet}(X):=H_X$.
\item For any morphism $f\in\mathcal{C}(X,Y)$, we define $H_{\bullet}(f)=H_f\in\mathrm{Set}^{\mathcal{C}^{\mathrm{op}}}(H_X,H_Y)$ to be the natural transformation with components $(H_f)_A:=f_*\in\mathrm{Set}(H_X(A),H_Y(A))$ taking any $p\in\mathcal{C}(A,X)$ to $f\circ p\in\mathcal{C}(A,Y)$, like in the following diagram:
\begin{center}
\begin{tikzcd}
X\ar[dd,"f"']&[-1em]&[4em]&[-1em]H_X(A)\ar[dd,"(H_f)_A"]\ar[r,phantom,"\ni"]&[-1.8em]p\ar[dd,mapsto,"f_*"]\\[-1em]
&\mathcal{C}^{\mathrm{op}}\ar[r,bend left=30,"H_X"{name=HX}]\ar[r,bend right=30,"H_Y"'{name=HY}]&\mathrm{Set}\ar[from=HX, to=HY, Rightarrow, "H_f"]\\[-1em]
Y&&&H_Y(A)\ar[r,phantom,"\ni"]&f\circ p
\end{tikzcd}
\end{center}
\end{itemize}
\end{definition}
\begin{remark} We have to prove that in the above definition, for every $f\in\mathcal{C}(X,Y)$, $H_f$ is indeed a natural transformation, i.e. that for every $g^{\mathrm{op}}\in\mathcal{C}^{\mathrm{op}}(A,B)$, the following diagram commutes:
\begin{center}
\begin{tikzcd}
A&&H_X(A)\ar[d,"H_X(g)"']\ar[r,"(H_f)_A"]&H_Y(A)\ar[d,"H_Y(g)"]\\
B\ar[u,"g"]&&H_X(B)\ar[r,"(H_f)_B"]&H_Y(B)
\end{tikzcd}
\end{center}
which does, since both directions take $p\in\mathcal{C}(A,X)$ to $f\circ p\circ g\in\mathcal{C}(B,Y)$.
\end{remark}

\begin{theorem}[Yoneda, 4.2.1] Let $\mathcal{C}$ be a locally small category. Moreover, let $X$ be any object of $\mathcal{C}$ and $F:\mathcal{C}^{\mathrm{op}}\to\mathrm{Set}$ be any functor. Then:
\[\mathrm{Set}^{\mathcal{C}^{\mathrm{op}}}(H_X,F)\cong F(X)\]
naturally in $(X,F)$ in $\mathcal{C}^{op}\times\mathrm{Set}^{\mathcal{C}^{\mathrm{op}}}$.
\end{theorem}
\begin{proof}[Sketch of the Proof] To remove some clutter in the notation, we just write $SP$ instead of $S(P)$ and $Su$ instead of $S(u)$ for every functor $S$, object $P$ and morphism $u$.

First of all, the ``$\cong$'' in the theorem is inside the category $\mathrm{Set}$, so it is a bijection. It being natural in $(X,F)$ means that we need to define a bijection
\[\psi_{X,F}:\mathrm{Set}^{\mathcal{C}^{\mathrm{op}}}(H_X,F)\to FX\]
and prove that it is a natural transformation between the following two functors:
\begin{center}
\begin{tikzcd}
\mathcal{C}^{op}\times\mathrm{Set}^{\mathcal{C}^{\mathrm{op}}}\ar[r,shift left,"(X{,}F)\mapsto\mathrm{Set}^{\mathcal{C}^{\mathrm{op}}}(H_X{,}F)"]\ar[r,shift right, "(X{,}F)\mapsto FX"']&[10em]\mathrm{Set}
\end{tikzcd}
\end{center}
In order to prove that this is a bijection, it suffices to define a function
\[\varphi_{X,F}:FX\to\mathrm{Set}^{\mathcal{C}^{\mathrm{op}}}(H_X,F)\]
and prove that it is the inverse of $\psi$.

So, the proof is going to have four steps: First we are going to define $\psi_{X,F}$, then we are going to define $\varphi_{X,F}$, then we are going to prove that these are inverses and finally we are going to prove that they are natural:
\begin{itemize}
\item There is the following ``natural'' choice for $\psi_{X,F}$. Let $\eta$ be any natural transformation $H_X\Rightarrow F$. Then define:
\[\psi_{X,F}(\eta):=\eta_X(1_X)\in FX\]
Let us unpack this: $\eta$ being a natural transformation means that it has components $\eta_A:H_XA\to FA$ for every $A$ in $\mathcal{C}$, but $H_XA$ is just $\mathcal{C}(A,X)$. Choosing $A=X$, we also get an obvious element $1_X\in\mathcal{C}(X,X)$, whose image under $\eta_X$ lies in the desired set $FX$.

\item There exists also a ``natural'' choice for $\varphi_{X,F}$. Let $x\in FX$ be any element of the set $FX$. We have to define a natural transformation $\varphi_{X,F}(x)=\theta^x$ from $H_X$ to $F$. Let us first define each component $\theta^x_A:H_XA\to FA$ as follows:
\[\theta^x_A(p)=(Fp)(x)\in FA\]
Let us also unpack this definition as well: We want to define $\theta^x_A$ on every element $p\in\mathcal{C}(A,X)$. We already have some element $x\in FX$ and for every such $p$, we can create $Fp\in\mathrm{Set}(FX,FA)$, since $F$ is a functor $\mathcal{C}^{\mathrm{op}}\to\mathrm{Set}$. So, the image of $x$ under $Fp$ lies in the desired set $FA$.

The definition of $\varphi$ is not over yet, since we have to prove that $\theta^x$ is indeed natural in $A$, i.e. that for every morphism $f^{\mathrm{op}}\in\mathcal{C}(A,B)$ the following diagram commutes:
\begin{center}
\begin{tikzcd}
A&&H_XA\ar[d,"H_Xf"']\ar[r,"\theta^x_A"]&FA\ar[d,"Ff"]\\
B\ar[u,"f"]&&H_XB\ar[r,"\theta^x_B"]&FB
\end{tikzcd}
\end{center}
which does, since both directions take $q$ to $(F(q\circ f))(x)$.

\item The next step is to prove that $\psi_{X,F}$ and $\varphi_{X,F}$ are inverse functions in the category of sets. The one direction is easy. Let $x\in FX$. Then:
\[\psi_{X,F}\big(\varphi_{X,F}(x)\big)=\psi_{X,F}(\theta^x)=\theta^x_X(1_X)=(F1_X)(x)=1_{FX}(x)=x\]

For the other direction, let $\eta$ be any natural transformation $H_X\to F$. Then:
\[\varphi_{X,F}\big(\psi_{X,F}(\eta)\big)=\varphi_{X,F}(\eta_X(1_X))=\theta^{\eta_X(1_X)}\]
In order to show that this is equal to $\eta$, we need to check the equality in every component. Let $A$ be any object of $\mathcal{C}$ and $p\in\mathcal{C}(A,X)$ any function. Then:
\[\theta^{\eta_X(1_X)}_A(p)=(Fp)\big(\eta_X(1_X)\big)\overset{(*)}{=}\eta_A\big((H_Xp)(1_X)\big)=\eta_A(1_X\circ p)=\eta_A(p)\]
where the $(*)$ holds because of the naturality of $\eta$, i.e. because this diagram commutes:
\begin{center}
\begin{tikzcd}
X&&H_XX\ar[r,"\eta_X"]\ar[d,"H_Xp"']&FX\ar[d,"Fp"]\\
A\ar[u,"p"]&&H_XA\ar[r,"\eta_A"]&FA
\end{tikzcd}
\end{center}

\item Now, it only remains to prove that $\psi$ and $\varphi$ are natural transformations. It suffices to prove this just for one of the two, say $\psi$. Also, being natural in $(X,F)\in\mathcal{C}^{\mathrm{op}}\times\mathrm{Set}^{\mathcal{C}^{\mathrm{op}}}$ is equivalent to being natural in $X\in\mathcal{C}^{\mathrm{op}}$ for every fixed $F$ and at the same time being natural in $F\in\mathrm{Set}^{\mathcal{C}^{\mathrm{op}}}$, for every fixed $X$.

Let $F$ be fixed and $g^{op}\in\mathcal{C}^{op}(X,Y)$. Then, we want to show that the following diagram commutes:
\begin{center}
\begin{tikzcd}
X&&H_X&&\mathrm{Set}^{\mathcal{C}^{\mathrm{op}}}(H_X,F)\ar[d,"-\circ H_g"']\ar[r,"\psi_{X,F}"]&FX\ar[d,"Fg"]\\
Y\ar[u,"g"]&&H_Y\ar[u,"H_g"]&&\mathrm{Set}^{\mathcal{C}^{\mathrm{op}}}(H_Y,F)\ar[r,"\psi_{Y,F}"]&FY
\end{tikzcd}
\end{center}
which is true, since for any natural transformation $\eta\in\mathrm{Set}^{\mathcal{C}^{\mathrm{op}}}(H_X,F)$, both directions lead to $\eta_Y(g)$. To prove this, use the naturality of $\eta$ and the definition of $H_g$.

Let now $X$ be fixed and $\alpha\in\mathrm{Set}^{\mathcal{C}^{\mathrm{op}}}(F,G)$. Then, we want to show that the following diagram commutes:
\begin{center}
\begin{tikzcd}
F\ar[d,"\alpha"]&&\mathrm{Set}^{\mathcal{C}^{\mathrm{op}}}(H_X,F)\ar[d,"\alpha\circ-"']\ar[r,"\psi_{X,F}"]&FX\ar[d,"\alpha_X"]\\
G&&\mathrm{Set}^{\mathcal{C}^{\mathrm{op}}}(H_X,G)\ar[r,"\psi_{X,G}"]&GX
\end{tikzcd}
\end{center}
which is true, since for any natural transformation $\eta\in\mathrm{Set}^{\mathcal{C}^{\mathrm{op}}}(H_X,F)$, both directions lead to $(\alpha\circ\eta)_X(1_X)$.\qedhere
\end{itemize}
\end{proof}
\begin{corollary} Let $\mathcal{C}$ be a locally small category. Moreover, let $X,Y$ be any two objects of $\mathcal{C}$. Then:
\[\mathrm{Set}^{\mathcal{C}^{\mathrm{op}}}(H_X,H_Y)\cong\mathcal{C}(X,Y)\]
naturally in $(X,Y)$ in $\mathcal{C}^{\mathrm{op}}\times \mathcal{C}^{\mathrm{op}}$.
\end{corollary}
\begin{proof} First of all, the bijection is trivially obtained from the Yoneda lemma, for $F=H_Y$. Moreover, since the map $Y\mapsto H_Y$ is functorial, as defined in Definition~\ref{def:functor}, the following diagram commutes:
\begin{center}
\begin{tikzcd}
Y_1\ar[d,"h"]&&H_{Y_1}\ar[d,"H_h"]&&\mathrm{Set}^{\mathcal{C}^{\mathrm{op}}}(H_X,H_{Y_1})\ar[d,"H_h\circ-"']\ar[r,"\psi_{X,H_{Y_1}}"]&H_{Y_1}X\ar[d,"(H_h)_X"]\\
Y_2&&H_{Y_2}&&\mathrm{Set}^{\mathcal{C}^{\mathrm{op}}}(H_X,H_{Y_2})\ar[r,"\psi_{X,H_{Y_2}}"]&H_{Y_2}X
\end{tikzcd}
\end{center}
just like in the proof of the Yoneda lemma.
\end{proof}
\begin{corollary}[4.3.7]\label{cor:ful_faith} Let $\mathcal{C}$ be a locally small category. Then, the functor
\[H_{\bullet}:\mathcal{C}\to\mathrm{Set}^{\mathcal{C}^{\mathrm{op}}}\]
as defined in Definition~\ref{def:functor} is full and faithful.
\end{corollary}
\begin{proof} $H_{\bullet}$ being full and faithful means that for every two objects $X,Y$ in $\mathcal{C}$, the map
\begin{center}
\begin{tikzcd}
\mathcal{C}(X,Y)\ar[r,"H_{\bullet}"]&\mathrm{Set}^{\mathcal{C}^{\mathrm{op}}}(H_X,H_Y)\\[-2em]
g\ar[r,mapsto]&H_g
\end{tikzcd}
\end{center}
is a bijection. We already know that $\varphi_{X,H_Y}$ is a bijection from the proof of the Yoneda lemma, so it suffices to show that $\varphi_{X,H_Y}(g)=H_g$ for every $g\in\mathcal{C}(X,Y)$, or equivalently $\psi_{X,H_Y}(H_g)=g$:
\[\psi_{X,H_Y}(H_g)=(H_g)_X(1_X)=g\circ 1_X=g\qedhere\]
\end{proof}
\begin{proposition}[4.3.10]\label{prop:unique} Let $\mathcal{C}$ be a locally small category. Moreover, let $X,Y$ be any two objects of $\mathcal{C}$. Then:
\[H_X\cong H_Y \Longleftrightarrow X\cong Y\]
\end{proposition}
\begin{proof} Since $H_{\bullet}$ is a functor, it takes an isomorphism to an isomorphism. For the other direction, it is easy to see that for any full and faithful functor $F$, $FA\cong FB$ gives $A\cong B$ and $H_{\bullet}$ was proven to be full and faithful in Corollary~\ref{cor:ful_faith}
\end{proof}

\subsection{Adjoint functors}
Let us now go through some definitions and results regarding the adjoints. Our reference for this subsection will mainly be the Chapter~2 of \cite{basic_cat}, but be warned that we heavily changed the notations and the proofs of this section to cover our needs.

\begin{definition}[2.1.1]\label{def:adj} Let $\mathcal{C},\mathcal{D}$ be two locally small categories and $L:\mathcal{D}\to\mathcal{C},\ R:\mathcal{C}\to\mathcal{D}$ be two functors. Then we say that \emph{$R$ is the right adjoint to $L$} and \emph{$L$ is the left adjoint to $R$}, if there exist a isomorphisms
\begin{equation}
\mathcal{C}(L(A),X)\cong\mathcal{D}(A,R(X))\label{adj}
\end{equation}
natural in $(A,X)\in\mathcal{D}^{op}\times\mathcal{C}$. We usually denote this by $L\dashv R$ or equivalently:
\vspace*{-0.2em}
\begin{center}
\begin{tikzcd}
\mathcal{C}\ar[r,bend right=30, "R"'{name=R}]&[3em]\mathcal{D}\ar[l,bend right=30, "L"'{name=L}]
\ar[phantom, from=L, to=R, "\dashv" rotate=-90]
\end{tikzcd}
\end{center}
\end{definition}
\begin{notation} Just like in the proof of Yoneda lemma, we will try in this subsection to omit the parenthesis when using a functor and write $FP$ instead of $F(P)$ etc. In order to avoid confusion as much as possible, we may reintroduce the parenthesis, sometimes when more than one functors are involved, as in $FG(P)$.
\end{notation}

\begin{proposition}[4.3.13] Let $\mathcal{C},\mathcal{D}$ be two locally small categories and $L:\mathcal{D}\to\mathcal{C}$. If $L$ has a right adjoint functor $R$, then $R$ is unique, up to isomorphism.
\end{proposition}
\begin{proof} Let $R,R'$ be two right adjoint functors of $L$. Then, for fixed $X$, there exist isomorphisms
\[\mathcal{D}(A,RX)\cong\mathcal{C}(LA,X)\cong\mathcal{D}(A,R'X)\]
natural in $A$. This means that there is an isomorphism $H_{RX}(A)\cong H_{R'X}(A)$ natural in $A$, which means $H_{RX}\cong H_{R'X}$ in $\mathrm{Set}^{\mathcal{D}^{\mathrm{op}}}$. Using Proposition~\ref{prop:unique}, we conclude that there exists an isomorphism $RX\cong R'X$. This construction is natural in $X$, so $R\cong R'$ as functors.
\end{proof}

Let us now unravel the Definition~\ref{def:adj}. Let $L$ and $R$ be an adjoint pair $L\dashv R$. This means that there exist some natural transformations $\psi$ and $\varphi$ with components:
\[\begin{aligned}
\psi_{A,X}&:\mathcal{C}(LA,X)\to\mathcal{D}(A,RX)\\[0.8em]
\varphi_{A,X}&:\mathcal{D}(A,RX)\to\mathcal{C}(LA,X)
\end{aligned}\qquad\text{ satisfying: }\qquad
\left\{\begin{aligned}
\psi_{A,X}\circ\varphi_{A,X}&=1_{\mathcal{D}(A,RX)}\\[0.8em]
\varphi_{A,X}\circ\psi_{A,X}&=1_{\mathcal{C}(LA,X)}
\end{aligned}\right.\]
To express the two naturality conditions, we fix some $p^{\mathrm{op}}\in\mathcal{D}^{\mathrm{op}}(A,B)$ and some $f\in\mathcal{C}(X,Y)$. Then, we have the following four commutative diagrams:
\begin{center}
\begin{tikzcd}
A&[1.2em]
\mathcal{C}(LA,X)\ar[d,"-\circ Lp"']\ar[r,"\psi_{A,X}","\cong"']&\mathcal{D}(A,RX)\ar[d,"-\circ p"]&[1.2em]
\mathcal{C}(LA,X)\ar[d,"-\circ Lp"']&\mathcal{D}(A,RX)\ar[l,"\varphi_{A,X}"',"\cong"]\ar[d,"-\circ p"]\\
B\ar[u,"p"']&
\mathcal{C}(LB,X)\ar[r,"\psi_{B,X}","\cong"']&\mathcal{D}(B,RX)&
\mathcal{C}(LB,X)&\mathcal{D}(B,RX)\ar[l,"\varphi_{B,X}"',"\cong"]\\
X\ar[d,"f"]&
\mathcal{C}(LA,X)\ar[d,"f\circ -"']\ar[r,"\psi_{A,X}","\cong"']&\mathcal{D}(A,RX)\ar[d,"Rf\circ -"]&
\mathcal{C}(LA,X)\ar[d,"f\circ -"']&\mathcal{D}(A,RX)\ar[l,"\varphi_{A,X}"',"\cong"]\ar[d,"Rf\circ -"]\\
Y&
\mathcal{C}(LA,Y)\ar[r,"\psi_{A,Y}","\cong"']&\mathcal{D}(A,RY)&
\mathcal{C}(LA,Y)&\mathcal{D}(A,RY)\ar[l,"\varphi_{A,Y}"',"\cong"]\\
\end{tikzcd}
\end{center}
i.e. for any $q\in\mathcal{D}(A,RX)$ and $g\in\mathcal{C}(LA,X)$, we have:
\begin{center}
\begin{minipage}{0.5\linewidth}
%\makeatletter\tagsleft@true\makeatother
\begin{align}
\psi_{B,X}(g\circ Lp)&=\psi_{A,X}(g)\circ p\label{psiL}\\[1em]
\psi_{A,Y}(f\circ g)&=Rf\circ \psi_{A,X}(g)\label{psiR}
\end{align}
\end{minipage}\begin{minipage}{0.5\linewidth}
\begin{align}
\varphi_{B,X}(q\circ p)&=\varphi_{A,X}(q)\circ Lp\label{phiL}\\[1em]
\varphi_{A,Y}(Rf\circ q)&=f\circ \varphi_{A,X}(q)\label{phiR}
\end{align}
\end{minipage}
\end{center}
Right now, we will focus on equations \eqref{phiL} and \eqref{psiR}, which we also write schematically as:
\begin{align*}
\varphi_{B,X}\left(B\overset{p}{\longrightarrow}A\overset{q}{\longrightarrow}RX\right)&=LB\overset{Lp}{\longrightarrow}LA\overset{\varphi_{A,X}(q)}{\longrightarrow}X\\[1em]
\psi_{A,Y}\left(LA\overset{g}{\longrightarrow}X\overset{f}{\longrightarrow}Y\right)&=A\overset{\psi_{A,X}(g)}{\longrightarrow}RX\overset{Rf}{\longrightarrow}RY
\end{align*}
Interestingly, this hints towards the following fact: The values of $\varphi$ (resp. $\psi$) on any $q\circ p$ (resp. $f\circ g$) only depend on $L$ (resp. $R$) and $\varphi_{A,X}(q)$ (resp. $\psi_{A,X}(g)$). So, by choosing $q$ (resp. $g$) as ``naturally'' as possible, we can describe what exactly $\varphi$ (resp. $\psi$) does on every other input. We thus choose $A=RX$, $q=1_{RX}$ for the equation involving $\varphi$ and $X=LA$, $g=1_{LA}$ for the equation involving $\psi$. This gives:
\begin{align*}
\varphi_{B,X}\left(B\overset{p}{\longrightarrow}RX\right)&=\varphi_{B,X}\left(B\overset{p}{\longrightarrow}RX\xrightarrow{1_{RX}}RX\right)=LB\overset{Lp}{\longrightarrow}LRX\xrightarrow{\varphi_{RX,X}(1_{RX})}X\\[1em]
\psi_{A,Y}\left(\mask{B\overset{p}{\longrightarrow}RX}{LA\overset{f}{\longrightarrow}Y}\right)&=\psi_{A,Y}\left(\mask{B\overset{p}{\longrightarrow}RX\xrightarrow{1_{RX}}RX}{LA\xrightarrow{1_{LA}}LA\overset{f}{\longrightarrow}Y}\right)=\mask{LB\overset{Lp}{\longrightarrow}LRX\xrightarrow{\varphi_{RX,X}(1_{RX})}X}{A\xrightarrow{\psi_{A,LA}(1_{LA})}RLA\overset{Rf}{\longrightarrow}RY}
\end{align*}
\begin{definition} Let $\mathcal{C},\mathcal{D}$ be two locally small categories and $L:\mathcal{D}\to\mathcal{C}$, $R:\mathcal{C}\to\mathcal{D}$ a pair of adjoint functors $L\dashv R$. Then the natural transformations
\[\varepsilon:LR\Rightarrow \mathbbm{1}_{\mathcal{C}}\qquad\text{ and }\qquad\eta:\mathbbm{1}_{\mathcal{D}}\Rightarrow RL\]
defined by:
\[\varepsilon_X:=\varphi_{RX,X}(1_{RX})\qquad\text{ and }\qquad\eta_A:=\psi_{A,LA}(1_{LA})\]
are called the \emph{counit} and the \emph{unit} of the adjunction, respectively.
\end{definition}
\begin{remark} In order for the counit and the unit to be well defined, we need to prove that they are indeed natural transformations, i.e. that for any $f\in\mathcal{C}(X,Y)$ and $p\in\mathcal{D}(B,A)$ the following diagrams commute:
\begin{center}
\begin{tikzcd}
X\ar[d,"f"']&[-1em]LR(X)\ar[r,"\varepsilon_X"]\ar[d,"LR(f)"']&X\ar[d,"f"]&[1.5em]
B\ar[d,"p"']&[-1em]B\ar[r,"\eta_B"]\ar[d,"p"']&RL(B)\ar[d,"RL(p)"]\\
Y&LR(Y)\ar[r,"\varepsilon_Y"]&Y&
A&A\ar[r,"\eta_A"]&RL(A)
\end{tikzcd}
\end{center}
\end{remark}
\begin{proof}
This is a direct consequence of the naturality of $\varphi,\psi$:
\begin{align*}
f\circ\varepsilon_X&\overset{\eqref{phiR}}{=}\varphi_{RX,Y}(1_{RY}\circ Rf\circ 1_{RX})\overset{\eqref{phiL}}{=}\varepsilon_Y\circ LRf\\[0.5em]
\eta_A\circ p&\overset{\eqref{psiL}}{=}\psi_{B,LA}(1_{LA}\circ Lp\circ 1_{LB})\overset{\eqref{psiR}}{=}RLp\circ\eta_B\qedhere
\end{align*}
\end{proof}

So, our discussion leads to the following fact. Let $f\in\mathcal{C}(LA,X)$ and $p\in\mathcal{D}(A,RX)$, then, equations \eqref{phiL} and \eqref{psiR} give:
\begin{align}
\begin{split}
\varphi_{A,X}(p)&=\varepsilon_X\circ Lp\\[0.8em]
\psi_{A,X}(f)&=Rf\circ\eta_A
\end{split}\label{iso_from_unit}
\end{align}

This shows that the isomorphism \eqref{adj} can be completely retrieved by its counit and unit. Our goal for the rest of this section is to give an alternative definition for $L\dashv R$, only based on the choice of a particular natural transformation $\varepsilon$ to play the role of the counit, from which we first will construct $\eta$ and then $\varphi$ and $\psi$. The reason we want to do this, is because this ``second definition'' will give us a way to more explicitly construct the right adjoint $R$ of a given $L$, if $L$ has a right adjoint.

First of all, notice that \eqref{iso_from_unit} on its own suffices to make $\varphi$ and $\psi$ natural, given that $\varepsilon$ and $\eta$ are:
\begin{proposition}\label{prop:nat_gives_nat} Let $\mathcal{C},\mathcal{D}$ be two locally small categories and $L:\mathcal{D}\to\mathcal{C}$, $R:\mathcal{C}\to\mathcal{D}$ any pair of functors. Moreover, let $\varepsilon:LR\Rightarrow\mathbbm{1}_{\mathcal{C}}$ and $\eta:\mathbbm{1}_{\mathcal{D}}\Rightarrow RL$ be any two natural transformations. If we define $\varphi_{A,X}$ and $\psi_{A,X}$ using the equations \eqref{iso_from_unit}, then $\varphi$ and $\psi$ are natural transformations in $(A,X)\in\mathcal{D}^{\mathrm{op}}\times\mathcal{C}$ and additionally the following equations are satisfied:
\begin{align}
\begin{split}
\varepsilon_X=\varphi_{RX,X}(1_{RX})\\[0.8em]
\eta_A=\psi_{A,LA}(1_{LA})
\end{split}\label{unit_from_iso}
\end{align}
\end{proposition}
\begin{proof} Let us fix some $p^{\mathrm{op}}\in\mathcal{D}^{\mathrm{op}}(A,B)$ and some $f\in\mathcal{C}(X,Y)$. Then it is easy to see that the equations \eqref{psiL}-\eqref{phiR} hold:
\begin{align*}
\psi_{B,X}(g\circ Lp)&=Rg\circ RLp\circ\eta_B=Rg\circ\eta_A\circ p=\psi_{A,X}(g)\circ p,&\text{for any } g&\in\mathcal{C}(LA,X)\\
\psi_{A,Y}(f\circ g)&=Rf\circ Rg\circ\eta_A=Rf\circ\psi_{A,X}(g),&\text{for any } g&\in\mathcal{C}(LA,X)\\
\varphi_{B,X}(q\circ p)&=\varepsilon_X\circ Lq\circ Lp=\varphi_{A,X}(q)\circ p,&\text{for any }q&\in\mathcal{D}(A,RX)\\
\varphi_{A,Y}(Rf\circ q)&=\varepsilon_Y\circ LRf\circ Lq=f\circ\varepsilon_X\circ Lq=f\circ\varphi_{A,X}(q),&\text{for any }q&\in\mathcal{D}(A,RX)
\end{align*}
We only used the equations \eqref{iso_from_unit} and the naturality of $\varepsilon,\eta$ in the first and last case. Moreover, it is again very easy to see that:
\begin{align*}
\varphi_{RX,X}(1_{RX})&=\varepsilon_X\circ L1_{RX}=\varepsilon_X\\
\psi_{A,LA}(1_{LA})&=R1_{LA}\circ\eta_A=\eta_A\qedhere
\end{align*}
\end{proof}

Our problem now is that if we choose some natural transformations $\varepsilon$ and $\eta$ randomly and construct $\varphi$ and $\psi$ using \eqref{iso_from_unit}, it is not at all guaranteed that $\varphi$ and $\psi$ are inverses of each other. It so happens that $\varepsilon$ and $\eta$ must satisfy a universal property in order to be the counit and unit of an adjunction. Let's start examining this at the following definition.
\begin{definition}[2.3.4] Let $\mathcal{A},\mathcal{B}$ be two locally small categories, $F:\mathcal{A}\to\mathcal{B}$ some functor and $P$ any object in $\mathcal{B}$. We then define the category $\commaCat{F}{P}$, as follows:
\begin{itemize}
\item $\mathrm{ob}(\commaCat{F}{P})$ is the class of all pairs $(A,f)$, where $A$ is an object of $\mathcal{A}$ and $f:FA\to P$ is a morphism in $\mathcal{B}$.
\item $(\commaCat{F}{P})\big((A_1,f_1),(A_2,f_2)\big)$ is the set of maps $a:A_1\to A_2$ in $\mathcal{A}$, such that $f_1=f_2\circ Fa$, i.e. making the following diagram commute:
\begin{center}
%\begin{tikzcd}
%&[-1.5em]\mathcal{B}&[-1.5em]&&\mathcal{A}\ar[lll,bend right=20, "F"']\\
%&&FA_1\ar[dd,"Fa"]\ar[dll,"f_1"']&&A_1\ar[dd,"a"]\\[-1em]
%P\\[-1em]
%&&FA_1\ar[ull,"f_2"]&&A_2\\
%\end{tikzcd}
\begin{tikzcd}
&FA_1\ar[dd,"Fa"]\ar[dl,"f_1"']& && &A_1\ar[dd,"a"]\\[-1em]
P& &\mathcal{B}&&\mathcal{A}\ar[ll,bend right=20, "F"']\\[-1em]
&FA_1\ar[ul,"f_2"]& && &A_2\\
\end{tikzcd}
\end{center}
\end{itemize}
\end{definition}
\begin{proposition}[co-2.3.5] Let $\mathcal{C},\mathcal{D}$ be locally small categories and $L:\mathcal{D}\to\mathcal{C}$, $R:\mathcal{C}\to\mathcal{D}$ a pair of adjoint functors $L\dashv R$. Then, the pair $(RX, \varepsilon_X)$ is a terminal object in $\commaCat{L}{X}$, where $\varepsilon$ is the counit of the adjunction.
\end{proposition}
\begin{proof} We need to show that for every object $A$ in $\mathcal{D}$ and every morphism $f:LA\to X$ in $\mathcal{C}$, there exists a unique morphism $p\in\mathcal{D}(A,RX)$ such that $f=\varepsilon_X\circ Lp$, i.e. the following diagram commutes:
\begin{center}
\begin{tikzcd}
&[-1.5em]\mathcal{C}\ar[rrr,bend right=20,"R"'{name=R}]&[-1.5em]&&\mathcal{D}\ar[lll,bend right=20, "L"'{name=L}]\ar[phantom, from=L, to=R, "\dashv" rotate=-90]\\
&&LA\ar[dd,"Lp"]\ar[dll,"f"']&&A\ar[dd,"p"]\\[-1em]
X\\[-1em]
&&LRX\ar[ull,"\varepsilon_X"]&&RX\\
\end{tikzcd}
\end{center}
Since $L\dashv R$, then there exist mutually inverse isomorphisms $\psi_{A,X}:\mathcal{C}(LA,X)\to\mathcal{D}(A,RX)$ and $\varphi_{A,X}:\mathcal{D}(A,RX)\to\mathcal{C}(LA,X)$ like above. If we let $p=\psi_{A,X}(f)\in\mathcal{D}(A,RX)$, then $f=\varphi_{A,X}(p)=\varepsilon_X\circ Lp$, since $\psi,\varphi$ are inverses and \eqref{iso_from_unit} holds. In order to prove the uniqueness, let $p':A\to RX$ be any other morphism satisfying the same identity. Then, $\varphi_{A,X}(p)=\varphi_{A,X}(p')$ which gives $p=p'$, since $\varphi_{A,X}$ is an isomorphism.
\end{proof}
We now want to prove that this is not only a property of the counit, but it fully characterizes it:
\begin{theorem}[co-2.3.6]\label{th:alternative_def} Let $\mathcal{C},\mathcal{D}$ be locally small categories and $L:\mathcal{D}\to\mathcal{C}$, $R:\mathcal{C}\to\mathcal{D}$ be a pair of functors. Moreover, let $\varepsilon:LR\Rightarrow\mathbbm{1}_{\mathcal{C}}$ be some natural transformation, such that for every object $X$ of $\mathcal{C}$, $(RX,\varepsilon_X)$ is a terminal object in $\commaCat{L}{X}$. Then $L\dashv R$ and $\varepsilon$ is the counit of this adjunction.
\end{theorem}
\begin{proof} First of all, let $A$ be any object of $\mathcal{D}$. Since $(RLA,\varepsilon_{LA})$ is a terminal object of $\commaCat{R}{LA}$, for $f=1_{LA}$, there exists a unique $\eta_A:A\to RLA$, such that $\varepsilon_{LA}\circ L\eta_A=1_{LA}$, i.e. making the following diagram commutative:
\begin{center}
\begin{tikzcd}
&[-1.5em]\mathcal{C}\ar[rrr,bend right=20,"R"'{name=R}]&[-1.5em]&&\mathcal{D}\ar[lll,bend right=20, "L"'{name=L}]\\
&&LA\ar[dd,"L\eta_A"]\ar[dll,"1_{LA}"']&&A\ar[dd,"\eta_A"]\\[-1em]
LA\\[-1em]
&&LRLA\ar[ull,"\varepsilon_{LA}"]&&RLA\\
\end{tikzcd}
\end{center}
Now the proof will have three steps. First we will prove that $\eta_A$ is natural in $A$. This will be enough to define some natural transformations $\varphi$ and $\psi$. The last two steps will be to prove that these two natural transformations are also inverses of each other.
\begin{itemize}
\item In order to prove that this $\eta$ we just defined is a natural transformation, fix some $p\in\mathcal{D}(A,B)$. Notice that the five subareas of the following diagram commute:
\begin{center}
\begin{tikzcd}
A\ar[dd,"p"']&&&LA\ar[dl,"Lp"']\ar[d,"1_{LA}"]\ar[r,"L\eta_A"]&LRLA\ar[dl,"\varepsilon_{LA}"]\ar[d,"LRLp"]\\[1em]
&&LB\ar[d,"L\eta_B"']\ar[dr,"1_{LB}"']&LA\ar[l,"Lp"']\ar[d,"Lp"']&LRLB\ar[dl,"\varepsilon_{LB}"]\\[1em]
B&&LRLB\ar[r,"\varepsilon_{LB}"']&LB
\end{tikzcd}
\end{center}
Two triangles because of the definition of $\eta$, the other two trivially and the square because of the naturality of $\varepsilon$. This gives us that the outer hexagon also commutes, and:
\[\varepsilon_{LB}\circ L(\eta_B\circ p)=Lp=\varepsilon_{LB}\circ L(RLp\circ\eta_A)\]
So, if we let
\[q=\eta_B\circ p\in\mathcal{D}(A,RLB)\qquad\text{ and }\qquad q'=RLp\circ\eta_A\in\mathcal{D}(A,RLB)\]
then, both $q$ and $q'$ satisfy the following commutative diagram:
\begin{center}
\begin{tikzcd}
&LA\ar[dd,"Lq=Lq'"]\ar[dl,"Lp"']&&A\ar[dd,shift right,"q"']\ar[dd,shift left,"q'"]\\[-1em]
LB\\[-1em]
&LRLB\ar[ul,"\varepsilon_{LB}"]&&RLB
\end{tikzcd}
\end{center}
Because $(RLB,\varepsilon_{LB})$ is terminal in $\commaCat{L}{LB}$, there is a unique morphism $(A,Lp)\to(RLB,\varepsilon_{LB})$, which means $q=q'$, so $\eta_B\circ p=RLp\circ\eta_A$, i.e. the following diagram commutes, as we wanted:
\begin{center}
\begin{tikzcd}
A\ar[d,"p"']&&A\ar[r,"\eta_A"]\ar[d,"p"']&RLA\ar[d,"RLp"]\\
B&&B\ar[r,"\eta_B"]&RLB
\end{tikzcd}
\end{center}
Since $\varepsilon:LR\Rightarrow\mathbbm{1}_{\mathcal{C}}$ and $\eta:\mathbbm{1}_{\mathcal{D}}\Rightarrow RL$, Proposition~\ref{prop:nat_gives_nat} gives that $\varphi,\psi$, defined by the equations \eqref{iso_from_unit} are natural transformations in $(A,X)\in\mathcal{D}^{\mathrm{op}}\times\mathcal{C}$.
\item Let $f\in\mathcal{C}(LA,X)$. Then, the naturality of $\varepsilon$ and the definition of $\eta$, give:
\[\varphi_{A,X}\big(\psi_{A,X}(f)\big)=\varphi_{A,X}(Rf\circ\eta_A)=\varepsilon_X\circ LRf\circ L\eta_A=f\circ\varepsilon_{LA}\circ L\eta_A=f\]
\item Let $p\in\mathcal{D}(A,RX)$. Then, the fact that
\begin{equation}\label{duel_eta_eps}
R\varepsilon_X\circ\eta_{RX}=1_{RX}
\end{equation}
and the naturality of $\eta$ give:
\[\psi_{A,X}\big(\varphi_{A,X}(p)\big)=\psi_{A,X}(\varepsilon_X\circ Lp)=R\varepsilon_X\circ RLp\circ\eta_A=R\varepsilon_X\circ\eta_{RX}\circ p=p\]
It only now remains to show \eqref{duel_eta_eps}. Notice that the triangle in the following diagram commutes because of the definition of $\eta$ and the square commutes because of the naturality of $\varepsilon$:
\begin{center}
\begin{tikzcd}
LRX\ar[d,"L\eta_{RX}"']\ar[dr,"1_{LRX}"]\\
LRLRX\ar[r,"\varepsilon_{LRX}"]\ar[d,"LR\varepsilon_X"']&LRX\ar[d,"\varepsilon_X"]\\
LRX\ar[r,"\varepsilon_X"]&X
\end{tikzcd}
\end{center}
This means that the outer pentagon also commutes, i.e.:
\[\varepsilon_X\circ L(R\varepsilon_X\circ\eta_{RX})=\varepsilon_X=\varepsilon_X\circ L(1_{RX})\]
So, if we let
\[q=R\varepsilon_X\circ\eta_{RX}\in\mathcal{D}(RX,RX)\qquad\text{ and }\qquad q'=1_{RX}\circ\eta_A\in\mathcal{D}(RX,RX)\]
then, both $q$ and $q'$ satisfy the following commutative diagram:
\begin{center}
\begin{tikzcd}
&LRX\ar[dd,"Lq=Lq'"]\ar[dl,"\varepsilon_X"']&&RX\ar[dd,shift right,"q"']\ar[dd,shift left,"q'"]\\[-1em]
X\\[-1em]
&LRX\ar[ul,"\varepsilon_X"]&&RX
\end{tikzcd}
\end{center}
Because $(RX,\varepsilon_X)$ is terminal in $\commaCat{L}{X}$, there is a unique morphism $(RX,\varepsilon_X)\to(RX,\varepsilon_X)$, which means $q=q'$, so $R\varepsilon_X\circ\eta_{RX}=1_{RX}$, which was our goal.
\end{itemize}
\end{proof}
Finally, this gives us the following way to construct $R$, given $L$:
\begin{corollary}[co-2.3.7]\label{cor:adj_construction} Let $\mathcal{C},\mathcal{D}$ be locally small categories and $L:\mathcal{D}\to\mathcal{C}$ be some functor. If $\commaCat{L}{X}$ has a terminal object for every object $X$ of $\mathcal{C}$, say $(\tilde{X},\varepsilon_X)$, then $L$ has a right adjoint functor $R:\mathcal{C}\to\mathcal{D}$ defined by:
\begin{itemize}
\item $RX$ to be equal to $\tilde{X}$, for every object $X$ in $\mathcal{C}$
\item $Rf$ to be equal to the unique morphism in $\commaCat{L}{Y}$ from $(RX,f\circ\varepsilon_X)$ to $(RY,\varepsilon_Y)$, for every $f\in\mathcal{C}(X,Y)$.
\end{itemize}
\end{corollary}
\begin{proof} Given \ref{th:alternative_def}, we only have to show that $R$ is indeed a functor and that $\varepsilon$ is a natural transformation:
\begin{itemize}
\item Let $X$ be an object in $\mathcal{C}$. Then $R1_X$ is the unique morphism in $\commaCat{L}{Y}$ from $(RX,\varepsilon_X)$ to $(RX,\varepsilon_X)$. Notice that $1_X$ is such a morphism:
\begin{center}
\begin{tikzcd}
X\ar[d,"1_X"']&LRX\ar[l,"\varepsilon_X"']\ar[d,"L1_{RX}",dashed]&&RX\ar[d,"1_{RX}"]\\
X&LRX\ar[l,"\varepsilon_X"']&&RX
\end{tikzcd}
\end{center}
Thus, the uniqueness gives $R1_X=1_X$. Moreover, let $f\in\mathcal{C}(X,Y)$ and $g\in\mathcal{C}(Y,Z)$. Then, $R(g\circ f)$ is the unique morphism in $\commaCat{L}{Z}$, from $(RX,g\circ f\circ\varepsilon_X)$ to $(RZ,\varepsilon_Z)$. Notice that $Rg\circ Rf$ is such a morphism:
\begin{center}
\begin{tikzcd}
X\ar[d,"f"']&LRX\ar[l,"\varepsilon_X"']\ar[d,"LRf"]\ar[dd,bend left=60,"L(Rg\circ Rf)",dashed]&&&RX\ar[d,"Rf"]\ar[dd,bend left=60,"Rg\circ Rf"]\\
Y\ar[d,"g"']&LRY\ar[l,"\varepsilon_Y"']\ar[d,"LRg"]&&&RY\ar[d,"Rg"]\\
Z&LRZ\ar[l,"\varepsilon_Z"']&&&RZ
\end{tikzcd}
\end{center}
Thus, the uniqueness gives $R(g\circ f)=Rg\circ Rf$.
\item Let $f\in\mathcal{C}(X,Y)$. Then, by definition of $Rf$, the following square commutes:
\begin{center}
\begin{tikzcd}
X\ar[d,"f"']&LRX\ar[l,"\varepsilon_X"']\ar[d,"LRf"]\\
Y&LRY\ar[l,"\varepsilon_Y"']
\end{tikzcd}
\end{center}
which means exactly that $\varepsilon$ is a natural transformation.\qedhere
\end{itemize}
\end{proof}
\begin{example} Let $\mathcal{C}$ a category that has binary coproducts and binary products, then:
\begin{center}
\begin{tikzcd}
\mathcal{C}\times\mathcal{C}\ar[d,bend right=50,shift right,"\amalg"'{name=coprod}]\ar[d,bend left=50,shift left,"\times"{name=prod}]\\[2em]
\mathcal{C}\ar[u,"\Delta"description,""{name=diagL}, ""'{name=diagR}]
\ar[from=coprod,to=diagL,"\dashv"description,phantom]
\ar[from=diagR,to=prod,"\dashv"description,phantom]
\end{tikzcd}
\end{center}
where $\Delta$ is the diagonal functor taking $X$ to $(X,X)$.
\end{example}
\begin{proof} We will prove the first adjunction using the definition and the second one using the characterization in Corollary~\ref{cor:adj_construction}.
\begin{itemize}
\item Let $(A_1,A_2)\in\mathcal{C}\times\mathcal{C}$ and let $A_1\amalg A_2$, together with the maps \[\imath_1:A_1\hookrightarrow A_1\amalg A_2\text{ and }\imath_2:A_2\hookrightarrow A_1\amalg A_2\]
be their coproduct. Then, for every $(q_1,q_2)\in\mathcal{C}\times\mathcal{C}\big((A_1,A_2),(X,X)\big)$ the universal property of the coproduct asserts the existance of a unique map $[q_1,q_2]:A_1\amalg A_2\to X$ such that $[q_1,q_2]\circ\imath_1=q_1$ and $[q_1,q_2]\circ\imath_2=q_2$, i.e. the following diagram commutes:
\begin{center}
\begin{tikzcd}
A_1\ar[r,hook,"\imath_1"]\ar[dr,"q_1"']&A_1\amalg A_2\ar[d,"{[}q_1{,}q_2{]}",near start]&A_2\ar[l,hook',"\imath_2"']\ar[dl,"q_2"]\\[1em]
&X
\end{tikzcd}
\end{center}
Moreover, for $(p_1,p_2)\in\mathcal{C}\times\mathcal{C}\big((B_1,B_2),(A_1,A_2)\big)$, let $p_1\amalg p_2:B_1\amalg B_2\to A_1\amalg A_2$ be the map $[\imath^A_1\circ p_1,\imath^A_2\circ p_2]$, i.e. the unique map making the following diagram commute:
\begin{center}
\begin{tikzcd}
A_1\ar[r,hook,"\imath^A_1"]&A_1\amalg A_2&A_2\ar[l,hook',"\imath^A_2"']\\[1em]
B_1\ar[r,hook,"\imath^B_1"]\ar[u,"p_1"']&B_1\amalg B_2\ar[u,"p_1\amalg p_2"]&B_2\ar[l,hook',"\imath^B_2"']\ar[u,"p_2"]
\end{tikzcd}
\end{center}
This defines a functor $L:\mathcal{C}\times\mathcal{C}\to\mathcal{C}$ taking $(A_1,A_2)$ to $A_1\amalg A_2$ and $(p_1,p_2)\in\mathcal{C}\times\mathcal{C}\big((B_1,B_2),(A_1,A_2)\big)$ to $p_1\amalg p_2$. We will prove that $R:\mathcal{C}\to\mathcal{C}\times\mathcal{C}$ taking $X$ to $(X,X)$ and $f\in\mathcal{C}(X,Y)$ to $(f,f):(X,X)\to (Y,Y)$ is the right adjoint of $L$. For this we will construct natural transformations $\psi,\varphi$ as in the Definition~\ref{def:adj}. Let
\[\psi_{(A_1,A_2),X}:\mathcal{C}(A_1\amalg A_2,X)\to\mathcal{C}\times\mathcal{C}\big((A_1,A_2),(X,X)\big)\]
be the map taking $g$ to $(g\circ\imath_1,g\circ\imath_2)$. And let
\[\varphi_{(A_1,A_2),X}:\mathcal{C}\times\mathcal{C}\big((A_1,A_2),(X,X)\big)\to\mathcal{C}(A_1\amalg A_2,X)\]
be the map taking $(q_1,q_2)$ to $[q_1,q_2]$. It is easy to see that $\psi_{(A_1,A_2),X}$ and $\varphi_{(A_1,A_2),X}$ are inverses of each other, so it remains to show that one of them, say $\varphi_{(A_1,A_2),X}$ is natural in $\big((A_1,A_2),X\big)\in(\mathcal{C}\times\mathcal{C})^{\mathrm{op}}\times\mathcal{C}$. Let
\[(p_1,p_2)\in\mathcal{C}\times\mathcal{C}\big((B_1,B_2),(A_1,A_2)\big)\text{ and }f\in\mathcal{C}(X,Y)\]
then the commutativity of the two diagrams defining $[q_1,q_2]$ and $p_1\amalg p_2$ and the uniqueness of $[q_1\circ p_1,q_2\circ p_2]$ gives:
\[[q_1,q_2]\circ p_1\amalg p_2=[q_1\circ p_1,q_2\circ p_2]\]
which is exactly the first naturality condition we wanted:
\begin{center}
\begin{tikzcd}
(A_1,A_2)&&\mathcal{C}(A_1\amalg A_2,X)\ar[d,"-\circ p_1\amalg p_2"']&[1.3cm]\mathcal{C}\times\mathcal{C}\big((A_1,A_2),(X,X)\big)\ar[l,"\varphi_{(A_1{,}A_2){,}X}"']\ar[d,"-\circ(p_1{,}p_2)"]\\[1em]
(B_1,B_2)\ar[u,"(p_1{,}p_2)"]&&\mathcal{C}(B_1\amalg B_2,X)&\mathcal{C}\times\mathcal{C}\big((B_1,B_2),(X,X)\big)\ar[l,"\varphi_{(B_1{,}B_2){,}X}"']
\end{tikzcd}
\end{center}
Moreover, since $[f\circ p_1,f\circ p_2]$ is unique, it is true that:
\[f\circ[q_1,q_2]=[f\circ q_1,f\circ q_2]\]
which is the second naturality condition we wanted:
\begin{center}
\begin{tikzcd}
X\ar[d,"f"']&&\mathcal{C}(A_1\amalg A_2,X)\ar[d,"f\circ-"']&[1.3cm]\mathcal{C}\times\mathcal{C}\big((A_1,A_2),(X,X)\big)\ar[l,"\varphi_{(A_1{,}A_2){,}X}"']\ar[d,"(f{,}f)\circ-"]\\[1em]
Y&&\mathcal{C}(A_1\amalg A_2,Y)&\mathcal{C}\times\mathcal{C}\big((A_1,A_2),(Y,Y)\big)\ar[l,"\varphi_{(A_1{,}A_2){,}Y}"']
\end{tikzcd}
\end{center}
\item For the second adjunction, let $L=\Delta:\mathcal{C}\to\mathcal{C}\times\mathcal{C}$ taking $A$ to $(A,A)$ and $p\in\mathcal{C}(B,A)$ to $(p,p):(B,B)\to(A,A)$. We will now try to construct its right adjoint, using Corollary~\ref{cor:adj_construction}. Let $X=(X_1,X_2)$ be any object of $\mathcal{C}\times\mathcal{C}$ and let $(\tilde{X},\varepsilon_X)$ be a terminal object in the category $\commaCat{L}{(X_1,X_2)}$, which means that $\varepsilon_X\in\mathcal{C}\times\mathcal{C}\big((\tilde{X},\tilde{X}),(X_1,X_2)\big)$, i.e. $\varepsilon_X=(\varepsilon^X_1,\varepsilon^X_2)$ for some maps $\varepsilon^X_1:\tilde{X}\to X_1$ and $\varepsilon^X_2:\tilde{X}\to X_2$. Then, the universal property $(\tilde{X},\varepsilon_X)$ should satisfy is the following:

For every object $A$ and every map $(g_1,g_2):(A,A)\to(X_1,X_2)$, there exists a unique map $\left<g_1,g_2\right>:A\to\tilde{X}$ such that $(g_1,g_2)=(\varepsilon^X_1,\varepsilon^X_2)\circ(\left<g_1,g_2\right>,\left<g_1,g_2\right>)$, which is exactly the universal property that $X_1\times X_2$, together with the projection maps $\pi_1,\pi_2$ satisfy:
\begin{center}
\begin{tikzcd}
&A\ar[dl,"g_1"']\ar[dr,"g_2"]\ar[d,"\left<g_1{,}g_2\right>",near end]\\[1em]
X_1&X_1\times X_2\ar[l,"\pi_1",two heads]\ar[r,"\pi_2"',two heads]&X_2
\end{tikzcd}
\end{center}
Moreover, let $(f_1,f_2)\in\mathcal{C}\times\mathcal{C}\big((X_1,X_2),(Y_1,Y_2)\big)$. Then, the morphism $f_1\times f_2:=\left<f_1\circ\pi^X_1,f_2\circ\pi^X_2\right>$ is the unique morphism in $\commaCat{L}{(Y_1,Y_2)}$
\[\big(X_1\times X_2,(f_1\circ\pi^X_1,f_2\circ\pi^X_2)\big)\to\big(Y_1\times Y_2,(\pi^Y_1,\pi^Y_2)\big)\]
since the following diagram commutes:
\begin{center}
\begin{tikzcd}
X_1\ar[d,"f_1"']&X_1\times X_2\ar[l,two heads,"\pi^X_1"']\ar[r,two heads,"\pi^X_2"]\ar[d,"f_1\times f_2"]&X_2\ar[d,"f_2"]\\[1em]
Y_1&Y_1\times Y_2\ar[l,two heads,"\pi^Y_1"']\ar[r,two heads,"\pi^Y_2"]&Y_2
\end{tikzcd}
\end{center}
This means that the functor $\times:\mathcal{C}\times\mathcal{C}\to\mathcal{C}$ taking $(X_1,X_2)$ to $X_1\times X_2$ and $(f_1,f_2)$ to $f_1\times f_2$ is the right adjoint of $\Delta$.\qedhere
\end{itemize}
\end{proof}
\begin{example} Let $\mathcal{C}$ a locally small category that has binary products and strict initial objects. Moreover, let $\mathcal{C}$ have objects $X_1,X_2$ and $Y$, such that $\mathcal{C}(X_1\times X_2,Y)$ has at least two elements $h_1,h_2$. Then the functor $\times:\mathcal{C}\times\mathcal{C}\to\mathcal{C}$, as defined above, does not have a right adjoint.
\end{example}
\begin{proof} An initial object $\varnothing$ is called a strict initial object, if the only morphisms into $\varnothing$ are isomorphisms, i.e. $A\to\varnothing$ implies $A\cong\varnothing$. We are going to use the following two facts:
\begin{itemize}
\item If $\varnothing$ is a strict initial object and $A$ any object, then $\varnothing\times A\cong A\times\varnothing\cong\varnothing$.
\item If $\top$ is a terminal object and $A$ any object, then $\top\times A\cong A\times\top\cong A$.
\end{itemize}
Suppose that the functor $\times:\mathcal{C}\times\mathcal{C}\to\mathcal{C}$ has a right adjoint sending $Y$ to $(Y_1,Y_2)$. Then, the characterization of Corollary~\ref{cor:adj_construction} requires $Y_1,Y_2$ to satisfy the following universal property:

There exists some $\varepsilon_Y:Y_1\times Y_2\to Y$ that satisfies the following: For every $(A,B)\in\mathcal{C}\times\mathcal{C}$ and every morphism $f:A\times B\to Y$, there exist unique morphisms $p:A\to Y_1$ and $q:B\to Y_2$ such that $f=\varepsilon_Y\circ(p\times q)$, i.e. the following diagram commutes:
\begin{center}
\begin{tikzcd}
&A\times B\ar[dl,"\forall f"']\ar[d,"p\times q"]&&(A,B)\ar[d,"\exists_{!}(p{,}q)"]\\
Y&Y_1\times Y_2\ar[l,"\varepsilon_Y"]&&(Y_1,Y_2)
\end{tikzcd}
\end{center}
For $A\cong\varnothing$, $A\times B\cong\varnothing$ so there is a unique map $\phi:A\times B\to Y$. The universal property gives us unique maps $\phi:A\to Y_1$ and $q:B\to Y_2$ such that $\phi=\varepsilon\circ(\phi\times q)$. But this equation is true for every choice of $q:B\to Y_2$, so for $q$ to be unique, there should be a unique map $B\to Y_2$ for every object $B$, which means that $Y_2\cong\top$. Then, choosing $B\cong\varnothing$ the same argument gives $Y_1\cong\top$. So, finally we have $Y_1\times Y_2\cong\top$, which means that there exists a unique map $t:X_1\times X_2\to Y_1\times Y_2$. For the maps $h_1,h_2:X_1\times X_2\to Y$ the universal property gives $h_1=\varepsilon_Y\circ t=h_2$, which is against the choice of $h_1,h_2$. This proves that the right adjoint cannot exist.
\end{proof}
\begin{remark} If we disregard the trivial cases, where the category $\mathcal{C}$ is so small that can afford the right adjoint to always be a terminal object, the above example shows that if a category has binary products and initial objects and its initial objects are strict, then there is not right adjoint to $\times$. Many categories we are interested in do have these properties, like $\mathrm{Set}$ and $\mathrm{Top}$, where the initial object is the empty set, which is not the codomain of any maps, except for the empty map $\emptyset:\emptyset\to\emptyset$.
\end{remark}
Although we failed to continue the line of adjunctions $\amalg\dashv\Delta\dashv\times$, there is a way to (sometimes) construct a right adjoint of the functor $-\times A:\mathcal{C}\to\mathcal{C}$, for some fixed object $A$ of $\mathcal{C}$.
\begin{definition}\label{def:exp_obj} Let $\mathcal{C}$ be a locally small category that has binary products and $A$ be some object of $\mathcal{C}$. Let $-\times A:\mathcal{C}\to\mathcal{C}$ be the functor defined by
$X\mapsto X\times A$ and $f\mapsto f\times1_A$. Its right adjoint, if it exists, is called an \emph{exponential object}, the counit of the adjunction is called an \emph{evaluation map} and they are denoted respectively by:
\[X^A\qquad\text{ and }\qquad\mathrm{ev}:X^A\times A\to X\]
\end{definition}
\begin{remark}\label{rem:universal_prop_exp} The exponential object together with the evaluation map $(X^A,\mathrm{ev})$ satisfy the following universal property (Corollary~\ref{cor:adj_construction}): For every object $Z$ and every morphism $g:Z\times A\to X$, there exists a unique morphism $g^{\#}:Z\to X^A$ such that $g=\mathrm{ev}\circ(g^{\#}\times1_A)$, i.e. the following diagram commutes:
\begin{center}
\begin{tikzcd}
&Z\times A\ar[dl,"g"']\ar[d,"g^{\#}\times1_A"]&&Z\ar[d,"g^{\#}"]\\
X&X^A\times A\ar[l,"\mathrm{ev}"']&&X^A
\end{tikzcd}
\end{center}
\end{remark}
The name and the notation of exponential objects is justified if we examine how they look in $\mathrm{Set}$:
\begin{proposition} In the category $\mathrm{Set}$ there exist all exponential objects and:
\begin{itemize}
\item $X^A$ is the set $\mathrm{Set}(A,X)$.
\item $\mathrm{ev}:X^A\times A\to X$ is the function taking $(f,a)$ to $f(a)$.
\end{itemize}
\end{proposition}
\begin{proof} Let $A,X$ be any sets. We will prove that the set $\mathrm{Set}(A,X)$ together with the function $\mathrm{eval}:\mathrm{Set}(A,X)\times A\to X$ taking $(f,a)$ to $f(a)$ is an exponential object. Let $Z$ be any set and $g:Z\times A\to X$ be any function. Let $g^{\#}(z)=g_z\in\mathrm{Set}(A,X)$, for any $z\in Z$. We want the following diagram to commute:
\begin{center}
\begin{tikzcd}
&Z\times A\ar[dl,"g"']\ar[d,"g^{\#}\times1_A"]&&Z\ar[d,"g^{\#}"]\\
X&\mathrm{Set}(A,X)\times A\ar[l,"\mathrm{eval}"']&&\mathrm{Set}(A,X)
\end{tikzcd}
\end{center}
This imposes the following condition for every $(z,a)\in Z\times A$:
\[g(z,a)=\mathrm{eval}(g^{\#}(z),1_A(a))=\mathrm{eval}(g_z,a)=g_z(a)\]
so, there is exactly one choice for the function $g_z$ that makes the diagram commute, which is to satisfy $g_z(a)=g(z,a)$ for every input $a\in A$.
\end{proof}

Everything we explored in this section has its dual analog. From the Yoneda Lemma to the alternative characterization of the right adjoint. It would be a good exercise for the reader to dualize everything and eventually get a characterization for the left adjoint of a given functor. The reason we chose to present the results in this asymmetric way is that sometimes this perfect symmetry on the categorical level can be misleading. The categories we are usually interested in are not always symmetrical under dualization. For example, we could categorically dualize the Definition~\ref{def:exp_obj} to get the notion of the \emph{co-exponential object}, but then the categories we are interested in, like $\mathrm{Set}$ and $\mathrm{Top}$ do not have co-exponential objects at all.

\section{Compact-Open Topology}
In this section we first define a natural way to topologize the set $\mathrm{Top}(A,X)$ and then we are going to restrict our attention to only some good topological spaces $A$, for which there exist the exponential objects $X^A$ for every $X$, i.e. the evaluation map and $g^{\#}$ are continuous. We are going to follow the beginning of Chapter~11 of the book \cite{top_rotman}.

\begin{definition} Let $A,X$ be two topological spaces. For every $A_0\subseteq A$ and $X_0\subseteq X$, define the following set of continuous functions:
\[S(A_0,X_0):=\big\{f\in\mathrm{Top}(A,X):f(A_0)\subseteq X_0\big\}\]
The \emph{compact-open topology} on $\mathrm{Top}(A,X)$ is the topology generated by the sub-basis:
\[\mathcal{B}:=\big\{S(K,U):K\subseteq A\text{ compact},\ U\subseteq X\text{ open}\big\}\]
\end{definition}

In this section we are going to be interested mostly in locally compact Hausdorff topological spaces:
\begin{definition} A topological space $X$ is called \emph{locally compact}, if every point has a compact neighborhood. That is, for every $x\in X$ there exists some open $U$ and compact $K$ such that
\[x\in U\subseteq K\]
\end{definition}

\begin{lemma}\label{lem:smaller_compact_nbh} Let $X$ be a locally compact, Hausdorff topological space. Moreover, let $x\in X$ and $U$ an open neighborhood of $x$. Then, there exists some open neighborhood $V$ of $x$, such that $V^-$ is compact and:
\[x\in V\subseteq V^-\subseteq U\]
\end{lemma}
\begin{proof} Let $x$ be any point of $X$. Since $X$ is locally compact, there exists some compact set $K$, such that $x\in K^{\circ}$. Define the open set $\tilde{U}:=U\cap K^{\circ}$. Since $X$ is Hausdorff, every compact subset of $X$ is also closed, so $(K^{\circ})^-=K$. This means that $\tilde{U}^-\subseteq U^-\cap K\subseteq K$. Since $X$ is Hausdorff, every closed subset of some compact set is also compact, so $\tilde{U}^-$ is compact. This gives us that $\partial\tilde{U}$ is also compact as a closed subset of $\tilde{U}^-$.

\begin{minipage}{.5\linewidth}
Since $X$ is Hausdorff, there exist open sets $A,B\subseteq X$ such that $A\cap B=\emptyset$, $x\in A$ and $\partial\tilde{U}\subseteq B$. Let
\[\tilde{A}:=A\cap\tilde{U}\text{ , }\tilde{B}:=B\cup\tilde{U}^c\]
and notice that $\partial\tilde{U}\subseteq B$ means that $\tilde{B}=B\cup(\tilde{U}^c)^{\circ}$ and thus $\tilde{B}$ is also open. So, $\tilde{A},\tilde{B}$ are also two open disjoint sets separating $x$ from $\partial\tilde{U}$. This gives:
\[x\in\tilde{A}\subseteq\tilde{A}^-\subseteq\tilde{B}^c=B^c\cap\tilde{U}\subseteq\tilde{U}\subseteq U\]
\end{minipage}%\hfill
\begin{minipage}{.5\linewidth}
\begin{center}
\begin{tikzpicture}
%	\draw[white] (-2,-2)--(5,5);
	\coordinate (x0) at (0,0) {};
	\coordinate (x1) at (3.0,0.0) {};
	\coordinate (x2) at (3.5,3.5) {};
	\coordinate (x3) at (0.0,3.0) {};
	\coordinate (x) at (1.5,1.5) {};
	\node[bag={1.1}{1.4}{1.0cm}] at (x) {};
	\node[bag={0.5}{0.8}{-0.5cm}] at (x) {};
	\node[vertex] at (x) {};
	\draw[dashed] (x0)--(x1)--(x2)--(x3)--cycle;
	\begin{pgfonlayer}{background}
		\draw[bubble={0.7cm}{0.02cm}] plot[closed hobby] coordinates {(x0) (x1) (x2) (x3)};
	\end{pgfonlayer}
	\node at (4.6,1.5) {$X$};
	\node at (3.7,2.8) {$\partial\tilde{U}$};
	\node[right] at (x) {$\tilde{A}$};
	\node[left] at (x) {$x$};
	\node[fill=transparent!2,opacity=0.8] at (1.6, 3.3) {$\tilde{B}$};
\end{tikzpicture}
\end{center}
\end{minipage}

which proves the assertion for $V=\tilde{A}$.
\end{proof}
\begin{corollary}\label{cor:sandwitch} Let $X$ be a locally compact Hausdorff space. Moreover, let $K,U\subseteq X$ such that $K$ is compact, $U$ is open and $K\subseteq U$. Then, there exists some open $V$, such that $V^-$ is compact and
\[K\subseteq V\subseteq V^-\subseteq U\]
\end{corollary}
\begin{proof}
For every $x\in K$, we can use lemma~\ref{lem:smaller_compact_nbh} and find some open neighborhood $V_x$ of $x$, such that $V_x^-$ is compact and $x\in V_x\subseteq V_x^-\subseteq U$. So, we have that $K\subseteq\bigcup_{x\in K}V_x$ and because $K$ is compact, there exists a finite set of indices $\{x_1,\ldots,x_n\}$, such that $K\subseteq\bigcup_{i=1}^nV_{x_i}$. Define $V=\bigcup_{i=1}^nV_{x_i}$ and notice that $V^-=\bigcup_{i=1}^nV_{x_i}^-$ is compact.
\end{proof}
\begin{theorem}\label{thm:top_exponential} Let $A$ be a locally compact, Hausdorff topological space. Then all exponential objects $(-)^A$ exist and:
\begin{itemize}
\item $X^A$ is the space $\mathrm{Top}(A,X)$ equipped with the compact-open topology.
\item $\mathrm{ev}:X^A\times A\to X$ is the map taking $(f,a)$ to $f(a)$.
\end{itemize}
\end{theorem}
\begin{proof} We are going to prove this in the following two steps:
\begin{enumerate}
\item The first step is to show that the map $\mathrm{ev}:\mathrm{Top}(A,X)\times A\to X$ is continuous. Let $(f,a)\in\mathrm{Top}(A,X)\times A$ and let $U\subseteq X$ be an open neighborhood of $\mathrm{ev}(f,a)=f(a)\in X$. Since $f$ is continuous, $f^{-1}(U)$ is open in $A$. Since $A$ is locally compact Hausdorff, we can use lemma~\ref{lem:smaller_compact_nbh} to find an open set $V$ such that $V^-$ is compact and $x\in V\subseteq V^-\subseteq f^{-1}(U)$. So, the set $W:=S(V^-,U)\times V$ is an open neighborhood of $(f,a)\in\mathrm{Top}(A,X)\times A$ and for any $(f',a')\in W$ we have:
\[\mathrm{ev}(f',a')=f'(a')\in f'(V)\subseteq f'(V^-)\subseteq U\]
which means that $\mathrm{ev}(W)\subseteq U$ and thus proves that $\mathrm{ev}$ is continuous.
\item The second step is to show that these choices for $X^A$ and $\mathrm{ev}$ satisfy the universal property of the remark~\ref{rem:universal_prop_exp}. Let $Z$ be any topological space and let $g:Z\times A\to X$ to be any continuous map. We then define $g^{\sharp}:Z\to\mathrm{Top}(A,X)$ to be the map taking $z$ to $g_z\in\mathrm{Top}(A,X)$, where $g_z$ is defined to be the unique map for which it is true that $g_z(a)=g(z,a)$ for every $a\in A$. Let us further divide the proof into steps:
\begin{enumerate}
\item First, we should show that $g^{\sharp}$ is well defined, i.e. that $g_z$ is a continuous function from $A$ to $X$, for every $z\in Z$. It is clear that $g_z$ has the right domain and codomain, so we only have to show that it is indeed continuous. Let us fix some $z\in Z$ and define $i_z:A\to Z\times A$ to be the function taking $a$ to $(z,a)$. This is continuous, as the composition of the isomorphism $\pi_2^{-1}:A\to\{z\}\times A$ and the inclusion $\{z\}\times A\to Z\times A$. So, $g_z$ is also continuous, as it is the composition $g_z=g\circ i_z$.
\item Then, we should show that $g^{\sharp}$ is a continuous map. For that, fix some $z\in Z$ and let $S(K,U)\subseteq\mathrm{Top}(A,X)$ be a subbasic open neighborhood of $g^{\sharp}(z)$. This means that $K\subseteq A$ is compact, $U\subseteq X$ is open and $g_z(K)\subseteq U$, i.e. $\{z\}\times K\subseteq g^{-1}(U)$. Since $g$ is continuous, $g^{-1}(U)\subseteq Z\times A$ is open. Our next goal is to find some open neighborhood $V$ of $z$ such that $V\times K\subseteq g^{-1}(U)$. For every $k\in K$ there exists some subbasic open set $V_k\times A_k\subseteq g^{-1}(U)$ containing $k$. This means that $\{z\}\times K\subseteq\bigcup_{k\in K}V_k\times A_k$ and since $\{z\}\times K\cong K$ is compact, there exists a finite set of indices $\{k_1,\ldots,k_n\}$ such that $\{z\}\times K\subseteq\bigcup_{i=1}^nV_{k_i}\times A_{k_i}$. Define $V:=\bigcap_{i=1}^nV_{k_i}$ and notice that for every $k'\in K$ there exists some $i$ such that $V\times\{k'\}\subseteq V_{k_i}\times A_{k_i}$. It is the same $i$ for which $(z,k')\in V_{k_i}\times A_{k_i}$. This means that $V\times K\subseteq g^{-1}(U)$ and so for every $z'\in V$ we have:
\[g_{z'}(K)=g(\{z'\}\times K)\subseteq g(V\times K)\subseteq U\]
and thus $g^{\sharp}(V)\subseteq S(K,U)$, which proves that $g^{\sharp}$ is continuous.\qedhere
\end{enumerate}
\end{enumerate}
\end{proof}

\begin{corollary}\label{cor:exp_obj_prop} Let $A$ be a locally compact, Hausdorff topological space and $X$ any topological space. If we topologize $\mathrm{Top}(A,X)$ with the compact-open topology, then the map $\mathrm{ev}:\mathrm{Top}(A,X)\times A\to X$ taking $(f,a)$ to $f(a)$ is continuous. Moreover, for every topological space $Z$ and any function $g:Z\times A\to X$, we can define the function $g^{\sharp}:Z\to\mathrm{Top}$ taking $z$ to $g_z:A\to X$ with $g_z(a)=g(z,a)$. It is then true that $g$ is continuous if and only if $g^{\sharp}$ is continuous.
\end{corollary}
\begin{proof} We already proved in lemma~\ref{lem:smaller_compact_nbh} that $\mathrm{ev}$ is continuous and that the continuity of $g$ implies the continuity of $g^{\sharp}$. We only have to show that if $g^{\sharp}$ is continuous, $g$ is also continuous, which is true, since $g=\mathrm{ev}\circ(g^{\sharp}\times1_A)$.
\end{proof}

We have now proved that there exist some sufficient conditions for $A$ in order for the sets $\mathrm{Top}(A,X)$ to have a nice topology, for all $X$. An interesting case arises when $X=A$ and when we restrict ourselves to the topological subspace of the automorphisms of $A$, denoted by $\mathrm{Hom}(A)$ or $\mathrm{Aut}(A)$, i.e. the set of all invertible bicontinuous functions from $A$ to $A$, topologised as a subspace of $\mathrm{Top}(A,A)$. In this set, there exists a natural binary operation, namely the composition, which can easily be proven to be continuous:
\begin{proposition} Let $A$ be a locally compact Hausdorff space and let $\mathrm{Hom}(A)$ be topologized with the compact-open topology. Then the composition:
\[\circ:\mathrm{Hom}(A)\times\mathrm{Hom}(A)\to\mathrm{Hom}(A)\]
is continuous.
\end{proposition}
\begin{proof} Let $f,g\in\mathrm{Hom}(A)$ and let $S(K,U)$ be a subbasic neighborhood of $f\circ g$. This means that $K$ is compact, $U$ is open and $g(K)\subseteq f^{-1}(U)$. Notice that $f^{-1}(U)$ is open and $g(K)$ is compact because $f,g$ are continuous. Thus, corollary~\ref{cor:sandwitch} gives us an open $V$, such that $V^-$ is compact and
\[g(K)\subseteq V\subseteq V^-\subseteq f^{-1}(U)\]
Then, $S(V^-,U)\times S(K,V)$ is an open neighborhood of $(f,g)$ and for every $(f',g')\in S(V^-,U)\times S(K,V)$ we have that $f'\circ g'\in S(K,U)$, which proves the assertion.
\end{proof}

It is only natural to ask ourselves if $\mathrm{Hom}(A)$ equipped with the compact-open topology, together with the composition is a topological group. For that we only need to check the continuity of the inverse function:
\[(-)^{-1}:\mathrm{Hom}(A)\to\mathrm{Hom}(A)\]
Interestingly enough this is not true in general, as exhibited in \cite{counterexample}. Based on their counterexample, we present a somewhat simpler construction in \ref{ex:counterexample} around the same pathology, also involving fiber bundles. Our goal here is to finish this chapter by giving some sufficient conditions for $\Hom(A)$ to be a topological group. The proof is based on the proof of Theorem~4 in \cite{top_group}. We first need the following definition.

\begin{definition} A topological space $X$ is called \emph{locally connected}, if for every $x\in X$ and $U$ an open neighborhood of $x$, there exists some open neighborhood $V$ of $x$, such that $V$ is connected and:
\[x\in V\subseteq U\]
\end{definition}
\begin{remark} Of course, local connectedness does not imply connectedness, but notice that the reverse isn't true as well. Take for example this subset of $\mathbb{R}^2$: \[\big([0,1]\times\{0\}\big)\cup\big(\{0\}\times[0,1]\big)\cup\bigcup_{n=1}^{\infty}\left(\left\{\frac{1}{n}\right\}\times[0,1]\right)\]
equipped with the subspace topology. Then, it is connected, but all open neighborhoods of $(1,0)$ contained in the open upper half-space are not connected.
\end{remark}
Notice that we can combine all properties in the following way:
\begin{lemma}\label{lem:nice_local_base} Let $X$ be a locally connected, locally compact Hausdorff space. Moreover, let $x\in X$ and $U$ an open neighborhood of $x$. Then there exists some open neighborhood $V$ of $x$, such that $V$ is connected, $V^-$ is compact and:
\[x\in V\subseteq V^-\subseteq U\]
\end{lemma}
\begin{proof} Indeed, since $X$ is locally compact Hausdorff, lemma~\ref{lem:smaller_compact_nbh} gives us some open $V_1$ such that $V_1^-$ is compact and $x\in V_1\subseteq V_1^-\subseteq U$. Then, since $X$ is locally connected we can find some open $V$ such that $V$ is connected and $V\subseteq V_1$, i.e.
\[x\in V\subseteq V^-\subseteq V_1^-\subseteq U\]
Then, since $X$ is Hausdorff and $V^-$ is a closed subset of the compact $V_1^-$, $V^-$ is also compact, which proves the assertion.
\end{proof}
In order to prove the final result, it will be useful to work with the following smaller subbase for the topology of $\mathrm{Hom}(A)$:
\begin{lemma}\label{lem:alt_subbasis} Let $A$ be locally connected, locally compact Hausdorff space. Then, the following set is a subbase of the compact-open topology of $\mathrm{Hom}(A)$:
\[\big\{S(V^-,U):U,V\subseteq A,\ V\neq\emptyset\text{ open and connected},\ V^-\text{ compact},\ U\text{ open}\big\}\]
\end{lemma}
\begin{proof} Let $f_0\in\mathrm{Hom}(A)$ and $S(K,U)$ be an open neighborhood of $f_0$ in the usual subbase. For every $x\in K$, since $A$ is locally connected, locally compact Hausdorff, lemma~\ref{lem:nice_local_base} gives us an open neighborhood $V_x$ of $x$, with $V_x$ being connected and $V_x^-$ being compact, such that $V_x^-\subseteq f_0^{-1}(U)$. So, we have $K\subseteq\bigcup_{x\in K}V_x$ and since $K$ is compact, there exists a finite set of indices $\{x_1,\ldots,x_n\}$ such that $K\subseteq\bigcup_{i=1}^nV_{x_i}$. It suffices now to prove that:
\[f_0\in\bigcup_{i=1}^nS(V_{x_i}^-,U)\subseteq S(K,U)\]
First, due to the choice of $V_x$, it is trivially true that $f_0\in S(V_{x_i})$ for every $i\in[n]$. Moreover, let $f\in\bigcup_{i=1}^nS(V_{x_i}^-,U)$ and $x\in K$. Then, there is some $i\in[n]$ such that $x\in V_{x_i}$ and thus $f(x)\in f(V_{x_i})\subseteq U$. So, $f(K)\subseteq U$ which proves the assertion.
\end{proof}
\begin{theorem}\label{thm:inverse} Let $A$ be a locally connected, locally compact Hausdorff space and let $\mathrm{Hom}(A)$ be topologised with the compact-open topology. Then the inverse function:
\[(-)^{-1}:\mathrm{Hom}(A)\to\mathrm{Hom}(A)\]
is continuous.
\end{theorem}
\begin{proof} Let $f\in\mathrm{Hom}(A)$ and let $S(V^-,U)$ be a subbasic open set in the neighborhood of $f^{-1}$, using the alternative subbasis defined in lemma~\ref{lem:alt_subbasis}. This means that $V$ is non-empty, open and connected, $V^-$ is compact, $U$ is open and $f^{-1}(V^-)\subseteq U$. Since $V\neq\emptyset$, there exists some $x\in V$ we are going to fix. Moreover, since $f^{-1}$ is continuous, $f^{-1}(V^-)$ is compact and using corollary~\ref{cor:sandwitch} twice we get some open $W,Y$ such that $W^-,Y^-$ are compact and a chain between $f^{-1}(V^-)$ and $U$, like this:
\[f^{-1}(x)\in f^{-1}(V)\subseteq f^{-1}(V^-)\subseteq W\subseteq W^-\subseteq Y\subseteq Y^-\subseteq U\]
We then define the open set
\[O:=S\big(W^c\cap Y^-,(V^-)^c\cap f(U)\big)\cap S\big(\{f^{-1}(x)\},V\big)\subseteq\mathrm{Hom}(A)\]
It is trivially true that $f\in O$ and it remains to show that for every $g\in O$ we have $g^{-1}\in S(V^-,U)$. Let us indeed choose some $g\in O$. The restrictions it satisfies are captured in the following picture:
\begin{center}
\begin{tikzpicture}[yscale=0.7]
\def\colorInside{red!40!white}
\def\colorAround{gray!50!white}
\coordinate (x)  at ( 0.0, 0.0);
\coordinate (V1) at ( 0.8,-0.2);
\coordinate (V2) at (-1.2, 1.0);
\coordinate (V3) at (-2.0,-0.7);
\coordinate (W1) at ( 3.2,-0.5);
\coordinate (W2) at ( 1.6, 1.4);
\coordinate (W3) at (-2.7, 1.2);
\coordinate (W4) at (-1.9,-2.0);
\coordinate (Y1) at ( 4.5,-0.4);
\coordinate (Y2) at ( 4.0, 2.5);
\coordinate (Y3) at ( 0.7, 1.6);
\coordinate (Y4) at (-3.0, 2.3);
\coordinate (Y5) at (-4.2,-1.7);
\coordinate (Y6) at (-0.4,-2.6);

\draw[bubble={1.5cm}{1pt}] plot[closed hobby] coordinates {(6,-1) (6,1) (0,3.2) (-6,1) (-6,-1) (0,-3.2)};
\node (A1) at (7.2,0) {$A$};
\draw[bubble={1.5cm}{1pt}] plot[closed hobby] coordinates {(Y1) (Y2) (Y3) (Y4) (Y5) (Y6)};
\node at (6,1) {$U$};
\draw[fill=\colorAround] (Y1)--(Y2)--(Y3)--(Y4)--(Y5)--(Y6)--cycle;
\node[right] at (Y1) {$Y^-$};
\draw[fill=white] (W1)--(W2)--(W3)--(W4)--cycle;
\node[above right] at (W1) {$W^-$};
\draw (V1)--(V2)--(V3)--cycle;
\node[right] at (V1) {$f^{-1}(V^-)$};
\node[vertex,label={left:$f^{-1}(x)$},draw=\colorInside,fill=\colorInside] at (x) {};

\begin{scope}[yshift=-9.5cm]
%\coordinate (ax)  at ( 1.0, 1.0);
\coordinate (aV1) at ( 2.0,-0.2);
\coordinate (aV2) at ( 1.7, 2.0);
\coordinate (aV3) at (-1.0, 0.9);
\coordinate (aV4) at (-2.0,-0.7);

\draw[bubble={1.5cm}{1pt}] plot[closed hobby] coordinates {(6,-1) (6,1) (0,3.2) (-6,1) (-6,-1) (0,-3.2)};
\node (A2) at (7.2,0) {$A$};
\draw[bubble={1.5cm}{1pt},fill=\colorAround,draw=\colorAround] plot[closed hobby] coordinates {(2.5,-2) (2,1) (0,2) (-2,0) (-2.5, -1) (0,-2)};
\node at (4,0.5) {$f(U)$};
\draw[fill=\colorInside] (aV1)--(aV2)--(aV3)--(aV4)--cycle;
\node[below] at (aV1) {$V^-$};
%\node[vertex,label={below:$x$}] at (ax) {};
\end{scope}

\draw[->] (A1)--(A2) node [midway,left=3pt] {$g$};
\end{tikzpicture}
\end{center}
The first restriction on $g$ is that $g\big(W^c\cap Y^-\big)\subseteq(V^-)^c\cap f(U)$. By taking the complement, we get that $g\big(W\cup(Y^-)^c\big)\supseteq V^-\cup f(U)^c$, i.e:
\[V^-\subseteq V^-\cup f(U)^c\subseteq g\big(W\cup(Y^-)^c\big)=g(W)\cup g\big((Y^-)^c\big) \]
Since $W\subseteq Y^-$ and $g$ is injective, $g(W)$ and $g\big((Y^-)^c\big)$ are two disjoint open subsets in $A$. Notice that $V^-$ is also connected, since $V$ is connected, which means that $V^-$ lies inside one of them. In order to choose, we check the second restriction on $g$, which is that $g(f^{-1}(x))\in V^-$. Since $f^{-1}(x)\in f^{-1}(V^-)\subseteq W$, we deduce that $V^-\cap g(W)\neq\emptyset$, which means that $V^-\subseteq g(U)$, or equivalently $g^{-1}\in S(V^-,U)$, which proves that $(-)^{-1}$ is continuous.
\end{proof}















%\section{CW Complexes}
%A CW structure on some space $X$ is usually defined recursively, as an inductive ``glueing'' of cells of some dimension $k$ to the previous, lower dimensional, skeleton of $X$, forming a new, $k$-dimensional, skeleton of $X$. A space $X$ may exhibit many different CW structures, but the existence of one suffices in order for $X$ to be characterized as CW complex. Here, we are going to use the following formal formulation of the above definition.
%
%\begin{definition} A topological space $X$ is a \ul{CW-complex}, if there exists some filtration
%\[\emptyset=X_{-1}\subseteq X_0\subseteq X_1\subseteq X_2\subseteq\cdots\subseteq X\]
%such that:
%\begin{b_item}
%\item $X=\varinjlim X_i$ with respect to all inclusion maps.
%\item For every $n\geq0$ there exists a pushout diagram in the category of topological spaces:
%\begin{center}
%\begin{tikzcd}
%\displaystyle\coprod_{e\in\pi_0(X_n\setminus{X_{n-1}})}S^{n-1}\ar[r,"\coprod_e\varphi_e"]\ar[d,hook,"\coprod_ej_e"']\ar[dr,phantom, very near start,"\ulcorner"]&[4em]X_{n-1}\ar[d,hook,"i_n"]\\[2em]
%\displaystyle\coprod_{e\in\pi_0(X_n\setminus{X_{n-1}})}D^n\ar[r,"\coprod_e\varphi_e"']&X_n\\
%\end{tikzcd}
%\end{center}
%where $j_e:S^{n-1}\to D^n$ is the usual inclusion map and $i_n:X_{n-1}\to X_n$ is the inclusion map given by the filtration.
%\end{b_item}
%\end{definition}
%
%A filtration of a topological space $X$, making $X$ a CW-complex is called a \ul{CW-structure} of $X$. Moreover, given a filtration of $X$ like in the above definition, the sets $\varphi_e\large({(D^n)}^{\circ}\large)$ (resp. $\varphi_e(D^n)$) are called the $n$-dimensional \ul{open} (resp. \ul{closed}) \ul{cells} of this CW-structure. Recall the following known facts regarding the dependencies between CW-complexes, structures and cells.
%
%\begin{notes}
%\begin{i_enum}
%\item A CW-complex $X$ can have more than one CW-structures, even structures having different number of $n$-dimensional open cells each.
%\item For a particular CW-structure of $X$, the maps $\varphi_e$ and $\varphi_e$ are not predetermined by the structure, which means that there can be more than one choices for them. For example, one could always precompose $\varphi_e$ with a disc homeomorphism.
%\item Even if the maps $\varphi_e$ and $\varphi_e$ can vary, the open and closed cells of a CW-structure are part of the structure (i.e.\ independent of the choice of the maps)
%\end{i_enum} \end{notes}
%
%\begin{remarks}
%\begin{i_enum}
%\item The property $X=\varinjlim X_i$ is equivalent to $X=\bigcup_{i\geq-1}X_i$ as a set, equipped with the final topology, with respect to all inclusion maps. In particular, a set $A$ is open (closed) in $X$, iff $A\cap X_i$ is open (closed) in $X_i$ for all $i\geq-1$, or equivalently, $A\cap\sigma$ is open (closed) in $\sigma$ for every open cell $\sigma$ of the CW structure. This property is what we usually refer to as the ``weak topology'' of $X$ (the ``W'' part of the CW).
%\item
%\end{i_enum}
%\end{remarks}
