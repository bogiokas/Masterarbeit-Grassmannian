\chapter{Bits and Pieces in Topology}
\section{Some Categorical Notions}
The goal of this small section is to motivate the definition of the categorical exponential object. In order to arrive there, we first go through the Yoneda lemma and then through the notion of the adjoint functors. For a more thorough introduction to the topics discussed here, refer to \cite{basic_cat}, or any other introductory category book.

We will use the notation $\mathrm{ob}(\mathcal{C})$ for the objects of $\mathcal{C}$ and $\mathcal{C}(A,B)$ for the morphisms from $A$ to $B$ in $\mathcal{C}$. Moreover, the following few statements will be about locally small categories:
\begin{definition} A category $\mathcal{C}$ is called \emph{locally small}, if for every two objects $A,B$ in $\mathcal{C}$, the morphisms $\mathcal{C}(A,B)$ form a set.
\end{definition}

\subsection{Yoneda Lemma}
\begin{definition} Let $\mathcal{C}$ be any locally small category. Then, we denote by $\mathrm{Set}^{\mathcal{C}}$ the \emph{category of set valued functors of $\mathcal{C}$}, which is defined as follows:
\begin{itemize}
\item $\mathrm{ob}\left(\mathrm{Set}^{\mathcal{C}}\right)$ is the class of all functors $F$ from $\mathcal{C}$ to $\mathrm{Set}$.
\item $\mathrm{Set}^{\mathcal{C}}(F,G)$ is the class of all natural transformations $F\overset{\eta}{\Rightarrow}G$.
\item For $G\overset{\eta}{\Rightarrow}H$ and $F\overset{\theta}{\Rightarrow}G$, the composition $F\overset{\eta\circ\theta}{\Rightarrow}H$ is the usual composition of natural transformations.
\end{itemize}
\end{definition}
\begin{remark} In the above definition, $\mathrm{Set}^{\mathcal{C}}$ need not be locally small. Indeed, let $\mathcal{C}$ be any large discrete category. Let for example, $\mathrm{ob}(\mathcal{C})$ be the same as $\mathrm{ob}(\mathrm{Set})$ and let
\[\mathcal{C}(A,B)=\left\{\begin{array}{ll}\left\{1_A\right\}&,A=B\\\emptyset&,A\neq B\end{array}\right.\]
This is obviously a locally small category. Moreover, define the functors $F,G$:
\begin{center}
\begin{tikzcd}
\mathcal{C}\ar[r,bend left=25, "F"{name=F}]\ar[r,bend right=25, "G"'{name=G}]&[2em]\mathrm{Set}
\end{tikzcd}
\end{center}
both to be the identity on $\mathrm{ob}(\mathcal{C})$. The only morphisms in $\mathcal{C}$ are the identity morphisms, for which there is no choice for $F,G$, since they are functors. They have to satisfy $F1_A=G1_A=1_A$ for every set $A$.

A natural transformation $\eta:F\Rightarrow G$ is a collection of choices $\eta_A:F(A)\to G(A)$, such that $(Ge)\circ\eta_A=\eta_B\circ(Fe)$ for every morphism $e\in\mathcal{C}(A,B)$. The only such morphisms are the identity arrows, so there aren't any restrictions on the choices when constructing $\eta$. This means that $\mathrm{Set}^{\mathcal{C}}(F,G)$ is the class of all $\eta$, each one determined by a collection of choices $\eta_A\in\mathrm{Set}(A,A)$ over all sets $A$. Since the class of all sets is a proper class, so is the class of all such collection of choices, which makes $\mathrm{Set}^{\mathcal{C}}$ not locally small.
\end{remark}

We start with some definitions and results from Chapter~4 in \cite{basic_cat}.
\begin{definition}[4.1.16] Let $\mathcal{C}$ be a locally small category. Moreover, let $X$ be any object of $\mathcal{C}$. We define the functor
\[H_X:\mathcal{C}^{\mathrm{op}}\to\mathrm{Set}\]
as follows:
\begin{itemize}
\item For any object $A$ of $\mathcal{C}^{\mathrm{op}}$, we define $H_X(A):=\mathcal{C}(A,X)$.
\item For any morphism $g^{\mathrm{op}}\in\mathcal{C}^{\mathrm{op}}(A,B)$, i.e. any morphism $g\in\mathcal{C}(B,A)$, we define $H_X(g):=g^*\in\mathrm{Set}(H_X(A),H_X(B))$ taking any $p\in\mathcal{C}(A,X)$ to $p\circ g\in\mathcal{C}(B,X)$, like in the following diagram:
\begin{center}
\begin{tikzcd}
A\ar[d,"g^{\mathrm{op}}"']&&A&&H_X(A)\ar[d,"H_X(g)"']\ar[r,phantom,":="]&[-1.5em]\mathcal{C}(A,X)\ar[r,phantom,"\ni"]&[-1.8em]p\ar[d,mapsto,"g^*"]\\
B&&B\ar[u,"g"]&&H_X(B)\ar[r,phantom,":="]&\mathcal{C}(B,X)\ar[r,phantom,"\ni"]&p\circ g
\end{tikzcd}
\end{center}
\end{itemize}
\end{definition}

\begin{definition}[4.1.21]\label{def:functor} Let $\mathcal{C}$ be a locally small category. We define the functor
\[H_{\bullet}:\mathcal{C}\to\mathrm{Set}^{\mathcal{C}^{\mathrm{op}}}\]
as follows:
\begin{itemize}
\item For any object $X$ of $\mathcal{C}$, we define $H_{\bullet}(X):=H_X$.
\item For any morphism $f\in\mathcal{C}(X,Y)$, we define $H_{\bullet}(f)=H_f\in\mathrm{Set}^{\mathcal{C}^{\mathrm{op}}}(H_X,H_Y)$ to be the natural transformation with components $(H_f)_A:=f_*\in\mathrm{Set}(H_X(A),H_Y(A))$ taking any $p\in\mathcal{C}(A,X)$ to $f\circ p\in\mathcal{C}(A,Y)$, like in the following diagram:
\begin{center}
\begin{tikzcd}
X\ar[dd,"f"']&[-1em]&[4em]&[-1em]H_X(A)\ar[dd,"(H_f)_A"]\ar[r,phantom,"\ni"]&[-1.8em]p\ar[dd,mapsto,"f_*"]\\[-1em]
&\mathcal{C}^{\mathrm{op}}\ar[r,bend left=30,"H_X"{name=HX}]\ar[r,bend right=30,"H_Y"'{name=HY}]&\mathrm{Set}\ar[from=HX, to=HY, Rightarrow, "H_f"]\\[-1em]
Y&&&H_Y(A)\ar[r,phantom,"\ni"]&f\circ p
\end{tikzcd}
\end{center}
\end{itemize}
\end{definition}
\begin{remark} We have to prove that in the above definition, for every $f\in\mathcal{C}(X,Y)$, $H_f$ is indeed a natural transformation, i.e. that for every $g^{\mathrm{op}}\in\mathcal{C}^{\mathrm{op}}(A,B)$, the following diagram commutes:
\begin{center}
\begin{tikzcd}
A&&H_X(A)\ar[d,"H_X(g)"']\ar[r,"(H_f)_A"]&H_Y(A)\ar[d,"H_Y(g)"]\\
B\ar[u,"g"]&&H_X(B)\ar[r,"(H_f)_B"]&H_Y(B)
\end{tikzcd}
\end{center}
which does, since both directions take $p\in\mathcal{C}(A,X)$ to $f\circ p\circ g\in\mathcal{C}(B,Y)$.
\end{remark}

\begin{theorem}[Yoneda, 4.2.1] Let $\mathcal{C}$ be a locally small category. Moreover, let $X$ be any object of $\mathcal{C}$ and $F:\mathcal{C}^{\mathrm{op}}\to\mathrm{Set}$ be any functor. Then:
\[\mathrm{Set}^{\mathcal{C}^{\mathrm{op}}}(H_X,F)\cong F(X)\]
naturally in $(X,F)$ in $\mathcal{C}^{op}\times\mathrm{Set}^{\mathcal{C}^{\mathrm{op}}}$.
\end{theorem}
\begin{proof}[Sketch of the Proof] To remove some clutter in the notation, we just write $SP$ instead of $S(P)$ and $Su$ instead of $S(u)$ for every functor $S$, object $P$ and morphism $u$.

First of all, the ``$\cong$'' in the theorem is inside the category $\mathrm{Set}$, so it is a bijection. It being natural in $(X,F)$ means that we need to define a bijection
\[\psi_{X,F}:\mathrm{Set}^{\mathcal{C}^{\mathrm{op}}}(H_X,F)\to FX\]
and prove that it is a natural transformation between the following two functors:
\begin{center}
\begin{tikzcd}
\mathcal{C}^{op}\times\mathrm{Set}^{\mathcal{C}^{\mathrm{op}}}\ar[r,shift left,"(X{,}F)\mapsto\mathrm{Set}^{\mathcal{C}^{\mathrm{op}}}(H_X{,}F)"]\ar[r,shift right, "(X{,}F)\mapsto FX"']&[10em]\mathrm{Set}
\end{tikzcd}
\end{center}
In order to prove that this is a bijection, it suffices to define a function
\[\phi_{X,F}:FX\to\mathrm{Set}^{\mathcal{C}^{\mathrm{op}}}(H_X,F)\]
and prove that it is the inverse of $\psi$.

So, the proof is going to have four steps: First we are going to define $\psi_{X,F}$, then we are going to define $\phi_{X,F}$, then we are going to prove that these are inverses and finally we are going to prove that they are natural:
\begin{itemize}
\item There is the following ``natural'' choice for $\psi_{X,F}$. Let $\eta$ be any natural transformation $H_X\Rightarrow F$. Then define:
\[\psi_{X,F}(\eta):=\eta_X(1_X)\in FX\]
Let us unpack this: $\eta$ being a natural transformation means that it has components $\eta_A:H_XA\to FA$ for every $A$ in $\mathcal{C}$, but $H_XA$ is just $\mathcal{C}(A,X)$. Choosing $A=X$, we also get an obvious element $1_X\in\mathcal{C}(X,X)$, whose image under $\eta_X$ lies in the desired set $FX$.

\item There exists also a ``natural'' choice for $\phi_{X,F}$. Let $x\in FX$ be any element of the set $FX$. We have to define a natural transformation $\phi_{X,F}(x)=\theta^x$ from $H_X$ to $F$. Let us first define each component $\theta^x_A:H_XA\to FA$ as follows:
\[\theta^x_A(p)=(Fp)(x)\in FA\]
Let us also unpack this definition as well: We want to define $\theta^x_A$ on every element $p\in\mathcal{C}(A,X)$. We already have some element $x\in FX$ and for every such $p$, we can create $Fp\in\mathrm{Set}(FX,FA)$, since $F$ is a functor $\mathcal{C}^{\mathrm{op}}\to\mathrm{Set}$. So, the image of $x$ under $Fp$ lies in the desired set $FA$.

The definition of $\phi$ is not over yet, since we have to prove that $\theta^x$ is indeed natural in $A$, i.e. that for every morphism $f^{\mathrm{op}}\in\mathcal{C}(A,B)$ the following diagram commutes:
\begin{center}
\begin{tikzcd}
A&&H_XA\ar[d,"H_Xf"']\ar[r,"\theta^x_A"]&FA\ar[d,"Ff"]\\
B\ar[u,"f"]&&H_XB\ar[r,"\theta^x_B"]&FB
\end{tikzcd}
\end{center}
which does, since both directions take $q$ to $(F(q\circ f))(x)$.

\item The next step is to prove that $\psi_{X,F}$ and $\phi_{X,F}$ are inverse functions in the category of sets. The one direction is easy. Let $x\in FX$. Then:
\[\psi_{X,F}\big(\phi_{X,F}(x)\big)=\psi_{X,F}(\theta^x)=\theta^x_X(1_X)=(F1_X)(x)=1_{FX}(x)=x\]

For the other direction, let $\eta$ be any natural transformation $H_X\to F$. Then:
\[\phi_{X,F}\big(\psi_{X,F}(\eta)\big)=\phi_{X,F}(\eta_X(1_X))=\theta^{\eta_X(1_X)}\]
In order to show that this is equal to $\eta$, we need to check the equality in every component. Let $A$ be any object of $\mathcal{C}$ and $p\in\mathcal{C}(A,X)$ any function. Then:
\[\theta^{\eta_X(1_X)}_A(p)=(Fp)\big(\eta_X(1_X)\big)\overset{(*)}{=}\eta_A\big((H_Xp)(1_X)\big)=\eta_A(1_X\circ p)=\eta_A(p)\]
where the $(*)$ holds because of the naturality of $\eta$, i.e. because this diagram commutes:
\begin{center}
\begin{tikzcd}
X&&H_XX\ar[r,"\eta_X"]\ar[d,"H_Xp"']&FX\ar[d,"Fp"]\\
A\ar[u,"p"]&&H_XA\ar[r,"\eta_A"]&FA
\end{tikzcd}
\end{center}

\item Now, it only remains to prove that $\psi$ and $\phi$ are natural transformations. It suffices to prove this just for one of the two, say $\psi$. Also, being natural in $(X,F)\in\mathcal{C}^{\mathrm{op}}\times\mathrm{Set}^{\mathcal{C}^{\mathrm{op}}}$ is equivalent to being natural in $X\in\mathcal{C}^{\mathrm{op}}$ for every fixed $F$ and at the same time being natural in $F\in\mathrm{Set}^{\mathcal{C}^{\mathrm{op}}}$, for every fixed $X$.

Let $F$ be fixed and $g^{op}\in\mathcal{C}^{op}(X,Y)$. Then, we want to show that the following diagram commutes:
\begin{center}
\begin{tikzcd}
X&&H_X&&\mathrm{Set}^{\mathcal{C}^{\mathrm{op}}}(H_X,F)\ar[d,"-\circ H_g"']\ar[r,"\psi_{X,F}"]&FX\ar[d,"Fg"]\\
Y\ar[u,"g"]&&H_Y\ar[u,"H_g"]&&\mathrm{Set}^{\mathcal{C}^{\mathrm{op}}}(H_Y,F)\ar[r,"\psi_{Y,F}"]&FY
\end{tikzcd}
\end{center}
which is true, since for any natural transformation $\eta\in\mathrm{Set}^{\mathcal{C}^{\mathrm{op}}}(H_X,F)$, both directions lead to $\eta_Y(g)$. To prove this, use the naturality of $\eta$ and the definition of $H_g$.

Let now $X$ be fixed and $\alpha\in\mathrm{Set}^{\mathcal{C}^{\mathrm{op}}}(F,G)$. Then, we want to show that the following diagram commutes:
\begin{center}
\begin{tikzcd}
F\ar[d,"\alpha"]&&\mathrm{Set}^{\mathcal{C}^{\mathrm{op}}}(H_X,F)\ar[d,"\alpha\circ-"']\ar[r,"\psi_{X,F}"]&FX\ar[d,"\alpha_X"]\\
G&&\mathrm{Set}^{\mathcal{C}^{\mathrm{op}}}(H_X,G)\ar[r,"\psi_{X,G}"]&GX
\end{tikzcd}
\end{center}
which is true, since for any natural transformation $\eta\in\mathrm{Set}^{\mathcal{C}^{\mathrm{op}}}(H_X,F)$, both directions lead to $(\alpha\circ\eta)_X(1_X)$.\qedhere
\end{itemize}
\end{proof}
\begin{corollary} Let $\mathcal{C}$ be a locally small category. Moreover, let $X,Y$ be any two objects of $\mathcal{C}$. Then:
\[\mathrm{Set}^{\mathcal{C}^{\mathrm{op}}}(H_X,H_Y)\cong\mathcal{C}(X,Y)\]
naturally in $(X,Y)$ in $\mathcal{C}^{\mathrm{op}}\times \mathcal{C}^{\mathrm{op}}$.
\end{corollary}
\begin{proof} First of all, the bijection is trivially obtained from the Yoneda lemma, for $F=H_Y$. Moreover, since the map $Y\mapsto H_Y$ is functorial, as defined in Definition~\ref{def:functor}, the following diagram commutes:
\begin{center}
\begin{tikzcd}
Y_1\ar[d,"h"]&&H_{Y_1}\ar[d,"H_h"]&&\mathrm{Set}^{\mathcal{C}^{\mathrm{op}}}(H_X,H_{Y_1})\ar[d,"H_h\circ-"']\ar[r,"\psi_{X,H_{Y_1}}"]&H_{Y_1}X\ar[d,"(H_h)_X"]\\
Y_2&&H_{Y_2}&&\mathrm{Set}^{\mathcal{C}^{\mathrm{op}}}(H_X,H_{Y_2})\ar[r,"\psi_{X,H_{Y_2}}"]&H_{Y_2}X
\end{tikzcd}
\end{center}
just like in the proof of the Yoneda lemma.
\end{proof}
\begin{corollary}[4.3.7]\label{cor:ful_faith} Let $\mathcal{C}$ be a locally small category. Then, the functor
\[H_{\bullet}:\mathcal{C}\to\mathrm{Set}^{\mathcal{C}^{\mathrm{op}}}\]
as defined in Definition~\ref{def:functor} is full and faithful.
\end{corollary}
\begin{proof} $H_{\bullet}$ being full and faithful means that for every two objects $X,Y$ in $\mathcal{C}$, the map
\begin{center}
\begin{tikzcd}
\mathcal{C}(X,Y)\ar[r,"H_{\bullet}"]&\mathrm{Set}^{\mathcal{C}^{\mathrm{op}}}(H_X,H_Y)\\[-2em]
g\ar[r,mapsto]&H_g
\end{tikzcd}
\end{center}
is a bijection. We already know that $\phi_{X,H_Y}$ is a bijection from the proof of the Yoneda lemma, so it suffices to show that $\phi_{X,H_Y}(g)=H_g$ for every $g\in\mathcal{C}(X,Y)$, or equivalently $\psi_{X,H_Y}(H_g)=g$:
\[\psi_{X,H_Y}(H_g)=(H_g)_X(1_X)=g\circ 1_X=g\qedhere\]
\end{proof}
\begin{proposition}[4.3.10]\label{prop:unique} Let $\mathcal{C}$ be a locally small category. Moreover, let $X,Y$ be any two objects of $\mathcal{C}$. Then:
\[H_X\cong H_Y \Longleftrightarrow X\cong Y\]
\end{proposition}
\begin{proof} Since $H_{\bullet}$ is a functor, it takes an isomorphism to an isomorphism. For the other direction, it is easy to see that for any full and faithful functor $F$, $FA\cong FB$ gives $A\cong B$ and $H_{\bullet}$ was proven to be full and faithful in Corollary~\ref{cor:ful_faith}
\end{proof}

\subsection{Adjoint functors}
Let us now go through some definitions and results regarding the adjoints. Our reference for this subsection will mainly be the Chapter~2 of \cite{basic_cat}, but be warned that we heavily changed the notations and the proofs of this section to cover our needs.

\begin{definition}[2.1.1]\label{def:adj} Let $\mathcal{C},\mathcal{D}$ be two locally small categories and $L:\mathcal{D}\to\mathcal{C},\ R:\mathcal{C}\to\mathcal{D}$ be two functors. Then we say that \emph{$R$ is the right adjoint to $L$} and \emph{$L$ is the left adjoint to $R$}, if there exist a isomorphisms
\begin{equation}
\mathcal{C}(L(A),X)\cong\mathcal{D}(A,R(X))\label{adj}
\end{equation}
natural in $(A,X)\in\mathcal{D}^{op}\times\mathcal{C}$. We usually denote this by $L\dashv R$ or equivalently:
\vspace*{-0.2em}
\begin{center}
\begin{tikzcd}
\mathcal{C}\ar[r,bend right=30, "R"'{name=R}]&[3em]\mathcal{D}\ar[l,bend right=30, "L"'{name=L}]
\ar[phantom, from=L, to=R, "\dashv" rotate=-90]
\end{tikzcd}
\end{center}
\end{definition}
\begin{notation} Just like in the proof of Yoneda lemma, we will try in this subsection to omit the parenthesis when using a functor and write $FP$ instead of $F(P)$ etc. In order to avoid confusion as much as possible, we may reintroduce the parenthesis, sometimes when more than one functors are involved, as in $FG(P)$.
\end{notation}

\begin{proposition}[4.3.13] Let $\mathcal{C},\mathcal{D}$ be two locally small categories and $L:\mathcal{D}\to\mathcal{C}$. If $L$ has a right adjoint functor $R$, then $R$ is unique, up to isomorphism.
\end{proposition}
\begin{proof} Let $R,R'$ be two right adjoint functors of $L$. Then, for fixed $X$, there exist isomorphisms
\[\mathcal{D}(A,RX)\cong\mathcal{C}(LA,X)\cong\mathcal{D}(A,R'X)\]
natural in $A$. This means that there is an isomorphism $H_{RX}(A)\cong H_{R'X}(A)$ natural in $A$, which means $H_{RX}\cong H_{R'X}$ in $\mathrm{Set}^{\mathcal{D}^{\mathrm{op}}}$. Using Proposition~\ref{prop:unique}, we conclude that there exists an isomorphism $RX\cong R'X$. This construction is natural in $X$, so $R\cong R'$ as functors.
\end{proof}

Let us now unravel the Definition~\ref{def:adj}. Let $L$ and $R$ be an adjoint pair $L\dashv R$. This means that there exist some natural transformations $\psi$ and $\phi$ with components:
\[\begin{aligned}
\psi_{A,X}&:\mathcal{C}(LA,X)\to\mathcal{D}(A,RX)\\[0.8em]
\phi_{A,X}&:\mathcal{D}(A,RX)\to\mathcal{C}(LA,X)
\end{aligned}\qquad\text{ satisfying: }\qquad
\left\{\begin{aligned}
\psi_{A,X}\circ\phi_{A,X}&=1_{\mathcal{D}(A,RX)}\\[0.8em]
\phi_{A,X}\circ\psi_{A,X}&=1_{\mathcal{C}(LA,X)}
\end{aligned}\right.\]
To express the two naturality conditions, we fix some $p^{\mathrm{op}}\in\mathcal{D}^{\mathrm{op}}(A,B)$ and some $f\in\mathcal{C}(X,Y)$. Then, we have the following four commutative diagrams:
\begin{center}
\begin{tikzcd}
A&[1.2em]
\mathcal{C}(LA,X)\ar[d,"-\circ Lp"']\ar[r,"\psi_{A,X}","\cong"']&\mathcal{D}(A,RX)\ar[d,"-\circ p"]&[1.2em]
\mathcal{C}(LA,X)\ar[d,"-\circ Lp"']&\mathcal{D}(A,RX)\ar[l,"\phi_{A,X}"',"\cong"]\ar[d,"-\circ p"]\\
B\ar[u,"p"']&
\mathcal{C}(LB,X)\ar[r,"\psi_{B,X}","\cong"']&\mathcal{D}(B,RX)&
\mathcal{C}(LB,X)&\mathcal{D}(B,RX)\ar[l,"\phi_{B,X}"',"\cong"]\\
X\ar[d,"f"]&
\mathcal{C}(LA,X)\ar[d,"f\circ -"']\ar[r,"\psi_{A,X}","\cong"']&\mathcal{D}(A,RX)\ar[d,"Rf\circ -"]&
\mathcal{C}(LA,X)\ar[d,"f\circ -"']&\mathcal{D}(A,RX)\ar[l,"\phi_{A,X}"',"\cong"]\ar[d,"Rf\circ -"]\\
Y&
\mathcal{C}(LA,Y)\ar[r,"\psi_{A,Y}","\cong"']&\mathcal{D}(A,RY)&
\mathcal{C}(LA,Y)&\mathcal{D}(A,RY)\ar[l,"\phi_{A,Y}"',"\cong"]\\
\end{tikzcd}
\end{center}
i.e. for any $q\in\mathcal{D}(A,RX)$ and $g\in\mathcal{C}(LA,X)$, we have:
\begin{center}
\begin{minipage}{0.5\linewidth}
%\makeatletter\tagsleft@true\makeatother
\begin{align}
\psi_{B,X}(g\circ Lp)&=\psi_{A,X}(g)\circ p\label{psiL}\\[1em]
\psi_{A,Y}(f\circ g)&=Rf\circ \psi_{A,X}(g)\label{psiR}
\end{align}
\end{minipage}\begin{minipage}{0.5\linewidth}
\begin{align}
\phi_{B,X}(q\circ p)&=\phi_{A,X}(q)\circ Lp\label{phiL}\\[1em]
\phi_{A,Y}(Rf\circ q)&=f\circ \phi_{A,X}(q)\label{phiR}
\end{align}
\end{minipage}
\end{center}
Right now, we will focus on equations \eqref{phiL} and \eqref{psiR}, which we also write schematically as:
\begin{align*}
\phi_{B,X}\left(B\overset{p}{\longrightarrow}A\overset{q}{\longrightarrow}RX\right)&=LB\overset{Lp}{\longrightarrow}LA\overset{\phi_{A,X}(q)}{\longrightarrow}X\\[1em]
\psi_{A,Y}\left(LA\overset{g}{\longrightarrow}X\overset{f}{\longrightarrow}Y\right)&=A\overset{\psi_{A,X}(g)}{\longrightarrow}RX\overset{Rf}{\longrightarrow}RY
\end{align*}
Interestingly, this hints towards the following fact: The values of $\phi$ (resp. $\psi$) on any $q\circ p$ (resp. $f\circ g$) only depend on $L$ (resp. $R$) and $\phi_{A,X}(q)$ (resp. $\psi_{A,X}(g)$). So, by choosing $q$ (resp. $g$) as ``naturally'' as possible, we can describe what exactly $\phi$ (resp. $\psi$) does on every other input. We thus choose $A=RX$, $q=1_{RX}$ for the equation involving $\phi$ and $X=LA$, $g=1_{LA}$ for the equation involving $\psi$. This gives:
\begin{align*}
\phi_{B,X}\left(B\overset{p}{\longrightarrow}RX\right)&=\phi_{B,X}\left(B\overset{p}{\longrightarrow}RX\xrightarrow{1_{RX}}RX\right)=LB\overset{Lp}{\longrightarrow}LRX\xrightarrow{\phi_{RX,X}(1_{RX})}X\\[1em]
\psi_{A,Y}\left(\mask{B\overset{p}{\longrightarrow}RX}{LA\overset{f}{\longrightarrow}Y}\right)&=\psi_{A,Y}\left(\mask{B\overset{p}{\longrightarrow}RX\xrightarrow{1_{RX}}RX}{LA\xrightarrow{1_{LA}}LA\overset{f}{\longrightarrow}Y}\right)=\mask{LB\overset{Lp}{\longrightarrow}LRX\xrightarrow{\phi_{RX,X}(1_{RX})}X}{A\xrightarrow{\psi_{A,LA}(1_{LA})}RLA\overset{Rf}{\longrightarrow}RY}
\end{align*}
\begin{definition} Let $\mathcal{C},\mathcal{D}$ be two locally small categories and $L:\mathcal{D}\to\mathcal{C}$, $R:\mathcal{C}\to\mathcal{D}$ a pair of adjoint functors $L\dashv R$. Then the natural transformations
\[\varepsilon:LR\Rightarrow 1_{\mathcal{C}}\qquad\text{ and }\qquad\eta:1_{\mathcal{D}}\Rightarrow RL\]
defined by:
\[\varepsilon_X:=\phi_{RX,X}(1_{RX})\qquad\text{ and }\qquad\eta_A:=\psi_{A,LA}(1_{LA})\]
are called the \emph{counit} and the \emph{unit} of the adjunction, respectively.
\end{definition}
\begin{remark} In order for the counit and the unit to be well defined, we need to prove that they are indeed natural transformations, i.e. that for any $f\in\mathcal{C}(X,Y)$ and $p\in\mathcal{D}(B,A)$ the following diagrams commute:
\begin{center}
\begin{tikzcd}
X\ar[d,"f"']&[-1em]LR(X)\ar[r,"\varepsilon_X"]\ar[d,"LR(f)"']&X\ar[d,"f"]&[1.5em]
B\ar[d,"p"']&[-1em]B\ar[r,"\eta_B"]\ar[d,"p"']&RL(B)\ar[d,"RL(p)"]\\
Y&LR(Y)\ar[r,"\varepsilon_Y"]&Y&
A&A\ar[r,"\eta_A"]&RL(A)
\end{tikzcd}
\end{center}
\end{remark}
\begin{proof}
This is a direct consequence of the naturality of $\phi,\psi$:
\begin{align*}
f\circ\varepsilon_X&\overset{\eqref{phiR}}{=}\phi_{RX,Y}(1_{RY}\circ Rf\circ 1_{RX})\overset{\eqref{phiL}}{=}\varepsilon_Y\circ LRf\\[0.5em]
\eta_A\circ p&\overset{\eqref{psiL}}{=}\psi_{B,LA}(1_{LA}\circ Lp\circ 1_{LB})\overset{\eqref{psiR}}{=}RLp\circ\eta_B\qedhere
\end{align*}
\end{proof}

So, our discussion leads to the following fact. Let $f\in\mathcal{C}(LA,X)$ and $p\in\mathcal{D}(A,RX)$, then, equations \eqref{phiL} and \eqref{psiR} give:
\begin{align}
\begin{split}
\phi_{A,X}(p)&=\varepsilon_X\circ Lp\\[0.8em]
\psi_{A,X}(f)&=Rf\circ\eta_A
\end{split}\label{iso_from_unit}
\end{align}

This shows that the isomorphism \eqref{adj} can be completely retrieved by its counit and unit. Our goal for the rest of this section is to give an alternative definition for $L\dashv R$, only based on the choice of a particular natural transformation $\varepsilon$ to play the role of the counit, from which we first will construct $\eta$ and then $\phi$ and $\psi$. The reason we want to do this, is because this ``second definition'' will give us a way to more explicitly construct the right adjoint $R$ of a given $L$, if $L$ has a right adjoint.

First of all, notice that \eqref{iso_from_unit} on its own suffices to make $\phi$ and $\psi$ natural, given that $\varepsilon$ and $\eta$ are:
\begin{proposition} Let $\mathcal{C},\mathcal{D}$ be two locally small categories and $L:\mathcal{D}\to\mathcal{C}$, $R:\mathcal{C}\to\mathcal{D}$ any pair of functors. Moreover, let $\varepsilon:LR\Rightarrow1_{\mathcal{C}}$ and $\eta:1_{\mathcal{D}}\Rightarrow RL$ be any two natural transformations. If we define $\phi_{A,X}$ and $\psi_{A,X}$ using the equations \eqref{iso_from_unit}, then $\phi$ and $\psi$ are natural transformations in $(A,X)\in\mathcal{D}^{\mathrm{op}}\times\mathcal{C}$ and additionally the following equations are satisfied:
\begin{align}
\begin{split}
\varepsilon_X=\phi_{RX,X}(1_{RX})\\[0.8em]
\eta_A=\psi_{A,LA}(1_{LA})
\end{split}\label{unit_from_iso}
\end{align}
\end{proposition}
\begin{proof} Let us fix some $p^{\mathrm{op}}\in\mathcal{D}^{\mathrm{op}}(A,B)$ and some $f\in\mathcal{C}(X,Y)$. Then it is easy to see that the equations \eqref{psiL}-\eqref{phiR} hold:
\begin{align*}
\psi_{B,X}(g\circ Lp)&=Rg\circ RLp\circ\eta_B=Rg\circ\eta_A\circ p=\psi_{A,X}(g)\circ p,&\text{for any } g&\in\mathcal{C}(LA,X)\\
\psi_{A,Y}(f\circ g)&=Rf\circ Rg\circ\eta_A=Rf\circ\psi_{A,X}(g),&\text{for any } g&\in\mathcal{C}(LA,X)\\
\phi_{B,X}(q\circ p)&=\varepsilon_X\circ Lq\circ Lp=\phi_{A,X}(q)\circ p,&\text{for any }q&\in\mathcal{D}(A,RX)\\
\phi_{A,Y}(Rf\circ q)&=\varepsilon_Y\circ LRf\circ Lq=f\circ\varepsilon_X\circ Lq=f\circ\phi_{A,X}(q),&\text{for any }q&\in\mathcal{D}(A,RX)
\end{align*}
We only used the equations \eqref{iso_from_unit} and the naturality of $\varepsilon,\eta$ in the first and last case. Moreover, it is again very easy to see that:
\begin{align*}
\phi_{RX,X}(1_{RX})&=\varepsilon_X\circ L1_{RX}=\varepsilon_X\\
\psi_{A,LA}(1_{LA})&=R1_{LA}\circ\eta_A=\eta_A\qedhere
\end{align*}
\end{proof}

Our problem now is that if we choose some natural transformations $\varepsilon$ and $\eta$ randomly and construct $\phi$ and $\psi$ using \eqref{iso_from_unit}, it is not at all guaranteed that $\phi$ and $\psi$ are inverses of each other. It so happens that $\varepsilon$ and $\eta$ must satisfy a universal property in order to be the counit and unit of an adjunction. Let's start examining this at the following definition.
\begin{definition}

\end{definition}



\section{Compact-Open Topology}


%\section{CW Complexes}
%A CW structure on some space $X$ is usually defined recursively, as an inductive ``glueing'' of cells of some dimension $k$ to the previous, lower dimensional, skeleton of $X$, forming a new, $k$-dimensional, skeleton of $X$. A space $X$ may exhibit many different CW structures, but the existence of one suffices in order for $X$ to be characterized as CW complex. Here, we are going to use the following formal formulation of the above definition.
%
%\begin{definition} A topological space $X$ is a \ul{CW-complex}, if there exists some filtration
%\[\emptyset=X_{-1}\subseteq X_0\subseteq X_1\subseteq X_2\subseteq\cdots\subseteq X\]
%such that:
%\begin{b_item}
%\item $X=\varinjlim X_i$ with respect to all inclusion maps.
%\item For every $n\geq0$ there exists a pushout diagram in the category of topological spaces:
%\begin{center}
%\begin{tikzcd}
%\displaystyle\coprod_{e\in\pi_0(X_n\setminus{X_{n-1}})}S^{n-1}\ar[r,"\coprod_e\phi_e"]\ar[d,hook,"\coprod_ej_e"']\ar[dr,phantom, very near start,"\ulcorner"]&[4em]X_{n-1}\ar[d,hook,"i_n"]\\[2em]
%\displaystyle\coprod_{e\in\pi_0(X_n\setminus{X_{n-1}})}D^n\ar[r,"\coprod_e\Phi_e"']&X_n\\
%\end{tikzcd}
%\end{center}
%where $j_e:S^{n-1}\to D^n$ is the usual inclusion map and $i_n:X_{n-1}\to X_n$ is the inclusion map given by the filtration.
%\end{b_item}
%\end{definition}
%
%A filtration of a topological space $X$, making $X$ a CW-complex is called a \ul{CW-structure} of $X$. Moreover, given a filtration of $X$ like in the above definition, the sets $\Phi_e\large({(D^n)}^{\circ}\large)$ (resp. $\Phi_e(D^n)$) are called the $n$-dimensional \ul{open} (resp. \ul{closed}) \ul{cells} of this CW-structure. Recall the following known facts regarding the dependencies between CW-complexes, structures and cells.
%
%\begin{notes}
%\begin{i_enum}
%\item A CW-complex $X$ can have more than one CW-structures, even structures having different number of $n$-dimensional open cells each.
%\item For a particular CW-structure of $X$, the maps $\phi_e$ and $\Phi_e$ are not predetermined by the structure, which means that there can be more than one choices for them. For example, one could always precompose $\Phi_e$ with a disc homeomorphism.
%\item Even if the maps $\phi_e$ and $\Phi_e$ can vary, the open and closed cells of a CW-structure are part of the structure (i.e.\ independent of the choice of the maps)
%\end{i_enum} \end{notes}
%
%\begin{remarks}
%\begin{i_enum}
%\item The property $X=\varinjlim X_i$ is equivalent to $X=\bigcup_{i\geq-1}X_i$ as a set, equipped with the final topology, with respect to all inclusion maps. In particular, a set $A$ is open (closed) in $X$, iff $A\cap X_i$ is open (closed) in $X_i$ for all $i\geq-1$, or equivalently, $A\cap\sigma$ is open (closed) in $\sigma$ for every open cell $\sigma$ of the CW structure. This property is what we usually refer to as the ``weak topology'' of $X$ (the ``W'' part of the CW).
%\item
%\end{i_enum}
%\end{remarks}
