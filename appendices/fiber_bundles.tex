\chapter{Fiber Bundles}
The Fiber Bundle is a topological object generalizing the notion of the product space. A fiber bundle $E$ can be thought like a ``twisted product'' of a base topological space $B$ and a fiber space $F$. This means that locally it always looks like the product of the two, but globally $F$ may (homeomorphically) change, while one is going ``around'' $B$. Before we give some explicit examples, we need the formal definition of a fiber bundle.

\begin{definition} Let $F$, $E$ and $B$ be some topological spaces. A continuous map $p:E\to B$ is called a \ul{fiber bundle with fiber $F$}, if for every $x\in B$ there exists a neighborhood $x\in U\subseteq B$, and a function $f_U:p^{-1}(U)\to U\times F$ (where $p^{-1}(U)$ has the subspace topology and $U\times F$ the product topology), such that:
\begin{i_enum}
\item $f_U$ is a homeomorphism.
\item The following diagram commutes:
\begin{center}
\begin{tikzcd}
E\ar[r,"\supseteq",phantom]&[-1em]p^{-1}(U)\ar[d,"p|_{p^{-1}(U)}"']\ar[r,"f_U"]&U\times F\ar[dl,"\pi_1"]\\[2em]
B\ar[r,"\supseteq",phantom]&U
\end{tikzcd}
\end{center}
i.e. $p|_{p^{-1}(U)}=\pi_1\circ f_U$.
\end{i_enum}
$B$ is then called \ul{base space} and $E$ \ul{total space} of the fiber bundle. Moreover, for every $x\in B$, the space $p^{-1}(\{x\})$ is called \ul{the fiber over $x$} and is denoted by $F_x$. Notice, that, sometimes, we refer to the fiber bundle just by its total space, if the definition of $p$ is clear.
\end{definition}

\begin{notation}
It is very common to denote a fiber bundle as:
\[F\to E\overset{p}{\to}B\]
which may remind to the reader the notion of a short exact sequence.
%%%%TODO: elaborate on the connection
\end{notation}

Notice that this is the smallest set of requirements one can demand of a fiber bundle. Most of the time though, people are interested in fiber bundles with additional structure imposed on the fiber space $F$, with the most noticeable example being a group action on $F$. These fiber bundles are said to then be equipped with a ``structure group'' $G$. Another interesting structure one can require is discussed on Chapter~\ref{chap:vec_bundles}, where we define and examine the notion of ``vector bundles'', i.e.\ fiber bundles having vector spaces as bundles.

It is worth noting at this point that the definition of the ``characteristic classes'' one will see in~\ref{def:char_class} originated from the study of ``sphere bundles'', which is the particular subclass of fiber bundles where the fibers are topologically spheres of some dimension.

The reader should thus keep in mind that all the definitions she will encounter in this chapter are (in a more specific setting) usually enriched. So, questions of bundle isomorphisms should always be answered carefully, by determining the underlying setting. For example, we will encounter fiber bundles which will be proven isomorphic in this simple setting, whereas one can distinguish between them in the setting of fiber bundles equipped with a structure group $G$.

\begin{proposition}\label{prop:same_fiber} Let $p:E\to B$ be a fiber bundle with fiber $F$, then $F_x\cong F$ for all $x\in B$.
\end{proposition}
\begin{proof}
Let $x\in B$. Then, there exists a neighborhood $U$ of $x$ inside $B$ and a homeomorphism $f_U:p^{-1}(U)\to U\times F$, such that
\[p|_{p^{-1}(U)}=\pi_1\circ f_U\]
Then, $f_U|_{F_x}:F_x\to f_U(F_x)$ is also a homeomorphism, as a restriction of a homeomorphism. Moreover, it is easy to see that $f_U(F_x)=\{x\}\times F$:
\begin{itemize}
\item Let $(x',a)\in f_U(F_x)$. This means that there exists some $v\in F_x$ with $f_U(v)=(x',a)$. Since $v\in F_x$, $p(v)=x$ by definition. Then, $x=p(v)=\pi_1\circ f_U(v)=\pi_1(x',a)=x'$, which means, $(x',a)=(x,a)\in\{x\}\times F$.
\item Let $(x,a)\in\{x\}\times F$. Since $f_U$ is surjective, there exists some $v\in p^{-1}(U)$ with $f_U(v)=(x,a)$. Then, $p(v)=\pi_1\circ f_U(v)=\pi_1(x,a)=x$ which means that $v\in p^{-1}(\{x\})=F_x$.
\end{itemize}
$\{x\}\times F$ is trivially homeomorphic to $F$, which proves the assertion.
\end{proof}

Now, one can see how bundles generalize products. Formally, one just needs to choose $U=B$ for every $x$ and $f_B=id$. In fact, products are in this setup the ``trivial'' example.
\begin{definition} Let $F, B$ be two topological spaces, then the projection
\[\pi_1:B\times F\to B\]
is called the \ul{trivial bundle over $B$ with fiber $F$}, where $\pi_1$ is the projection on the first coordinate, i.e.\ $\pi_1(x,a)=x$.
\end{definition}

Historically, the most famous non-trivial example of a fiber bundle has been the M\"obius strip:
\begin{example} Let $B=S^1$, $F=[0,1]$ and $E=\sfrac{[0,1]\times F}{(0,a)\sim(1,1-a)}$. Then $p:E\to B$, with:
\[p([(x,a)]):= x,\qquad\forall [(x,a)]\in E\]
is a fiber bundle.
\end{example}
Notice how this fiber bundle is not the ``same'' as the trivial bundle over the circle with the same fiber. We are going to soon define the notion of bundle maps and isomorphisms, but before going into that let us try the same M\"obius like constructions, i.e.\ let us fix $B=S^1$ and examine how the bundles look like for different choices of fibers $F$ and different equivalence relations $\sim$, which define $E=\sfrac{[0,1]\times F}{\sim}$. The bundle in these examples will always be $p:E\to B$ with $p([(x,a)])=x$.

\begin{examples}
\begin{i_enum}
\item For $F=\{0,1\}$, and $(0,a)\sim(1,1-a)$, $E$ is obviously the boundary of the M\"obius strip, which is topologically a circle. The image of $p$ then traces the circle with twice the speed of the input variable. Notice, that this is different from the trivial bundle with the same fiber over $B$, since the trivial bundle is a disjoint union of two circles.
\item $F=\mathbb{R}$ or $F=(0,1)$, and again $(0,a)\sim(1,1-a)$. This is the open version of the M\"obius strip.
\item $F=S^1\subseteq\mathbb{C}$, and $(0,z)\sim(1,\overline{z})$. This is maybe the second non-trivial example one usually sees, namely the Klein bottle.
\item $F=S^1$ again, but this time $(0,z)\sim(1,-z)$. Notice that this bundle contains the boundary of the M\"obius strip (and in fact the boundary of every rotation of the M\"obius strip around $B$). Moreover, notice that ``turning'' the circle $\{1\}\times S^1$ by $\pi$, before gluing it back to $\{0\}\times S^1$, just produces a total space, which is topologically a torus, i.e.\ $E$ is the ``same'' as the trivial bundle (although it contains a non-trivial one).
\end{i_enum}
\end{examples}

Notice how in the above discussion we first fixed a base space $B$, before starting to ask questions about similarity of bundles. This is so common, that the first definition of a maps between bundles assumes that both involved bundles have the same base space.
\begin{definition}
Let $p_1:E_1\to B$ and $p_2:E_2\to B$ be two fiber bundles with fibers $F_1$ and $F_2$ respectively. A continuous map $\phi:E_1\to E_2$ is a \ul{bundle map from $p_1$ to $p_2$ over $B$}, if the following diagram commutes:
\begin{center}
\begin{tikzcd}
E_1\ar[rd,"p_1"']\ar[rr,"\phi"]&[-2em]&[-2em]E_2\ar[ld,"p_2"]\\
&B
\end{tikzcd}
\end{center}
i.e. if $p_1=p_2\circ\phi$.
\end{definition}

\begin{proposition}\label{prop:fiberwise}
Let $p_1:E_1\to B$ and $p_2:E_2\to B$ be two fiber bundles and $\phi:E_1\to E_2$ be any continuous map, then $\phi$ is a bundle map, if and only if
\[\phi\big({(F_1)}_x\big)\subseteq{(F_2)}_x,\qquad\forall x\in B\]
i.e.\ iff $\phi$ maps every fiber over $x$, of the left fiber bundle, to the fiber over the same $x$, of the right fiber bundle.
\end{proposition}
\begin{proof}
\begin{b_item}
\item[($\Rightarrow$)] Let $\phi:E_1\to E_2$ be a bundle map, $x\in B$ and $v\in {(F_1)}_x$. Then $(p_2\circ\phi)(v)=p_1(v)=x$, which proves that $\phi(v)\in{(F_2)}_x$.
\item[($\Leftarrow$)] Let $\phi:E_1\to E_2$ be a continuous map, with $\phi\big({(F_1)}_x\big)\subseteq{(F_2)}_x$ for every $x\in B$. Moreover, let $v\in E_1$. Then, trivially, $v\in {(F_1)}_{p_1(v)}$, which gives $\phi(v)\in {(F_2)}_{p_1(v)}$, which means exactly that $(p_2\circ\phi)(v)=p_1(v)$.\qedhere
\end{b_item}
\end{proof}

\begin{definition}
Let $p_1:E_1\to B$ and $p_2:E_2\to B$ be two fiber bundles with fibers $F_1$ and $F_2$ respectively and let $\phi:E_1\to E_2$ be a bundle map from $p_1$ to $p_2$ over $B$. The map $\phi$ is called a \ul{bundle isomorphism}, if $\phi$ is a homeomorphism.
\end{definition}

\begin{proposition}
Let $p_1:E_1\to B$ and $p_2:E_2\to B$ be two fiber bundles with fibers $F_1$ and $F_2$ respectively and let $\phi:E_1\to E_2$ be a bundle isomorphism map from $p_1$ to $p_2$ over $B$. Then, for every $x\in B$, the map
\[\phi_x:{\left(F_1\right)}_x\to{\left(F_2\right)}_x\]
with $\phi_x(v)=\phi(v)$ for every $v\in\left(F_1\right)_x$ is a homeomorphism with $(\phi_x)^{-1}=\big(\phi^{-1}\big)_x$.
\end{proposition}
\begin{proof} First of all, the map $\phi_x$ is well defined because of Proposition~\ref{prop:fiberwise}. Moreover, $\phi_x$ is surjective. Indeed, let $v_2\in (F_2)_x$. Since $\phi$ is surjective, there exists some $v_1\in E_1$ such that $\phi(v_1)=v_2$. Since $\phi$ is a bundle map, it is true that $p_1(v_1)=p_2\circ\phi(v_1)=p_2(v_2)=x$, which means that $v_1\in(F_1)_x$. This proves that
\[\phi_x=\phi|_{{(F_1)}_x}:{(F_1)}_x\to\phi\big({(F_1)}_x\big)\]
and thus is a homeomorphism as a restriction of one. The fact that $\phi_x^{-1}$ is the restriction of $\phi^{-1}$ on ${(F_2)}_x$ follows by symmetry.
\end{proof}

We want now to investigate if the opposite is true. Given two fiber bundles $p_1, p_2$ over some space $B$ that have isomorphic fibers over each $b\in B$, could we ``glue'' these isomorphisms to a global one between the fiber bundles themselves? We immediately see that the information we get only from the fibers is not enough to reconstruct the whole bundle, since it is clear from the above examples that there are non-isomorphic fiber bundles $E_1, E_2$ over a space $B$, which have the same fiber $F$. In such cases, we cannot even construct a map $\phi:E_1\to E_2$ making the following diagram commutative, let alone construct an isomorphism.
\begin{center}
\begin{tikzcd}
E_1\ar[rd,"p_1"']\ar[rr,"\phi"]&[-2em]&[-2em]E_2\ar[ld,"p_2"]\\
&B
\end{tikzcd}
\end{center}
In order to resolve this, we should at least require that there already exists some bundle map $\phi$, the slices of which are the fiber-wise isomorphisms and ask if this is enough for $\phi$ to be an isomorphism.

\begin{example} This is an example of a fiber bundle involving nice spaces, that shows the converse to be false. It is based on an example given in \cite{counterexample} of an exponentiable space, whose set of homeomorphisms does not form a topological group, because the inverse map isn't continuous.

Let $B\subseteq[0,1]$ be a converging sequence, together with its limit:
\[B=\left\{\frac{1}{n}:n\in\mathbb{N}\right\}\cup\left\{0\right\}\]
and $F\subseteq[0,1]$ be a Cantor set, without its leftmost point:
\[F=\left\{\sum_{i=1}^{\infty}\frac{\alpha_n}{3^n}:\alpha_n\in\{0,2\}\right\}\setminus\{0\}\]
both equipped with the subspace topology. Moreover let $E_1,E_2$ be the trivial bundle over $B$, with fiber $F$, i.e. $E_1=E_2=B\times F$. Define now the following function $f:E_1\to E_2$:
\begin{center}
\begin{tabular}{ccc}
\begin{tikzcd}
f\ar[r,phantom,":"]&[-.8cm]B\times F\ar[r]&B\times F\\[-1.2em]
&(b,x)\ar[r,mapsto]&(b,f_b(x))
\end{tikzcd}
&,&
$\displaystyle f_b=\left\{\begin{array}{ll}
1_F&,b=0\\[1.2em]
h_n&,b=\frac{1}{n}
\end{array}\right.$
\end{tabular}
\end{center}
where $h_n:F\to F$ is defined as follows:
\begin{center}
\begin{tabular}{cc}
$\displaystyle h_b(x)=\left\{\begin{array}{ll}
3x&,x\in\left(0,\frac{1}{3^{n+1}}\right]\cap F\\[1em]
x+1-\frac{1}{3^n}&,x\in\left[\frac{2}{3^{n+1}},\frac{1}{3^n}\right]\cap F\\[1em]
x&,x\in\left[\frac{2}{3^n},1-\frac{2}{3^n}\right]\cap F\\[1em]
\frac{1}{3}x+\frac{2}{3}-\frac{2}{3^{n+1}}&,x\in\left[1-\frac{1}{3^n},1\right]\cap F
\end{array}\right.$
&
\end{tabular}
\begin{tikzpicture}[baseline=0, decoration=Cantor set]
\def\d{5}
\node at (0,0) {$h_2$};
\coordinate (O) at (-\d/2,-\d/2);
%\draw[->,very thin,gray,dotted] (O) + (-.2,0) -- +(\d+0.2,0);
%\draw[->,very thin,gray,dotted] (O) + (0,-.2) -- +(0,\d+0.2);
\draw[thick] decorate{ decorate{ decorate{ decorate{ (O) -- +(\d,0) }}}};
\draw[thick] decorate{ decorate{ decorate{ decorate{ (O) -- +(0,\d) }}}};

\draw[red, thick] decorate{ decorate{ (O) -- +(\d/27,\d/9) }};
\draw[red, thick] decorate{ (O) +(2*\d/27,\d-\d/27) -- +(\d/9,\d) };
\draw[red, thick] decorate{ decorate{ (O) +(2*\d/9,2*\d/9) -- +(\d/3,\d/3) }};
\draw[red, thick] decorate{ decorate{ (O) +(\d-\d/3,\d-\d/3) -- +(\d-2*\d/9,\d-2*\d/9) }};
\draw[red, thick] decorate{ decorate{ (O) +(\d-\d/9,\d-\d/9) -- +(\d,\d-2*\d/27) }};

\draw[very thin, gray, dotted] (O) +(\d/27,0) -- +(\d/27,\d/9);
\draw[very thin, gray, dotted] (O) +(2*\d/27,0) -- +(2*\d/27,\d-\d/27);
\draw[very thin, gray, dotted] (O) +(\d/9,0) -- +(\d/9,\d);
\draw[very thin, gray, dotted] (O) +(2*\d/9,0) -- +(2*\d/9,2*\d/9);
\draw[very thin, gray, dotted] (O) +(\d-2*\d/9,0) -- +(\d-2*\d/9,\d-2*\d/9);
\draw[very thin, gray, dotted] (O) +(\d-\d/9,0) -- +(\d-\d/9,\d-\d/9);
\draw[very thin, gray, dotted] (O) +(\d,0) -- +(\d,\d-2*\d/27);

\draw[very thin, gray, dotted] (O) +(0,\d/9) -- +(\d/27,\d/9);
\draw[very thin, gray, dotted] (O) +(0,\d-\d/27) -- +(2*\d/27,\d-\d/27);
\draw[very thin, gray, dotted] (O) +(0,\d) -- +(\d/9,\d);
\draw[very thin, gray, dotted] (O) +(0,2*\d/9) -- +(2*\d/9,2*\d/9);
\draw[very thin, gray, dotted] (O) +(0,\d-2*\d/9) -- +(\d-2*\d/9,\d-2*\d/9);
\draw[very thin, gray, dotted] (O) +(0,\d-\d/9) -- +(\d-\d/9,\d-\d/9);
\draw[very thin, gray, dotted] (O) +(0,\d-2*\d/27) -- +(\d,\d-2*\d/27);
\end{tikzpicture}
\end{center}
We will prove that $f$ is a bundle map, but it isn't a homeomorphism, although $f_b$ is a homeomorphism for every $b\in B$.

First of all, $f_0$ is trivially a homeomorphism and it is easy to see that $h_n$ is always a homeomorphism, since it is bijective and linear on disconnected subsets of $F$. Now, let us prove that $f$ is continuous. It then immediately will follow that $f$ is also a bundle map, since $\pi_1\circ f=\pi_1$. We will first examine the function $g_f:B\times F\to F$, with $g_f:=\pi_2\circ f$. Since $F$ is Hausdorff and locally compact, the space $Top(F,F)$, topologised with the compact-open topology has the following properties, as we proved in Proposition~\ref{prop:exponential_object}:
\begin{itemize}
\item The evaluation map $\mathrm{Top}(F,F)\times F\overset{ev}{\to}F$, defined by $(f,x)\mapsto f(x)$ is continuous.
\item For every topological space $T$ and continuous map $g:T\times F\to F$, there exists a unique $g^{\#}:T\to\mathrm{Top}(F,F)$ making the following diagram commute:
\begin{center}
\begin{tikzcd}
T\times F\ar[r,"g^{\#}\times 1_F"]\ar[d,"g"]&\mathrm{Top}(F,F)\times F\ar[dl,"ev"]\\
F
\end{tikzcd}
\end{center}
\end{itemize}
In such cases, $g$ is continuous if and only if $g^{\#}$ is continuous, as we proved in Theorem~\ref{th:associated_map}.

In our example, if we let $\tilde{g}_f:B\to\mathrm{Top}(F,F)$ to be the function taking $b$ to $f_b$, we have the following commutative diagram:
\begin{center}
\begin{tikzcd}
B\times F\ar[r,"\tilde{g}_f\times 1_F"]\ar[d,"g_f"]&\mathrm{Top}(F,F)\times F\ar[dl,"ev"]\\
F
\end{tikzcd}
\end{center}
which proves that $g_f^{\#}=\tilde{g}_f$, due to the uniqueness property. Thus, $g_f$ is continuous, if and only if $\tilde{g}_f$ is continuous., which is equivalent to $\lim_{n\to\infty}h_n=1_F$ in $\mathrm{Top}(F,F)$. The sets
\[\Big\{S(K,O):=\{f:K\to O\,|\,f(K)\subseteq O\}\ \Big|\ K,O\subseteq F,\ K\text{ compact},\ O\text{ open}\Big\}\]
form a subbasis of the topology of $\mathrm{Top}(F,F)$.
Let $S(K,O)$ be such a set in the neighborhood of $1_F$. This means that moreover $K\subseteq O$. We will prove that $h_n\in S(K,O)$ for every big enough $n$. First of all, notice, that there exists some $n_1\in\mathbb{N}$ such that $\frac{1}{3^n}<\min K$ for all $n\geq n_1$. If $K\not\ni 1$, then there exists also a $n_2\in\mathbb{N}$ such that $1-\frac{1}{3^n}>\max K$ for all $n\geq n_2$. This means that $K\subseteq[\frac{2}{3^n},1-\frac{2}{3^n}]$ for all $n\geq\max\{n_1,n_2\}$, which means that $h_n(K)=K\subseteq O$, i.e. $h_n\in S(K,O)$. Else, if $K\ni 1$, then there exists some $n_2'\in\mathbb{N}$ such that $V_n:=[1-\frac{1}{3^n},1]\cap F\subseteq O$ for every $n\geq n_2'$. Then, $h_n(V_n\cap K)\subseteq h_n(V_n)\subseteq V_n\subseteq O$ for every $n\geq n_2'$ and also, like in the previous case, $h_n(K\setminus V_n)= h_n(K\cap[\frac{2}{3^n},1-\frac{2}{3^n}]\cap F)=K\cap[\frac{2}{3^n},1-\frac{2}{3^n}]\cap F\subseteq O$ for every $n\geq n_1$. Thus, $h_n\in S(K,O)$ for every $n\geq\max\{n_1,n_2'\}$. This proves the continuity of $\tilde{g}_f$, which implies the continuity of $g$. So, we now have the following commutative diagram:
\begin{center}
\begin{tikzcd}
B&B\times F\ar[l,"\pi_1"']\ar[r,"\pi_2"]&F\\
&B\times F\ar[ul,"1_B"]\ar[u,"f"]\ar[ur,"g_f"']
\end{tikzcd}
\end{center}
i.e. $f=1_B\times g$ by the universal property of the product, which means in particular that $f$ is continuous.

The functions $f_b^{-1}$ are also continuous and we again have that $f^{-1}$ is continuous if and only if $g_{f^{-1}}$ is continuous, which in turn is the case if and only if $g_{f^{-1}}^{\#}=\tilde{g}_{f^{-1}}:B\to\mathrm{Top}(F,F)$, taking $b$ to $f_b^{-1}$ is continuous, which isn't. Indeed, let $V_n:=[1-\frac{1}{3^n},1]\cap F$. This is both open and compact, so $S(V_n,V_n)$ is a subbasic neighborhood of $1_F$. For every $n\in\mathbb{N}$ we have that $h_n^{-1}(K)\ni h_n^{-1}(1)=\frac{1}{3^n}<1-\frac{1}{3^n}=\min O$, which means that for every $n\in\mathbb{N}$, $h^{-1}_n\not\in S(V_n,V_n)$.
\end{example}

The above example fails because $F$ although Hausdorff and locally compact, wasn't locally connected. We will now prove the following theorem, which is based on Theorem~4 in \cite{top_group} which proves the continuity of the inverse operator on $\mathrm{Top}(F,F)$, provided that $F$ has some good properties

\begin{theorem} Let $F$ be a Hausdorff, locally compact, locally connected topological space and let $p_1: E_1\to B$ and $p_2: E_2\to B$ be two fiber bundles with fibers $F_1$ and $F_2$ respectively, both isomorphic to $F$. Moreover, let $\phi:E_1\to E_2$ be a bundle map such that its restriction
\[\phi_x:(F_1)_x\to (F_2)_x\]
with $\phi_x(v)=\phi(v)$ is a homeomorphism for every $x\in B$. Then $\phi$ is also a homeomorphism.
\end{theorem}
\begin{proof}

\end{proof}





\begin{example} Define the topological spaces $B=(\mathbb{R}, \tau)$ and $B'=(\mathbb{R}, \tau')$, where $\tau$ is the usual topology on $\mathbb{R}$ generated by the open intervals and $\tau'$ is the topology on $\mathbb{R}$ generated by the semi-open intervals, i.e. of all intervals of the form $[a,b)$. Then examine the following diagram:
\begin{center}
\begin{tikzcd}
B'\ar[rd,"p_1\ =\ id_{\mathbb{R}}"']\ar[rr,"\phi\ =\ id_{\mathbb{R}}"]&[-1.5em]&[-1.5em]B\ar[ld,"p_2\ =\ id_{\mathbb{R}}"]\\
&B
\end{tikzcd}
\end{center}
\end{example}

Now that we have defined what an isomorphism of two fiber bundles is, let us examine what it means for a fiber bundle to be isomorphic to the trivial bundle.

\begin{remark}
Let $p:E\to B$ be a fiber bundle with fiber $F$. Then, $p$ is isomorphic to a trivial fiber bundle, if and only if there exists a map
\[\phi:F\times B\to E\]
such that, for every $x\in B$:
\begin{i_enum}
\item $p\circ\phi(a,x)=x$ (i.e. $\phi$ is a bundle map), and
\item The restriction map $p_x:F\times\{x\}\to p^{-1}(\{x\})$ is an homeomorphism. (i.e. $\phi$ is a bundle isomorphism).
\end{i_enum}
\end{remark}



%%%%TODO
%- pullback





%---------
%%%TODO: build proof environment for equivalences
%
%
%%%TODO: check if lex order is used correctly
%
%
%
