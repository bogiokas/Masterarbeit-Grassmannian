\documentclass[12pt]{book}
%%%%%%%%%%%%%%%%%%%%%%%%%%%%%%%%%%%%%%%%%%%%%%%%%%%%%
%%%%%%%%%%%  MATH  %%%%%%%%%%%%%%%%%%%%%%%%%%%%%%%%%%
\usepackage{amsmath,amssymb,amsthm,amsfonts} %math
\usepackage{bbm} %\mathbbm{1}
\usepackage{array}
\usepackage{xfrac} %sfrac
\usepackage{faktor}
%\everymath{\displaystyle} %to show every math big
%%%%%%%%%%%%%%%%%%%%%%%%%%%%%%%%%%%%%%%%%%%%%%%%%%%%%
%%%%%%%%%%%  OTHER PACKAGES  %%%%%%%%%%%%%%%%%%%%%%%%
\usepackage[utf8]{inputenc} %encoding
\usepackage{makeidx} %indexing
\usepackage{tabularx} %tabular + expand cols
\usepackage{booktabs} %hlines in tabular
\usepackage{colortbl} %color table background
\usepackage{graphicx} %inclusion of graphics
\usepackage{subcaption} %manages subcaptions
\usepackage{wrapfig} %wrap text around figures
\usepackage[usenames,dvipsnames,svgnames]{xcolor} %colornames
\usepackage{hyperref} %cross reference
 \hypersetup{colorlinks=true,linkcolor=MidnightBlue,citecolor=BrickRed} %setup hyper
\usepackage{xstring} %manipulate strings
\usepackage{xparse} %define overloaded commands
\usepackage{remreset} %remove control of a counter over another
\usepackage{blindtext} %produce crap
\usepackage{mfirstuc} %make first upper case
\usepackage{ytableau}
%%%%%%%%%%%%%%%%%%%%%%%%%%%%%%%%%%%%%%%%%%%%%%%%%%%%%
%%%%%%%%%%%  TIKZ 4 LIFE  %%%%%%%%%%%%%%%%%%%%%%%%%%%
\usepackage{tikz}
\usetikzlibrary{cd} %commutative diagrams
\usetikzlibrary{decorations} %curved lines
\usetikzlibrary{decorations.fractals} %cantor
\usetikzlibrary{positioning} %coordinates positioning
\usetikzlibrary{3d,calc} %coordinate calculations & 3d
\usetikzlibrary{backgrounds} %bg of tikzpicture
\usetikzlibrary{calc}
\usetikzlibrary{hobby}
\tikzset{%
	vertex/.style={%
		circle, thick, draw=black, fill, inner sep=0, outer sep=0, minimum height=0.1cm
	},%
	bag/.style n args={3}{% {minRand}{maxRand}{bleed}
		append after command={%
			\pgfextra{%
				\let\thisNode\tikzlastnode
				\foreach\a in {0,60,...,300} {%
					\pgfmathparse{#1 + (#2 - #1)*random()}
					\coordinate (\thisNode\a) at ($(\thisNode) + (\a:\pgfmathresult)$);
					\coordinate (\thisNode\a-Bleed) at ($(\thisNode\a) + (\a:#3)$);
				}
				\ifdim#3>0cm\relax%
				\filldraw[draw=white,thick,closed hobby,inner color=transparent!10, outer color=transparent!0] plot coordinates {(\thisNode0-Bleed) (\thisNode60-Bleed) (\thisNode120-Bleed) (\thisNode180-Bleed) (\thisNode240-Bleed) (\thisNode300-Bleed)};
				\draw[closed hobby, fill=white] plot coordinates {(\thisNode0) (\thisNode60) (\thisNode120) (\thisNode180) (\thisNode240) (\thisNode300)};
				\else%
				\filldraw[closed hobby,inner color=transparent!0, outer color=transparent!3] plot coordinates {(\thisNode0) (\thisNode60) (\thisNode120) (\thisNode180) (\thisNode240) (\thisNode300)};
				\filldraw[draw=white,thick,closed hobby, fill=white] plot coordinates {(\thisNode0-Bleed) (\thisNode60-Bleed) (\thisNode120-Bleed) (\thisNode180-Bleed) (\thisNode240-Bleed) (\thisNode300-Bleed)};
				\fi
			}
		}
	},%
	bubble/.style 2 args={% {buffer}{linewidth}
		preaction={draw,line width=#1 + #2},
		white,fill,draw,line width=#1,
	},%
	elt/.style={
		draw, circle
	}
}
\pgfdeclarelayer{background}
\pgfsetlayers{background,main}
%%%%%%%%%%%%%%%%%%%%%%%%%%%%%%%%%%%%%%%%%%%%%%%%%%%%%
%%%%%%%%%%%  GEOMETRY  %%%%%%%%%%%%%%%%%%%%%%%%%%%%%%
%\usepackage[Glenn]{fncychap} %fancy chapters, with option Glenn
%\ChNameVar{\bfseries\Large} %edits word "chapter"
\usepackage{geometry} %page size and margins
\geometry{headheight=15pt}
\setlength\parindent{0pt} %no indentation
\setlength\parskip{0.5em} %hspace between par
%%%%%%%%%%%%%%%%%%%%%%%%%%%%%%%%%%%%%%%%%%%%%%%%%%%%%
%%%%%%%%%%%  TITLES %%%%%%%%%%%%%%%%%%%%%%%%%%%%%%%%%
\usepackage{titlesec} %customize chapters/sections/...
\usepackage[nottoc]{tocbibind} %add bibliography to toc
\usepackage[titletoc]{appendix} %add appendix to toc
\setcounter{tocdepth}{1} %only include chapters and sections in toc
%\usepackage{biblatex}
% \addbibresource{bibliography/bibsample.bib}
\renewcommand\theequation{\arabic{equation}}
\newcommand\mychapter{\thechapter\quad} %for titleformat
\newcommand\mysection{\thesection\quad} %and chaptermark
\titleformat{\chapter}[hang]{\bfseries\Huge}{}{0pt}{\mychapter}
\titleformat{name=\chapter,numberless}[hang]{\bfseries\Huge}{}{0pt}{}
\titleformat{\section}[hang]{\bfseries\Large}{}{0pt}{\mysection}
\titleformat{name=\section,numberless}[hang]{\bfseries\Large}{}{0pt}{}
\newenvironment{myappendix}{%
	\appendix%
	\renewcommand\chaptername{Appendix}%
	\renewcommand\mychapter{\chaptername\ \thechapter\\}%
	\begin{appendices}
}{%
	\end{appendices}%
	\renewcommand\chaptername{Chapter}%
	\renewcommand\mychapter{\thechapter\quad}%
}
%%%%%%%%%%%%%%%%%%%%%%%%%%%%%%%%%%%%%%%%%%%%%%%%%%%%%
%%%%%%%%%%%  FOOTER / HEADER  %%%%%%%%%%%%%%%%%%%%%%%
\usepackage{fancyhdr} %control of footers/headers
\pagestyle{fancy} %change style to fancy
\renewcommand\chaptermark[1]{\markboth{\mychapter #1}{}}
\renewcommand\sectionmark[1]{\markright{\mysection #1}}
\fancyhf{} %clear header/footer
\fancyhead[LE,RO]{\thepage}
\fancyhead[LO]{\nouppercase{\textit{\leftmark}}}
\fancyhead[RE]{\nouppercase{\textit{\rightmark}}}
\renewcommand{\headrulewidth}{0pt}
\renewcommand{\footrulewidth}{0pt}
\fancypagestyle{plain}{%define plain page style
	\fancyhf{}
	\fancyfoot[LE,RO]{\thepage}
	\renewcommand{\headrulewidth}{0pt}
	\renewcommand{\footrulewidth}{0pt}
}
%%%%%%%%%%%%%%%%%%%%%%%%%%%%%%%%%%%%%%%%%%%%%%%%%%%%%
%%%%%%%%%%%  ENUMERATE / ITEMIZE  %%%%%%%%%%%%%%%%%%%
\usepackage{enumerate} %enumerate/itemize (has to be first)
\usepackage{enumitem} %customize enumerate/itemize
% \setlist{nolistsep} %<=> [nosep] <=> Kills vert sep <=> ![listsep]
% \setlist{noitemsep} % <=> itemsep=0pt, parsep=0pt
% \setlist{topsep=0pt} %before and after list
% \setlist{partopsep=0pt} %before and after list if it starts new paragraph
% \setlist{itemsep=0pt} %between items
% \setlist{parsep=0pt} %between paragaphs inside list
\newenvironment{n_enum}{%
	\begin{enumerate}[label=(\arabic{*})]
}{%
	\end{enumerate}
}%1,2,3,...
\newenvironment{i_enum}{%
	\begin{enumerate}[label=(\roman{*})]
}{%
	\end{enumerate}
} %i,ii,iii,...
\newenvironment{a_enum}{%
	\begin{enumerate}[label=(\alph{*})]
}{%
	\end{enumerate}
} %a,b,c,...
\newenvironment{b_item}{%
	\begin{itemize}
}{%
	\end{itemize}
} %bullets
%\newenvironment{equivproof}{%
%	\begin{itemize}[($\Rightarrow$),($\Leftarrow$)]
%}{%
%	\end{itemize}
%}
%%%%%%%%%%%%%%%%%%%%%%%%%%%%%%%%%%%%%%%%%%%%%%%%%%%%%
%%%%%%%%%%%  MATH NOTATIONS  %%%%%%%%%%%%%%%%%%%%%%%%
\newcommand\conv{\mathrm{conv }}
\newcommand\Hom{\mathrm{Hom}}
\newcommand\Tor{\mathrm{Tor}}
\newcommand\Ext{\mathrm{Ext}}
\newcommand\Mod{\mathrm{Mod}}
\newcommand\Ab{\mathrm{Ab}}
\newcommand\map{\mathrm{map }}
\newcommand\im{\mathrm{im }}
\newcommand\tensor{\otimes}
%\newcommand\ker{\mathrm{ker}}
\newcommand\coim{\mathrm{coim }}
\newcommand\coker{\mathrm{coker }}
\newcommand\cupdot{\mathbin{\mathaccent\cdot\cup}}
\NewDocumentCommand\StO{mgg}{%
	\mathrm{V_0}
	\IfNoValueTF{#3}{}{_{,#3}}
	\left(#1\IfNoValueTF{#2}{}{,#2}\right)
}
\NewDocumentCommand\St{mgg}{%
	\mathrm{V}
	\IfNoValueTF{#3}{}{_{#3}}
	\left(#1\IfNoValueTF{#2}{}{,#2}\right)
}
\NewDocumentCommand\Gr{mgg}{%
	\mathrm{Gr}
	\IfNoValueTF{#3}{}{_{#3}}
	\left(#1\IfNoValueTF{#2}{}{,#2}\right)
}
\newcommand\tworows[8]{\left(\begin{array}{cccc}#1&#2&#3&#4\\{#5}&#6&#7&#8\\\end{array}\right)}
%%%%%%%%%%%%%%%%%%%%%%%%%%%%%%%%%%%%%%%%%%%%%%%%%%%%%
%%%%%%%%%%%  OTHER NOTATIONS  %%%%%%%%%%%%%%%%%%%%%%%
\newcommand\ul[1]{\emph{#1}}
\newcommand\nospace{\hspace*{-0.5em}}
\newcommand\HRule{\rule{\linewidth}{0.1mm}}
\newcommand\commaCat[2]{#1\mkern-5mu\downarrow\mkern-5mu #2}
%% Usage: \mask{long text creating space}{actual test}
\newlength\wantedwidth
\newcommand{\mask}[2]{%
  \settowidth{\wantedwidth}{\ensuremath{#1}}%
  \makebox[\wantedwidth]{\ensuremath{#2}}%
}
%%%%%%%%%%%%%%%%%%%%%%%%%%%%%%%%%%%%%%%%%%%%%%%%%%%%%
%%%%%%%%%%%  COUNTERS  %%%%%%%%%%%%%%%%%%%%%%%%%%%%%%
\usepackage{aliascnt} %alias between counters
\newtheorem{common}{}[chapter] %the common counter
\numberwithin{common}{section}
\newcommand\mynewtheorem[1]{
	\newaliascnt{#1}{common} %counter = common_counter
	\newtheorem{#1}[#1]{\makefirstuc{#1}}
	\aliascntresetthe{#1} %???
	\expandafter\def\csname #1autorefname\endcsname{\makefirstuc{#1}}
}%\def\lemmaautorefname{Lemma}
%%theoremstyle
\mynewtheorem{computation}
\mynewtheorem{computations}
\mynewtheorem{conjecture}
\mynewtheorem{conjectures}
\mynewtheorem{corollary}
\mynewtheorem{corollaries}
\mynewtheorem{example}
\mynewtheorem{examples}
\mynewtheorem{exercise}
\mynewtheorem{exercises}
\mynewtheorem{fact}
\mynewtheorem{facts}
\mynewtheorem{lemma}
\mynewtheorem{lemmata}
\mynewtheorem{notation}
\mynewtheorem{notations}
\mynewtheorem{observation}
\mynewtheorem{observations}
\mynewtheorem{problem}
\mynewtheorem{problems}
\mynewtheorem{proposition}
\mynewtheorem{propositions}
\mynewtheorem{question}
\mynewtheorem{questions}
\mynewtheorem{reminder}
\mynewtheorem{reminders}
\mynewtheorem{theorem}
\mynewtheorem{theorems}
%%theoremstyle
\theoremstyle{definition}
\mynewtheorem{axiom}
\mynewtheorem{axioms}
\mynewtheorem{definition}
\mynewtheorem{definitions}
%%theoremstyle
\theoremstyle{remark}
\mynewtheorem{goal}
\mynewtheorem{goals}
\mynewtheorem{remark}
\mynewtheorem{remarks}
\mynewtheorem{note}
\mynewtheorem{notes}
%% when adding foo, add also \begin{foo} inside [user]\AppData\Roaming\texstudio\completion\user

\makeatletter %for the @ in next line
\@removefromreset{equation}{chapter} %ch-nmb not resetting eq-nmb
\makeatother %for the @ in previous line
%%%%%%%%%%%%%%%%%%%%%%%%%%%%%%%%%%%%%%%%%%%%%%%%%%%%%
%%%%%%%%%%%  MAKE TITLE  %%%%%%%%%%%%%%%%%%%%%%%%%%%%
%\title{\vspace{-30pt}\rule{320pt}{0.5mm}\\\vspace{10pt}Cohomology of Grassmann Manifold\\\rule{320pt}{0.5mm}}
%\author{}
%\date{}
\newcommand\mytitle{Cohomology ring of the real Grassmannians\\
with coefficients from $\mathbb{Z}_2$}
\renewcommand{\maketitle}{%
	\begin{titlepage}
	\newgeometry{left=0cm, right=0cm, top=0cm, bottom=0cm}
	\vspace*{0.3\textheight}
	\begin{center}
	\rule{400pt}{.5mm}\\
	\vspace{10pt}
	\LARGE\mytitle\\
	\rule{400pt}{.5mm}\\
	\end{center}
	\end{titlepage}
}
%%%%%%%%%%%%%%%%%%%%%%%%%%%%%%%%%%%%%%%%%%%%%%%%%%%%%
%%%%%%%%%%%  BEGIN DOCUMENT  %%%%%%%%%%%%%%%%%%%%%%%%
\begin{document}
%cover
	\renewcommand{\thepage}{C\arabic{page}}
	\thispagestyle{empty}
	\maketitle
	\newpage
	\thispagestyle{empty}
%everything that goes in the beginning
\frontmatter
	\pagenumbering{roman}
	%\\\vspace{100pt}\chapter*{Introduction\markboth{Introduction}{Introduction}}
\addcontentsline{toc}{chapter}{Introduction}
In this thesis our ultimate goal is to study the $\mathbb{F}_2$ Cohomology ring of the real Grassmannians $\Gr{k}{n}{\mathbb{R}}$ and to examine a fascinating connection between this algebraic structure and the set of all real vector bundles of $\Gr{k}{n}{\mathbb{R}}$, which is a purely geometric structure. We mainly follow the book \cite{char_class} but we try to present the theory in a more detailed and concise way. The thesis is divided in three chapters.
The first two chapters are going to build the language we need. They are dedicated, respectively, to the two sides of the coin, namely the topology and the geometry involved. Each one of these two chapters is going to be self-contained and thus a completely independent and sufficient read. This means in particular, that these two chapters do not aim at all towards the proof of the main theorem (this is stated and proved in the third chapter), but their goal is instead for the reader to build a stable entry point to the theory on Grassmannians and Vector bundles, respectively. Most importantly, this includes forming a sufficiently good intuition on the topics involved. Thus, we have enriched the theory with plenty of examples and helpful comments. On the other side, we have tried to remove hand-waving as much as possible of our proofs. This has of course made some of the proofs more precise than in the literature and thus lengthier. Our advice to the reader would be to skip some of them in the first read, since we always enclose the proofs inside more intuitive comments about the main arguments in the proof.
The reader can either choose to start at Chapter 1 or Chapter 2, in accordance with her interests and her level of comfort. 

In the first chapter we define the Grassmannians and immediately prove that they are indeed Manifolds. We also prove some first topological properties of them, in order to establish that they are "good" topological spaces which deserve our attention. Next, we introduce a very natural cell structure making them CW-manifolds. These cells are going to be named ``Schubert cells''. This cell decomposition plays a very important role in the study of the (co)homology of the Grassmannians. Namely, one can define a very interesting product on the set of the Schubert classes, which -you could already guess- are very close related to Schubert cells. One then speaks for a whole area called ``Schubert calculus''. We get involved with Schubert calculus just enough to convince the reader that this is a really deep and interesting area, but our approach of this topic is very introductory and not even close to be characterized as complete.

In the second chapter we define the vector bundles in general and prove some basic first results, mainly, in order to understand how different bundles can get combined to form new ones. The highlight of this chapter is going to be the notion of the ``characteristic classes'', already giving the reader an idea how one can bridge the gap between the set of all vector bundles of a space and its topology.

In the last chapter we are going to unleash the combined power we have been gathering in the first two chapters of this thesis in order to prove an amazing result, fully characterising the cohomology algebra of any Grassmannian. At this point we first examine and fully prove the infinite case and then we prove the general result involving the finite Grassmannians.

As far as we know, the results of the third chapter are well known and widely used in the community, but we have yet to see some written proof of the finite case, i.e. the last part of the thesis.

We hope that this over-analytic approach is going to help future students introduce themselves to such beautiful and deep mathematical concepts.

	\tableofcontents
%main document
\mainmatter
	\chapter*{Introduction\markboth{Introduction}{Introduction}}
\addcontentsline{toc}{chapter}{Introduction}
In this thesis our ultimate goal is to study the $\mathbb{F}_2$ Cohomology ring of the real Grassmannians $\Gr{k}{n}{\mathbb{R}}$ and to examine a fascinating connection between this algebraic structure and the set of all real vector bundles of $\Gr{k}{n}{\mathbb{R}}$, which is a purely geometric structure. We mainly follow the book \cite{char_class} but we try to present the theory in a more detailed and concise way. The thesis is divided in three chapters.
The first two chapters are going to build the language we need. They are dedicated, respectively, to the two sides of the coin, namely the topology and the geometry involved. Each one of these two chapters is going to be self-contained and thus a completely independent and sufficient read. This means in particular, that these two chapters do not aim at all towards the proof of the main theorem (this is stated and proved in the third chapter), but their goal is instead for the reader to build a stable entry point to the theory on Grassmannians and Vector bundles, respectively. Most importantly, this includes forming a sufficiently good intuition on the topics involved. Thus, we have enriched the theory with plenty of examples and helpful comments. On the other side, we have tried to remove hand-waving as much as possible of our proofs. This has of course made some of the proofs more precise than in the literature and thus lengthier. Our advice to the reader would be to skip some of them in the first read, since we always enclose the proofs inside more intuitive comments about the main arguments in the proof.
The reader can either choose to start at Chapter 1 or Chapter 2, in accordance with her interests and her level of comfort. 

In the first chapter we define the Grassmannians and immediately prove that they are indeed Manifolds. We also prove some first topological properties of them, in order to establish that they are "good" topological spaces which deserve our attention. Next, we introduce a very natural cell structure making them CW-manifolds. These cells are going to be named ``Schubert cells''. This cell decomposition plays a very important role in the study of the (co)homology of the Grassmannians. Namely, one can define a very interesting product on the set of the Schubert classes, which -you could already guess- are very close related to Schubert cells. One then speaks for a whole area called ``Schubert calculus''. We get involved with Schubert calculus just enough to convince the reader that this is a really deep and interesting area, but our approach of this topic is very introductory and not even close to be characterized as complete.

In the second chapter we define the vector bundles in general and prove some basic first results, mainly, in order to understand how different bundles can get combined to form new ones. The highlight of this chapter is going to be the notion of the ``characteristic classes'', already giving the reader an idea how one can bridge the gap between the set of all vector bundles of a space and its topology.

In the last chapter we are going to unleash the combined power we have been gathering in the first two chapters of this thesis in order to prove an amazing result, fully characterising the cohomology algebra of any Grassmannian. At this point we first examine and fully prove the infinite case and then we prove the general result involving the finite Grassmannians.

As far as we know, the results of the third chapter are well known and widely used in the community, but we have yet to see some written proof of the finite case, i.e. the last part of the thesis.

We hope that this over-analytic approach is going to help future students introduce themselves to such beautiful and deep mathematical concepts.

	\chapter{Grassmannians}
Since the topology of projective spaces has been thoroughly studied and characterized, a logical generalization is imposing a natural topology on the set of $k$-dimensional subspaces of a vector space for any $k\geq1$, which we are going call a \ul{Grassmannian}.

So, the goal of this chapter is to present the basic properties of the Grassmann spaces, or Grassmannians. This chapter will be self-contained and used as a point of reference for the rest of the thesis. Our aim is for the reader to get to know the combinatorial structure of a Grassmannian and be able to eventually do basic computations using the Schubert decomposition of these manifolds. A good point to start would be with their definition.

Recall the definition of the real projective spaces as topological spaces:
\[P\mathbb{R}^n\cong\faktor{\mathbb{R}^{n+1}\setminus\{0\}}{\sim}\]
where two vectors are equivalent, if they span the same line. Notice that in this definition we need to take a proper subset of the whole vector space, in order for the quotient to be well defined. Namely we need the set of all vectors, which span a line. In accordance with that, for the case of the Grassmannians, we need $\St{k}{n}\subsetneq{\left(\mathbb{R}^n\right)}^k$, the space of all $k$-tuples of vectors in $\mathbb{R}^n$, spanning a $k$-dimensional vector space (i.e.\ $k$-frames), as defined and discussed in Appendix~\ref{app:stiefel}.

\begin{definition} Let $0<k<n$ be some natural numbers. Then, the real \ul{Grassmann space} $\Gr{k}{n}{\mathbb{R}}=\Gr{k}{n}$ is the set of all linear $k$-dimensional subspaces of $\mathbb{R}^n$, equipped with the following quotient topology:
\[\Gr{k}{n}{\mathbb{R}}:=\faktor{\St{k}{n}{\mathbb{R}}}{\sim}\]
where $(\vec{v}_1,\ldots,\vec{v}_k)\sim(\vec{u}_1,\ldots,\vec{u}_k)$, if $\left<\vec{v}_1,\ldots,\vec{v}_k\right>=\left<\vec{u}_1,\ldots,\vec{u}_k\right>$.
\end{definition}

If the reader has still in mind the case of the projective spaces, it would be of no surprise that we are about to give an alternative definition of the Grassmannians. We know that the projective space of some dimension could alternatively be defined as the quotient over the unit sphere, rather than over the set of every nonzero vector. The analog of the set of all unit vectors would be here the Stiefel manifold $\StO{k}{n}$, which is the space of all \ul{orthonormal} $k$-frames, as defined and discussed again in Appendix~\ref{app:stiefel}.

\begin{proposition} Let $0<k<n$ be some natural numbers. Let $q:\St{k}{n}\to\Gr{k}{n}$ be the quotient map used in the definition of the Grassmannians and let $q_0:=q|_{\StO{k}{n}}:\StO{k}{n}\to\Gr{k}{n}$ be its restriction to the Stiefel manifold. Then $q_0$ is surjective and continuous. In other words, $q_0$ and $q$ induce the same quotient topology on the set of all $k$-planes in $\mathbb{R}^n$.
\end{proposition}
\begin{proof}
Let $i:\StO{k}{n}\to\St{k}{n}$ be the inclusion and $\mathfrak{gs}:\St{k}{n}\to\StO{k}{n}$ be the Gram-Schmidt process. Then, the following diagram commutes:
\begin{center}
\begin{tikzcd}
\StO{k}{n}\ar[r,hook,"i"]\ar[rd,"q_0"']&\St{k}{n}\ar[r,two heads,"\mathfrak{gs}"]\ar[d,"q"']&\StO{k}{n}\ar[dl,"q_0"]\\
&\Gr{k}{n}
\end{tikzcd}
\end{center}
The left part of this diagram implies $q_0=q\circ i$, which means in particular that $q_0$ is continuous. Moreover, the right part gives $q=q_0\circ\mathfrak{gs}$, which gives us that $q_0$ is a surjective map as well, since $q$ is surjective. Thus, the map $q_0$ is a quotient map. In other words, the maps $q$ and $q_0$ induce the same quotient topology on the set of all $k$-dim subspaces of $\mathbb{R}^n$.
\end{proof}

\section{Grassmannians are Manifolds}
Our first real goal is to prove that the Grassmannians are in fact compact topological manifolds.
\begin{lemma} For each pair of natural numbers $k,n$, with $0<k<n$, the space $\Gr{k}{n}$ is compact, Hausdorff and locally homeomorphic to $\mathbb{R}^{k(n-k)}$. 
\end{lemma}

\begin{proof} \begin{b_item}
\item The Stiefel manifold $\StO{k}{n}$ is a compact topological space, as proven in the Appendix in Lemma~\ref{lem:StO_compact} and $q_0$ is a continuous map. Thus, $\Gr{k}{n}=q_0\left(\StO{k}{n}\right)$ is also compact.
\item In order to show that $\Gr{k}{n}$ is Hausdorff, it suffices to show that it is completely Hausdorff, i.e.\ that any two distinct points in $\Gr{k}{n}$ can be separated by a continuous function $\Gr{k}{n}\to\mathbb{R}$. First, for every vector $v\in\mathbb{R}^n$, we define the map $\phi_v:\StO{k}{n}\to\mathbb{R}$ which takes each orthonormal $k$-frame $(v_1,\ldots,v_k)$ to the square of the distance between $v$ and the linear space spanned by $(v_1,\ldots,v_k)$, i.e.:
\[\phi_v(v_1,\ldots,v_k)=v\cdot v-\sum_{i=1}^k{\left(v\cdot v_i\right)}^2\]
This is a continuous map, which depends only on the spanned $k$-plane, i.e.\ if two $k$-frames have the same image under $q_0$, they also have the same image under $\phi_v$. The universal property of the quotient map $q_0$ implies that there exists a unique continuous map $\psi_v$ making the following diagram commute:

\begin{center}
\begin{tikzcd}
\StO{k}{n}\ar[d,"q_0"']\ar[dr,"\phi_v"]\\
\Gr{k}{n}\ar[r,"\psi_v"',dotted]&\mathbb{R}
\end{tikzcd}
\end{center}

Moreover, since $\phi_v(v_1,\ldots,v_k)=0$ iff $v\in\left<v_1,\ldots,v_k\right>$, we have that $\psi_v(H)=0$ iff $v\in H$. Let now $H_1,H_2\in\Gr{k}{n}$ be two distinct $k$-planes and $v\in H_1\setminus H_2$. Then, we get $\psi_v(H_1)=0\neq\psi_v(H_2)$, proving that $\Gr{k}{n}$ is completely Hausdorff.


\item We will prove that for every $H\in\Gr{k}{n}$, the following subset of $\mathbb{R}^n$ is a neighborhood of $H$, homeomorphic to $\mathbb{R}^{k(n-k)}$:
\[\mathcal{U}_H:=\left\{K\in\Gr{k}{n}:K\cap H^{\perp}=\{0\}\right\}\]
First of all, one can fix an orthonormal basis $\{u_1,\ldots,u_k\}$ of $H$ and an orthonormal basis $\{\bar{u}_1,\ldots,\bar{u}_{n-k}\}$ of $H^{\perp}$. It will be also convenient to regard $\mathbb{R}^n$ as the direct sum $H\oplus H^{\perp}$ for this proof. We also define the orthogonal projections $p_H,p_{H^{\perp}}$ from $H\oplus H^{\perp}$ to each component.

One can now consider $\mathcal{U}_H$ to be the set of all $k$-planes $K$ in $H\oplus H^{\perp}$ for which the map $\left.p_H\right|_{K}$ is a homeomorphism. Indeed, the subspace $K\cap H^{\perp}$ is at least one dimensional, if and only if $\left.p_H\right|_K$ is not injective, if and only if $\left.p_H\right|_K$ is not surjective.

Each $K\in\mathcal{U}_H$ can now be considered as the graph of some linear transformation $T_K$ inside $\Hom_{\mathbb{R}}\left(H,H^{\perp}\right)$ (which has the desired dimension as a vector space over $\mathbb{R}$). Our goal is to define rigorously the function taking $K$ to $T_K$ and prove that this is bicontinuous. We urge the reader to convince herself at this point that $T_K$ depends continuously on $K$ and vice versa, because our approach towards proving this fact is going to be rather technical.

In particular, we are going to construct the desired function using the universal property of the quotient map $q:\St{k}{n}\to\Gr{k}{n}$. Moreover, since we need the topology of $\Hom_{\mathbb{R}}\left(H,H^{\perp}\right)$, we are going to use the fact that its topology is also an induced topology by an isomorphism to $\mathbb{R}^{k(n-k)}$. Finally, this last space, being a direct product, is also topologized through the initial topology with respect to the orthogonal projections on each coordinate. One can summarize these steps in the following commutative diagram:

\begin{center}
\begin{tikzcd}
&[-2em]&[-2em] (a_1,\ldots,a_k)\ar[rr,mapsto]&[-1em]&[-7em]\bar{u}_j^t\cdot p_{H^{\perp}}\circ{\left(\left.p_H\right|_{\left<a_1,\ldots,a_k\right>}\right)}^{-1}\left(u_i\right)&[-7em]\\[-2em]
{\left(\mathbb{R}^n\right)}^k\ar[r,phantom,"\supseteq"]&\St{k}{n}\ar[d,"q"]\ar[r,phantom,"\supseteq"]&q^{-1}\left(\mathcal{U}_H\right)\ar[d,"\left.q\right|"]\ar[dr,dotted,"\tilde{T}"']\ar[drrr,dotted,"f"']\ar[rrr,"f_{i,j}"']&&&\mathbb{R}\\[2em]
&\Gr{k}{n}\ar[r,phantom,"\supseteq"]&\mathcal{U}_H\ar[r,dotted,"T"]&\Hom_{\mathbb{R}}\left(H,H^{\perp}\right)\ar[rr,"\phi","\cong"']&&\mathbb{R}^{[k]\times[n-k]}\ar[u,"\pi_{i,j}"]\\[-2em]
&&K\ar[r,mapsto,dotted]&p_{H^{\perp}}\circ\left(\left.p_H\right|_K\right)^{-1}
\end{tikzcd}
\end{center}

Remember that we are trying to define the map $T$ taking each $k$-space $K$ to the function with graph $K$. Thus, we first should define the map $\tilde T$ which is later going to be equal with $T\circ\left.q\right|$. For now we just define it to be:
\[\tilde T(a_1,\ldots,a_k)=p_{H^{\perp}}\circ{\left(\left.p_H\right|_{\left<a_1,\ldots,a_k\right>}\right)}^{-1}\]
Notice that $\tilde T$ is well defined, since $q^{-1}\left(\mathcal{U}_H\right)$ is exactly the set of all $k$-frames, for which the map $\left.p_H\right|_{\left<a_1,\ldots,a_k\right>}$ is a homeomorphism. Also, notice that $\tilde T$ is going to have the same value for any two $k$-frames that span the same $k$-space, since on input $(a_1,\ldots,a_k)$ the output only depends on $\left<a_1,\ldots,a_k\right>$.

Next, we want to argue that $\tilde T$ is a continuous map. In order to do this, we need to remember how $\Hom_{\mathbb{R}}\left(H,H^{\perp}\right)$ is topologized. Having fixed the bases ${\{u_i\}}_{i\in[k]}$ and ${\{\bar{u}_j\}}_{j\in[n-k]}$, we can define an isomorphism of vector spaces $\phi$, which takes a linear map $L$ to
\[\phi(L)={\left(\bar{u}_j^t\cdot L(u_i)\right)}_{(i,j)\in[k]\times[n-k]}\in\mathbb{R}^{[k]\times[n-k]}\]
Then, the natural topology on $\Hom_{\mathbb{R}}\left(H,H^{\perp}\right)$ is the induced topology by $\phi$. This means in particular, that our $\tilde T$ is continuous if and only if $f:=\phi\circ\tilde T$ is continuous.

Finally, since $\mathbb{R}^{[k]\times[n-k]}=\prod_{(i,j)\in[k]\times[n-k]}\mathbb{R}$ is a categorical product, $\mathbb{R}^{[k]\times[n-k]}$ is equipped with the initial topology, with respect to all orthogonal projections ${\left(\pi_{i,j}\right)}_{(i,j)\in[k]\times[n-k]}$. This means in particular, that the function $f$ that interests us is continuous if and only if every $f_{i,j}:=\pi_{i,j}\circ f$ is continuous. In order to show that every $f_{i,j}$ is a continuous map, we have to write it down in the language of linear algebra:

First, define $A$ to be the matrix of the linear function taking $u_i$ to $a_i$ for each $i\in[k]$, expressed in the bases $\{u_i\}$ and $\{\bar{u}_j\}$:
\[A=\left(\begin{array}{ccc}
a^t_1\cdot u_1&\cdots&a^t_k\cdot u_1\\
\vdots&\ddots&\vdots\\
a^t_1\cdot u_k&\cdots&a^t_k\cdot u_k\\
\midrule
a^t_1\cdot \bar{u}_1&\cdots&a^t_k\cdot \bar{u}_1\\
\vdots&\ddots&\vdots\\
a^t_1\cdot \bar{u}_{n-k}&\cdots&a^t_k\cdot \bar{u}_{n-k}\\
\end{array}\right)=:
\left(\begin{array}{c}A_H\\\midrule A_{H^{\perp}}\\
\end{array}\right)
\in\mathbb{R}^{n\times{k}}\]
The two blocks $A_H\in\mathbb{R}^{k\times k}$ and $A_{H^{\perp}}\in\mathbb{R}^{(n-k)\times k}$ of $A$ correspond to the maps taking $u_i$ to the projections of $a_i$ inside $H$ and inside $H^{\perp}$ respectively.

Notice that the matrix corresponding to the linear map $p_{H^{\perp}}$, with regard to the bases we have fixed is
\[\left(0_{(n-k)\times k}|I_{n-k}\right)\in\mathbb{R}^{(n-k)\times n}\]
and the matrix corresponding to the linear map ${\left(\left.p_H\right|_{\left<a_1,\ldots,a_k\right>}\right)}^{-1}$, with regard to the same bases is
\[A\cdot A_H^{-1}=\left(\begin{array}{c}
I_k\\
\midrule
A_{H^{\perp}}\cdot A_H^{-1}\\
\end{array}\right)
\in\mathbb{R}^{n\times{k}}\]
Indeed, since $\left<a_1,\ldots,a_k\right>\in\mathcal{U}_H$, we know that $\left\{p_H(a_1),\ldots,p_H(a_k)\right\}$ is a basis of $K$, which means that $A_H$ is invertible and thus $A\cdot A_H^{-1}$ is well defined. Moreover, we can easily compute that the application of this matrix to any $p_H(a_r)$ gives us $a_r$ which is exactly what the map ${\left({\left.p_H\right|}_{\left<a_1,\ldots,a_k\right>}\right)}^{-1}$ does to the same basis, which proves our assertion.

This means, that the map $f_{ij}$ takes the element $\left(a_1,\ldots,a_k\right)$ to the real number
\[\begin{array}{rcl}f_{ij}\left(a_1,\ldots,a_k\right)
&=&\bar{u}_j^t\cdot p_{H^{\perp}}\circ{\left(\left.p_H\right|_{\left<a_1,\ldots,a_k\right>}\right)}^{-1}(u_i)\\
&=&\bar{u}_j^t\cdot\left(\left.0_{(n-k)\times k}\right|I_{n-k}\right)\cdot\left(\begin{array}{c}I_k\\\midrule A_{H^{\perp}}\cdot A_H^{-1}\\\end{array}\right)\cdot u_i\\
&=&\bar{u}_j^t\cdot A_{H^{\perp}}\cdot A_H^{-1}\cdot u_i\\
\end{array}\]
Since inner product, matrix multiplication and inversion are continuous, $f_{ij}$ is a continuous function for every $i,j\in[k]\times[n-k]$. This proves that $f$ is continuous as well, which proves that $\tilde T$ is also continuous. Since $\tilde T$ is a continuous function which depends only on the $k$-plane spanned by its input, the universal property of the quotient spaces ensures the existence of a continuous map $T:\mathcal{U}_H\to\Hom_{\mathbb{R}}\left(H,H^{\perp}\right)$ such that $T\circ \left.q\right|=\tilde T$. The uniqueness part of this property ensures that the map $T$ is the one taking $K$ to $p_{H^{\perp}}\circ{\left(\left.p_H\right|_K\right)}^{-1}$, as we wanted.

If we think again this function as taking $K$ to the linear function, whose graph is $K$, we can easily see that this function is both one to one and onto, since two linear maps are different if and only if they have different graphs. This makes $T$ a homeomorphism, proving that finally there exists the homeomorphism
\[\mathcal{U}_H\overset{\phi\circ T}{\cong}\mathbb{R}^{n(n-k)}\]
which finishes the proof of the lemma.\qedhere
\end{b_item}
\end{proof}
In the next section we are going to further examine these spaces topologically and build the appropriate language in order to tackle problems regarding their Homology and Cohomology structures. Before we dive in into this topic though, it would be useful to notice a first duality between these spaces, arising from the duality between a $k$-space inside $\mathbb{R}^n$ and its $n-k$ complement.

We urge the reader now to get convinced that the space $K^{\perp}\in\Gr{n-k}{n}$ depends continuously on $K\in\Gr{k}{n}$, because in order to show this fact, we are going to use again the open sets $\mathcal{U}_H$ defined in the proof of the previous lemma.

\begin{lemma} For each pair of natural numbers $k,n$, with $0<k<n$, the space $\Gr{k}{n}$ is homeomorphic to $\Gr{n-k}{n}$, with the homeomorphism taking some $k$-space to its orthogonal complement inside $\mathbb{R}^n$.
\end{lemma}

\begin{proof} The orthogonal-complement function $\perp:\Gr{k}{n}\to\Gr{n-k}{n}$ is obviously one to one and onto. Thus, it suffices to show that it is continuous. (Since ${\left(K^{\perp}\right)}^{\perp}=K$, for all spaces, continuity for every $0<k<n$ implies bicontinuity.) We are going to prove first that for any $H\in\Gr{k}{n}$ the restriction of this function in $\mathcal{U}_H$ is continuous. For this proof we are going to use the following commutative diagram:
\begin{center}
\begin{tikzcd}
\left(\mathbb{R}^n\right)^k\ar[r,"\supseteq",phantom]&[-2em]q_0^{-1}\left(\mathcal{U}_H\right)\ar[d,"\left.q_0\right|"]\ar[r,"\mathfrak{gs}'"]\ar[rrd,dotted,"\left.\tilde\perp\right|"]&\StO{n}{n}\ar[r,"\pi_{[k+1,n]}"]&[5em]\StO{n-k}{n}\ar[d,"q_0"]\\[2em]
&\mathcal{U}_H\ar[rr,dotted,"\left.\perp\right|"]&&\Gr{n-k}{n}\\[-2em]
&K\ar[rr,mapsto]&&K^{\perp}
\end{tikzcd}
\end{center}
In this diagram, $\mathfrak{gs}'$ is the function that takes an orthonormal $k$-frame $(a_1,\ldots,a_k)$ to the orthonormal $n$-frame constructed after applying the Gram-Schmidt process to the $n$-basis $\left(a_1,\ldots,a_k,\bar{u}_1,\ldots,\bar{u}_{n-k}\right)$ where $\left(\bar{u}_j\right)$ is an orthonormal basis of $H^{\perp}$, just like in the previous Lemma. The next map $\pi_{[k+1,n]}$ is just the orthogonal projection in the last $n-k$ coordinates. Both of these maps are continuous and well defined, and thus we get a continuous map $\left.\tilde{\perp}\right|$, like in the diagram. This map depends only on the plane spanned by the input, and thus the universal property of the quotient spaces ensures the continuity of the map $\left.\perp\right|$.

After establishing the continuity of $\left.\perp\right|_{\mathcal{U}_H}$ for every $H$, notice that
\[\Gr{k}{n} = \bigcup_{H\in\left\{\left<B\right>:B\in\binom{\{e_1,\ldots,e_n\}}{k}\right\}}\mathcal{U}_H\]
The union is over all $k$-planes spanned by any $k$ vectors among $\{e_1,\ldots,e_n\}$, where $e_1,\ldots,e_n$ is the standard basis of $\mathbb{R}^n$. This means that $\#\binom{\{e_1,\ldots,e_n\}}{k}=\binom{n}{k}<\infty$ sets are participating in the union and thus one can use the pasting lemma, proving that $\perp$ is continuous as a function from the whole space $\Gr{k}{n}$ to $\Gr{n-k}{n}$.
\end{proof} 

It would be helpful at this point to mention what are the ``small'' examples of Grassmannians. We already know that of course $\Gr{1}{n}\cong\mathbb{P}^{n-1}$ and because of the last lemma we also know that $\Gr{n-1}{n}\cong\mathbb{P}^{n-1}$. This already takes care of the cases $n=2,3$:
\[\Gr{1}{2}\cong\mathbb{P}^1\qquad\Gr{1}{3}\cong\Gr{2}{3}\cong\mathbb{P}^2\]
This forces us to always consider $\Gr{2}{4}$ as the smallest non-trivial case in our further discussion.

\section{They are also CW-Complexes}
In the previous section we proved that the finite Grassmannians are compact topological manifolds. Our goal now is to prove that they are in fact finite CW complexes. For this, we need to define a cell decomposition of each Grassmannian, which we are going to do next. Before we start laying out the formal definition, it would be best for the reader to have in mind the analogous cell decomposition of the projective space $\mathbb{P}^{n-1}\cong\Gr{1}{n}$, consisting of the following $n$ cells:
\[\left\{l\subseteq\mathbb{R}^1\right\}\cong\mathbb{R}^0\ ,\ \left\{l\subseteq\mathbb{R}^2\setminus\mathbb{R}^1\right\}\cong\mathbb{R}^1\ ,\ \ldots\ ,\ \left\{l\subseteq\mathbb{R}^n\setminus\mathbb{R}^{n-1}\right\}\cong\mathbb{R}^{n-1}\]
This cell decomposition seems natural, but it depends heavily on our basis choice for $\mathbb{R}^n$. This fact does not bother us, since for a different choice we get essentially the same decomposition, in terms of the homology classes we are going to eventually compute. This freedom of choice is going to play an important role though, towards the end of this chapter, when we are going to use different decompositions (i.e.\ depending on different bases) in order to understand the multiplicative structure of the cohomology ring of the Grassmannians. Thus, we first need to define what flags are in an $n$-dimensional vector space.

\begin{definition} Let $V$ be an $n$-dimensional vector space over a field $k$. A \ul{flag} $\mathbb{F}_{\bullet}$ for $V$ is a sequence $\left(\mathbb{F}_0,\mathbb{F}_1,\mathbb{F}_2,\ldots,\mathbb{F}_n\right)$, such that $\dim_k\mathbb{F}_i = i$ for all $i\in\{0,1,\ldots,n\}$ and:
\[0=\mathbb{F}_0\subset\mathbb{F}_1\subset\mathbb{F}_2\subset\cdots\subset\mathbb{F}_n=V\]
Given a flag $\mathbb{F}_{\bullet}$ of $V$, denote an orthonormal basis $f_{\bullet}=(f_1,\ldots,f_n)$ of $V$ to be \ul{compatible} with $\mathbb{F}_{\bullet}$, if
\[f_i\in\mathbb{F}_i\]
for every $i\in[n]$.
\end{definition}

In the future we are going to use the non-standard notation ${[n]}_0$ to denote the set $[n]\cup\{0\}=\{0,1,\ldots,n\}$.

An obvious example of flag is the one we used above, namely the flag with $\mathbb{F}_i=\mathbb{R}_i=\left<e_1,\ldots,e_i\right>$. This is sometimes referred to as \ul{standard flag}, but since we avoid to fix some basis of $\mathbb{R}^n$ in this section, we are going to treat every flag equally.

An obvious remark is that given a flag, one can always find a compatible basis with this flag. In fact, there always exist $2^n$ different compatible orthonormal bases and fixing one is like fixing an ``orientation'' of the flag.

Notice, that the role of the flags on our example above is to distinguish between all the different ways a line can intersect this flag. This is exactly the role a flag is going to play in the general definition of Schubert cells.

\begin{definition} Let $k,n\in\mathbb{N}$ with $0<k<n$. Moreover let $\mathbb{F}_{\bullet}$ be a flag of $\mathbb{R}^n$. Then, for each $k$-element subset $\mathbf{j}=\left\{j_1<j_2<\cdots<j_k\right\}$ of $[n]$ the \ul{Schubert cell} $\mathcal{C}_{\mathbf{j}}\left(\mathbb{F}_{\bullet}\right)$ is defined to be the following subset of $\Gr{k}{n}$:
\[\mathcal{C}_{\mathbf{j}}\left(\mathbb{F}_{\bullet}\right):=\big\{H\in\Gr{k}{n}:\ \dim\left(H\cap\mathbb{F}_i\right)=\max\{l\in{[k]}_0:j_l\leq i\}\ \ \forall i\in{[n]}_0\big\}\]
where we define $j_0$ to be $0$.
\end{definition}

Before we start examining mathematically this definition, let us write down the Schubert cells of the first non-trivial example we have, $\Gr{2}{4}$, with respect to the standard flag of $\mathbb{R}^4$:
\[\begin{array}{rcl}
\mathcal{C}_{\{1,2\}}&=&\big\{H:\ \dim(H\cap\mathbb{R}^{0})=0,\ \dim(H\cap\mathbb{R}^{1})=1,\ \dim(H\cap\mathbb{R}^{2,3,4})=2\big\}\\
&=&\big\{\mathbb{R}^2\big\}\\[.6em]
\mathcal{C}_{\{1,3\}}&=&\big\{H:\ \dim(H\cap\mathbb{R}^{0})=0,\ \dim(H\cap\mathbb{R}^{1,2})=1,\ \dim(H\cap\mathbb{R}^{3,4})=2\big\}\\
&=&\big\{H:\ \mathbb{R}^1\subseteq H\subseteq\mathbb{R}^3,\ H\neq\mathbb{R}^2\big\}\\[.6em]
\mathcal{C}_{\{1,4\}}&=&\big\{H:\ \dim(H\cap\mathbb{R}^{0})=0,\ \dim(H\cap\mathbb{R}^{1,2,3})=1,\ \dim(H\cap\mathbb{R}^{4})=2\big\}\\
&=&\big\{H:\ \mathbb{R}^1\subseteq H,\ H\not\subseteq\mathbb{R}^3\big\}\\[.6em]
\mathcal{C}_{\{2,3\}}&=&\big\{H:\ \dim(H\cap\mathbb{R}^{0,1})=0,\ \dim(H\cap\mathbb{R}^{2})=1,\ \dim(H\cap\mathbb{R}^{3,4})=2\big\}\\
&=&\big\{H:\ H\subseteq\mathbb{R}^3,\ \mathbb{R}^1\not\subseteq H\big\}\\[.6em]
\mathcal{C}_{\{2,4\}}&=&\big\{H:\ \dim(H\cap\mathbb{R}^{0,1})=0,\ \dim(H\cap\mathbb{R}^{2,3})=1,\ \dim(H\cap\mathbb{R}^{4})=2\big\}\\
&=&\big\{H:\ \dim(H\cap\mathbb{R}^2)=1,\ \mathbb{R}^1\not\subseteq H,\ H\not\subseteq\mathbb{R}^3\big\}\\[.6em]
\mathcal{C}_{\{3,4\}}&=&\big\{H:\ \dim(H\cap\mathbb{R}^{0,1,2})=0,\ \dim(H\cap\mathbb{R}^{3})=1,\ \dim(H\cap\mathbb{R}^{4})=2\big\}\\
&=&\big\{H:\ H\cap\mathbb{R}^2=\{0\}\big\}\\[.6em]
\end{array}\]
Although we see that there exist dimension restrictions in the definitions of the cells which can be omitted (for example that $H\cap\mathbb{R}^0=0$), the final form doesn't feel intuitive either. Let us take a step back for a moment and see what the Schubert cell decomposition of the (well-known) projective space is. Take for example $\Gr{1}{4}\cong\mathbb{P}^2$:
\[\begin{array}{rcl}
\mathcal{C}_{\{1\}}&=&\big\{l:\ \dim(l\cap\mathbb{R}^{0})=0,\ \dim(l\cap\mathbb{R}^{1,2,3,4})=1\big\}\\
&=&\big\{\mathbb{R}^1\big\}\\[.6em]
\mathcal{C}_{\{2\}}&=&\big\{l:\ \dim(l\cap\mathbb{R}^{0,1})=0,\ \dim(l\cap\mathbb{R}^{2,3,4})=1\big\}\\
&=&\big\{l:\ l\subseteq\mathbb{R}^2,\ l\neq\mathbb{R}^1\big\}\\[.6em]
\mathcal{C}_{\{3\}}&=&\big\{l:\ \dim(l\cap\mathbb{R}^{0,1,2})=0,\ \dim(l\cap\mathbb{R}^{3,4})=1\big\}\\
&=&\big\{l:\ l\subseteq\mathbb{R}^3,\ l\not\subseteq\mathbb{R}^2\big\}\\[.6em]
\mathcal{C}_{\{4\}}&=&\big\{l:\ \dim(l\cap\mathbb{R}^{0,1,2,3})=0,\ \dim(l\cap\mathbb{R}^{4})=1\big\}\\
&=&\big\{l:\ l\not\subseteq\mathbb{R}^3\big\}\\[.6em]
\end{array}\]
We can easily predict how the general Schubert cell (with respect to some flag $\mathbb{F}_{\bullet}$) of any projective space looks like: It will be the set of all lines contained in $\mathbb{F}_k\setminus\mathbb{F}_{k-1}$. This gives us a serial way to think of the cells of a particular projective space, which is the result of the total order that the set $\binom{[n]}{1}$ naturally has. Since $\binom{[n]}{k}$ is in general naturally a poset (inheriting the coordinate-wise ordering on the set of $k$ element sequences ${[n]}^k$), it is now of no surprise that the same poset structure lies behind the Schubert decomposition. We are going to more precisely investigate into this structure, when we examine the closure of these cells we just defined.

Our goal now is to convince the reader that this is a meaningful decomposition of the Grassmannians, i.e.\ to prove eventually that this makes every $\Gr{k}{n}$ into a CW complex. Let us start with proving that ${\left\{\mathcal{C}_{\mathbf{j}}(\mathbb{F}_{\bullet})\right\}}_{\mathbf{j}\in\binom{[n]}{k}}$ is indeed a decomposition. The following proof makes sense, if one conceptualizes a $k$-subset of $[n]$, as the $k$ ``jump points'' of the dimension of the intersections $H\cap\mathbb{F}_i$, for the various $i$.

\begin{lemma} For any integers $0<k<n$ and for every flag $\mathbb{F}_{\bullet}$ for $\mathbb{R}^n$, the set of all Schubert cells ${\left\{\mathcal{C}_{\mathbf{j}}(\mathbb{F}_{\bullet})\right\}}_{\mathbf{j}\in\binom{[n]}{k}}$ is a partition of the Grassmannian $\Gr{k}{n}$.
\end{lemma}

\begin{proof} It is rather obvious that two cells are disjoint, since each set of $k$ elements in $[n]$ describes uniquely $k$ jump points of the dimensions of $H\cap\mathbb{F}_0,H\cap\mathbb{F}_1,\ldots,H\cap\mathbb{F}_n$. Moreover, for a $k$-plane $H$ the dimensions in this sequence of intersections can increase at most by $1$ in each step. Indeed, because of the short exact sequence
\[0\to H\cap\mathbb{F}_{i-1}\to H\cap\mathbb{F}_i\to\faktor{H\cap\mathbb{F}_i}{H\cap\mathbb{F}_{i-1}}\to0\]
we get, for every $i\in[n]$:
\[\dim_{\mathbb{R}}\left(H\cap\mathbb{F}_i\right)-\dim_{\mathbb{R}}\left(H\cap\mathbb{F}_{i-1}\right)=\dim_{\mathbb{R}}\faktor{H\cap\mathbb{F}_i}{H\cap\mathbb{F}_{i-1}}\]
Using the second isomorphism theorem for vector spaces, we get:
\[\begin{array}{>{\displaystyle}r>{\displaystyle}c>{\displaystyle}l}\faktor{H\cap\mathbb{F}_i}{H\cap\mathbb{F}_{i-1}}
&=&\faktor{H\cap\mathbb{F}_i}{H\cap\mathbb{F}_i\cap\mathbb{F}_{i-1}}\cong\faktor{(H\cap\mathbb{F}_i)+\mathbb{F}_{i-1}}{\mathbb{F}_{i-1}}\\[1.5em]
&\cong&\faktor{(H+\mathbb{F}_{i-1})\cap(\mathbb{F}_i+\mathbb{F}_{i-1})}{\mathbb{F}_{i-1}}\cong\faktor{(H+\mathbb{F}_{i-1})\cap\mathbb{F}_i}{\mathbb{F}_{i-1}}\\
\end{array}\]
with the last vector space being obviously a subspace of $\faktor{\mathbb{F}_i}{\mathbb{F}_{i-1}}$, which gives finally:
\[\dim_{\mathbb{R}}\faktor{H\cap\mathbb{F}_i}{H\cap\mathbb{F}_{i-1}}\leq\dim_{\mathbb{R}}\faktor{\mathbb{F}_i}{\mathbb{F}_{i-1}}=1\]
Which means that there exist exactly $k$ jump points in the sequence of the dimensions of $H\cap\mathbb{F}_0,\ldots,H\cap\mathbb{F}_n$, putting $H$ in some Schubert cell.
\end{proof}

We now want to prove that these cells are indeed building blocks we can use, i.e.\ that they are in fact homeomorphic to open balls of various dimensions. Before writing down the formula for the dimension of a general Schubert cell, let us revisit the case of $\Gr{2}{4}$ and try to compute the dimension of the cells by hand-waving:

\begin{example} In $\Gr{2}{4}$ we have the Schubert decomposition we discussed earlier, w.r.t.\ the standard flag. Let us represent each plane $H$ in $\mathbb{R}^4$ by the unique corresponding $2\times 4$ matrix whose rows span $H$ and the matrix is in its reduced echelon form. Then, we get the following picture if we look at the form of the matrices for each cell. Remember, the index $\mathbf{j}\subseteq\binom{[4]}{2}$ refers to the dimension jump points:
\begin{center}
\begin{tikzcd}
\mathcal{C}_{1,2}\ar[r,leftrightarrow]&\tworows{1}{0}{0}{0}{0}{1}{0}{0}&\mathcal{C}_{1,3}\ar[r,leftrightarrow]&\tworows{1}{0}{0}{0}{0}{*}{1}{0}\\
\mathcal{C}_{1,4}\ar[r,leftrightarrow]&\tworows{1}{0}{0}{0}{0}{*}{*}{1}&\mathcal{C}_{2,3}\ar[r,leftrightarrow]&\tworows{*}{1}{0}{0}{*}{0}{1}{0}\\
\mathcal{C}_{2,4}\ar[r,leftrightarrow]&\tworows{*}{1}{0}{0}{*}{0}{*}{1}&\mathcal{C}_{3,4}\ar[r,leftrightarrow]&\tworows{*}{*}{1}{0}{*}{*}{0}{1}\\
\end{tikzcd}
\end{center}
In fact, every matrix of a given form gives a unique plane in the according cell. Thus, we get a bijection and the dimension of the cells equals the number of the choices we have in each matrix, i.e.\ the number of stars. Notice how the jumps in the dimensions now correspond to pivot elements of the rows. Moreover, notice that the first row has always $j_1-1$ stars and the second $j_2-2$.
\end{example}

We are now going to compute the dimension in general. Given a set $\mathbf{j}=\{j_1<\cdots<j_k\}\in\binom{[n]}{k}$, define the number
\[d(\mathbf{j})=(j_1-1)+(j_2-2)+\cdots+(j_k-k)\]
Using the reduced echelon form for a matrix written in a suitable base of $\mathbb{R}^n$, depending on the flag $\mathbb{F}_{\bullet}$, one can easily argue that $\mathcal{C}_{\mathbf{j}}\left(\mathbb{F}_{\bullet}\right)$ is an open cell of dimension $d(\mathbf{j})$, for any $\mathbf{j}\in\binom{[n]}{k}$, but our goal is to eventually prove that these open cells also give the Grassmannian a CW structure. For this we need to find maps from the closed cells of the appropriate dimensions into the Grassmannian and unfortunately it is not easy to work with the compactification of the matrices defined above. Thus, the approach in the bibliography may seem more artificial and it is also the one we employ here:

Our main goal is to prove Lemma~\ref{lem:dim_of_cells}. The approach is:
\begin{i_enum}
\item First to define appropriate sets $\tilde{\mathcal{C}}_{\mathbf{j}}(\mathbb{F}_{\bullet})$ living in the Stiefel manifold, each one ``above'' the matching Schubert cell. (Definition~\ref{def:lift_of_Schubert_cells})
\item Then to prove that each closure $\tilde{\mathcal{C}}_{\mathbf{j}}{\left(\mathbb{F}_{\bullet}\right)}^-$ is homeomorphic with a closed disk of the right dimension. (Lemma~\ref{lem:shub_dim})
\item Finally, to prove that $q_0$ maps $\tilde{\mathcal{C}}_{\mathbf{j}}(\mathbb{F}_{\bullet})$ homeomorphically onto $\mathcal{C}_{\mathbf{j}}(\mathbb{F}_{\bullet})$. (Lemma~\ref{lem:back_to_Schubert_cells})
\end{i_enum}
Let us begin with the definition:

\begin{definition}\label{def:lift_of_Schubert_cells} Let $k,n\in\mathbb{N}$, with $0<k<n$. Moreover, let $\mathbb{F}_{\bullet}$ be a flag of $\mathbb{R}^n$ and $f_{\bullet}$ an orthonormal basis of $\mathbb{R}^n$, compatible with $\mathbb{F}_{\bullet}$. Then, for each $\mathbf{j}=\{j_1<j_2<\cdots<j_k\}\in\binom{[n]}{k}$ define $\tilde{\mathcal{C}}_{\mathbf{j}}(\mathbb{F}_{\bullet})=\tilde{\mathcal{C}}_{\mathbf{j}}(\mathbb{F}_{\bullet},f_{\bullet})$ to be the following subset of $\StO{k}{n}$:
\[\tilde{\mathcal{C}}_{\mathbf{j}}(\mathbb{F}_{\bullet},f_{\bullet}):=\left\{(v_1,v_2,\ldots,v_k)\in\StO{k}{n}:v_l\in\mathbb{H}_{j_l}(\mathbb{F}_{\bullet},f_{\bullet})\ \forall l\in[k]\right\}\]
where $\mathbb{H}_i(\mathbb{F}_{\bullet},f_{\bullet})$ is defined to be the ``positive open halfspace'' of $\mathbb{F}_{i}$, w.r.t.\ the orientation defined by $f_i$:
\[\mathbb{H}_i(\mathbb{F}_{\bullet},f_{\bullet}):=\left\{v\in\mathbb{F}_i:v\cdot f_i>0\right\}\]
for every $i\in[n]$.
\end{definition}

Although we are going to prove that the images of these subspaces of $\StO{k}{n}$ are the Schubert decomposition of $\Gr{k}{n}$, notice that for a fixed basis $f_{\bullet}$, they do not even cover $\StO{k}{n}$ and if we regard all possible compatible bases, they do cover the whole space, but we take most of these sets multiple times.

Let us prove now that these sets have the right dimension and that their closure (now much easier to handle than the matrices in echelon form) is homeomorphic to a closed cell of this dimension. In order to do so in Lemma~\ref{lem:shub_dim}, we first need the following trivial (but lengthy) assertion:

\begin{lemma} For any integers $0<k<n$, for any flag $\mathbb{F}_{\bullet}$ of $\mathbb{R}^n$, for any orthonormal basis $f_{\bullet}$ of $\mathbb{R}^n$, compatible with $\mathbb{F}_{\bullet}$ and for any set $\mathbf{j}\in\binom{[n]}{k}$ the closure of $\tilde{\mathcal{C}}_{\mathbf{j}}{\left(\mathbb{F}_{\bullet},f_{\bullet}\right)}$ inside ${\left(\mathbb{R}^n\right)}^k$ is:
\[{\tilde{\mathcal{C}}_{\mathbf{j}}{\left(\mathbb{F}_{\bullet},f_{\bullet}\right)}}^-=\left\{(v_1,v_2,\ldots,v_k)\in\StO{k}{n}:v_l\in{\mathbb{H}_{j_l}(\mathbb{F}_{\bullet},f_{\bullet})}^-\ \forall l\in[k]\right\}\]
where:
\[{\mathbb{H}_i(\mathbb{F}_{\bullet},f_{\bullet})}^-:=\left\{v\in\mathbb{F}_i:v\cdot f_i\geq 0\right\}\]
\end{lemma}
\begin{proof} Using simple point-set topology and the fact that $\StO{k}{n}$ is closed inside ${\left(\mathbb{R}^n\right)}^k$ as proven in Proposition~\ref{prop:StO_dim_closed}, we have:
\[\begin{array}{rcl}\tilde{\mathcal{C}}_{\mathbf{j}}{\left(\mathbb{F}_{\bullet},f_{\bullet}\right)}^-
&=&{\left\{(v_1,v_2,\ldots,v_k)\in\StO{k}{n}:v_l\in\mathbb{H}_{j_l}(\mathbb{F}_{\bullet},f_{\bullet})\ \forall l\in[k]\right\}}^-\\
&=&{\big(\StO{k}{n}\cap\mathbb{H}_{j_1}(\mathbb{F}_{\bullet},f_{\bullet})\times\cdots\times\mathbb{H}_{j_k}(\mathbb{F}_{\bullet},f_{\bullet})\big)}^-\\
&\subseteq&\StO{k}{n}\cap{\big(\mathbb{H}_{j_1}(\mathbb{F}_{\bullet},f_{\bullet})\times\cdots\times\mathbb{H}_{j_k}(\mathbb{F}_{\bullet},f_{\bullet})\big)}^-\\
&=&\StO{k}{n}\cap\mathbb{H}_{j_1}{\left(\mathbb{F}_{\bullet},f_{\bullet}\right)}^-\times\cdots\times\mathbb{H}_{j_k}{\left(\mathbb{F}_{\bullet},f_{\bullet}\right)}^-\\
&=&\left\{(v_1,v_2,\ldots,v_k)\in\StO{k}{n}:v_l\in{\mathbb{H}_{j_l}(\mathbb{F}_{\bullet},f_{\bullet})}^-\ \forall l\in[k]\right\}\\
\end{array}\]
The inclusion in the third line is actually an equality, as we are going to prove. Indeed, let $(v_1,\ldots,v_k)\in\StO{k}{n}\cap{\big(\mathbb{H}_{j_1}(\mathbb{F}_{\bullet},f_{\bullet})\times\cdots\times\mathbb{H}_{j_k}(\mathbb{F}_{\bullet},f_{\bullet})\big)}^-$. This means that there exists a sequence ${(v_1^m,\ldots,v_k^m)}_m\in\mathbb{H}_{j_1}(\mathbb{F}_{\bullet},f_{\bullet})\times\cdots\times\mathbb{H}_{j_k}(\mathbb{F}_{\bullet},f_{\bullet})$ converging to $(v_1,\ldots,v_k)\in\StO{k}{n}\subseteq\St{k}{n}$, inside ${\left(\mathbb{R}^n\right)}^k$. Since $\St{k}{n}$ is open, as proven in Proposition~\ref{prop:St_open}, there exists some $m_0\in\mathbb{N}$, such that ${(v_1^m,\ldots,v_k^m)}_m\in\St{k}{n}$ for all $m\geq m_0$. For each such tuple, define now ${(w_1^m,\ldots,w_k^m)}_m\in\StO{k}{n}$ to be the result of the Gram-Schmidt process on input ${(v_1^m,\ldots,v_k^m)}_m$ for every $m\geq m_0$:
\[{(w_1^m,\ldots,w_k^m)}_m:=\mathfrak{gs}\big({(v_1^m,\ldots,v_k^m)}_m\big)\]
First of all, $\mathfrak{gs}:\St{k}{n}\to\StO{k}{n}$ is a continuous map, which means that
\[{(w_1^m,\ldots,w_k^m)}_m=\mathfrak{gs}\big({(v_1^m,\ldots,v_k^m)}_m\big)\overset{m\to\infty}{\to}\mathfrak{gs}(v_1,\ldots,v_k)=(v_1,\ldots,v_k)\]
It now suffices to show that $\mathfrak{gs}(v_1',\ldots,v_k')\in\mathbb{H}_{j_1}(\mathbb{F}_{\bullet},f_{\bullet})\times\cdots\times\mathbb{H}_{j_k}(\mathbb{F}_{\bullet},f_{\bullet})$ for any $(v_1',\ldots,v_k')\in\mathbb{H}_{j_1}(\mathbb{F}_{\bullet},f_{\bullet})\times\cdots\times\mathbb{H}_{j_k}(\mathbb{F}_{\bullet},f_{\bullet})$. We are going to show this recursively, using the recursive nature of the Gram-Schmidt process. Assume that the assertion holds for every dimension between $1$ and $k-1$, let $(v_1',\ldots,v_k')\in\mathbb{H}_{j_1}(\mathbb{F}_{\bullet},f_{\bullet})\times\cdots\times\mathbb{H}_{j_k}(\mathbb{F}_{\bullet},f_{\bullet})$ and let also $(w_1',\ldots,w_{k-1}')=\mathfrak{gs}(v_1',\ldots,v_{k-1}')$. Then:
\[\mathfrak{gs}(v_1',\ldots,v_k')=\left(w_1',\ldots,w_{k-1}',\frac{1}{\left\|v_k'-\sum_{l=1}^{k-1}(w_l'\cdot v_k')w_l'\right\|}\left(v_k'-\sum_{l=1}^{k-1}(w_l'\cdot v_k')w_l'\right)\right)\]
Because of the inductive hypothesis, we only need to take care of the last vector, which we will denote as $w_k'$. Because $\mathbb{F}_{\bullet}$ is a flag, we have that $\mathbb{F}_{j_1}\subseteq\mathbb{F}_{j_2}\subseteq\cdots\subseteq\mathbb{F}_{j_k}$ and thus: $w_1',w_2',\ldots,w_{k-1}'\in\mathbb{F}_{j_k}$. Since $v_k'\in\mathbb{F}_{j_k}$ as well, we have that:
\[w_k'=\frac{1}{\left\|v_k'-\sum_{l=1}^{k-1}(w_l'\cdot v_k')w_l'\right\|}\left(v_k'-\sum_{l=1}^{k-1}(w_l'\cdot v_k')w_l'\right)\in\mathbb{F}_{j_k}\]
Moreover, it is true that $v\cdot f_i=0$, if $v\in\mathbb{F}_{i-1}$ for some $i\in[n]$. Indeed, $f_{\bullet}$ is compatible with $\mathbb{F}_{\bullet}$, which means that $v\perp f_i$ for every $v\in\mathbb{F}_{i-1}$. Since $w_1',\ldots,w_{k-1}'\in\mathbb{F}_{j_{k-1}}$, we have:
\[w_k'\cdot f_{j_k}=\frac{1}{\left\|v_k'-\sum_{l=1}^{k-1}(w_l'\cdot v_k')w_l'\right\|}v_k'\cdot f_{j_k}>0\]
This proves that $w_k'\in\mathbb{H}_{j_k}(\mathbb{F}_{\bullet},f_{\bullet})$, which proves in turn:
\[\mathfrak{gs}(v_1',\ldots,v_k')\in\mathbb{H}_{j_1}(\mathbb{F}_{\bullet},f_{\bullet})\times\cdots\times\mathbb{H}_{j_k}(\mathbb{F}_{\bullet},f_{\bullet})\]
Thus, the sequence ${(w_1^m,\ldots,w_k^m)}_m$ is a sequence inside $\StO{k}{n}\cap\mathbb{H}_{j_1}(\mathbb{F}_{\bullet},f_{\bullet})\times\cdots\times\mathbb{H}_{j_k}(\mathbb{F}_{\bullet},f_{\bullet})$ converging to $(v_1,\ldots,v_k)$, which means, finally that:
\[(v_1,\ldots,v_k)\in{\big(\StO{k}{n}\cap\mathbb{H}_{j_1}(\mathbb{F}_{\bullet},f_{\bullet})\times\cdots\times\mathbb{H}_{j_k}(\mathbb{F}_{\bullet},f_{\bullet})\big)}^-\]
which proves the desired inclusion.
\end{proof}

\begin{lemma}\label{lem:shub_dim} For any integers $0<k<n$, for any flag $\mathbb{F}_{\bullet}$ of $\mathbb{R}^n$, for any orthonormal basis $f_{\bullet}$ of $\mathbb{R}^n$, compatible with $\mathbb{F}_{\bullet}$ and for any set $\mathbf{j}\in\binom{[n]}{k}$, there exists a homeomorphism
\[\tilde{\Phi}_{\mathbf{j}}:D^{d(\mathbf{j})}\to\tilde{C}_{\mathbf{j}}{\left(\mathbb{F}_{\bullet},f_{\bullet}\right)}^-\]
\end{lemma}
\begin{proof} For this proof we are going to use the approach of Hatcher~\cite{vec_bundles} (p.37), in order to already familiarize ourselves with the notion of a trivial fiber bundle. 





%%%%%%%TODO
\newpage





We will construct the desired homeomorphism inductively in $k$. For $k=1$ we have:
\[\tilde{\mathcal{C}}_{\{j_1\}}{\left(\mathbb{F}_{\bullet},f_{\bullet}\right)}^-=\left\{v\in\mathbb{F}_{j_1}:v\cdot v=1\ ,\ v\cdot f_{j_1}\geq0\right\}\]

It is known that the closed unit semisphere of $\mathbb{R}^p$ is homeomorphic to the closed disc $D^{p-1}$ for every $p\in\mathbb{N}$. Let us fix now once and for all, for every $p\in\mathbb{N}$, a homeomorphism
\[\psi_p:D^{p-1}\to\left\{v\in\mathbb{R}^p:v\cdot v=1\ ,\ v\cdot e_p\geq 0\right\}\]
One such homeomorphism would be the inverse of the restriction of the usual projection $\mathbb{R}^p\to\mathbb{R}^{p-1}$.

\[\St{k}{n}\cap\cdots=
\{(v_1,\ldots,v_k):(v_1,\ldots,v_{k-1})\in\St{k}{n}{F}\ v_l\in F_{j_l}\ v_k\cdot v_l=\delta_{k,l}\}\]



\end{proof}

\begin{lemma}\label{lem:dim_of_cells} For any integers $0<k<n$, for any flag $\mathbb{F}_{\bullet}$ of $\mathbb{R}^n$ and for any set $\mathbf{j}\in\binom{[n]}{k}$ there exists a map
\[\Phi_{\mathbf{j}}:D^{d(\mathbf{j})}\to\Gr{k}{n}\]
such that:
\begin{i_enum}
\item $\Phi_{\mathbf{j}}\big({\left(D^{d(\mathbf{j})}\right)}^{\circ}\big)\subseteq\mathcal{C}_{\mathbf{j}}(\mathbb{F}_{\bullet})$ and
\item $\Phi_{\mathbf{j}}|_{{\left(D^{d(\mathbf{j})}\right)}^{\circ}}\to\mathcal{C}_{\mathbf{j}}(\mathbb{F}_{\bullet})$ is a homeomorphism.
\end{i_enum}
\end{lemma}





	%!TEX root = Cohomology of real Grassmannians.tex
\chapter{Vector Bundles}\label{chap:vector_bundles}
The discussion in this chapter is largely based on the Chapters 2,3,4 of \cite{char_class} and its goal is to introduce the notion of vector bundles, how to combine them to new bundles and how to detect vector bundle isomorphisms. In the first section, the more general notion of fiber bundles is also introduced, since some basic notions regarding them are used later in Chapter 3. Near the end of this Chapter, the focus lies on vector bundles that fiber-wise admit also an inner product, which is globally continuous.

\section{Basic Notions}
\begin{definition}\label{def:vector_bundle} Let $n\in\mathbb{N}$ and $E,B$ be some topological spaces. A continuous function $\xi:E\to B$ is an \emph{$n$-plane vector bundle}, if for every $x\in B$ the set $\xi^{-1}(x)$ is a vector space and moreover for every $x\in B$ there exists an open $U\subseteq B$ containing $x$ and a continuous function $f_U:\xi^{-1}(U)\to U\times\mathbb{R}^n$, called \emph{local trivialization of $\xi$}, where $\xi^{-1}(U)$ has the subspace topology and $U\times\mathbb{R}^n$ the product topology, such that
\vspace*{-1em}
\begin{center}
\begin{minipage}{0.65\textwidth}
\begin{i_enum}
\item $f_U$ is a homeomorphism,
\item $\xi|_{\xi^{-1}(U)}=\pi_1\circ f_U$, i.e. the diagram on the right commutes and
\item $f_U|_{\xi^{-1}(x)}:\xi^{-1}(x)\to\{x\}\times\mathbb{R}^n$ is linear for every $x\in U$.
\end{i_enum}
\end{minipage}
\begin{minipage}{0.34\textwidth}
\begin{center}
\begin{tikzcd}
E\ar[r,"\supseteq",phantom]&[-1.5em]\xi^{-1}(U)\ar[d,"\xi|_{\xi^{-1}(U)}"']\ar[r,"f_U"]&U\times\mathbb{R}^n\ar[dl,"\pi_1"]\\[1em]
B\ar[r,"\supseteq",phantom]&U
\end{tikzcd}
\end{center}
\end{minipage}
\end{center}
$B=B(\xi)$ is then called \emph{base space} and $E=E(\xi)$ \emph{total space} of $\xi$. Moreover, for every $x\in B$, the space $\xi^{-1}(x)$ is called \emph{the fiber over $x$}.
\end{definition}
\begin{definition}\label{def:fiber_bundle} Let $B$, $E$ and $F$ be some topological spaces. A continuous function $p:E\to B$ is a \emph{fiber bundle with fiber $F$}, if for every $x\in B$ there exists an open $U\subseteq B$ containing $x$, and a continuous function $f_U:p^{-1}(U)\to U\times F$ called \emph{local trivialization of $p$}, where $p^{-1}(U)$ has the subspace topology and $U\times F$ the product topology, such that $f_U$ is a homeomorphism and $p|_{p^{-1}(U)}=\pi_1\circ f_U$. $B=B(p)$ is then called the \emph{base space} and $E=E(p)$ the \emph{total space} of $p$. Moreover, for every $x\in B$, the space $p^{-1}(\{x\})$ is called \ul{the fiber over $x$} and is denoted by $F_x$.
\end{definition}
\begin{remark}\label{rem:subcover} In both cases it is clear that for any open $V\subseteq U$, $f_U|_V$ is also a local trivialization.
\end{remark}

One way to visualize fiber or vector bundles is to ``attach'' to a base space $B$ a copy of the fiber on each point.
\begin{proposition}\label{prop:same_fiber_fb} Let $p:E(p)\to B$ be a fiber bundle with fiber $F$, then $F_x\cong F$ for all $x\in B$.
\end{proposition}
\begin{proof}
Let $x\in B$. Then, there exists a neighborhood $U$ of $x$ inside $B$ and a homeomorphism $f_U:p^{-1}(U)\to U\times F$, such that $p|_{p^{-1}(U)}=\pi_1\circ f_U$. Then, $f_U|_{F_x}:F_x\to f_U(F_x)$ is also a homeomorphism, as a restriction of a homeomorphism. It is clear that $f_U(F_x)=\{x\}\times F\cong F$.
\end{proof}
\begin{remark}\label{rem:same_fiber_vb} A vector bundle $\xi:E(\xi)\to B$ is in particular a fiber bundle and thus $\xi^{-1}(x)\cong\mathbb{R}^n$ as topological spaces for every $x\in B$. Moreover, since the local trivialization maps are fiberwise linear, $\xi^{-1}(x)$ is also isomorphic to $\mathbb{R}^n$ as vector spaces.
\end{remark}

The morphisms and the isomorphisms in the category of the fiber and vector bundles are defined as expected. Notice that a vector bundle isomorphism, although not explicitly requested in the definition is fiber-wise a vector space isomorphism.
\begin{definitions}\label{def:bundle_map}\begin{b_item}
\item Let $\xi_1:E_1\to B$ be an $n$-plane vector bundle and $\xi_2:E_2\to B$ be an $m$-plane vector bundle. A continuous map $\phi:E_1\to E_2$ is a \emph{vector bundle map from $\xi_1$ to $\xi_2$ over $B$}, if $\xi_1=\xi_2\circ\phi$ and $\phi|_{\xi_1^{-1}(x)}:\xi_1^{-1}(x)\to\xi_2^{-1}(x)$ is linear for every $x\in B$.

\item Let $p_1:E_1\to B$ and $p_2:E_2\to B$ be two fiber bundles with fibers $F_1$ and $F_2$ respectively. A continuous map $\phi:E_1\to E_2$ is a \ul{bundle map from $p_1$ to $p_2$ over $B$}, if $p_1=p_2\circ\phi$.

\item Let $\xi_1:E_1\to B$ be an $n$-plane bundle and $\xi_2:E_2\to B$ be an $m$-plane bundle. Then a vector bundle map $\phi:E_1\to E_2$ is a \emph{vector bundle isomorphism} if $\phi$ is a homeomorphism.

\item Let $p_1:E_1\to B$ and $p_2:E_2\to B$ be two fiber bundles with fibers $F_1$ and $F_2$ respectively. A bundle map $\phi:E_1\to E_2$ is a \ul{bundle isomorphism}, if $\phi$ is a homeomorphism.
\end{b_item}
\end{definitions}

Now we can already give the trivial example for both vector and fiber bundles.
\begin{definition}\label{def:trivial_vb} Let $n\in\mathbb{N}$, $B$ a topological space and $\xi:E(\xi)\to B$ a vector bundle, then $\xi$ is a \emph{trivial vector bundle over $B$ with fiber $F$}, if $\xi\cong\varepsilon_B^n$, where $\varepsilon_B^n:B\times\mathbb{R}^n\to B$ with $\varepsilon_B^n(x,v)=x$.
\end{definition}
\begin{definition}\label{def:trivial_fb} Let $F,B$ be two topological spaces and $p:E(p)\to B$ a fiber bundle, then $p$ is a \emph{trivial fiber bundle over $B$ with fiber $F$}, if $p\cong\varepsilon_B^F$, where $\varepsilon_B^F:B\times F\to B$ with $\varepsilon_B^F(x,a)=x$.
\end{definition}

At this point we prove a very useful proposition that is true for the vector bundles but not for the fiber bundles in general. This proposition lets us construct vector bundle isomorphisms locally, if we already have a bundle map. The reason that this proof does not work for general fiber bundles is that it boils down to inverting automorphisms of the fiber continuously. Since the group of automorphisms of a real vector space, is $GL(n)$, i.e. a topological group, continuously inverting is successful. For an example of a space $F$, whose group of automorphisms can be topologised, but the inverse map is not continuous refer to \cite{counterexample}. From this, one can build a trivial fiber bundle and a continuous bijection to itself that is a counterexample for the next proposition.
\begin{proposition}\label{prop:local_to_global_iso_vector} Let $\xi_1:E_1\to B$ and $\xi_2:E_2\to B$ be two $n$-plane vector bundles. Moreover, let $\phi:E_1\to E_2$ be a vector bundle map such that its restriction $\phi_x:\xi_1^{-1}(x)\to\xi_2^{-1}(x)$ with $\phi_x(v)=\phi(v)$ is a vector space isomorphism for every $x\in B$. Then $\phi$ is also a homeomorphism and thus a vector bundle isomorphism.
\end{proposition}
\begin{proof} First of all, it is easy to see that $\phi$ is a bijection.
Let $u_1,u_2\in E_1$ and $\phi(u_1)=\phi(u_2)$. Since $\phi$ is a vector bundle map,
$\xi_1(u_1)=\xi_2(\phi(u_1))=\xi_2(\phi(u_2))=\xi_1(u_2)=:x_0$.
So, $u_1,u_2\in\xi_1^{-1}(x_0)$ and $\phi_{x_0}(u_1)=\phi(u_1)=\phi(u_2)=\phi_{x_0}(u_2)$. Since $\phi_{x_0}$ is injective, $u_1=u_2$, which proves that $\phi$ is injective as well.
Let $v\in E_2$. For $x_0:=\xi_2(v)$, $v\in\xi_2^{-1}(x_0)$. Since $\phi_{x_0}$ is surjective, there exists some $u\in\xi_1^{-1}(x_0)\subseteq E_1$ with $\phi(u)=\phi_{x_0}(u)=v$, proving that $\phi$ is also surjective.

It now remains to prove that $\phi^{-1}$ is continuous as well. For every $x\in B$, there exist open neighborhoods $U_1,U_2$ of $x$, and local trivializations $f_1:\xi_1^{-1}(U_1)\to U_1\times\mathbb{R}^n,f_2:\xi_2:\xi_2^{-1}(U_2)\to U_2\times\mathbb{R}^n$. Then, as seen in Remark~\ref{rem:subcover}, for $U_x:U_1\cap U_2$, $g_1:=f_1|_{U_x}$ and $g_2:=f_2|_{U_x}$ are local trivializations of $E_1$ and $E_2$ respectively. Let $\mathcal{U}=\{U_x:x\in B\}$ and notice that $\{\xi_2^{-1}(U):U\in\mathcal{U}\}$ forms an open cover of $E_2$. Thus, it suffices to show that $\phi^{-1}|_{\xi_2^{-1}(U)}$ is continuous for every $U\in\mathcal{U}$ and then apply the glueing lemma. Let us fix such a $U\in\mathcal{U}$. Then, define $\psi:U\times\mathbb{R}\to U\times\mathbb{R}$ as the following composition.
\vspace*{-1em}
\begin{center}
\begin{tikzcd}
U\times\mathbb{R}^n\ar[r,"\cong"',"g_1^{-1}"]\ar[drr,"\pi_1"',near start]\ar[rrrr,bend left=15,dotted,"\psi"]&[2em]\xi_1^{-1}(U)\ar[rr,"\phi|_{\xi_1^{-1}(U)}"]\ar[dr,"\xi_1"',near start]&[-1em]&[-1em]\xi_2^{-1}(U)\ar[r,"\cong"',"g_2"]\ar[dl,"\xi_2",near start]&[2em]U\times\mathbb{R}^n\ar[dll,"\pi_1",near start]\\[2em]
&&U
\end{tikzcd}
\end{center}
Hence the map $\psi$ is continuous and a bijection. Moreover for every $x\in U$ there exists some linear map $\psi_x:\mathbb{R}^n\to\mathbb{R}^n$ such that $\psi(x,v)=(x,\psi_x(v))$, and $\psi_x\in GL(n)$, since the following diagram commutes for every $x\in U$:
\vspace*{-1em}
\begin{center}
\begin{tikzcd}
\mathbb{R}^n\ar[r,"\cong"']\ar[rrrrr,bend left=15,"\psi_x"]&\{x\}\times\mathbb{R}^n\ar[r,"\cong"']&\xi_1^{-1}(x)\ar[r,"\cong"',"\phi_x"]&\xi_2^{-1}(x)\ar[r,"\cong"']&\{x\}\times\mathbb{R}^n\ar[r,"\cong"']&\mathbb{R}^n
\end{tikzcd}
\end{center}
Also, $\phi^{-1}|_{\xi_2^{-1}(U)}=(\phi|_{\xi_1^{-1}(U)})^{-1}$ is continuous if and only if $\psi^{-1}$ is continuous and essentially, the problem is reduced to the case of $E_1,E_2$ both being the trivial vector bundle $U\times\mathbb{R}^n$ and $\psi:U\times\mathbb{R}^n\to U\times\mathbb{R}^n$.

We now prove that the function $U\to GL(n)$ with $x\mapsto\psi_x$ is continuous. Let us fix a basis $e_1,\ldots,e_n$ of $\mathbb{R}^n$ and write $\psi_x=(a_{i,j}(x))_{i,j\in[n]}\in GL(n)$ as a matrix in that basis. Then, it suffices to prove that $a_{i,j}:U\to\mathbb{R}$ is continuous for every $i,j\in[n]$, since $GL(n)$ is topologised as a subset of $\mathbb{R}^{n^2}$. This holds, since $a_{i,j}$ can be written as the following composition:
\begin{center}
\begin{tikzcd}
U\ar[r,hook]&U\times\mathbb{R}^n\ar[r,"\psi"]&U\times\mathbb{R}^n\ar[r,"\pi_2"]&\mathbb{R}^n\ar[r,two heads,"\pi_i"]&\mathbb{R}\\[-1.5em]
x\ar[r,mapsto]&(x,e_j)\ar[r,mapsto]&(x,\psi_xe_j)\ar[r,mapsto]&\psi_xe_j\ar[r,mapsto]&e_i^t\psi_xe_j\ar[r,phantom,"="]&[-2em]a_{i,j}(x)
\end{tikzcd}
\end{center}

Now, notice that $GL(n)$ is a topological group, so the inverse function is continuous, i.e. the function $U\to GL(n)$ with $x\mapsto\psi_x^{-1}$ is continuous. This means that $\psi^{-1}$ is also continuous, as it is the following composition:
\begin{center}
\begin{tikzcd}
U\times\mathbb{R}^n\ar[r]&U\times GL(n)\times\mathbb{R}^n\ar[r]&U\times\mathbb{R}^n\\[-1.5em]
(x,v)\ar[r,mapsto]&(x,\psi_x^{-1},v)\ar[r,mapsto]&(x,\psi_x^{-1}v)\ar[r,phantom,"="]&[-2em]\psi^{-1}(x,v)
\end{tikzcd}
\end{center}
Thus, $\phi^{-1}$ is also continuous, which completes this proof.
\end{proof}

In general, in order to distinguish between non-isomorphic vector bundles we need to examine them globally and a natural way to do this is by sections. This is a continuous choice of an element inside every fiber over all points of the base space.
\begin{definition} Let $\xi:E\to B$ be a vector bundle. A \emph{cross section} or just \emph{section} $s:B\to E$ is a continuous right inverse of $\xi$, i.e. a continuous function taking each $x\in B$ to an element in $\xi^{-1}(x)$.
\end{definition}
Notice that the same definition of a section applies generally for fiber bundles, but in the case of vector bundles we can use the vector structure to measure how many independent sections can be found in a space:

\begin{definition} Let $\xi:E\to B$ be a vector bundle and $s_1,\ldots,s_k:B\to E$ $k$ sections of $\xi$. Then $s_1,\ldots,s_k$ are \emph{nowhere dependent sections} if for each $x\in B$ the vectors $s_1(x),\ldots,s_k(x)$ are linearly independent. In particular, a nowhere dependent section $s$ is also called \emph{nowhere zero section}.
\end{definition}

Clearly, an $n$-plane vector bundle cannot have more than $n$ nowhere dependent sections and the next lemma asserts that the interesting cases arise when there exist less than $n$ nowhere dependent sections.

\begin{lemma}\label{lem:n_sections_makes_trivial} Let $\xi:E\to B$ be an $n$-plane vector bundle. Then $\xi\cong\varepsilon_B^n$ if and only if there exist $n$ nowhere dependent sections of $\xi$.
\end{lemma}
\begin{proof} In order to construct $n$ nowhere dependent sections of $\varepsilon_B^n=B\times\mathbb{R}^n$, fix some basis $(e_1,\ldots,e_n)$ of $\mathbb{R}^n$ and let $s_i(x):=(x,e_i)$ for every $x\in B$ and every $i\in[n]$. On the other hand, let $s_1,\ldots,s_n$ be $n$ nowhere dependent sections of $\xi$. Then, define $\phi:B\times\mathbb{R}^n\to\varepsilon_B^n$ as follows: Fix $e_1,\ldots,e_n$ to be a basis of $\mathbb{R}^n$, define $\phi(x,e_i):=s_i(x)\in\xi^{-1}(x)$. Then extend $\phi$ linearly over each $x\in B$ to get $\phi\left(x,\sum_{i\in[n]}a_ie_i\right)=\sum_{i\in[n]}a_i\phi(x,e_i)=\sum_{i\in[n]}a_is_i(x)\in\xi^{-1}(x)$. Then, $\phi$ clearly continuous and by definition linear fiber-wise, i.e. $\phi$ is a vector bundle map. Each restriction $\phi_x:\{x\}\times\mathbb{R}^n\to\xi^{-1}(x)$ is a linear isomorphism, since $s_1(x),\ldots,s_n(x)$ are linearly independent and $\xi^{-1}(x)$ is $n$-dimensional. Thus, using Proposition~\ref{prop:local_to_global_iso_vector} we get that $\phi$ is a vector bundle isomorphism.
\end{proof}

\begin{examples}\label{ex:vector_bundles}
\begin{i_enum}
\item Let $M$ be a smooth $n$-manifold. Then its \emph{tangent bundle} $\tau_M:TM\to M$ is the $n$-plane bundle constructed as follows: Let $(U_{\alpha})_{\alpha\in I}$ be an open cover of $M$ and $\mathcal{A}=\big\{(U_{\alpha},\phi_{\alpha}:U_{\alpha}\to\mathbb{R}^n)_{\alpha\in I}\big\}$ be a smooth atlas of $M$. Fix any point $x\in M$ and some $(U,\phi)\in\mathcal{A}$ such that $x\in U$. Then define the tangent space at $x$ to be the set of all possible \emph{tangent vectors} at $x$ inside $M$, i.e. the set of equivalence classes
$T_xM:=\left\{\gamma:\mathbb{R}\to M\ |\ \gamma(0)=x\text{ and }\phi\circ\gamma\text{ smooth}\right\}/\sim$,
where $\gamma_1\sim\gamma_2$ if and only if $(\phi\circ\gamma_1)'(0)=(\phi\circ\gamma_2)'(0)$. This set can be given an $\mathbb{R}$-vector space structure as,
$\lambda_1\cdot[\gamma_1]+\lambda_2\cdot[\gamma_2]:=\left[\phi^{-1}\circ\big(\lambda_1(\phi\circ\gamma_1)+\lambda_2(\phi\circ\gamma_2)\big)\right]$.

Notice that $T_xM$ does not depend on the choice of $(U,\phi)\in\mathcal{A}$. Indeed, since the atlas is smooth and $U_{\alpha}$ are open, every transition map $\phi_{\alpha}\circ\phi_{\beta}:\mathbb{R}^n\to\mathbb{R}^n$ has an invertible Jacobi matrix and thus the equivalence relation is the same under any $\phi$. Then, we define the total space of the tangent bundle of $M$ to be the set of all possible tangent vectors at any point of $M$, i.e. the set
$TM:=\big\{(x,[\gamma]):x\in M\text{ and }[\gamma]\in T_xM\big\}$
which is now topologised as follows: Fix some $(U,\phi)\in\mathbb{A}$, then define the following map:
\begin{center}
\begin{tikzcd}
TM\ar[r,phantom,"\supseteq"]&[-2em]\pi_1^{-1}(U)\ar[r]&U\times\mathbb{R}^n\\[-1.5em]
&(x,[\gamma])\ar[r,mapsto]&(x,(\phi\circ\gamma)'(0))
\end{tikzcd}
\end{center}
This induces a topology on $\pi_1^{-1}(U)$ and since the transition maps are open maps, this topology is well defined on every $U_{\alpha}\cap U_{\beta}$. This defines a topology on $TM$. The vector bundle $\tau_{M}:TM\to M$ is exactly the function $\pi_1(x,[\gamma])=x$ we used above, so the local triviality is true by the definition of the topology.
\item Let $M$ be a smooth $n$-manifold embedded in $\mathbb{R}^d$. Then its \emph{normal bundle} $\nu_{M,d}:N_dM\to M$ is the $d-n$-plane bundle \emph{orthogonal} to $\tau_M$, i.e.
\[N_dM:=\big\{(x,v)\in M\times\mathbb{R}^d:x\in M\text{ and }\forall[\gamma]\in T_xM\ v\perp\gamma'(0)\in\mathbb{R}^d\big\}\]
and it is easy to see how the fibers are vector spaces and how the total space is topologised. The vector bundle $\nu_{M,d}$ is again the projection on the first coordinate, but notice that this vector bundle depends on the ambient space the $M$ lives in, while the tangent bundle defined above is an intrinsic construction, i.e. independent of any embedding.
\item\label{ex:tautological_line_bundle} There exists a natural $1$-plane bundle (also \emph{line bundle}) over each $n$ dimensional real projective space, the \emph{tautological} line bundle
\begin{center}
\begin{tikzcd}
\gamma_n^1:\big\{(l,v)\in\mathbb{R}P^n\times\mathbb{R}^{n+1}:v\in l\big\}\ar[r]&\mathbb{R}P^n\\[-1.5em]
(l,v)\ar[r,mapsto]&l
\end{tikzcd}
\end{center}
The total space of this bundle is topologised as a subspace of $\mathbb{R}P^n\times\mathbb{R}^{n+1}$, where the $n$-dimensional projective space is defined to be the set of all lines through the origin in $\mathbb{R}^{n+1}$, topologized as the quotient
$\mathbb{R}P^n=\{l\subseteq\mathbb{R}^{n+1}:l\text{ vector space with }\dim l=1\}\cong S^n/x\sim-x$.

We need to show that the total space is locally homeomorphic to a trivial bundle with the homeomorphism being linear isomorphism on each fiber. We use the standard open cover of $\mathbb{R}P^n$ which makes it a manifold, the one also used in Lemma~\ref{prop:gr_manifold}: Let $l\in\mathbb{R}P^n$ and define $U:=\{l'\in\mathbb{R}P:l'\cap l^{\perp}=\{0\}\}$. Fix also some $v\in l\setminus\{0\}$. Then, define $f_U:(\gamma_n^1)^{-1}(U)\to\mathbb{R}P^n\times\mathbb{R}$ as
$f_U(l',v')=(l',v'^tv)$.
This is continuous and linear over each fiber and since $v'^tv\neq0$ for $v'\neq0$, $f_U$ is also a homeomorphism.
\end{i_enum}
\end{examples}

\begin{proposition} Let $n\geq1$ be any natural number, then $\gamma_n^1\not\cong\varepsilon_{\mathbb{R}P^n}^1$.
\end{proposition}
\begin{proof} It suffices to prove that there does not exist any section $s$ of $\gamma_n^1$ that is nowhere zero. We prove this by contradiction. Suppose there exists some continuous map $s:\mathbb{R}P^n\to\mathbb{R}P^n\times\mathbb{R}^{n+1}$ such that $s(l)=(l,\tilde{s}(l))$, where $\tilde{s}(l)\in l\setminus\{0\}$ for every $l\in\mathbb{R}P^n$. Next, define $\hat{s}=\tilde{s}\circ q:S^n\to\mathbb{R}^{n+1}$, where $q:S^n\to\mathbb{R}P^n$ is the quotient map such that $q(v)=\spans\{v\}\in\mathbb{R}P^n$. Notice that $\hat{s}(v)$ is a non-zero vector in the line spanned by $v$, which means in particular that its inner product with $v$ is not zero. This lets us define $t:S^n\to\mathbb{R}\setminus\{0\}$ with $t(v)=\hat{s}(v)^tv$. Notice now that $t$ is an antipodal map, indeed for any $v\in\mathbb{R}^n$, $t(-v)=\hat{s}(-v)^t(-v)=-\tilde{s}(\spans\{-v\})^tv=-\tilde{s}(\spans\{v\})^tv=-\hat{s}(v)^tv=-t(v)$. Since $n\geq1$, $S^n$ is connected and the existence of such a continuous $t$ leads to a contradiction due to the intermediate value theorem. Notice that this is the easiest case of the (BU1b)-version of the Borsuk Ulam theorem as stated in Theorem~2.1.1 in \cite{BU_Matousek}.
\end{proof}

\section{Creating new bundles}
The goal of this section is to describe some ways to create new bundles by combining already known ones. First new bundles are constructed based on continuous maps between the base spaces, disregarding the vector space structure on the fibers and then we construct new bundles by fixing the base space and doing operations on the vector spaces fiber-wise.

\subsection{By combining the base spaces}
In order to briefly ignore the vector space structure of the fibers we describe the next few constructions in terms of fiber bundles, in general.

\begin{proposition} Let $p:E\to B$ be a fiber bundle with fiber $F$, let $B'$ be any topological space and let $g:B'\to B$ be any continuous map. Moreover, let $E':=\{(x,v)\in B'\times E:g(x)=p(v)\}$, $g^*p(x,v)=x$ and $\bar{g}(x,v)=v$ be the space and the maps making the following a pullback square of topological spaces:
\begin{center}
\begin{tikzcd}
E'\ar[r,dotted,"\bar{f}"]\ar[d,dotted,"g^*p"']\ar[rd,phantom,"\lrcorner",near start]&E\ar[d,"p"]\\
B'\ar[r,"g"]&B
\end{tikzcd}
\end{center}
Then $g^*p:E'\to B'$ is a fiber bundle with fiber $F$.
\end{proposition}
\begin{proof} We have to show that $g^*p$ as defined above is locally trivial. Let $x\in B'$, then, since $p$ is a fiber bundle, there exists some open $U$ with $g(x)\in U\subseteq B$ and a homeomorphism $f_U:p^{-1}(U)\to U\times F$ such that $p=\pi_1\circ f_U$. Let $U':=g^{-1}(U)$. This contains $x$ and is open, since $g$ is continuous. Notice that
$(g^*p)^{-1}(U')=\{(x,v)\in U'\times E: g(x)=p(v)\}
=\{(x,v)\in U'\times p^{-1}(U):g(x)=p(v)\}$.
So, we define $f'_{U'}:= id\times(\pi_2\circ f_U):(g^*p)^{-1}(U')\to U'\times F$. This is a continuous map and clearly satisfies the equation $g^*p(x,a)=x=\pi_1\circ f'_{U'}$. In order to prove that it is also a homeomorphism, we prove that it has a continuous inverse map. Define $h_{U'}:U'\times F\to (g^*p)^{-1}(U')$ as follows:
\begin{center}
\begin{tikzcd}
h_{U'}\ar[r,phantom,":"]&[-2em]U'\times F\ar[r,"\left<id_{U'}{,}g\right>\times id_F"]&[3em]U'\times U\times F\ar[r,"id_{U'}\times f_U^{-1}"]&[3em](g^*p)^{-1}(U')\\[-1.5em]
&(x,a)\ar[r,mapsto]&(x,g(x),a)\ar[r,mapsto]&(x,f_U^{-1}(g(x),a))
\end{tikzcd}
\end{center}
This map is continuous well defined. Indeed, $p\big(f_U^{-1}(g(x),a)\big)=\pi_1(g(x),a)=g(x)$, and thus $(x,f_U^{-1}(g(x),a))\in (g^*p)^{-1}(U')$ for every $(x,a)\in U'\times F$. Moreover, notice that $(f'_{U'}\circ h_{U'})(x,a)=f'_{U'}(x,f_U^{-1}(g(x),a))=(x,\pi_2\circ f_U\circ f_U^{-1}(g(x),a))=(x,a)$ for every $(x,a)\in U'\times F$. Also, for every $(x,v)\in (g^*p)^{-1}(U')$,
\begin{align*}
(h_{U'}\circ f'_{U'})(x,v)&=h_{U'}(x,\pi_2\circ f_U(v))=\big(x,f_U^{-1}(g(x),\pi_2\circ f_U(v))\big)\\
&=\big(x,f_U^{-1}(p(v),\pi_2\circ f_U(v))\big)=\big(x,f_U^{-1}(\pi_1\circ f_U(v),\pi_2\circ f_U(v))\big)\\
&=\big(x,f_U^{-1}\circ f_U(v)\big)=(x,v)
\end{align*}
which completes the proof that $(f'_{U'})^{-1}=h_{U'}$
\end{proof}

\begin{definition} Let $p:E\to B$ be a fiber bundle with fiber $F$ and $g:B'\to B$ any continuous map as in the above proposition. The \emph{induced} fiber bundle of $p$ over $g$ is $g^*p:E'\to B'$, where $E':=\{(x,a)\in B'\times E:g(x)=p(a)\}$ and $g^*p(x,a)=x$.
\end{definition}
The way to imagine the pullback fiber bundle is that we attach to a point $x\in B'$ a copy of $F$ ``in the same way'' it is attached to $g(x)$.

\begin{remark} In the special case where $g:B'\to B$ is injective, then $E'\cong p^{-1}(g(B'))\subseteq E$ and $(g^*p)(v)=g^{-1}(p(v))$ for every $v\in p^{-1}(g(B'))$.
\end{remark}
\begin{proof} The function $g$ being injective means that it has a left inverse, which we denote by $g^{-1}$ here. Hence,
$E'=\{(x,v)\in B'\times E:g(x)=p(v)\}=\{(x,v)\in B'\times E:x=(g^{-1}\circ p)(v)\}
\cong\{v\in E:(g^{-1}\circ p)(v)\in B'\}=p^{-1}(g(B))$.
Factoring $g^*p$ through this homeomorphism we exactly get $(g^*p)(v)=x=g^{-1}(p(v))$, which proves the assertion.
\end{proof}

\begin{definition} Let $p:E\to B$ be a fiber bundle and $A\subseteq B$ any subset of $B$. Then, the \emph{restriction bundle of $p$ on $A$} is $p|A:E'\to A$, where $E'=p^{-1}(A)$ and $g^*p(v)=p|_{p^{-1}(A)}(v)$.
\end{definition}
Notice that $p|A$ has fiber $F$, which means that we restricted $B$ but not $F$.

\begin{proposition} Let $p_1:E_1\to B_1$ and $p_2:E_2\to B_2$ be two fiber bundles with fibers $F_1$ and $F_2$ respectively. Then $p_1\times p_2:E_1\times E_2\to B_1\times B_2$ is a fiber bundle with fiber $F_1\times F_2$.
\end{proposition}
\begin{proof} We have to show that $p_1\times p_2$ is locally trivial. Let $(x_1,x_2)\in B_1\times B_2$. Then, since $p_1$ and $p_2$ are fiber bundles there exist some open sets $U_1\subseteq E_1$ and $U_2\subseteq E_2$ with $x_1\in U_1$ and $x_2\in U_2$ and homeomorphisms $f_{U_1}:p_1^{-1}(U_1)\to U_1\times F_1$ and $g_{U_2}:p_2^{-1}(U_2)\to U_2\times F_2$ such that $p_1=\pi_1\circ f_{U_1}$ and $p_2=\pi_1\circ g_{U_2}$. Let $U=U_1\times U_2$. This contains $(x_1,x_2)$ and is open. Notice that $(p_1\times p_2)^{-1}(U)=p_1^{-1}(U_1)\times p_2^{-1}(U_2)$. So, we define $h_U:=\tau_{2,3}\circ(f_{U_1}\times g_{U_2}):(p_1\times p_2)^{-1}(U)\to U_1\times U_2\times F_1\times F_2$, where $\tau_{2,3}:U_1\times F_1\times U_2\times F_2\to U_1\times U_2\times F_1\times F_2$ is the homeomorphism swapping coordinates $2$ and $3$. This makes $h_U$ a homeomorphism and it is easy to check that $\pi_1\circ h_U(v_1,v_2)=(p_1\times p_2)(v_1,v_2)\in B_1\times B_2$, where $\pi_1$ is the projection on the first two coordinates.
\end{proof}

\begin{definition} Let $p_1:E_1\to B_1$ and $p_2:E_2\to B_2$ be two fiber bundles with fibers $F_1$ and $F_2$ respectively. Then, the \emph{product fiber bundle of $p_1$ and $p_2$} is the usual product map $p_1\times p_2:E_1\times E_2\to B_1\times B_2$.
\end{definition}

\begin{proposition} Let $p_1:E_1\to B_1$ and $p_2:E_2\to B_2$ be two fiber bundles both with fiber $F$. Then $p_1\amalg p_2:E_1\amalg E_2\to B_1\amalg B_2$ is a fiber bundle with fiber $F$.
\end{proposition}
\begin{proof} We have to show that $p_1\amalg p_2:E_1\amalg E_2\to B_1\amalg B_2$ is locally trivial. Let $x\in B_1\amalg B_2$. Then $x\in B_i$ for some $i\in\{1,2\}$ and there exists some open $U\subseteq B_i$ containing $x$ and a homeomorphism $f_U:p_i^{-1}(U)\to U\times F$. Notice that $(p_1\amalg p_2)^{-1}(U)=p_i^{-1}(U)$, so $f_U$ itself is a local homeomorphism for $p_1\amalg p_2$.
\end{proof}

\begin{definition} Let $p_1:E_1\to B_1$ and $p_2:E_2\to B_2$ be two fiber bundles both with fiber $F$. Then, the \emph{coproduct fiber bundle of $p_1$ and $p_2$} is the usual coproduct map $p_1\amalg p_2:E_1\amalg E_2\to B_1\amalg B_2$.
\end{definition}
Notice that the product of the base spaces always defines a product of fiber bundles, but the disjoint union of two base spaces defines a fiber bundle only if the fibers of the two components were already isomorphic.

\subsection{By combining the fibers}
Aiming to exploit the vector space structure of the fibers, we fix some base space $B$ and define new bundles over $B$ using fiber-wise operations. In this section we construct some vector bundle isomorphisms, by constructing a bundle map that is linear on the fibers, proving that the desired isomorphism is true fiberwise and then using proposition~\ref{prop:local_to_global_iso_vector} to argue that this bundle map is in fact a bundle isomorphism.

\begin{definition} Let $\eta:E(\eta)\to B$ be an $n$-plane vector bundle and $\xi:E(\xi)\to B$ a $k$-plane vector bundle, where $E(\xi)\subseteq E(\eta)$. Then $\xi$ is a \emph{subbundle of $\eta$} and we write $\xi\leq\eta$, if $\xi^{-1}(x)<\eta^{-1}(x)$ as vector spaces for every $x\in B$. Notice that in this case $k\leq n$.
\end{definition}

\begin{definition} Let $\xi_1:E(\xi_1)\to B$ be an $n_1$-plane vector bundle, $\xi_2:E(\xi_2)\to B$ an $n_2$-plane vector bundle and $\eta:E(\eta)\to B$ an $n$-plane vector bundle, where $E(\xi_1),E(\xi_2)\subseteq E(\eta)$. Then $\eta$ is the \emph{whitney sum of $\xi_1$ and $\xi_2$} and we write $\eta=\xi_1\oplus\xi_2$, if $\eta^{-1}(x)=\xi_1^{-1}(x)\oplus\xi_2^{-1}(x)$ as vector spaces, for every $x\in B$. Notice that in this case $n_1+n_2=n$.
\end{definition}

The next two propositions exhibit the connection between products and whitney sums of fiber bundles.
\begin{proposition} Let $\xi_1:E(\xi_1)\to B$ be an $n_1$-plane vector bundle and $\xi_2:E(\xi_2)\to B$ an $n_2$-plane vector bundle. Then, $\xi_1\oplus\xi_2\cong d^*(\xi_1\times\xi_2)$, where $d:B\to B\times B$ is the diagonal map taking $x\in B$ to $(x,x)\in B\times B$.
\end{proposition}
\begin{proof} Notice that $E(d^*(\xi_1\times\xi_2)=\big\{(x,v_1,v_2)\in B\times E(\xi_1)\times E(\xi_2):(x,x)=(\xi_1(v_1),\xi_2(v_2))\big\}$ and define the following map:
\begin{center}
\begin{tikzcd}
E(d^*(\xi_1\times\xi_2))\ar[r,"\phi"]&[5em]E(\xi_1\oplus\xi_2)\\[-1.5em]
(x,v_1,v_2)\ar[r,mapsto]&v_1+v_2
\end{tikzcd}
\end{center}
It is easy to see that this is a bundle map and linear over each fiber. Next, notice that its restriction on the fibers $\phi_x=\phi|:\big(d^*(\xi_1\times\xi_2)\big)^{-1}(x)\to(\xi_1\oplus\xi_2)^{-1}(x)$ is a linear isomorphism. Indeed, for any $x\in B$, its inverse is the map $\psi_x$ taking some $v=v_1+v_2\in(\xi_1\oplus\xi_2)^{-1}(x)=\xi_1^{-1}(x)\oplus\xi_2^{-1}(x)$ to $(x,v_1,v_2)\in E(d^*(\xi_1\times\xi_2))$. Due to Proposition~\ref{prop:local_to_global_iso_vector}, $\phi$ is a vector bundle isomorphism.
\end{proof}
\begin{remark} If we define $\zeta:\big\{(v_1,v_2)\in E(\xi_1)\times E(\xi_2):\xi_1(v_1)=\xi_2(v_2)\big\}\to B$ to be the vector bundle with $\zeta(v_1,v_2)=\xi_1(v_1)$, then $\xi_1\oplus\xi_2\cong\zeta$.
\end{remark}
\begin{proof} Notice that
$E(d^*(\xi_1\times\xi_2))=\big\{(x,v_1,v_2)\in B\times E(\xi_1)\times E(\xi_2):(x,x)=(\xi_1(v_1),\xi_2(v_2))\big\}
=\big\{(\xi_1(v_1),v_1,v_2)\in B\times E(\xi_1)\times E(\xi_2):\xi_1(v_1)=\xi_2(v_2)\big\}
\cong\big\{(v_1,v_2)\in E(\xi_1)\times E(\xi_2):\xi_1(v_1)=\xi_2(v_2)\big\}=E(\zeta)$,
where the homeomorphism is a vector bundle isomorphism.
\end{proof}
This is the description used in the remainder of the thesis.
%TODO: move this after the def of induced and change the text above
\begin{proposition}\label{prop:product_bundle_to_whitney_sum} Let $\xi_1:E(\xi_1)\to B_1$ be an $n_1$-plane vector bundle and $\xi_2:E(\xi_2)\to B_2$ an $n_2$-plane vector bundle. Then, $\xi_1\times\xi_2\cong (\pi_1^*\xi_1)\oplus(\pi_2^*\xi_2)$, where $\pi_1:B_1\times B_2\to B_1$ and $\pi_2:B_1\times B_2\to B_2$ are the usual projections.
\end{proposition}
\begin{proof} Note that $E((\pi_1^*\xi_1)\oplus(\pi_2^*\xi_2))=\big\{(x_1,x_2,v_1,y_1,y_2,v_2)\in B_1\times B_2\times E(\xi_1)\times B_1\times B_2\times E(\xi_2):\pi_1(x_1,x_2)=\xi_1(v_1)\text{ and }\pi_2(y_1,y_2)=\xi_2(v_2)\text{ and }(x_1,x_2)=(y_1,y_2)\big\}\cong\big\{(x_1,x_2,v_1,v_2)\in B_1\times B_2\times E(\xi_1)\times E(\xi_2):x_1=\xi_1(v_1)\text{ and }x_2=\xi_2(v_2)\}=E(\xi_1\times\xi_2)$.
Clearly, this homeomorphism is a bundle map and a linear isomorphism over each fiber.
\end{proof}

\begin{lemma}\label{lem:induced_sum} Let $\eta,\xi$ be vector bundles over $B$ and $f:B'\to B$ be any continuous function. Then $f^*(\eta\oplus\xi)\cong(f^*\eta)\oplus(f^*\xi)$.
\end{lemma}
\begin{proof} Notice that $E\big((f^*\eta)\oplus(f^*\xi)\big)=\big\{(x,a,y,b)\in B'\times E(\eta)\times B'\times E(\xi):f(x)=\eta(a)\text{ and }f(y)=\xi(b)\text{ and }x=y\big\}\cong\big\{(x,a,b)\in B'\times E(\eta)\times E(\xi):\eta(a)=\xi(b)=x\big\}=E\big(f^*(\eta\oplus\xi)\big)$, where the homeomorphism is a bundle map and a linear isomorphism fiber-wise. So, using Proposition~\ref{prop:local_to_global_iso_vector} it is a vector bundle isomorphism.
\end{proof}

\begin{proposition} Let $\xi_1,\xi_2,\eta_1,\eta_2$ be vector bundles over $B$ and $\phi_1:E(\xi_1)\to E(\eta_1)$, $\phi_2:E(\xi_2)\to E(\eta_2)$ vector bundle maps. Moreover, let $\phi_1\oplus\phi_2:E(\xi_1\oplus\xi_2)\to E(\eta_1\oplus\eta_2)$ be the function taking $(v_1,v_2)$ to $(\phi_1(v_1),\phi_2(v_2))$. Then $\phi_1\oplus\phi_2$ is a vector bundle map.
\end{proposition}
\begin{proof} First of all, $\phi_1\oplus\phi_2$ is well defined, since $(\eta_1\circ\phi_1)(v_1)=\xi_1(v_1)=\xi_2(v_2)=(\eta_2\circ\phi_2)(v_2)$. Also it is a bundle map, since $\big((\eta_1\oplus\eta_2)\circ(\phi_1\oplus\phi_2)\big)(v_1,v_2)=(\eta_1\circ\phi_1)(v_1)=\xi_1(v_1)=(\xi_1\oplus\xi_2)(v_1,v_2)$. It is also trivial to check that $\phi_1\oplus\phi_2$ is linear on each fiber.
\end{proof}
\begin{definition}\label{def:wh_sum_map} Let $\xi_1,\xi_2,\eta_1,\eta_2$ be vector bundles over $B$ and $\phi_1:E(\xi_1)\to E(\eta_1)$, $\phi_2:E(\xi_2)\to E(\eta_2)$ vector bundle maps. Then the \emph{whitney sum of $\phi_1,\phi_2$} is $\phi_1\oplus\phi_2:E(\xi_1\oplus\xi_2)\to E(\eta_1\oplus\eta_2)$, where $\phi_1\oplus\phi_2(v_1,v_2)=(\phi_1(v_1),\phi_2(v_2))$.
\end{definition}

\begin{definition} Let $\eta:E(\eta)\to B$ be an $n$-plane vector bundle, $\xi:E(\xi)\to B$ a $k$-plane vector bundle and $\theta:E(\theta)\to B$ an $m$-plane vector bundle, where $E(\xi)\subseteq E(\eta)$. Then $\theta$ is the \emph{quotient vector bundle of $\eta$ and $\xi$} and we write $\theta=\eta/\xi$, if $\theta^{-1}(x)=\eta^{-1}(x)/\xi^{-1}(x)$ as vector spaces, for every $x\in B$. Notice that in this case $n-k=m$.
\end{definition}
Next we examine how to explicitly define the quotient bundles given a bundle and a subbundle.
\begin{proposition}\label{prop:quotient_bundle} Let $\eta:E(\eta)\to B$ be an $n$-plane vector bundle and $\xi:E(\xi)\to B$ a $k$-plane vector subbundle, i.e. $\xi\leq\eta$. Moreover, let $\zeta:E(\eta)/\sim\to B$, where $v_1\sim v_2$ if and only if $\eta(v_1)=\eta(v_2)$ and $v_1-v_2\in\xi^{-1}(\eta(v))$ be the vector bundle with $\zeta([v])=\xi(v)$. Then, $\eta/\xi\cong\zeta$.
\end{proposition}
\begin{proof} First of all notice that $\zeta$ is well defined, since $\xi(v)$ is the same for every choice of class representative $v$. Next, define the following map:
\begin{center}
\begin{tikzcd}
 E(\eta)/\sim\ar[r,"\phi"]&E(\eta/\xi)\\[-1.5em]
[v]\ar[r,mapsto]&v+\xi^{-1}(\eta(v))
\end{tikzcd}
\end{center}
Notice that this is well defined, by the definition of the equivalence relation. Moreover it is easy to see that it is a bundle map and also linear over each fiber. Next, we prove that it is in fact a vector isomorphism over each fiber. To do this, we define for every $x\in B$ a continuous map $\psi_x:(\eta/\xi)^{-1}(x)\to\zeta^{-1}(x)$ and prove that it is the inverse of $\phi_x$. Let $v+\xi^{-1}(x)\in \eta^{-1}(x)/\xi^{-1}(x)$, then we define $\psi_x(v+\xi^{-1}(x))=[v]$. This is again well defined and is easy to see that it is the desired inverse. Using remark~\ref{rem:subcover} we get that $\phi$ is a vector bundle isomorphism.
\end{proof}

At this point it is natural to wonder if given two bundles $\xi\leq\eta$ it is always possible to decompose $\eta$ in the whitney sum of $\xi$ and $\eta/\xi$. Trying to build a vector bundle isomorphism from $\xi(\eta/\xi)$ to $\eta$, we see however that we need to be able to choose an inner product on each fiber that changes continuously. So the next question arises naturally:

\section{Do we always have a dot product?}
\begin{definition} Let $V$ be an $\mathbb{R}$ vector space. Then $V$ is an \emph{Euclidean vector space}, if there exists a symmetric, bilinear function $\beta:V\times V\to\mathbb{R}$ such that $\beta(v,v)\geq0$ and $\beta(v,v)=0$ only if $v=0$. In this case $\beta$ is called \emph{inner product} and $\beta(u,v)$ denoted by $u\cdot v$.
\end{definition}

\begin{examples}\label{ex:infinite_euclidean_space}
\begin{i_enum}\item For every $n\in\mathbb{N}$, the usual inner product on $\mathbb{R}^n$ is defined by $\beta_n(u,v)=u^tv$.
\item Notice that the inclusion $\mathbb{R}^n\subseteq\mathbb{R}^{n+1}$ taking $v$ to $\bar{v}:=(v^t|0)^t$ respects the usual inner product, i.e. $\beta_n(u,v)=\beta_{n+1}(\bar{u},\bar{v})$. This means, that for every $u,v\in\mathbb{R}^{\infty}$, we can define the inner product $u^tv=\beta(u,v):=\beta_n(u,v)$ for some $n$ big enough such that $u,v\in\mathbb{R}^n\subseteq\mathbb{R}^{\infty}$. This makes $\mathbb{R}^{\infty}$ a euclidean space.
\end{i_enum}
\end{examples}

\begin{definition} Let $\xi:E\to B$ be a vector bundle. Then $\xi$ is a \emph{Euclidean vector bundle}, if there exists a continuous function $\beta:E(\xi\oplus\xi)\to\mathbb{R}$ the restriction of which on $(\xi\oplus\xi)^{-1}(x)$ makes $\xi^{-1}(x)$ a euclidean vector space for each $x\in B$.
\end{definition}

\begin{example} Let $B$ be any topological space and $\varepsilon_B^n:B\times\mathbb{R}^n\to B$ the trivial vector bundle over $B$. Then the usual inner product of $\mathbb{R}$ makes $\varepsilon_B^n$ a euclidean vector bundle: Let $(x,a),(x,b)\in\{x\}\times\mathbb{R}^n$. Then $(x,a)\cdot(x,b)=a^tb\in\mathbb{R}$.
\end{example}

\begin{definition} Let $\xi:E\to B$ be a euclidean vector bundle and $s_1,\ldots,s_k:B\to E$ be $k$ sections of $\xi$. Then $s_1,\ldots,s_k$ are \emph{orthonormal sections} if for each $x\in B$ the vectors $s_1(x),\ldots,s_k(x)$ are orthonormal, i.e. $s_i(x)\cdot s_j(x)=\delta_{i,j}$.
\end{definition}

If $V$ is a finitely dimensional euclidean vector space, then we can always transform a basis of $V$ into an orthonormal basis of $V$ using the Gram-Schmidt process. This also holds for euclidean vector bundles:
\begin{lemma}\label{lem:orthonormal_sections} Let $\xi:E\to B$ be a euclidean $n$-plane vector bundle which has $k$ nowhere dependent sections. Then, there exist $k$ orthonormal sections of $\xi$.
\end{lemma}
\begin{proof} Let $s_1,\ldots,s_k:B\to E$ be any $k$ nowhere dependent sections of $\xi$. Then, we define $t_1,\ldots,t_k:B\to E$ inductively by using the Gram-Schmidt process:
\[t_l(x):=\frac{1}{\left\|s_l(x)-\sum_{i=1}^{l-1}(t_i(x)\cdot s_i(x))t_i(x)\right\|}\left(s_l(x)-\sum_{i=1}^{l-1}(t_i(x)\cdot s_i(x))t_i(x)\right)\]
where the linear operations are well defined, since $s_i(x),t_i(x)\in\xi^{-1}(x)$. Since $t_i:B\to E$ is continuous for every $i$, $t_1,\ldots,t_k$ are $k$ sections for which $t_i(x)\cdot t_j(x)=\delta_{i,j}$ by the properties of the Gram-Schmidt process, which proves the assertion.
\end{proof}
\begin{remark}\label{rem:trivial_euclidian} In particular, using Lemma~\ref{lem:n_sections_makes_trivial}, we get that an $n$-plane euclidean vector bundle $\xi$ is trivial, i.e. $\xi\cong\varepsilon_B^n$ if and only if there exist $n$ orthonormal sections of $\xi$.
\end{remark}

In the particular case of euclidean vector bundles, we can always define a natural complementary bundle of a given subbundle:
\begin{proposition} Let $\eta:E(\eta)\to B$ be an $n$-plane euclidean vector bundle and $\xi:E(\xi)\to B$ a $k$-plane euclidean vector subbundle of $\eta$. Moreover, let $E:=\{v\in E(\eta):\forall u\in\xi^{-1}(\eta(v))\ v\cdot u=0\}$ and $\zeta:E\to B$ the map taking $v\in E$ to $\eta(v)$. Then $\zeta$ is a $(n-k)$-plane vector bundle.
\end{proposition}
\begin{proof} We need to show that $\zeta$ is locally trivial. Let us fix some $x_0\in B$. Then, there exist open sets $U_1,U_2\subseteq B$ both containing $x_0$ such that $\eta|U_1$ is an $n$-plane trivial vector bundle and $\xi|U_2$ is a $k$-plane trivial vector bundle. Using Remark~\ref{rem:subcover} we restrict them further on $U:=U_1\cap U_2$ and they remain trivial. For $\eta$ we use the definition of local triviality, which gives a vector bundle isomorphism $f_U:\eta^{-1}(U)\to U\times\mathbb{R}^n$, whereas for $\xi$ we use Remark~\ref{rem:trivial_euclidian}, which gives $k$ orthonormal sections of $\xi$, namely $s_1,\ldots,s_k:U\to\xi^{-1}(U)$. This means that for each $x\in U$ the vectors $v_1(x):=(\pi_2\circ f_U\circ s_1)(x),\ldots,v_k(x):=(\pi_2\circ f_U\circ s_k)(x)$ are $k$ linearly independent vectors in $\mathbb{R}^n$. We extend this $k$-frame to a basis, specifically at the point $x=x_0$, i.e. we choose any $n-k$ vectors $v_{k+1},\ldots,v_n\in\mathbb{R}^n$ such that
$\det\big(v_1(x_0),\ldots,v_k(x_0),v_{k+1},\ldots,v_n\big)>0$.

Define $D:U\to\mathbb{R}$ to be $D(x):=\det\big(v_1(x),\ldots,v_k(x),v_{k+1},\ldots,v_n\big)$ for all $x\in U$. Since the determinant, the $f_U$ and the sections are continuous, $D$ is also continuous and thus there exists some open neighborhood $V\subseteq U$ of $x_0$ such that $D(x)>0$ for every $x\in V$, i.e. $v_1(x),\ldots,v_k(x),v_{k+1},\ldots,v_n$ are linearly independent for every $x\in B$. Define now $\bar{s}_i(x):=f_U^{-1}(x,v_i)\in\eta^{-1}(x)$ for every $x\in B$ and every $i\in\{k+1,\ldots,n\}$. Since $f_U$ is a linear isomorphism on every fiber, $s_1(x),\ldots,s_k(x),\bar{s}_{k+1}(x),\ldots,\bar{s}_n(x)$ are $n$ nowhere dependent sections. Using the Gramm-Schmidt process, as in Lemma~\ref{lem:orthonormal_sections}, we construct $n-k$ sections $s_{k+1},\ldots,s_n$ such that $s_1,\ldots,s_n$ are $n$ orthonormal sections. This means that $s_{k+1},\ldots,s_n$ are $n-k$ orthonormal sections of $\zeta|V$, i.e. $\zeta|V$ is isomorphic to the trivial $(n-k)$-plane vector bundle.
\end{proof}

\begin{definition} Let $\eta:E(\eta)\to B$ be an $n$-plane euclidean vector bundle and $\xi:E(\xi)\to B$ a $k$-plane euclidean vector subbundle of $\eta$. Then, the \emph{orthogonal complement of $\xi$ in $\eta$} is $\xi^{\perp_{\eta}}:E\to B$ (or just $\xi^{\perp}$, when $\eta$ is clear) where $E:=\{v\in E(\eta):\forall u\in\xi^{-1}(\eta(v))\ v\cdot u=0\}$ and $\xi^{\perp_{\eta}}(v)=\eta(v)$.
\end{definition}

\begin{proposition}\label{prop:tangent_normal_vb} Let $M$ be a smooth manifold, embedded in $\mathbb{R}^d$, $\tau_M:TM\to M$ its tangent bundle and $\nu_{M,d}:N_dM\to M$ its normal bundle. Then $\nu_{M,d}\cong\tau_M^{\perp_{\varepsilon_M^d}}$
\end{proposition}
\begin{proof}
%TODO
\end{proof}

\begin{lemma} Let $\eta:E(\eta)\to B$ be an $n$-plane euclidean vector bundle and $\xi:E(\xi)\to B$ a $k$-plane euclidean vector subbundle of $\eta$. Then $\xi^{\perp_{\eta}}\cong \eta/\xi$.
\end{lemma}
\begin{proof} Using Proposition~\ref{prop:quotient_bundle}, we write $E(\eta/\xi)=E(\eta)/\sim$ for $v_1\sim v_2$ if and only if $\eta(v_1)=\eta(v_2)$ and $v_1-v_2\in\xi^{-1}(\eta(v_1))$. Moreover, we also write $\eta/\xi([v])=\eta(v)$ and define the map
\begin{center}
\begin{tikzcd}
E(\xi^{\perp_{\eta}})\ar[r,"\phi"]&E/\sim\\[-1.5em]
v\ar[r,mapsto]&\left[v\right].
\end{tikzcd}
\end{center}
This is a bundle map, linear on each fiber. In order to prove that this is in fact a linear isomorphism fiberwise, notice that $\phi_x(v)=0$ means that $v\in\xi^{-1}(x)$, but since $v\in E(\xi^{\perp_{\eta}})$, we also have that $v\cdot u=0$ for every $u\in\xi^{-1}(x)$. Thus, $u=0$. Hence, $\ker\phi_x=\{0\}$ and since $\dim_{\mathbb{R}}(\xi^{\perp_{\eta}})^{-1}(x)=\dim_{\mathbb{R}}(\sfrac{\eta}{\xi})^{-1}(x)=n-k$, $\phi_x$ is a linear isomorphism. Due to Proposition~\ref{prop:local_to_global_iso_vector} $\phi$ is a vector bundle isomorphism.
\end{proof}

This lets us decompose some euclidean vector bundle to a whitney sum, provided we find some vector subbundle of it:
\begin{proposition}\label{prop:vb_decomposition} Let $\eta:E(\eta)\to B$ be an $n$-plane euclidean vector bundle and $\xi:E(\xi)\to B$ a $k$-plane euclidean vector subbundle of $\eta$. Then $\eta\cong\xi\oplus\xi^{\perp_{\eta}}$.
\end{proposition}
\begin{proof} Let us define the map
\begin{center}
\begin{tikzcd}
E(\xi\oplus\xi^{\perp_{\eta}})\ar[r,"\phi"]&E(\eta)\\[-1.5em]
(v_1,v_2)\ar[r,mapsto]&v_1+v_2.
\end{tikzcd}
\end{center}
It is easy to see that this is a bundle map which is also a linear isomorphism on each fiber. Thus, due to Proposition~\ref{prop:local_to_global_iso_vector}, $\phi$ is a vector bundle isomorphism.
\end{proof}
\begin{remark}\label{rem:vb_decomposition} For every euclidean vector bundle $\eta$ and for any $\xi\leq\eta$ we have the following decomposition $\eta\cong\xi\oplus\eta/\xi$.
\end{remark}

At this point the question about decomposing a vector bundle is positively answered, provided that the vector bundle in question is a euclidean vector bundle. A very natural question now is how restrictive the notion of a euclidean vector is. As it is shown at the end of the next section, a vector bundle admits a euclidean vector bundle structure at least when the base space is paracompact.

\section{Paracompactness}
\begin{definition} Let $X$ be a topological space and $\mathcal{U}=\{U_i\}_{i\in I}$ be a collection of subsets of $X$. Then $\mathcal{U}$ is \emph{locally finite} if for every $x\in X$ there exists some open $V\subseteq X$ containing $x$ such that $U_i\cap V\neq\emptyset$ only for finitely many $i$.
\end{definition}
\begin{definition} Let $X$ be a topological space and $\mathcal{U},\mathcal{V}$ be two collection of subsets of $X$. Then $\mathcal{V}$ is a \emph{refinement of $\mathcal{U}$} if for every $V\in\mathcal{V}$ there exists some $U\in\mathcal{U}$ such that $V\subseteq U$.
\end{definition}
\begin{definition} A topological space $X$ is \emph{paracompact} if $X$ is Hausdorff and for every open cover $\mathcal{U}$ of $X$ there exists some locally finite open cover $\mathcal{V}$ of $X$ that is a refinement of $\mathcal{U}$.
\end{definition}

\begin{proposition}\label{prop:paracompact_closed_subset} Let $X$ be a paracompact space and $A\subset X$ any closed subset. Then $A$ is paracompact.
\end{proposition}
\begin{proof} Indeed, let $\{U_i\}_{i\in I}$ be an open cover of $A$, i.e. there exists some open collection $\{V_i\}_i$ of subsets of $X$, such that $U_i=A\cap V_i$. Let $W_i:=V_i\cup(X\setminus A)$. Then $\{W_i\}_{i\in I}$ is an open cover of $X$. Since $X$ is paracompact, there exists some locally finite open cover $\{Z_j\}_{j\in J}$ of $X$, that is a refinement of $\{W_i\}_{i\in I}$. Then $\{Z_j\cap A\}_{j\in J}$ is a locally finite open cover of $A$, that is a refinement of $\{U_i\}_{i\in I}$.
\end{proof}

\begin{proposition}\label{prop:paracompact_times_compact} Let $X$ be a paracompact space and $K$ a compact space. Then $X\times K$ is a paracompact space.
\end{proposition}
\begin{proof} Let $\{U_i\}_{i\in I}$ be an open cover of $X\times K$. Then, for each $(x,a)\in X\times K$ there exists some open $U_i\ni(x,k)$. So, there exists some open $A_{x,k}\subseteq X$ and open $B_{x,k}\subseteq K$, such that $A_{x,k}\times B_{x,k}\subseteq U_i$. Notice that $\{A_{x,k}\times B_{x,k}\}_{(x,k)\in X\times K}$ is an open cover of $X\times K$ that is a refinement of $\{U_i\}_{i\in I}$. Let us now fix some $x_0\in X$ and notice that the family $\{B_{x_0,k}\}_{k\in K}$ is an open cover of $K$. Since $K$ is compact, there exist only finitely many $k_1(x_0),\ldots,k_{r(x_0)}(x_0)\in K$ such that $\{B_{x_0,k}\}_{k\in\{k_1(x_0),\ldots,k_{r(x_0)}(x_0)\}}$ is an open cover of $K$. Define now $A_{x_0}:=\bigcap_{k\in\{k_1(x_0),\ldots,k_{r(x_0)}(x_0)\}}A_{x_0,k}\subseteq X$. This is a finite intersection of open sets, which contains $x_0$, so it is an open set. So, we get an open cover $\{A_x\}_{x\in X}$ of $X$. Since $X$ is paracompact, there exists some locally finite open cover $\{L_j\}_{j\in J}$ of $X$ that is a refinement of $\{A_x\}_{x\in X}$, i.e. for every $j\in J$ there exists some $x=x(j)\in X$ such that $L_j\subseteq A_{x(j)}$. So, we now define $\big\{L_j\times B_{x(j),k}\big\}_{j\in J, k\in\{k_1(x(j)),\ldots,k_{r(x(j))}(x(j))\}}$, which is a locally finite open cover of $X\times K$ and a refinement of $\{A_{x,k}\times B_{x,k}\}_{(x,k)\in X\times K}$.
\end{proof}

Paracompact spaces were first defined in \cite{paracompactness}, where it was also proven that they are $T_4$:
\begin{theorem}[Dieudonn\'e] Let $X$ be a paracompact space, then $X$ is normal.
\end{theorem}

Using the fact that paracompact spaces are normal together with Urysohn's lemma, it can be proven that any given cover not only has a finite open subcover, but there exists one such subcover that also has a subordinate partition of unity. For the proof of this, the reader can refer to Theorem~12.8 of \cite{bredon}.
\begin{definition} Let $X$ be a topological space and $\mathcal{F}=\{f_i:X\to[0,1]\}_{i\in I}$ be a collection of continuous maps. Then, $\mathcal{F}$ is a \emph{partition of unity of $X$} if for every $x\in X$ there exists some open $U\subseteq X$ containing $x$ such that $f_i|_U\equiv0$ for all but finitely many $i\in I$ and $\sum_{i\in I}f_i(x)=1$.
\end{definition}
\begin{remark} For every $f\in\mathcal{F}$ we are usually interested in the \emph{support of $f$}, which is defined as $\mathrm{supp}(f):=\left\{x\in X:f(x)\neq0\right\}^-$.
\end{remark}
\begin{definition} Let $X$ be a topological space, $\mathcal{U}=\{U_i\}_{i\in I}$ be a collection of subsets of $X$ and $\mathcal{F}=\{f_i:X\to[0,1]\}_{i\in I}$ a partition of unity of $X$ on the same indices. Then $\mathcal{F}$ is \emph{subordinate of $\mathcal{U}$}, if $\mathrm{supp}(f_i)\subseteq U_i$ for every $i\in I$.
\end{definition}
\begin{theorem}\label{thm:paracompact_partition_of_unity} Let $X$ be a paracompact topological space and let $\mathcal{U}$ be an open cover of $X$, then there exists some locally finite open cover $\mathcal{V}$ of $X$ that is a refinement of $\mathcal{U}$ and a partition of unity $\mathcal{F}$ that is subordinate of $\mathcal{V}$.
\end{theorem}

\begin{proposition}\label{prop:paracompact_is_euclidean} Let $B$ be a paracompact topological space and $\xi:E(\xi)\to B$ an $n$-plane vector bundle. Then, $\xi$ is a euclidean vector bundle.
\end{proposition}
\begin{proof} Our goal is to define a continuous $\beta:E(\xi\oplus\xi)\to\mathbb{R}$, the restriction of which on $(\xi\oplus\xi)^{-1}(x)$ is an inner product in $\xi^{-1}(x)$. For each $x\in B$ there exists some open $U_x\subseteq B$ containing $x$ and a vector bundle isomorphism $f_{U_x}:\xi^{-1}(U_x)\to U_x\times\mathbb{R}^n$. Notice that $\{U_x\}_{x\in B}$ is an open cover of $X$. Since $B$ is paracompact, Theorem~\ref{thm:paracompact_partition_of_unity} gives us a locally finite open cover $\mathcal{V}=\{V_i\}_{i\in I}$ of $X$ that is a refinement of $\{U_x\}_{x\in B}$ and a partition of unity $\mathcal{F}=\{h_i\}_{i\in I}$ that is subordinate of $\mathcal{V}$. Since $\mathcal{V}$ is a refinement, every $V_i\in\mathcal{V}$ is contained inside some $U_x$ and thus, using Remark~\ref{rem:subcover} the restriction $f_{V_i}:\xi^{-1}(V_i)\to V_i\times\mathbb{R}^n$ of $f_{U_x}$ is a vector bundle isomorphism. Recall that we can write $E(\xi\oplus\xi)=\left\{(v_1,v_2)\in E(\xi)\times E(\xi):\xi(v_1)=\xi(v_2)\right\}$ and $(\xi\oplus\xi)(v_1,v_2)=\xi(v_1)$. Then, notice that $(\xi\oplus\xi)^{-1}(V_i)=E(\xi|V_i\oplus\xi|V_i)$ and let $f_{V_i}\oplus f_{V_i}:E(\xi|V_i\oplus\xi|V_i)\to V_i\times\mathbb{R}^n\times\mathbb{R}^n$ as in Definition~\ref{def:wh_sum_map}, taking $(v_1,v_2)$ to $(\xi_1(v_1),(\pi_2\circ f_{V_i})(v_1),(\pi_2\circ f_{V_i})(v_2))$. Moreover, let $\gamma:\mathbb{R}^n\times\mathbb{R}^n\to\mathbb{R}$ be the usual dot product of $\mathbb{R}^n$, i.e. $\gamma(u_1,u_2)=u_1^tu_2$. Then, we can define the following continuous map:
\begin{center}
\begin{tikzcd}
\beta_i\ar[r,phantom,":"]&[-2.5em](\xi\oplus\xi)^{-1}(V_i)\ar[r,"f_{V_i}\oplus f_{V_i}"]&V_i\times\mathbb{R}^n\times\mathbb{R}^n\ar[r,"(h_i\circ\pi_1)\cdot(\gamma\circ\pi_{[2,3]})"]&\mathbb{R}\\[-1.5em]
&(v_1,v_2)\ar[rr,mapsto]&&h_i(\xi_1(v_1))\cdot(\pi_2\circ f_{V_i})(v_1)^t(\pi_2\circ f_{V_i})(v_2)
\end{tikzcd}
\end{center}
If we fix some $x\in V_i$, $\beta_i$ restricted on $\xi^{-1}(x)\times\xi^{-1}(x)$ is bilinear and symmetric, since $h_i(\xi_1(v_1))$ is constant. Moreover, $\beta_i(v,v)\geq0$, since $h_i(x)\geq0$. Now, we extend $\beta_i$ to be $\equiv0$ for every $v\in E(\xi\oplus\xi)\setminus\big((\xi\oplus\xi)^{-1}(V_i)\big)$. Since $\mathrm{supp}(h_i)\subseteq V_i$, this is a continuous extension, which we also call $\beta_i$. Now, we define $\beta:=\sum_{i\in I}\beta_i:E(\xi\oplus\xi)\to\mathbb{R}$, where the sum is finite for each input, since $\mathcal{V}$ is locally finite and $\mathcal{F}$ is subordinate of $\mathcal{V}$. It is also continuous, non-negative and over each fiber symmetric and bilinear. It only remains to show that $\beta(v,v)\neq0$ for every $v\neq0$. Let us fix a $v\in E(\xi)$ and an open $U\subseteq B$ containing $\xi(v)$ such that $U\cap V_i\neq\emptyset$ only for finitely many $i\in I$. Let $V_{i_1},\ldots,V_{i_k}\in\mathcal{V}$ be the only sets non-trivially intersecting $U$. Then,  $\beta(v,v)=\beta_{i_1}(v,v)+\cdots+\beta_{i_k}(v,v)=\sum_{j=1}^kh_{i_j}(\xi_1(v))\cdot\big\|(\pi_2\circ f_{V_{i_j}})(v)\big\|^2$.

Define now $a=\min_{j\in[k]}\left\{\big\|(\pi_2\circ f_{V_{i_j}})(v)\big\|\right\}$ and notice that $a>0$, since $\pi_2\circ f_{V_i}$ is a linear isomorphism for every $i\in I$ and $v\neq0$. Hence, $\beta(v,v)\geq a^2\sum_{j=1}^kh_{i_j}(\xi(v))=a^2\sum_{i\in I}h_i(\xi(v))=a^2>0$, since $\mathcal{F}$ is a partition of unity.
\end{proof}

\begin{lemma}\label{lem:paracompact_countable_cover} Let $B$ be a paracompact space and $\xi:E(\xi)\to B$ any vector bundle. Then, there exists some locally finite countable open cover $\{U_k\}_{k\in\mathbb{N}}$ of $B$ such that $\xi|U_k$ is a trivial vector bundle for every $k\in\mathbb{N}$.
\end{lemma}
\begin{proof} Since $\xi$ is a vector bundle, there exists some open cover $\{U_x\}_{x\in B}$ such that $\xi|U_x$ is trivial for every $x\in B$. Using theorem~\ref{thm:paracompact_partition_of_unity}, we can find a locally finite open cover $\mathcal{V}=\{V_{\alpha}\}_{\alpha\in A}$ of $B$ that is a refinement of $\{U_x\}_{x\in B}$ and a partition of unity $\mathcal{F}=\{u_{\alpha}:B\to[0,1]\}$, that is subordinate of $\mathcal{V}$. Notice that Remark~\ref{rem:subcover} gives us that $\xi|V$ is a trivial vector bundle for every $V\in\mathcal{V}$, since it is a subcover of $\{U_x\}_{x\in B}$. If $A$ is finite, $\mathcal{V}$ is a bundle with the desired properties. Otherwise, for every finite $S\subseteq A$ we define now
\[U(S):=\big\{x\in B: \forall\alpha\in S\ \forall\beta\in A\setminus S\ u_{\alpha}(x)>u_{\beta}(x)\big\}.\]
Notice that $U(S)$ is open. Indeed, for $x\in U(S)$, since $\mathcal{F}$ is a partition of unity, there exists an open $X\subseteq B$ and only finitely many indices $\alpha\in A$ such that $u_{\alpha}|_X>0$, say $\{\alpha_1,\ldots,\alpha_k\}$. Hence, $S\subseteq\{\alpha_1,\ldots,\alpha_k\}$ and if we set $T=\{\alpha_1,\ldots,\alpha_k\}\setminus S$, we get that $x\in X\cap\bigcap_{\alpha\in S,\beta\in T}(u_{\alpha}|_X-u_{\beta}|_X)^{-1}\big((0,1]\big)\subseteq U(S)$ and since $S,T$ are finite, this is an open set. Since this is true for every $x\in U(S)$, $U(S)$ is open.

Moreover, notice that $\{U(S)\}_{S\in\mathcal{S}}$ covers $B$, where $\mathcal{S}:=\{S\subseteq A:|S|\in\mathbb{N}\}$ is the set of all finite subsets of $A$. Indeed, for every $x\in B$ there exist only finitely many indices $\{\alpha_1,\ldots,\alpha_k\}\subseteq A$ with $u_{\alpha}(x)>0$, as we argued before and thus $x\in U(\{\alpha_1,\ldots,\alpha_k\})$. Also, it is true that $U(S)\subseteq V_{\alpha}$ for every $\alpha\in S$. Indeed, if $\alpha\in S$ and $x\in U(S)$, then it should be true that $u_{\alpha}(x)>0$, because otherwise we would have $S=A$, but $A$ is not finite. So, $\{U_S\}_{S\in\mathcal{S}}$ is an open subcover of $\mathcal{V}$ and thus, it is a locally finite open cover of $B$ and again due to Remark~\ref{rem:subcover} $\xi|_{U(S)}$ is a trivial vector bundle for every $S\in\mathcal{S}$.

Last, for every $k\in\mathbb{N}$ we define $U_k:=\bigcup_{S\in\mathcal{S}_k}U(S)$, where $\mathcal{S}_k=\{S\in\mathcal{S}:|S|=k\}$. Then, $\{U_k\}_{k\in\mathbb{N}}$ is clearly a cover of $B$ with $U_k$ open for every $k\in\mathbb{N}$. Moreover, it is true that $U_k$ is the disjoint union of all $U(S)$ for $S\in\mathcal{S}_k$. Indeed, for any two $S,S'\in\mathcal{S}_k$ with $S\neq S'$, there exist $\alpha\in S\setminus S'$ and $\beta\in S'\setminus S$. If there was some $x\in U(S)\cap U(S')$, then for this $x$ we would have simultaneously that $u_{\alpha}(x)>u_{\beta}(x)$ and $u_{\beta}(x)>u_{\alpha}(x)$, which is impossible. So, $\xi|U_k\cong\amalg_{S\in\mathcal{S}_k}\xi|_{U(S)}$ and thus is a trivial vector bundle. Also, for every $x\in B$ there exists some $X\subseteq B$ containing $x$, such that $X\cap U(S)\neq\emptyset$ only for finitely many $S\in\mathcal{S}$, since $\{U(S)\}_{S\in\mathcal{S}}$. This means that $X\cap U_k\neq\emptyset$ only for finitely many $k\in\mathbb{N}$. Hence $\{U_k\}_{k\in\mathbb{N}}$ is a locally finite, countable, open cover of $B$ making $\xi$ locally trivial.
\end{proof}
\begin{remark}\label{lem:paracompact_countable_cover_fb} At no point in this proof did we need that $\xi$ is linear over each fiber, so exactly the same proof can be done for a fiber bundle $p:E(p)\to B$. Namely, if $B$ is paracompact, then there exists some locally finite countable open cover $\{U_k\}_{k\in\mathbb{N}}$ of $B$ such that $p|U_k$ is a trivial fiber bundle for every $k\in\mathbb{N}$.
\end{remark}

%TODO_MAYBE: tangent and normal vb
%TODO_MAYBE: define parallelizable spaces
%TODO_MAYBE: counterexample of vb over non-paracompact space w/o inner product
%TODO_MAYBE: refer to thms for generality of paracompact spaces

	\chapter{Cohomology computation}
This chapter starts by a direct combination of the Grassmannians discussed in Chapter~1 and the vector bundles discussed in Chapter~2. Namely, the tautological vector bundle is defined over the Grassmannian and additionaly it is proven that this is a universal bundle, in the sense that any other vector bundle over almost any topological space is a subbundle of the tautological bundle. Then, using the universal line bundles we define the Stiefel-Whitney class of any vector bundle to be a sequence of elements in the cohomology of the base space and prove that it is invariant under bundle isomorphisms. At the end of the chapter, we compute the Stiefel-Whitney classes of the tautological vector bundle and prove that it generates the Cohomology of the Grassmannian.

\begin{theorem}\label{thm:projective_spaces_cohomology} There exist the following isomorphisms of graded $\mathbb{Z}_2$-algebras:
\begin{b_item}
\item $H^*(\mathbb{R}P^{\infty};\mathbb{Z}_2)\cong\mathbb{Z}_2[z]$, where $\deg(z)=1$ and
\item $H^*(\mathbb{R}P^n;\mathbb{Z}_2)\cong\mathbb{Z}_2[z_n]/(z_n^{n+1})$, where $\deg(z_n)=1$, for every $n\in\mathbb{N}$.
\end{b_item}
Moreover, for every $n\in\mathbb{N}$, the inclusion $\iota_{1,n+1}:\mathbb{R}P^n\hookrightarrow\mathbb{R}P^{\infty}$ induces an epimorphism $\mathbb{Z}_2[z]\to\mathbb{Z}_2[z_n]/(z_n^{n+1})$ with $\iota_{1,n+1}^*z=z_n$.
\end{theorem}

\section{Tautological vector bundle}
\begin{proposition} Let $k,n\in\mathbb{N}$, with $0<k<n$. Moreover let $\xi:\big\{(H,v)\in\Gr{k}{n}\times\mathbb{R}^n:v\in H\}\to\Gr{k}{n}$ be the function with $\xi(H,v)=H$. Also, for $H_0\in\Gr{k}{n}$ identify $\xi^{-1}(H_0)=\{H_0\}\times H_0$ with $H_0$ as vector spaces. Then $\xi$ is a $k$-plane vector bundle.
\end{proposition}
\begin{proof} We only need to show the local triviality of $\xi$. Let $H_0\in\Gr{k}{n}$ and $U_{H_0}:=\{K\in\Gr{k}{n}:K\cap H_0^{\perp}=\{0\}\}$ as we also defined in the proof of \ref{lem:gr_manifold}. Notice that $\xi^{-1}(U_{H_0})=\{(K,v)\in\Gr{k}{n}\times\mathbb{R}^n:K\cap H_0^{\perp}=\{0\}\text{ and }v\in K\}$. Fix $v_1,\ldots,v_k\in H_0$ to be any basis of $H_0$. We define then the map $\phi:\xi^{-1}(U_{H_0})\to U_{H_0}\times\mathbb{R}^k$ as follows: $\phi(K,v):=(K,v^tv_1,v^tv_2,\ldots,v^tv_k)$.

This map is obviously a continuous bundle map. Also, over each fiber it is linear, since $\phi(K,\lambda u+\mu v)=(K,(\lambda u+\mu v)^tv_1,\ldots,(\lambda u+\mu v)^tv_1)=\lambda(K,u)+\mu(K,u)=\lambda\phi(K,u)+\mu\phi(K,v)$. Notice that since $K\cap H_0^{\perp}=\{0\}$, $v^tv_i=0$ for all $i\in[k]$ if and only if $v=0\in K$, so $\phi(K,v)=0$ if and only if $v=0$. This proves that the kernel of the linear map $\phi_K:=\phi(K,-)$ vanishes and since $\dim_{\mathbb{R}}K=k=\dim_{\mathbb{R}}\mathbb{R}^k$, $\phi_K$ is a linear isomorphism. Thus, Proposition~\ref{prop:local_to_global_iso_vector} gives us that $\phi$ is a vector bundle isomorphism, proving the assertion.
\end{proof}
\begin{remark} Similarly, for $\xi:\{(H,v)\in\Gr{k}\times\mathbb{R}^{\infty}\}\to\Gr{k}$ with $\xi(H,v)=H$, $\xi$ is a $k$-plane vector bundle over $\Gr{k}$. Indeed, for the local triviality we again define $U_{H_0}=\{K\in\Gr{k}:K\cap H_0^{\perp}=\{0\}\}$, since there is a well defined inner product on $\mathbb{R}^{\infty}$, as we saw in Example~\ref{ex:infinite_euclidean_space} and then, the arguments are exactly the same.
\end{remark}

\begin{definition} Let $k,n\in\mathbb{N}$, with $0<k<n$. Then the \emph{tautological} vector bundle on $\Gr{k}{n}$ is the map $\gamma_n^k:\big\{(H,v)\in\Gr{k}{n}\times\mathbb{R}^n:v\in H\big\}$ with $\gamma_n^k(H,v)=H$.
\end{definition}
\begin{remark} This is a generalization of the tautological line bundle over the projective space we defined at Example~\ref{ex:vector_bundles}~\!\ref{ex:tautological_line_bundle}.
\end{remark}

\begin{definition} For $k\in\mathbb{N}$, the \emph{tautological}, or \emph{universal} vector bundle on $\Gr{k}$ is the map $\gamma^k:\big\{(H,v)\in\Gr{k}\times\mathbb{R}^{\infty}:v\in H\big\}$ with $\gamma^k(H,v)=H$.
\end{definition}

\begin{proposition}\label{prop:restriction} For any $k,n\in\mathbb{N}$, with $0<k<n$ it is true that $\gamma_n^k\cong\iota^*_{k,n}\gamma^k$, where $\iota_{k,n}:\Gr{k}{n}\hookrightarrow\Gr{k}$.
\end{proposition}
\begin{proof} Notice that $E(\iota_{k,n}^*\gamma^k)=\big\{(H,H',v)\in\Gr{k}{n}\times\Gr{k}\times\mathbb{R}^{\infty}:v\in H'\text{ and }H=H'\big\}\cong\big\{(H,v)\in\Gr{k}{n}\times\mathbb{R}^{\infty}:v\in H\big\}\cong\big\{(H,v)\in\Gr{k}{n}\times\mathbb{R}^n:v\in H\big\}=E(\gamma_n^k)$, where the homeomorphism is clearly a bundle isomorphism and linear over each fiber.
\end{proof}

\begin{theorem}\label{thm:universal} Let $B$ be any paracompact topological space and $\xi:E(\xi)\to B$ any $k$-plane vector bundle. Then, there exists a continuous function $f:B\to\Gr{k}$ such that $\xi\cong f^*\gamma^k$.
\end{theorem}

\begin{lemma}\label{lem:injection_from_total_space} Let $\xi:E(\xi)\to B$ be a vector bundle and $\hat{f}:E(\xi)\to\mathbb{R}^{\infty}$ be a continuous map such that its restriction $\hat{f}_x:\xi^{-1}(x)\to\mathbb{R}^{\infty}$ is a linear injection for every $x\in B$. Then, for the function $f:B\to\Gr{k}$ with $f(x):=\hat{f}(\xi^{-1}(x))$ it is true that $\xi\cong f^*\gamma^k$. Conversely, given $f$ the function $\hat{f}$ can be uniquely retrieved.
\end{lemma}
\begin{proof} Let $\hat{f}:E(\xi)\to\mathbb{R}^{\infty}$, such that its restriction $\hat{f}_x:\xi^{-1}(x)\to\mathbb{R}^{\infty}$ is a linear injection for every $x\in B$. This means that $\dim_{\mathbb{R}}(\hat{f}(\xi^{-1}(x)))=k$, which lets us define the function $f:B\to\Gr{k}$ such that $f(x)=\hat{f}(\xi^{-1}(x))\in\Gr{k}$. In order to prove that this is a continuous function, we will factor $f$ through $q:V{k}\to\Gr{k}$ locally. For $x\in B$, there exists some open $U\subseteq B$ containing $x$, such that $\xi|U$ is a trivial vector bundle. Equivalently, there exist $k$ nowhere dependent sections $s_1,\ldots,s_k:U\to\xi^{-1}(U)$. Notice that $f|_U(x)=q\big(\hat{f}(s_1(x)),\hat{f}(s_2(x)),\ldots,\hat{f}(s_k(x))\big)$, which is continuous. This means that $f:B\to\Gr{k}$ is continuous.

Next, we define $\bar{f}:E(\xi)\to E(\gamma^k)=\big\{(H,v)\in\Gr{k}\times\mathbb{R}^{\infty}:v\in H\big\}$ to be the continuous function $\bar{f}(v):=\big(f(\xi(v)),\hat{f}(v)\big)$. It is well defined, since $\hat{f}(v)\in\hat{f}(\xi^{-1}(\xi(v)))=f(\xi(v))$. Moreover, since $(f\circ\xi)(v)=(\gamma^k\circ\bar{f})(v)$, it is true that $\bar{f}(\xi^{-1}(x))\subseteq(\gamma^k)^{-1}(f(x))$. Also, since $\hat{f}$ is fiber-wise linear, so is $\bar{f}$ as well. Now, define $\phi:E(\xi)\to E(f^*\gamma^k)=\big\{(x,H,v)\in B\times\Gr{k}\times\mathbb{R}^{\infty}:v\in H\text{ and }f(x)=H\big\}$ by $\phi(v):=(\xi(v),\bar{f}(v))=(\xi(v),f(\xi(v)),\hat{f}(v))$. This is clearly a vector bundle map. For the restriction $\phi_x:\xi^{-1}(x)\to\{x\}\times\{f(x)\}\times f(x)\cong f(x)\subseteq\mathbb{R}^{\infty}$, it is clear that $\phi_x(v)=(x,f(x),\hat{f}(v))$, which is a linear isomorphism, since $\hat{f}$ is injective and $\dim\xi^{-1}(x)=k=\dim f(x)$. Hence, Proposition~\ref{prop:local_to_global_iso_vector} ensures that $\phi$ is a vector bundle isomorphism.

For the converse, given $f:B\to\Gr{k}$ and a vector bundle isomorphism $\phi:E(\xi)\to E(f^*\gamma)$, we can always define the function $\bar{f}:E(f^*\gamma^k)\to E(\gamma_k)$ which completes the pushout square and define $\hat{f}:=\pi_2\circ\bar{f}\circ\phi:E(\xi)\to\mathbb{R}^{\infty}$. It is trivial to check that $\hat{f}$ is a linear injection on each fiber and that $f(x)=\hat{f}(\xi^{-1}(x))$.
\end{proof}

\begin{proof}[Proof of Theorem~\ref{thm:universal}] Because of Lemma~\ref{lem:injection_from_total_space}, it suffices to define a continuous, fiber-wise linear injection map $\hat{f}:E(\xi)\to\mathbb{R}^{\infty}$.

Since $B$ is paracompact, Lemma~\ref{lem:paracompact_countable_cover} gives us a locally finite countable open cover $\{U_i\}_{i\in\mathbb{N}}$ of $B$, such that $\xi|U_i$ is a trivial vector bundle for every $i\in\mathbb{N}$. Let $f_{U_i}:\xi^{-1}(U_i)\to U_i\times\mathbb{R}^k$ be a bundle isomorphism for every $i\in\mathbb{N}$. Also, Theorem~\ref{thm:paracompact_partition_of_unity} lets us find a subcover $\mathcal{V}$ with a subordinate partition of unity $\mathcal{F}=\{u_i\}_{i\in\mathbb{N}}$. Then, $\mathcal{V}$ is also a locally finite countable open cover making $\xi$ trivial, so without loss of generality we will assume that $\mathcal{F}$ is subordinate of $\{U_i\}_{i\in\mathbb{N}}$.

For every $i\in\mathbb{N}$ we define $h_i:\xi^{-1}(U_i)\to\mathbb{R}^k$ by
\begin{center}
\begin{tikzcd}
h_i\ar[r,phantom,":"]&[-2em]\xi^{-1}(U_i)\ar[r,"f_{U_i}","\cong"']&U_i\times\mathbb{R}^k\ar[r,"(u_i\circ\pi_1)\cdot(\pi_2)"]&\mathbb{R}^k\\[-1.5em]
&v\ar[r,mapsto]&(\xi(v),(\pi_2\circ f_{U_i})(v))\ar[r,mapsto]&u_i(\xi(v))\cdot(\pi_2\circ f_{U_i})(v)
\end{tikzcd}
\end{center}
This map is clearly continuous and linear over $\xi^{-1}(x)$ for $x\in U_i$. We extend $h_i$ to the whole $E(\xi)$ by setting $h_i(v)=0$ for every $v\in E(\xi)\setminus\xi^{-1}(U_i)$. We will abuse the notation and use $h_i$ for the extended function as well. Notice that $h_i$ remains continuous and linear over each fiber as a function $h_i:E(\xi)\to\mathbb{R}^k$.

Then, let $\hat{f}:E(\xi)\to\mathbb{R}^{\infty}$ be the function with $\hat{f}(v):=\big(h_1(v),h_2(v),\ldots\big)$ for every $v\in E(\xi)$. This is well defined, since $\{U_i\}_{i\in\mathbb{N}}$ is a locally finite cover of $B$, which means that for every $v\in E(\xi)$ there are only finitely many indices $i\in\mathbb{N}$ such that $h_i(\xi)\neq0$. This means that for every $v\in E(\xi)$ there exists some $n_0\in\mathbb{N}$ with $\hat{f}(v)\in\mathbb{R}^{kn_0}\subseteq\mathbb{R}^{\infty}$. Moreover, $\hat{f}$ is clearly continuous and its restriction $\hat{f}_x:\xi^{-1}(x)\to\mathbb{R}^{\infty}$ is linear.

Next, notice that $\hat{f}_x:\xi^{-1}(x)\to\mathbb{R}^{\infty}$ is injective. Indeed, let $v\in\ker\hat{f}_x$, i.e. $h_1(v)=h_2(v)=\ldots=0\in\mathbb{R}^k$. Since $\mathcal{F}$ is a partition of unity on $B$, there exists some $i_0\in\mathbb{N}$ with $u_{i_0}(\xi(v))>0$. Then $\xi(v)\in U_{i_0}$ and thus $h_{i_0}(v)=0$ means $f_{U_{i_0}}(v)=(\xi(v),0)$. Since the restriction of $f_{U_{i_0}}|:\xi^{-1}(\xi(v))\to\{\xi(v)\}\times\mathbb{R}^k$ is a homeomorphism, $v=0\in\xi^{-1}(\xi(v))$, which proves the assertion.
\end{proof}

\begin{theorem}\label{thm:universal_uniqueness} Let $B$ be any paracompact topological space and $f,g:B\to\Gr{k}$ any two continuous maps, such that $f^*\gamma^k\cong g^*\gamma^k$ as vector bundles. Then $f\simeq g$.
\end{theorem}

\begin{lemma}\label{lem:even_odd} Let $b_0,b_1:\mathbb{R}^{\infty}\to\mathbb{R}^{\infty}$ be the following linear maps
\begin{align*}
b_0(x_1,x_2,x_3,x_4,x_5,\ldots)&=(0,x_1,0,x_2,0,x_3,0,x_4,\ldots)\\
b_1(x_1,x_2,x_3,x_4,x_5,\ldots)&=(x_1,0,x_2,0,x_3,0,x_4,0,\ldots)
\end{align*}
Then, $b_0,b_1$ induce some $d_0,d_1:\Gr{k}\to\Gr{k}$, for which it is true that $d_i^*\gamma_k\cong\gamma$ and $d_i\simeq id_{\Gr{k}}$ for both $i\in\{0,1\}$.
\end{lemma}
\begin{proof} Let us fix an $i\in\{0,1\}$ for the whole proof. First, we define $d_i$ as follows: Let $b_i^k:\St{k}\to\St{k}$ be the induced product map, i.e. let $b_i^k(v_1,\ldots,v_k)=(b_i(v_1),\ldots,b_i(v_k))$. This map is well defined, since $v_1,\ldots,v_k$ are linearly independent if and only if $b_i(v_1),\ldots,b_i(v_k)$ are linearly independent. Moreover, notice that for any two $(v_1,\ldots,v_k),(v_1',\ldots,v_k')\in\St{k}$ with $\mathrm{span}\{v_1,\ldots,v_k\}=\mathrm{span}\{v_1',\ldots,v_k'\}$ we have $\mathrm{span}\{b_i(v_1),\ldots,b_i(v_k)\}=\mathrm{span}\{b_i(v_1'),\ldots,b_i(v_k')\}$, i.e. the map $q\circ b_i^k:\St{k}\to\Gr{k}$ induces a continuous map $d_i:\Gr{k}\to\Gr{k}$.

Next, notice that $E(d_i^*\gamma^k)=\big\{(H_1,H_2,v)\in\Gr{k}\times\Gr{k}\times\mathbb{R}^{\infty}:v\in H_2\text{ and }d_i(H_1)=H_2\big\}\cong\big\{(H,v)\in\Gr{k}\times\mathbb{R}^{\infty}:v\in d_i(H)\big\}$, with $(d_i^*\gamma^k)(H,v)=H$ and $\bar{d}_i(H,v)=(d_i(H),v)$. In order to show that $d_i^*\gamma^k\cong\gamma^k$ we construct the following map $\phi_i:E(\gamma^k)\to E(d_i^*\gamma^k)$ which takes $(H,v)$ to $(H,b_i(v))$. This is well defined, since $b_i(v)\in d_i(H)$. Moreover, it is clearly continuous and linear isomorphism over the fibers. Thus, Proposition~\ref{prop:local_to_global_iso_vector} gives us that $\phi_i$ is a vector bundle isomorphism.

It now remains to define a homotopy $h:\Gr{k}\times[0,1]\to\Gr{k}$, such that $h(H,0)=H$ and $h(H,1)=d_i(H)$. For the maps $d_i,id:\Gr{k}\to\Gr{k}$, we define $\hat{d}_i,\hat{id}:E(\gamma^k)\to \mathbb{R}^{\infty}$ using Lemma~\ref{lem:injection_from_total_space}, i.e. $\hat{id}(H,v)=v$ and
\begin{center}
\begin{tikzcd}
\hat{d}_i\ar[r,phantom,":"]&[-2em]E(\gamma^k)\ar[r,"\phi_i"]&E(d_i^*\gamma^k)\ar[r,"\bar{d}_i"]&E(\gamma^k)\ar[r,"\pi_2"]&\mathbb{R}^{\infty}\\[-1.5em]
&(H,v)\ar[r,mapsto]&(H,b_i(v))\ar[r,mapsto]&(d_i(H),b_i(v))\ar[r,mapsto]&b_i(v)
\end{tikzcd}
\end{center}
Next, let $\hat{h}:E(\gamma^k)\times[0,1]\to\mathbb{R}^{\infty}$ be the homotopy $\hat{h}(v,t)=(1-t)v+tb_i(v)$. Notice that each $\hat{h}_t:E(\gamma^k)\to\mathbb{R}^{\infty}$ is continuous and its restriction $(\hat{h}_t)_H:(\gamma^k)^{-1}(H)\to\mathbb{R}^{\infty}$ is linear. In fact $(\hat{h}_t)_H$ is injective. We already this is the case for $t=0$ and $t=1$ so fix a $t\in(0,1)$ and let $v\in(\gamma^k)^{-1}(H)$ such that $(1-t)v+tb_i(v)=0$, i.e. $b_i(v)=\frac{t-1}{t}v$. Since $v\in\mathbb{R}^{\infty}$, there exists some $n_0$ such that $v_n=0$ for every $n\geq n_0$, where $v_n$ is the $n$-th coordinate of $v$. Let us choose the minimum such $n_0$. This means that $b_i(v)_n=0$ for every $n\geq n_0$. For the case of $b_0$ this means that $v_n=0$ for $n\geq\left\lceil\frac{n_0}{2}\right\rceil$ but if $n_0>1$, then $n_0>\left\lceil\frac{n_0}{2}\right\rceil$ which contradicts the minimality of $n_0$. So we have that $n_0=1$ and $v=0$. For the case of $b_1$ this means that $v_n=0$ for $n\geq\left\lceil\frac{n_0+1}{2}\right\rceil$ but if $n_0>2$, then $n_0>\left\lceil\frac{n_0+1}{2}\right\rceil$ which contradicts the minimality of $n_0$. If $n_0\leq2$, then we have that $v_1=\frac{t-1}{t}v_1$, which again means $n_0=1$ and $v=0$.

Using Lemma~\ref{lem:injection_from_total_space} again, we can thus define continuous $h_t:\Gr{k}\to\Gr{k}$ by $h_t(H):=\hat{h}_t((\gamma^k)^{-1}(H))$ and finally, we define $h:\Gr{k}\times[0,1]\to\Gr{k}$ by $h(H,t)=\hat{h}(\xi^{-1}(H),t)$, which is a homotopy between $d_i$ and $id$.
\end{proof}

\begin{proof}[Proof of Theorem~\ref{thm:universal_uniqueness}] We will use the maps $d_0,d_1:\Gr{k}\to\Gr{k}$ defined in Lemma~\ref{lem:even_odd}. Since $d_i\simeq id$, it is true that $f\simeq d_0\circ f$ and $g\simeq d_1\circ g$. Moreover, since $d_i^*\gamma^k\cong\gamma^k$, it is also true that $(d_i\circ f)^*\gamma^k=f^*\gamma^k$ and similarly $(d_i\circ g)^*\gamma^k=g^*\gamma^k$. So, without loss of generality we will suppose that $f(x)\subseteq\{(x_1,x_2,\ldots)\in\mathbb{R}^{\infty}:\forall n\in\mathbb{N}\ x_{2n}=0\}$ and $g(x)\subseteq\{(x_1,x_2,\ldots)\in\mathbb{R}^{\infty}:\forall n\in\mathbb{B}\ x_{2n-1}=0\}$. Let $\xi\cong f^*\gamma^k\cong g^*\gamma^k$ and $\hat{f},\hat{g}:E(\xi)\to\mathbb{R}^{\infty}$ be the associated maps of $f,g:B\to\Gr{k}$, as defined in Lemma~\ref{lem:injection_from_total_space}.

Like in the proof of the Lemma~\ref{lem:even_odd}, we first define the homotopy $\hat{h}:E(\xi)\times[0,1]\to\mathbb{R}^{\infty}$ with $\hat{h}_t(v)=(1-t)\hat{f}+t\hat{g}$. This makes $\hat{h}$ continuous and its restriction $(\hat{h}_t)_x:\xi^{-1}(x)\to\mathbb{R}^{\infty}$ linear. In fact, $\hat{h}_t$ is a linear injection. Indeed, fix some $t\in[0,1]$ and let $v\in E(\xi)$ such that $\hat{h}_t(v)=(1-t)\hat{f}(v)+t\hat{g}(v)=0$. But, $\hat{h}_t(v)_n=(1-t)\hat{f}(v)_n+t\hat{g}(v)_n$ which equals $(1-t)\hat{f}(v)_n$ if $n$ is odd and $t\hat{g}(v)_n$ if $n$ is even. This means that $\hat{f}(v)=\hat{g}(v)=0$, i.e. $v=0$ since both $\hat{f}$ and $\hat{g}$ are injective.

Hence, using Lemma~\ref{lem:injection_from_total_space} again, we can now define a continuous $h_t:B\to\Gr{k}$ by $h_t(x)=\hat{h}_t(\xi^{-1}(x))$ and thus $h:B\times[0,1]\to\Gr{k}$ with $h(x,t)=\hat{h}(\xi^{-1}(x),t)$ is a homotopy between $f$ and $g$.
\end{proof}

\section{Stiefel-Whitney classes}
\begin{axioms} Let $B$ be a topological space and $\xi:E(\xi)\to B$ any $k$-plane vector bundle. Then, an element
\[w(\xi)=w_0(\xi)+w_1(\xi)+w_2(\xi)+\cdots\in H^{\prod}(B;\mathbb{Z}_2)\]
where $w_i(\xi)\in H^i(B;\mathbb{Z}_2)$ for every $i\in\mathbb{N}_0$ \emph{satisfies the Stiefel-Whitney Axioms} if it satisfies the following four Axioms:
\begin{b_item}
\item[(SW1)] $w_0(\xi)=1$ and $w_{k+1}(\xi)=w_{k+2}(\xi)=\cdots=0$, for every $k$-plane $\xi:E(\xi)\to B$.
\item[(SW2)] $w(f^*\xi)=f^*w(\xi)$, for every $f:B'\to B$ and every $\xi:E(\xi)\to B$.
\item[(SW3)] $w(\eta\oplus\xi)=w(\eta)w(\xi)$, for every $\eta:E(\eta)\to B$ and every $\xi:E(\xi)\to B$.
\item[(SW4)] $w(\gamma_1^1)\neq0$.
\end{b_item}
\end{axioms}

For our discussion about Stiefel-Whitney classes, we are heavily relying on ~\cite{husemoller}, Chapter~17, Sections~2-6.

\subsection{Basic properties of Stiefel Whitney classes}
In this subsection, we assume that there exists some $w(\xi)$ satisfying the Stiefel-Whitney axioms and examine the nice properties that it also satisfies.
\begin{proposition}\label{prop:trivial_sw} For all $k\in\mathbb{N}$, $w(\varepsilon_B^k)=1$, i.e. $w_i(\varepsilon_B^k)=0$ for every $i>0$.
\end{proposition}
\begin{proof} Let $c_B:B\to\{*\}$ be the map to the one-point set. Then $\varepsilon_B^n\cong c_B^*\varepsilon_{\{*\}}^n$. Indeed, $E(c_B^*\varepsilon_{\{*\}}^n)\cong B\times\mathbb{R}^n=E(\varepsilon_B^n)$. Let $i>0$, then using (SW1) we then get $w_i(\varepsilon_B^n)=c_B^*w_i(\varepsilon_{\{*\}}^n)=0$, since $w_i(\varepsilon_{\{*\}}^n)\in H^i(\{*\},\mathbb{Z}_2)=0$.
\end{proof}

\begin{proposition} Let $\eta:E(\eta)\to B$, $\xi:E(\xi)\to B$ two vector bundles such that $\eta\cong\xi$. Then $w(\eta)=w(\xi)$.
\end{proposition}
\begin{proof} Let $\phi:E(\eta)\to E(\xi)$ be a vector bundle isomorphism, then $\eta=(id)^*\xi$. Then, (SW1) gives $w(\eta)=w((id)^*\xi)=(id)^*w(\xi)=w(\xi)$.
\end{proof}

\begin{proposition} Let $\xi:E(\xi)\to B$ and $n\in\mathbb{N}$, then $w(\xi\oplus\varepsilon_B^n)=w(\xi)$.
\end{proposition}
\begin{proof} Using (SW2) and Proposition~\ref{prop:trivial_sw} we get $w(\xi\oplus\varepsilon_B^n)=w(\xi)w(\varepsilon_B^n)=w(\xi)$.
\end{proof}

\begin{corollary} Let $\eta:E(\eta)\to B$, $\xi:E(\xi)\to B$ two vector bundles such that $\eta\oplus\varepsilon_B^k\cong\xi\oplus\varepsilon_B^n$ for some $k,n\in\mathbb{N}$. Then $w(\eta)=w(\eta\oplus\varepsilon_B^k)=w(\xi\oplus\varepsilon_B^n)=w(\xi)$.
\end{corollary}

\begin{proposition} Let $\xi:E(\xi)\to B$ be an $n$-plane euclidean vector bundle and $s_1,\ldots,s_k:B\to E(\xi)$ $k$ nowhere linearly dependent sections. Then $w_{n-k+1}(\xi)=w_{n-k+2}=\cdots=w_n(\xi)=0$.
\end{proposition}
\begin{proof} Given $k$ nowhere linearly dependent sections of $\xi$, we can define a $k$-plane trivial vector bundle $\eta$ with $\eta\leq\xi$. Indeed, let $E(\eta)=\big\{v\in E(\xi):v\in\mathrm{span}\{s_1(\xi(v)),\ldots,s_k(\xi(v))\}\subseteq E(\xi)$ and $\eta(v)=\xi(v)$. Then $\phi:E(\eta)\to E(\varepsilon_B^k)$ takes $v=a_1s_1(\xi(v))+\cdots+a_ks_k(\xi(v))$ to $(\xi(v),a_1,\ldots,a_k)$. Since $\xi$ is euclidean, Lemma~\ref{prop:vb_decomposition} gives us that $\xi\cong\eta\oplus\eta^{\perp_{\xi}}$. Thus, (SW2) and Proposition~\ref{prop:trivial_sw} give: $w(\xi)=w(\eta)w(\eta^{\perp_{\xi}})=w(\eta^{\perp_{\xi}})$ and (SW0) gives $w_i(\eta^{\perp_{\xi}})=0$ for every $i>n-k$, since $\eta^{\perp_{\xi}}$ is an $(n-k)$-plane vector bundle.
\end{proof}

\begin{lemma} For any topological space $B$, the space $\big\{1+a_1+a_2+\cdots\in H^{\prod}(B;\mathbb{Z}_2)\big\}$ is a group with respect to the cup product.
\end{lemma}
\begin{proof} We only have to check that every $1+a_1+a_2+\cdots\in H^{\prod}(B;\mathbb{Z}_2)$ is invertible. We can construct the inverse element $1+\bar{a}_1+\bar{a}_2+\cdots$ inductively by:
\[\bar{a}_n=-a_n-\sum_{i=k}^{n-1}a_k\bar{a}_{n-k}\qedhere\]
\end{proof}

\begin{definition} Let $\xi:E(\xi)\to B$ be a $k$-plane vector bundle. Then, we define the \emph{dual of the stiefel whitney classes of $\xi$} to be $\bar{w}(\xi):=(\bar{w}(\xi))^{-1}\in H^{\prod}(B;\mathbb{Z}_2)$.
\end{definition}

\begin{proposition} Let $M$ be a smooth manifold embedded in $\mathbb{R}^d$, $\tau_M:TM\to M$ be its tangent bundle and $\nu_{M,d}:N_dM\to M$ its normal bundle. Then $\bar{w}(\tau_M)=w(\nu_{M,d})$
\end{proposition}
\begin{proof} Proposition~\ref{prop:tangent_normal_vb} gives us that $\nu_{M,d}=\tau_M^{\perp_{\mathbb{R}^d}}$ and thus, because of Proposition~\ref{prop:vb_decomposition}, we have that $\varepsilon_M^d\cong\tau_M\oplus\nu_{M,d}$. Using (SW2) we get $w(\tau_M)w(\nu_{M,d})=1$, i.e. $\bar{w}(\tau_M)=w(\nu_{M,d})$.
\end{proof}

\subsection{Definition of Stiefel Whitney classes}
In this subsection we define the Stiefel-Whitney classes for every $k$-plane bundle over a paracompact space.
\begin{definition} A fiber bundle $p:E\to B$ is \emph{of finite type}, if there exists a finite open cover $\{U_i\}_{i\in[n]}$ such that $p|U_i$ is the trivial fiber bundle.
\end{definition}
\begin{theorem}[Leray-Hirsch]\label{thm:leray_hirsch} Let $B$ be a topological space and $p:E(p)\to B$ be a fiber bundle of finite type with fiber $F$. For each $x\in B$ fix some homeomorphism $j_x:F\to p^{-1}(x)$. Let $a_1\in H^{n_1}(E;\mathbb{Z}_2),\ldots,a_r\in H^{n_r}(E;\mathbb{Z}_2)$ such that for every $x\in B$
\begin{b_item}
\item the elements $j_x^*a_1,\ldots,j_x^*a_r$ are $\mathbb{Z}_2$-linearly independent in $H^*(F;\mathbb{Z}_2)$ and
\item $\mathbb{Z}_2\left<j_x^*a_1,\ldots,j_x^*a_r\right>\cong H^*(F;\mathbb{Z}_2)$ as $\mathbb{Z}_2$-modules.
\end{b_item}
Then,
\begin{b_item}
\item the elements $a_1,\ldots,a_r$ are $H^*(B;\mathbb{Z}_2)$-linearly independent in $H^*(E;\mathbb{Z}_2)$ and
\item $H^*(B;\mathbb{Z}_2)\left<a_1,\ldots,a_r\right>\cong H^*(E;\mathbb{Z}_2)$ as $H^*(B;\mathbb{Z}_2)$-modules,
\end{b_item}
where the $H^*(B;\mathbb{Z}_2)$-module structure on $H^*(E;\mathbb{Z}_2)$ is defined by $p^*:H^*(B;\mathbb{Z}_2)\to H^*(E;\mathbb{Z}_2)$. This means that the isomorphism is given by
\begin{center}
\begin{tikzcd}
H^*(B;\mathbb{Z}_2)\left<a_1,\ldots,a_r\right>\ar[r]&H^*(E;\mathbb{Z}_2)\\[-1.5em]
\displaystyle\sum_{i=0}^{k-1}x_i\cdot a_{\xi}^i\ar[r,mapsto]&p^*(x_i)a_{\xi}^i
\end{tikzcd}
\end{center}
In particular, $p^*$ is a monomorphism.
\end{theorem}
\begin{proof}
%TODO
\end{proof}

\begin{proposition}\label{prop:proj_bundle} Let $\xi:E(\xi)\to B$ be a $k$-plane vector bundle. Moreover, let $E_0:=\{v\in E:v\neq 0\}$ and $q_E:E_0\to\sfrac{E_0}{\sim}$, where $v\sim v'$ if and only if $\xi(v)=\xi(v')$ and $v=\lambda v'$ for some $\lambda\in\mathbb{R}\setminus\{0\}$. Also, let $p:E':=\sfrac{E_0}{\sim}\to B$ be the function $p([v])=\xi(v)$. Then $p$ is a fiber bundle with fiber $\mathbb{R}P^{k-1}$. Moreover, if for some $x\in B$ we fix a linear isomorphism $f_x:\xi^{-1}(x)\to\mathbb{R}^k$, then there exists an induced homeomorphism $g_x:p^{-1}(x)\to\mathbb{R}P^{k-1}$ such that $q\circ f_x(v)=g_x([v])$ for every $v\in E_0\cap\xi^{-1}(b)$, where $q:\mathbb{R}^k\to\mathbb{R}P^{k-1}$ is the usual quotient map.
\end{proposition}
\begin{proof} First of all $p$ is well defined and continuous, since $v\sim v'$ means in particular that $\xi(v)=\xi(v')$. So, it only remains to show that $p$ is locally trivial. Let $x\in B$, then since $\xi$ is locally trivial, there exists some open $U\subseteq B$ containing $x$ and a vector bundle isomorphism $f_U:\xi^{-1}(U)\to U\times\mathbb{R}^k$. Then, for the function $(id\times q)\circ f_U:E_0\cap\xi^{-1}(U)\to U\times\mathbb{R}P^{k-1}$ and for any $v\sim v'$ it is true that $((id\times q)\circ f_U)(v)=((id\times q)\circ f_U)(v')$, since $f_U$ is linear on each fiber and $q(a)=q(\lambda a)$. So there exists a unique continuous function $g_U:p^{-1}(U)\to U\times\mathbb{R}P^{k-1}$ such that $g_U([v])=((id\times q)\circ f_U)(v)$. The map $g_U$ is a homeomorphism, since its inverse is: $g_U^{-1}(x,[a]):=[f_U^{-1}(x,a)]$. Indeed, this is well defined and continuous, since for $q(a)=q(a')$ we have that $a=\lambda a'$ for some $\lambda\in\mathbb{R}\setminus\{0\}$ and $f_U^{-1}$ is linear over each fiber.

Let us fix some $x\in B$ and a linear isomorphism $f_x:\mathbb{R}^n\to\xi^{-1}(x)$. Then, using the same arguments as for $f_U$, the function $g_x:p^{-1}(x)\to\mathbb{R}P^{k-1}$ with $g_x([v]):=q\circ f_x(v)$ is well defined and continuous. To prove that this is a homeomorphism, we notice that its inverse $g_x^{-1}([a])=[f_x^{-1}(a)]$ is also continuous and well defined.
\end{proof}

\begin{definition} Let $\xi:E(\xi)\to B$ be a $k$-plane vector bundle. Then, the \emph{projective bundle associated with $\xi$} is the fiber bundle $P\xi:E(P\xi)\to B$ with fiber $\mathbb{R}P^{k-1}$, where $E(P\xi)=\sfrac{\{v\in E:v\neq0\}}{\sim}$, $v\sim v'$ if and only if $\xi(v)=\xi(v')$ and $v=\lambda v'$ for some $\lambda\in\mathbb{R}\setminus\{0\}$ and $(P\xi)([v])=\xi(v)$.
\end{definition}

\begin{definition} For any $k$-plane vector bundle $\xi:E(\xi)\to B$, we define the line bundle $\lambda_{\xi}:E(\lambda_\xi)\to E(P\xi)$ as follows. First, notice that we can write $E(P\xi)=\big\{L\subseteq E(\xi):\exists v\in E(\xi)\text{ such that }L=\mathrm{span}(v)\big\}$. Then, we construct the induced vector bundle $(P\xi)^*\xi:E((P\xi)^*\xi)\to E(P\xi)$. Recall that
\[E((P\xi)^*\xi)=\big\{(L,v)\in E(P\xi)\times E(\xi):(P\xi)(L)=\xi(v)\big\}\]
and define $\lambda_{\xi}\leq(P\xi)^*\xi$ with $E(\lambda_{\xi}):=\big\{(L,v)\in E(P\xi)\times E(\xi):(P\xi)(L)=\xi(v)\text{ and }v\in L\big\}\subseteq E((P\xi)^*\xi)$ and $\lambda_{\xi}(L,v)=\xi(v)$.
\begin{center}
\begin{tikzcd}
E(\lambda_{\xi})\ar[dr,"\lambda_{\xi}"']\ar[r,"inc",hook]&E((P\xi)^*\xi)\ar[d,"(P\xi)^*\xi"']\ar[r,"\pi_2"]\ar[dr,phantom,"\lrcorner",near start]&E(\xi)\ar[d,"\xi"]\\[1.7em]
&E(P\xi)\ar[r,"P\xi"]&B
\end{tikzcd}
\end{center}
\end{definition}
\begin{remark}\label{rem:lambda_for_line_bundle} If $\xi:E(\xi)\to B$ is a line bundle, then $\lambda_\xi\cong\xi$.
\end{remark}
\begin{proof} First, notice that for every $x\in B$ and $v_1,v_2\in\xi^{-1}(x)$ with $v_1,v_2\neq 0$ it is true that $[v_1]=[v_2]$. So $P\xi:E(\xi)\to \{*\}\times B$ is a bundle isomorphism and thus $(P\xi)^*\xi\cong(id_B)^*\xi\cong\xi$. Then, since $\lambda_{\xi}$ is also a line bundle with $\lambda_{\xi}\leq\xi$, $\lambda_{\xi}\cong\xi$.
\end{proof}

\begin{lemma}\label{lem:projective_paracompact} Let $B$ be a paracompact space and $\xi:E(\xi)\to B$ a vector bundle. Then $E(P\xi)$ is also paracompact.
\end{lemma}
\begin{proof} Since $B$ is paracompact, Proposition~\ref{lem:paracompact_countable_cover} gives us a locally finite, countable open cover $\mathcal{U}=\{U_i\}_{i\in\mathbb{N}}$ of $B$, such that $\xi|U_i$ is a trivial vector bundle. Let $f_{U_i}:\xi^{-1}(U_i)\to U_i\times\mathbb{R}^k$. Next, we use \ref{thm:paracompact_partition_of_unity} to find a subordinate partition of unity $\mathcal{F}=\{u_i\}_{i\in\mathbb{N}}$ of an open subcover of $\mathcal{U}$. Without loss of generality, we suppose that $\mathcal{F}$ is directly subordinate of $\mathcal{U}$, since it is already locally finite. In particular, if we let $V_i:=\mathrm{supp}(u_i)$, then $\mathcal{V}$ is a closed locally finite countable cover of $B$ such that $\xi|V_i$ is a trivial vector bundle and $\{V_i^{\circ}\}_{i\in\mathbb{N}}$ also covers $B$. This means that $\xi^{-1}(V_i)\cong V_i\times\mathbb{R}^k$. As we proved in Proposition~\ref{prop:proj_bundle}, this gives us a homeomorphism $p^{-1}(V_i)\cong V_i\times\mathbb{R}P^{k-1}$. Proposition~\ref{prop:paracompact_closed_subset} gives us that $V_i$ is paracompact and then, Proposition~\ref{prop:paracompact_times_compact} gives us that $V_i\times\mathbb{R}P^{k-1}\cong p^{-1}(V_i)$ is paracompact. If we write $E(P\xi)=\bigcup_{i\in\mathbb{N}}p^{-1}(V_i)$ and notice that $\{p^{-1}(V_i)^{\circ}\}_{i\in\mathbb{N}}$ also covers $E(P\xi)$, then we get that $E(P\xi)$ is paracompact.%TODO write this better
\end{proof}

\begin{remark} Let $B$ be paracompact, $\xi:E(\xi)\to B$ be any $k$-plane vector bundle and $\lambda_{\xi}:E(\lambda_{\xi})\to E(P\xi)$ the line bundle with $E(\lambda_{\xi})=\big\{(L,v)\in E(P\xi)\times E(\xi):(P\xi)(L)=\xi(v)\text{ and }v\in L\big\}$ as defined above. Since $B$ is paracompact, Lemma~\ref{lem:projective_paracompact} gives us that $E(P\xi)$ is paracompact as well. Then, using Theorem~\ref{thm:universal} we find a map $u_{\xi}:E(P\xi)\to\Gr{1}\cong\mathbb{R}P^{\infty}$ such that $\lambda_{\xi}\cong u_{\xi}^*\gamma^1$.
\begin{center}
\begin{tikzcd}
E(\lambda_{\xi})\ar[d,"\lambda_{\xi}\cong u_{\xi}^*\gamma^1"']\ar[r,"\bar{f}_{\xi}"]\ar[dr,phantom,"\lrcorner",near start]&E(\gamma^1)\ar[d,"\gamma^1"]\\[1.7em]
E(P\xi)\ar[r,"u_{\xi}"]&\mathbb{R}P^{\infty}
\end{tikzcd}
\end{center}
Define now $a_{\xi}:=u_{\xi}^*(z)\in H^*(E(P\xi);\mathbb{Z}_2)$, where $z$ is the generator of $H^*(\mathbb{RP}^{\infty};\mathbb{Z}_2)$, as defined in Theorem~\ref{thm:projective_spaces_cohomology}. Then Theorem~\ref{thm:universal_uniqueness} gives us that $u_{\xi}$ is unique, up to homotopy and thus $a_{\xi}\in H^*(E(P\xi);\mathbb{Z}_2)$ is well defined.
\end{remark}

\begin{proposition}\label{prop:basis_of_EP} Let $B$ be paracompact and $\xi:E(\xi)\to B$ any $k$-plane vector bundle and also let $a_{\xi}:=u_{\xi}^*z\in H^*(E(P\xi);\mathbb{Z}_2)$ as defined in the previous remark. Then the elements $1,a_{\xi},a_{\xi}^2,\ldots,a_{\xi}^{k-1}$ are a basis of the $H^*(B;\mathbb{Z}_2)$-module $H^*(E(P\xi);\mathbb{Z}_2)$, where the $H^*(B;\mathbb{Z}_2)$-module structure on $H^*(E(P\xi);\mathbb{Z}_2)$ is defined by the function $(P\xi)^*$.
\end{proposition}
\begin{proof} The strategy is to use Leray-Hirsch Theorem (\ref{thm:leray_hirsch}) for the fiber bundle $P\xi:E(P\xi)\to B$, which has fiber $F=\mathbb{R}P^{k-1}$. Since $B$ is paracompact, so is $E(P\xi)$ as well, as we saw in Lemma~\ref{lem:projective_paracompact}. So, we need to construct for every $x\in B$ a homeomorphism $j_x:\mathbb{R}P^{k-1}\to(P\xi)^{-1}(x)$, such that $\{j_x^*(1),j_x^*(a_{\xi}),\ldots,j_x^*(a_{\xi}^{k-1})\}$ is a $\mathbb{Z}_2$-basis of $H^*(\mathbb{R}^{k-1};\mathbb{Z}_2)$.

Let $x\in B$. We fix a linear isomorphism $f_x:\xi^{-1}(x)\to\mathbb{R}^k$ and then we define the homeomorphism $j_x([a]):=g_x^{-1}([a])=[f_x^{-1}(a)]$, as in Proposition~\ref{prop:proj_bundle}. We examine now the line bundle $j_x^*u_{\xi}^*\gamma^1=j_x^*\lambda_{\xi}:E(j_x^*\lambda_{\xi})\to\mathbb{R}P^{k-1}$. Notice that $j_x(\mathbb{R}P^{k-1})=(P\xi)^{-1}(x)\subseteq E(P\xi)$ and thus $j_x^*\lambda_{\xi}=j_x^*(inc^*(\lambda_{\xi}))$:
\begin{center}
\begin{tikzcd}
E(j_x^*\lambda_{\xi})\ar[d,"j_x^*\lambda_{\xi}"']\ar[r,"\bar{j}_x"]\ar[dr,phantom,"\lrcorner",near start]&E(inc^*\lambda_{\xi})\ar[d,"inc^*\lambda_{\xi}"']\ar[r,"inc",hook]\ar[dr,phantom,"\lrcorner",near start]&E(\lambda_{\xi})\ar[d,"\lambda_{\xi}"']\ar[r,"\bar{u}_{
\xi}"]\ar[dr,phantom,"\lrcorner",near start]&E(\gamma^1)\ar[d,"\gamma^1"]\\[1.7em]
\mathbb{R}P^{k-1}\ar[r,"j_x"]&(P\xi)^{-1}(x)\ar[r,"inc",hook]&E(P\xi)\ar[r,"u_{\xi}"]&\mathbb{R}P^{\infty}
\end{tikzcd}
\end{center}
Notice that we can write $E(inc^*\lambda_{\xi})=\big\{(L,v)\in (P\xi)^{-1}\times\xi^{-1}:v\in L\big\}$ and thus $E(j^*\lambda_{\xi})=\big\{([a],L,v)\in\mathbb{R}P^{k-1}\times(P\xi)^{-1}(x)\times\xi^{-1}(x):v\in L\text{ and }j_x([a])=L\big\}\cong\big\{([a],v)\in\mathbb{R}P^{k-1}\times\xi^{-1}(x):v\in j_x([a])\big\}\cong\big\{([a],b)\in\mathbb{R}P^{k-1}\times\mathbb{R}^k:f_x^{-1}(b)\in j_x([a])\big\}$. But $f_x^{-1}(b)\in j_x([a])$ if and only if $[f_x^{-1}(b)]=j_x([a])$ and since $j_x$ is induced by $f_x$, this is equivallent to $j_x([b])=j_x([a])$, i.e. $[b]=[a]$ since $j_x$ is a homemorphism which is equivalent to $b\in[a]$. This proves that $j_x^*\lambda_{\xi}\cong\gamma_k^1$, i.e. $(u_{\xi}\circ j_x)^*(\gamma^1)\cong\gamma_k^1=\iota_{1,k}^*(\gamma^1)$, because of Proposition~\ref{prop:restriction}. So, using Theorem~\ref{thm:universal_uniqueness} $u_{\xi}\circ j_x\simeq\iota_{1,k}$ and thus $j_x^*u_{\xi}^*=\iota_{1,k}^*:H^*(\mathbb{R}P^{\infty};\mathbb{Z}_2)\to H^*(\mathbb{R}P^{k-1};\mathbb{Z}_2)$. Thus, due to the definition of $a_{\xi}$ and Theorem~\ref{thm:projective_spaces_cohomology}, we get $j_x^*(a_{\xi})=j_x^*(u_{\xi}^*(z))=\iota_{1,k}(z)=z_{k-1}\in H^1(\mathbb{R}P^{k-1};\mathbb{Z}_2)$ and hence $j_x^*(a_{\xi}^i)=j_x^*(a_{\xi})^i=z_{k-1}^i$. This means that $\{j_x^*(1),j_x^*(a_{\xi}),j_x^*(a_{\xi}^2),\ldots,j_x^*(a_{\xi}^{k-1})\}=\{1,z_{k-1},z_{k-1}^2,\ldots,z_{k-1}^{k-1}\}$ is a $\mathbb{Z}_2$-linear independent set that generates the $\mathbb{Z}_2$-module $H^*(\mathbb{R}P^{k-1})$.

Leray-Hirsch theorem gives us now that the set $\{1,a_{\xi},\ldots,a_{\xi}^{k-1}\}$ is a $H^*(B;\mathbb{Z}_2)$-linearly independent set in $H^*(E(P\xi);\mathbb{Z}_2)$ and that
\[H^*(B;\mathbb{Z}_2)\left<1,a_{\xi},a_{\xi}^2,\ldots,a_{\xi}^{k-1}\right>\cong H^*(E(P\xi);\mathbb{Z}_2)\]
as $H^*(B;\mathbb{Z}_2)$-modules, where the isomorphism is given by:
\[\sum_{i=0}^{k-1}x_i\cdot a_{\xi}^i\ \longmapsto\ \sum_{i=0}^{k-1}(P\xi)^*(x_i)a_{\xi}^i\qedhere\]
\end{proof}

\begin{definition}\label{def:SW} Let $B$ be paracompact and $\xi:E(\xi)\to B$ any $k$-plane vector bundle and also let $a_{\xi}:=u_{\xi}^*z\in H^*(E(P\xi);\mathbb{Z}_2)$ as before. Then, because of Proposition~\ref{prop:basis_of_EP}, for each $i\in[k]$ there exists a $w_i(\xi)\in H^i(B;\mathbb{Z}_2)$, such that
\[a_{\xi}^k=\sum_{i=0}^{k-1}\Big((P\xi)^*\big(w_{k-i}(\xi)\big)\Big)a_{\xi}^i\]
Then, $w_i(\xi)$ is called \emph{the $i$-th Stiefel-Whitney class of $\xi$} and $w(\xi):=1+w_1(\xi)+\ldots+w_k(\xi)$ is called the \emph{total Stiefel-Whitney class of $\xi$}.
\end{definition}

\subsection{Uniqueness of Stiefel Whitney classes}
In this subsection we prove that if some $w(\xi)$ satisfies the the Stiefel-Whitney axioms, then this is unique.



\subsection{Existence of Stiefel Whitney classes}
In this subsection we prove that the Stiefel-Whitney classes defined previously satisfy the Stiefel-Whitney axioms.
\begin{proposition} Let $B$ be paracompact and $\xi:E(\xi)\to B$ be any $k$-plane vector bundle. Then, $w_0(\xi)=1$ and $w_{k+1}(\xi)=w_{k+2}(\xi)=\cdots=0$.
\end{proposition}
\begin{proof} This is immediate from the definition.
\end{proof}

\begin{proposition} Let $B',B$ be two paracompact spaces, $f:B'\to B$ be any continuous function and $\xi:E(\xi)\to B$ be any $k$-plane vector bundle. Then, $w(f^*\xi)=f^*w(\xi)$.
\end{proposition}
\begin{proof} Recall that $E(f^*\xi)=\big\{(x,v)\in B'\times E(\xi):f(x)=\xi(v)\big\}$ and define $\bar{f}:E(f^*\xi)\to E(\xi)$ with $\bar{f}(x,v):=v$, completing the pullback square. Then, define $g:E(P(f^*\xi))\to E(P\xi)$ with $g([(x,v)]):=[\bar{f}(x,v)]=[v]$. This is continuous and well defined. Indeed, let $(x_1,v_1),(x_2,v_2)\in E(f^*\xi)$ with $v_1,v_2\neq 0$ and $[(x_1,v_1)]=[(x_2,v_2)]$ in $E(P(f^*\xi))$. Then $x_1=x_2$, i.e. $\xi(v_1)=f(x_1)=f(x_2)=\xi(v_2)$ and $v_1=\lambda v_2$ for some $\lambda\in\mathbb{R}\setminus\{0\}$. Thus, $[v_1]=[v_2]$ in $E(Pf)$. So, we have the following two commutative diagrams:
\begin{center}
\begin{tikzcd}
&[-7em]&[-3em]&[-10em]\faktor{\big\{(x,v)\in B'\times E(\xi):f(b)=\xi(v)\text{ and }v\neq0\big\}}{\left\{\substack{(x_1,v_1)\sim(x_2,v_2)\text{ iff }\\[0.2em]x_1=x_2\text{ and }v_1=\lambda v_2}\right\}}\ar[dd,phantom,"\cong"{yshift=1em,rotate=90}]&[-10em]\\[-3.2em]
&\big\{(x,v)\in B'\times E(\xi):f(b)=\xi(v)\big\}\ar[d,phantom,"\cong"rotate=90]&\\[-1.2em]
&E(f^*\xi)\ar[r,"\bar{f}"]\ar[d,"f^*\xi"']&E(\xi)\ar[d,"\xi"]&[4em]E(P(f^*\xi))\ar[r,"g"]\ar[d,"P(f^*\xi)"']&E(P\xi)\ar[d,"P\xi"]\\[1.3em]
&B'\ar[r,"f"]&B&B'\ar[r,"f"]&B
\end{tikzcd}
\end{center}
Next, we will prove that $\lambda_{f^*\xi}\cong g^*\lambda_{\xi}$, i.e. we will define a map $\phi:E(\lambda_{f^*\xi})\to E(g^*\lambda_{\xi})$, where
\begin{align*}
E(\lambda_{f^*\xi})&=\big\{(L',v')\in E(P(f^*\xi))\times E(f^*\xi):(P(f^*\xi))(L')=f^*\xi(v')\text{ and }v'\in L'\big\},\\
E(g^*\lambda_{\xi})&=\big\{(L',L,v)\in E(P(f^*\xi))\times E(P\xi)\times E(\xi):(P\xi)(L)=\xi(v)\text{ and }v\in L\text{ and }g(L')=L\big\}\\
&\cong\big\{(L',v)\in E(P(f^*\xi))\times E(\xi):(P\xi)(g(L'))=\xi(v)\text{ and }v\in g(L')\big\}.
\end{align*}
Let $\phi(L',v'):=(L',\bar{f}(v'))$. Then, $\phi$ is well defined. Indeed, for $(L',v')\in E(\lambda_{f^*\xi})$ we first have $(P\xi)(g(L'))=f(P(f^*\xi(L')))=f(f^*\xi(v'))=\xi(\bar{f}(v'))$ and if we write $v'=(x,a)$ and $L'=[(y,b)]$, then we also have that $v'\in L'$ means $[(x,a)]=[(y,b)]$ which gives that $[a]=[b]$. It is clear that $\phi$ is a fiber bundle map, which is a linear isomorphism fiber-wise, since $\bar{f}$ is a linear isomorphism fiber-wise. So, the following diagram is commutative and both squares are pullback squares:
\begin{center}
\begin{tikzcd}
E(\lambda_{f^*\xi})\ar[r]\ar[d,"\lambda_{f^*\xi}=g^*\lambda_{\xi}"']\ar[dr,phantom,"\lrcorner",near start]&[2em]E(\lambda_{\xi})\ar[r]\ar[d,"\lambda_{\xi}"']\ar[dr,phantom,"\lrcorner",near start]&[2em]E(\gamma^1)\ar[d,"\gamma^1"]\\[1.3em]
E(P(f^*\xi))\ar[r,"g"]\ar[rr,dotted,bend right=10,"u_{f^*\xi}"']&E(P\xi)\ar[r,"u_{\xi}"]&\mathbb{R}P^{\infty}
\end{tikzcd}
\end{center}
So, it is true that $\lambda_{f^*\xi}=g^*\lambda_{\xi}=g^*u_{\xi}^*\gamma^1=(u_{\xi}\circ g)^*\gamma^1$, but we also have that $\lambda_{f^*\xi}=u^*_{f^*\xi}\gamma^1$, so Theorem~\ref{thm:universal_uniqueness} gives us that $u_{\xi}\circ g\simeq u_{f^*\xi}$, which means that $g^*u^*_{\xi}=u^*_{f^*\xi}:H^*(\mathbb{R}P;\mathbb{Z}_2)\to H^*(E(P(f^*\xi));\mathbb{Z}_2)$. This gives: $a_{f^*\xi}=u_{f^*\xi}z=g^*u^*_{\xi}z=g^*a_{\xi}$ and thus:
\[g^*a^k_{\xi}=a^k_{f^*\xi}=\sum_{i=0}^{k-1}\Big((P(f^*\xi))^*w_{k-i}(f^*\xi)\Big)a_{f^*\xi}^i=\sum_{i=0}^{k-1}\Big((P(f^*\xi))^*w_{k-i}(f^*\xi)\Big)g^*a_{\xi}^i\]
On the other hand:
\[g^*a^k_{\xi}=g^*\left(\sum_{i=0}^{k-1}\Big((P\xi)^*w_{k-i}(\xi)\Big)a_{\xi}^i\right)=\sum_{i=0}^{k-1}\Big(g^*(P\xi)^*w_{k-i}(\xi)\Big)g^*a_{\xi}^i=\sum_{i=0}^{k-1}\Big((P(f^*\xi))^*f^*w_{k-i}(\xi)\Big)g^*a_{\xi}^i\]
which gives $w_i(f^*\xi)=f^*w_i(\xi)$ for all $i\in[k]$.
\end{proof}

\begin{proposition} Let $B$ be a paracompact space, $\eta:E(\eta)\to B$ be any $k_1$-plane vector bundle and $\xi:E(\xi)\to B$ be any $k_2$-plane vector bundle. Then, $w(\eta\oplus\xi)=w(\eta)w(\xi)$.
\end{proposition}

\begin{proposition} For the tautological line bundle $\gamma_1^1:\big\{(L,v)\in\mathbb{R}P^1\times\mathbb{R}^2:v\in L\big\}\to$ with $\gamma_1^1(L,v)=L$ it is true that $w(\gamma_1^1)\neq0$.
\end{proposition}
\begin{proof} First, using Remark~\ref{rem:lambda_for_line_bundle}, we get that the map $(P\gamma_1^1)^*:H^*(\mathbb{R}P^1;\mathbb{Z}_2)\to H^*(E(P\gamma_1^1);\mathbb{Z}_2)$ is an isomorphism and that $\lambda_{\gamma_1^1}\cong\gamma_1^1$. Then, using Proposition~\ref{prop:restriction}, we are able to express $\lambda_{\gamma_1^1}$ as a pullback: $\gamma_1^1=\iota_{1,2}^*\gamma^1$. By the uniqueness Thereom~\ref{thm:universal_uniqueness}, $u_{\gamma_1^1}\sim\iota_{1,2}$ and thus $u_{\gamma_1^1}^*=\iota_{1,2}^*:H^*(\mathbb{R}P^{\infty};\mathbb{Z}_2)\to H^*(\mathbb{R}P^1;\mathbb{Z}_2)$. So, using Theorem~\ref{thm:projective_spaces_cohomology}, we can explicitly compute $a_{\gamma_1^1}=u_{\gamma_1^1}z=\iota_{1,2}z=z_1$, which gives us $z_1=a_{\gamma_1^1}=\sum_{i=0}^0(P\gamma_1^1)^*(w_{1-i}(\gamma_1^1))a_{\gamma_1^1}^k=(P\gamma_1^1)^*w_1(\gamma_1^1)$, i.e. $w_1(\gamma_1^1)=z_1\in\mathbb{Z}_2[z_1]\cong H^*(E(P\gamma_1^1);\mathbb{Z}_2)$. In particular, $w_1(\gamma_1^1)\neq0$.
\end{proof}


\section{Computation of the Cohomology}
%* elementary schubert calculus?


\begin{definition} Let $k\in\mathbb{N}$. Define then the following graded commutative $\mathbb{Z}_2$-algebra:
\[A_k=\mathbb{Z}_2[w_1,\ldots,w_k]\]
with $\deg(w_i)=i$ for every $i\in[k]$.
\end{definition}
\begin{theorem} For any $k\in\mathbb{N}$, there exists a graded algebra isomorphism:
\[\phi_k:A_k\to H^*(\Gr{k};\mathbb{Z}_2)\]
with $\phi_k(w_i)=w_i(\gamma^k)$.
\end{theorem}
\begin{proof} TODO - Chapter 7 in \cite{char_class}.
\end{proof}
\begin{definition} Let $0<k<n$ be some natural numbers. Then, define the following graded commutative $\mathbb{Z}_2$-algebra:
\[A_{k,n}=\sfrac{\mathbb{Z}_2[w_1,\ldots,w_k,\bar{w}_1,\ldots,\bar{w}_{n-k}]}{I_{k,n}}\]
where $\deg(w_i)=i$ for every $i\in[k]$, $\deg(\bar{w}_j)=j$ for every $j\in[n-k]$ and
\[I_{k,n}=\big((1+w_1+\cdots+w_k)(1+\bar{w}_1+\cdots+\bar{w}_{n-k})+1\big)\]
\end{definition}
\begin{proposition} We can also think of $A_{k,n}$ as:
\[A_{k,n}=\sfrac{\mathbb{Z}_2[w_1,w_2,\ldots,w_k,\bar{w}_1,\bar{w}_2,\ldots]}{I_k+(\bar{w}_{n-k+1})+(\bar{w}_{n-k+2})+\cdots+(\bar{w}_{n})}\]
where
\[I_k=\big((1+w_1+w_2+\cdots+w_k)(1+\bar{w}_1+\bar{w}_2+\cdots)+1\]
\end{proposition}
\begin{proof} Notice that for fixed $k$, for any $j\in[n-k]$, the graded ideal $I_{k,n}$ gives:
\[\bar{w}_j+\bar{w}_{j-1}w_1+\cdots+\bar{w}_{j-k}w_k=0\]
i.e. if $k$ consecutive $\bar{w}_j$ are zero, then all $\bar{w}_j$ are zero from that point on.
\end{proof}
\begin{theorem} For any natural numbers $0<k<n$, there exists a graded algebra isomorphism:
\[\phi_{k,n}:A_{k,n}\to H^*(\Gr{k}{n};\mathbb{Z}_2)\]
with $\phi_{k,n}(w_i)=w_i(\gamma^k_n)$ and $\phi_{k,n}(\bar{w}_j)=w_j({\gamma^k_n}^{\perp})$.
\end{theorem}
\begin{proof}\begin{b_item}
\item $\phi_{k,n}$ is well defined, since the Stiefel-Whitney classes satisfy $I_{k,n}$.
\item $\phi_{k,n}$ is surjective. TODO Indeed, first of all for the Manifold $\Gr{k}{n}$, we have
\[|C_i|=|H_i|=|C^i|=|H^i|\]
for every $i\in[k(n-k)]$. Moreover, one can prove that the inclusion map
\[i_{k,n}:\Gr{k}{n}\to\Gr{k}\] is a cell map, taking a shubert cell to one with the same symbol. This makes the induced map on the cohomologies surjective:
\[i^*_{k,n}:H^*(\Gr{k})\to H^*(\Gr{k}{n})\] This inclusion map induces a bundle map $\gamma^k_n\to\gamma^k$ and because of naturality of the SW classes, we get $i^*(w_i)=w_i$, which means that $i^*_{k,n}$ factors through $\phi_{k,n}$.
\item $\phi_{k,n}$ is injective. For this, it suffices to show that
\[\dim_{\mathbb{Z}_2}(A_{k,n}^i)=\dim_{\mathbb{Z}_2}(H^i(\Gr{k}{n}))\] for every $i\geq0$, which will be proven in the next lemma.
\end{b_item}
\end{proof}
\begin{lemma} The Hilbert series of the graded algebras $A_{k,n}$ and $H^*(\Gr{k}{n})$ are equal.
\end{lemma}
\begin{proof} Let
\begin{align*}
p_{k,n}^i:=&\dim_{\mathbb{Z}_2}(H^i(\Gr{k}{n}))=|C_i|\\
=&\#\{\text{Young tableau of at most }k\text{ rows and at most }n-k\text{ columns.}\}
\end{align*}
Which makes its Hilbert series:
\[P_{k,n}(t):=\sum_{i\geq0}p_{k,n}^it^i=\binom{n}{k}_t\]
as proven in the Lemma \ref{lem:hilbert_series_of_young}. On the other hand, let
\[H_{k,n}(t)=\sum_{i\geq0}\dim_{\mathbb{Z}_2}(A_{k,n}^i)t^i\]
It suffices to prove that $H_{k,n}(t)$ satisfies the same recursive conditions as $P_{k,n}(t)$. First of all, we easily get $H_{1,n}(t)=P_{1,n}(t)$ and then, using the additivity of the following SES:
\begin{center}
\begin{tikzcd}
0\ar[r]& A_{k,n-1}^{[-k]}\ar[r,"\cdot w_k", "f"']&A_{k,n}\ar[r,"w_k\mapsto0","g"']&A_{k-1,n-1}\ar[r]&0
\end{tikzcd}
\end{center}
we get:
\[H_{k,n}(t)=t^kH_{k,n-1}(t)+H_{k-1,n-1}(t)\]
which is the same recursion that defines $P_{k,n}(t)$. The fact that this is indeed a SES is proven in the following lemma:
\end{proof}
\begin{lemma} The following is a SES of graded commutative $\mathbb{Z}_2$-algebras:
\begin{center}
\begin{tikzcd}
0\ar[r]& A_{k,n-1}^{[-k]}\ar[r,"\cdot w_k", "f"']&A_{k,n}\ar[r,"w_k\mapsto0","g"']&A_{k-1,n-1}\ar[r]&0
\end{tikzcd}
\end{center}
where ${A_{k,n}^{[-k]}}^i:=A_{k,n}^{i-k}$ is the proper shift to make $f$ a homomorphism of graded algebras.
\end{lemma}
\begin{proof}
Let
\[R_{k,n}:=\sfrac{\mathbb{Z}_2[w_1,w_2,\ldots,w_k,\bar{w}_1,\bar{w}_2,\ldots]}{I_k+(\bar{w}_{n-k+1})+(\bar{w}_{n-k+2})+\cdots+(\bar{w}_{n-1})}\]
then, the algebras can be written as:
\begin{center}
\begin{tikzcd}
0\ar[r]& \sfrac{R_{k,n}}{(\bar{w}_{n-k})}^{[-k]}\ar[r,"\cdot w_k", "f"']&\sfrac{R_{k,n}}{(\bar{w}_{n})}\ar[r,"w_k\mapsto0","g"']&\sfrac{R_{k,n}}{(w_k)}\ar[r]&0
\end{tikzcd}
\end{center}
\begin{b_item}
\item $f,g$ are well defined. Indeed, notice that
\[\bar{w}_{n}=\bar{w}_{n-1}w_1+\bar{w}_{n-2}w_2+\cdots+\bar{w}_{n-k+1}w_{k-1}+\bar{w}_{n-k}w_k=\bar{w}_{n-k}w_k,\text{ inside }R_{k,n}\]
\item $g$ is surjective and $\im(f)\subseteq\ker(g)$ are easy to prove.
\item $f$ is injective and $\im(f)\supseteq\ker(g)$ -- TODO!!
\end{b_item}
\end{proof}


\begin{definition} We write:
\[\binom{n}{k}_t:=\frac{[n]_t!}{[k]_t!\cdot[n-k]_t!}\]
where:
\[[x]_t:=\sum_{i=0}^{x-1}t^i=\frac{t^x-1}{t-1}\]
and:
\[[x]_t!=[x]_t\cdot[x-1]_t\cdots[1]_t\]
\end{definition}

	\begin{myappendix}
%	\chapter{Bits and Pieces in Topology}
\section{Some Categorical Notions}
The goal of this small section is to motivate the definition of the categorical exponential object. In order to arrive there, we first go through the Yoneda lemma and then through the notion of the adjoint functors. For a more thorough introduction to the topics discussed here, refer to \cite{basic_cat}, or any other introductory category book.

We will use the notation $\mathrm{ob}(\mathcal{C})$ for the objects of $\mathcal{C}$ and $\mathcal{C}(A,B)$ for the morphisms from $A$ to $B$ in $\mathcal{C}$. Moreover, the following few statements will be about locally small categories:
\begin{definition} A category $\mathcal{C}$ is called \emph{locally small}, if for every two objects $A,B$ in $\mathcal{C}$, the morphisms $\mathcal{C}(A,B)$ form a set.
\end{definition}

\subsection{Yoneda Lemma}
\begin{definition} Let $\mathcal{C}$ be any locally small category. Then, we denote by $\mathrm{Set}^{\mathcal{C}}$ the \emph{category of set valued functors of $\mathcal{C}$}, which is defined as follows:
\begin{itemize}
\item $\mathrm{ob}\left(\mathrm{Set}^{\mathcal{C}}\right)$ is the class of all functors $F$ from $\mathcal{C}$ to $\mathrm{Set}$.
\item $\mathrm{Set}^{\mathcal{C}}(F,G)$ is the class of all natural transformations $F\overset{\eta}{\Rightarrow}G$.
\item For $G\overset{\eta}{\Rightarrow}H$ and $F\overset{\theta}{\Rightarrow}G$, the composition $F\overset{\eta\circ\theta}{\Rightarrow}H$ is the usual composition of natural transformations.
\end{itemize}
\end{definition}
\begin{remark} In the above definition, $\mathrm{Set}^{\mathcal{C}}$ need not be locally small. Indeed, let $\mathcal{C}$ be any large discrete category. Let for example, $\mathrm{ob}(\mathcal{C})$ be the same as $\mathrm{ob}(\mathrm{Set})$ and let
\[\mathcal{C}(A,B)=\left\{\begin{array}{ll}\left\{1_A\right\}&,A=B\\\emptyset&,A\neq B\end{array}\right.\]
This is obviously a locally small category. Moreover, define the functors $F,G$:
\begin{center}
\begin{tikzcd}
\mathcal{C}\ar[r,bend left=25, "F"{name=F}]\ar[r,bend right=25, "G"'{name=G}]&[2em]\mathrm{Set}
\end{tikzcd}
\end{center}
both to be the identity on $\mathrm{ob}(\mathcal{C})$. The only morphisms in $\mathcal{C}$ are the identity morphisms, for which there is no choice for $F,G$, since they are functors. They have to satisfy $F1_A=G1_A=1_A$ for every set $A$.

A natural transformation $\eta:F\Rightarrow G$ is a collection of choices $\eta_A:F(A)\to G(A)$, such that $(Ge)\circ\eta_A=\eta_B\circ(Fe)$ for every morphism $e\in\mathcal{C}(A,B)$. The only such morphisms are the identity arrows, so there aren't any restrictions on the choices when constructing $\eta$. This means that $\mathrm{Set}^{\mathcal{C}}(F,G)$ is the class of all $\eta$, each one determined by a collection of choices $\eta_A\in\mathrm{Set}(A,A)$ over all sets $A$. Since the class of all sets is a proper class, so is the class of all such collection of choices, which makes $\mathrm{Set}^{\mathcal{C}}$ not locally small.
\end{remark}

We start with some definitions and results from Chapter~4 in \cite{basic_cat}.
\begin{definition}[4.1.16] Let $\mathcal{C}$ be a locally small category. Moreover, let $X$ be any object of $\mathcal{C}$. We define the functor
\[H_X:\mathcal{C}^{\mathrm{op}}\to\mathrm{Set}\]
as follows:
\begin{itemize}
\item For any object $A$ of $\mathcal{C}^{\mathrm{op}}$, we define $H_X(A):=\mathcal{C}(A,X)$.
\item For any morphism $g^{\mathrm{op}}\in\mathcal{C}^{\mathrm{op}}(A,B)$, i.e. any morphism $g\in\mathcal{C}(B,A)$, we define $H_X(g):=g^*\in\mathrm{Set}(H_X(A),H_X(B))$ taking any $p\in\mathcal{C}(A,X)$ to $p\circ g\in\mathcal{C}(B,X)$, like in the following diagram:
\begin{center}
\begin{tikzcd}
A\ar[d,"g^{\mathrm{op}}"']&&A&&H_X(A)\ar[d,"H_X(g)"']\ar[r,phantom,":="]&[-1.5em]\mathcal{C}(A,X)\ar[r,phantom,"\ni"]&[-1.8em]p\ar[d,mapsto,"g^*"]\\
B&&B\ar[u,"g"]&&H_X(B)\ar[r,phantom,":="]&\mathcal{C}(B,X)\ar[r,phantom,"\ni"]&p\circ g
\end{tikzcd}
\end{center}
\end{itemize}
\end{definition}

\begin{definition}[4.1.21]\label{def:functor} Let $\mathcal{C}$ be a locally small category. We define the functor
\[H_{\bullet}:\mathcal{C}\to\mathrm{Set}^{\mathcal{C}^{\mathrm{op}}}\]
as follows:
\begin{itemize}
\item For any object $X$ of $\mathcal{C}$, we define $H_{\bullet}(X):=H_X$.
\item For any morphism $f\in\mathcal{C}(X,Y)$, we define $H_{\bullet}(f)=H_f\in\mathrm{Set}^{\mathcal{C}^{\mathrm{op}}}(H_X,H_Y)$ to be the natural transformation with components $(H_f)_A:=f_*\in\mathrm{Set}(H_X(A),H_Y(A))$ taking any $p\in\mathcal{C}(A,X)$ to $f\circ p\in\mathcal{C}(A,Y)$, like in the following diagram:
\begin{center}
\begin{tikzcd}
X\ar[dd,"f"']&[-1em]&[4em]&[-1em]H_X(A)\ar[dd,"(H_f)_A"]\ar[r,phantom,"\ni"]&[-1.8em]p\ar[dd,mapsto,"f_*"]\\[-1em]
&\mathcal{C}^{\mathrm{op}}\ar[r,bend left=30,"H_X"{name=HX}]\ar[r,bend right=30,"H_Y"'{name=HY}]&\mathrm{Set}\ar[from=HX, to=HY, Rightarrow, "H_f"]\\[-1em]
Y&&&H_Y(A)\ar[r,phantom,"\ni"]&f\circ p
\end{tikzcd}
\end{center}
\end{itemize}
\end{definition}
\begin{remark} We have to prove that in the above definition, for every $f\in\mathcal{C}(X,Y)$, $H_f$ is indeed a natural transformation, i.e. that for every $g^{\mathrm{op}}\in\mathcal{C}^{\mathrm{op}}(A,B)$, the following diagram commutes:
\begin{center}
\begin{tikzcd}
A&&H_X(A)\ar[d,"H_X(g)"']\ar[r,"(H_f)_A"]&H_Y(A)\ar[d,"H_Y(g)"]\\
B\ar[u,"g"]&&H_X(B)\ar[r,"(H_f)_B"]&H_Y(B)
\end{tikzcd}
\end{center}
which does, since both directions take $p\in\mathcal{C}(A,X)$ to $f\circ p\circ g\in\mathcal{C}(B,Y)$.
\end{remark}

\begin{theorem}[Yoneda, 4.2.1] Let $\mathcal{C}$ be a locally small category. Moreover, let $X$ be any object of $\mathcal{C}$ and $F:\mathcal{C}^{\mathrm{op}}\to\mathrm{Set}$ be any functor. Then:
\[\mathrm{Set}^{\mathcal{C}^{\mathrm{op}}}(H_X,F)\cong F(X)\]
naturally in $(X,F)$ in $\mathcal{C}^{op}\times\mathrm{Set}^{\mathcal{C}^{\mathrm{op}}}$.
\end{theorem}
\begin{proof}[Sketch of the Proof] To remove some clutter in the notation, we just write $SP$ instead of $S(P)$ and $Su$ instead of $S(u)$ for every functor $S$, object $P$ and morphism $u$.

First of all, the ``$\cong$'' in the theorem is inside the category $\mathrm{Set}$, so it is a bijection. It being natural in $(X,F)$ means that we need to define a bijection
\[\psi_{X,F}:\mathrm{Set}^{\mathcal{C}^{\mathrm{op}}}(H_X,F)\to FX\]
and prove that it is a natural transformation between the following two functors:
\begin{center}
\begin{tikzcd}
\mathcal{C}^{op}\times\mathrm{Set}^{\mathcal{C}^{\mathrm{op}}}\ar[r,shift left,"(X{,}F)\mapsto\mathrm{Set}^{\mathcal{C}^{\mathrm{op}}}(H_X{,}F)"]\ar[r,shift right, "(X{,}F)\mapsto FX"']&[10em]\mathrm{Set}
\end{tikzcd}
\end{center}
In order to prove that this is a bijection, it suffices to define a function
\[\phi_{X,F}:FX\to\mathrm{Set}^{\mathcal{C}^{\mathrm{op}}}(H_X,F)\]
and prove that it is the inverse of $\psi$.

So, the proof is going to have four steps: First we are going to define $\psi_{X,F}$, then we are going to define $\phi_{X,F}$, then we are going to prove that these are inverses and finally we are going to prove that they are natural:
\begin{itemize}
\item There is the following ``natural'' choice for $\psi_{X,F}$. Let $\eta$ be any natural transformation $H_X\Rightarrow F$. Then define:
\[\psi_{X,F}(\eta):=\eta_X(1_X)\in FX\]
Let us unpack this: $\eta$ being a natural transformation means that it has components $\eta_A:H_XA\to FA$ for every $A$ in $\mathcal{C}$, but $H_XA$ is just $\mathcal{C}(A,X)$. Choosing $A=X$, we also get an obvious element $1_X\in\mathcal{C}(X,X)$, whose image under $\eta_X$ lies in the desired set $FX$.

\item There exists also a ``natural'' choice for $\phi_{X,F}$. Let $x\in FX$ be any element of the set $FX$. We have to define a natural transformation $\phi_{X,F}(x)=\theta^x$ from $H_X$ to $F$. Let us first define each component $\theta^x_A:H_XA\to FA$ as follows:
\[\theta^x_A(p)=(Fp)(x)\in FA\]
Let us also unpack this definition as well: We want to define $\theta^x_A$ on every element $p\in\mathcal{C}(A,X)$. We already have some element $x\in FX$ and for every such $p$, we can create $Fp\in\mathrm{Set}(FX,FA)$, since $F$ is a functor $\mathcal{C}^{\mathrm{op}}\to\mathrm{Set}$. So, the image of $x$ under $Fp$ lies in the desired set $FA$.

The definition of $\phi$ is not over yet, since we have to prove that $\theta^x$ is indeed natural in $A$, i.e. that for every morphism $f^{\mathrm{op}}\in\mathcal{C}(A,B)$ the following diagram commutes:
\begin{center}
\begin{tikzcd}
A&&H_XA\ar[d,"H_Xf"']\ar[r,"\theta^x_A"]&FA\ar[d,"Ff"]\\
B\ar[u,"f"]&&H_XB\ar[r,"\theta^x_B"]&FB
\end{tikzcd}
\end{center}
which does, since both directions take $q$ to $(F(q\circ f))(x)$.

\item The next step is to prove that $\psi_{X,F}$ and $\phi_{X,F}$ are inverse functions in the category of sets. The one direction is easy. Let $x\in FX$. Then:
\[\psi_{X,F}\big(\phi_{X,F}(x)\big)=\psi_{X,F}(\theta^x)=\theta^x_X(1_X)=(F1_X)(x)=1_{FX}(x)=x\]

For the other direction, let $\eta$ be any natural transformation $H_X\to F$. Then:
\[\phi_{X,F}\big(\psi_{X,F}(\eta)\big)=\phi_{X,F}(\eta_X(1_X))=\theta^{\eta_X(1_X)}\]
In order to show that this is equal to $\eta$, we need to check the equality in every component. Let $A$ be any object of $\mathcal{C}$ and $p\in\mathcal{C}(A,X)$ any function. Then:
\[\theta^{\eta_X(1_X)}_A(p)=(Fp)\big(\eta_X(1_X)\big)\overset{(*)}{=}\eta_A\big((H_Xp)(1_X)\big)=\eta_A(1_X\circ p)=\eta_A(p)\]
where the $(*)$ holds because of the naturality of $\eta$, i.e. because this diagram commutes:
\begin{center}
\begin{tikzcd}
X&&H_XX\ar[r,"\eta_X"]\ar[d,"H_Xp"']&FX\ar[d,"Fp"]\\
A\ar[u,"p"]&&H_XA\ar[r,"\eta_A"]&FA
\end{tikzcd}
\end{center}

\item Now, it only remains to prove that $\psi$ and $\phi$ are natural transformations. It suffices to prove this just for one of the two, say $\psi$. Also, being natural in $(X,F)\in\mathcal{C}^{\mathrm{op}}\times\mathrm{Set}^{\mathcal{C}^{\mathrm{op}}}$ is equivalent to being natural in $X\in\mathcal{C}^{\mathrm{op}}$ for every fixed $F$ and at the same time being natural in $F\in\mathrm{Set}^{\mathcal{C}^{\mathrm{op}}}$, for every fixed $X$.

Let $F$ be fixed and $g^{op}\in\mathcal{C}^{op}(X,Y)$. Then, we want to show that the following diagram commutes:
\begin{center}
\begin{tikzcd}
X&&H_X&&\mathrm{Set}^{\mathcal{C}^{\mathrm{op}}}(H_X,F)\ar[d,"-\circ H_g"']\ar[r,"\psi_{X,F}"]&FX\ar[d,"Fg"]\\
Y\ar[u,"g"]&&H_Y\ar[u,"H_g"]&&\mathrm{Set}^{\mathcal{C}^{\mathrm{op}}}(H_Y,F)\ar[r,"\psi_{Y,F}"]&FY
\end{tikzcd}
\end{center}
which is true, since for any natural transformation $\eta\in\mathrm{Set}^{\mathcal{C}^{\mathrm{op}}}(H_X,F)$, both directions lead to $\eta_Y(g)$. To prove this, use the naturality of $\eta$ and the definition of $H_g$.

Let now $X$ be fixed and $\alpha\in\mathrm{Set}^{\mathcal{C}^{\mathrm{op}}}(F,G)$. Then, we want to show that the following diagram commutes:
\begin{center}
\begin{tikzcd}
F\ar[d,"\alpha"]&&\mathrm{Set}^{\mathcal{C}^{\mathrm{op}}}(H_X,F)\ar[d,"\alpha\circ-"']\ar[r,"\psi_{X,F}"]&FX\ar[d,"\alpha_X"]\\
G&&\mathrm{Set}^{\mathcal{C}^{\mathrm{op}}}(H_X,G)\ar[r,"\psi_{X,G}"]&GX
\end{tikzcd}
\end{center}
which is true, since for any natural transformation $\eta\in\mathrm{Set}^{\mathcal{C}^{\mathrm{op}}}(H_X,F)$, both directions lead to $(\alpha\circ\eta)_X(1_X)$.\qedhere
\end{itemize}
\end{proof}
\begin{corollary} Let $\mathcal{C}$ be a locally small category. Moreover, let $X,Y$ be any two objects of $\mathcal{C}$. Then:
\[\mathrm{Set}^{\mathcal{C}^{\mathrm{op}}}(H_X,H_Y)\cong\mathcal{C}(X,Y)\]
naturally in $(X,Y)$ in $\mathcal{C}^{\mathrm{op}}\times \mathcal{C}^{\mathrm{op}}$.
\end{corollary}
\begin{proof} First of all, the bijection is trivially obtained from the Yoneda lemma, for $F=H_Y$. Moreover, since the map $Y\mapsto H_Y$ is functorial, as defined in Definition~\ref{def:functor}, the following diagram commutes:
\begin{center}
\begin{tikzcd}
Y_1\ar[d,"h"]&&H_{Y_1}\ar[d,"H_h"]&&\mathrm{Set}^{\mathcal{C}^{\mathrm{op}}}(H_X,H_{Y_1})\ar[d,"H_h\circ-"']\ar[r,"\psi_{X,H_{Y_1}}"]&H_{Y_1}X\ar[d,"(H_h)_X"]\\
Y_2&&H_{Y_2}&&\mathrm{Set}^{\mathcal{C}^{\mathrm{op}}}(H_X,H_{Y_2})\ar[r,"\psi_{X,H_{Y_2}}"]&H_{Y_2}X
\end{tikzcd}
\end{center}
just like in the proof of the Yoneda lemma.
\end{proof}
\begin{corollary}[4.3.7]\label{cor:ful_faith} Let $\mathcal{C}$ be a locally small category. Then, the functor
\[H_{\bullet}:\mathcal{C}\to\mathrm{Set}^{\mathcal{C}^{\mathrm{op}}}\]
as defined in Definition~\ref{def:functor} is full and faithful.
\end{corollary}
\begin{proof} $H_{\bullet}$ being full and faithful means that for every two objects $X,Y$ in $\mathcal{C}$, the map
\begin{center}
\begin{tikzcd}
\mathcal{C}(X,Y)\ar[r,"H_{\bullet}"]&\mathrm{Set}^{\mathcal{C}^{\mathrm{op}}}(H_X,H_Y)\\[-2em]
g\ar[r,mapsto]&H_g
\end{tikzcd}
\end{center}
is a bijection. We already know that $\phi_{X,H_Y}$ is a bijection from the proof of the Yoneda lemma, so it suffices to show that $\phi_{X,H_Y}(g)=H_g$ for every $g\in\mathcal{C}(X,Y)$, or equivalently $\psi_{X,H_Y}(H_g)=g$:
\[\psi_{X,H_Y}(H_g)=(H_g)_X(1_X)=g\circ 1_X=g\qedhere\]
\end{proof}
\begin{proposition}[4.3.10]\label{prop:unique} Let $\mathcal{C}$ be a locally small category. Moreover, let $X,Y$ be any two objects of $\mathcal{C}$. Then:
\[H_X\cong H_Y \Longleftrightarrow X\cong Y\]
\end{proposition}
\begin{proof} Since $H_{\bullet}$ is a functor, it takes an isomorphism to an isomorphism. For the other direction, it is easy to see that for any full and faithful functor $F$, $FA\cong FB$ gives $A\cong B$ and $H_{\bullet}$ was proven to be full and faithful in Corollary~\ref{cor:ful_faith}
\end{proof}

\subsection{Adjoint functors}
Let us now go through some definitions and results regarding the adjoints. Our reference for this subsection will mainly be the Chapter~2 of \cite{basic_cat}, but be warned that we heavily changed the notations and the proofs of this section to cover our needs.

\begin{definition}[2.1.1]\label{def:adj} Let $\mathcal{C},\mathcal{D}$ be two locally small categories and $L:\mathcal{D}\to\mathcal{C},\ R:\mathcal{C}\to\mathcal{D}$ be two functors. Then we say that \emph{$R$ is the right adjoint to $L$} and \emph{$L$ is the left adjoint to $R$}, if there exist a isomorphisms
\begin{equation}
\mathcal{C}(L(A),X)\cong\mathcal{D}(A,R(X))\label{adj}
\end{equation}
natural in $(A,X)\in\mathcal{D}^{op}\times\mathcal{C}$. We usually denote this by $L\dashv R$ or equivalently:
\vspace*{-0.2em}
\begin{center}
\begin{tikzcd}
\mathcal{C}\ar[r,bend right=30, "R"'{name=R}]&[3em]\mathcal{D}\ar[l,bend right=30, "L"'{name=L}]
\ar[phantom, from=L, to=R, "\dashv" rotate=-90]
\end{tikzcd}
\end{center}
\end{definition}
\begin{notation} Just like in the proof of Yoneda lemma, we will try in this subsection to omit the parenthesis when using a functor and write $FP$ instead of $F(P)$ etc. In order to avoid confusion as much as possible, we may reintroduce the parenthesis, sometimes when more than one functors are involved, as in $FG(P)$.
\end{notation}

\begin{proposition}[4.3.13] Let $\mathcal{C},\mathcal{D}$ be two locally small categories and $L:\mathcal{D}\to\mathcal{C}$. If $L$ has a right adjoint functor $R$, then $R$ is unique, up to isomorphism.
\end{proposition}
\begin{proof} Let $R,R'$ be two right adjoint functors of $L$. Then, for fixed $X$, there exist isomorphisms
\[\mathcal{D}(A,RX)\cong\mathcal{C}(LA,X)\cong\mathcal{D}(A,R'X)\]
natural in $A$. This means that there is an isomorphism $H_{RX}(A)\cong H_{R'X}(A)$ natural in $A$, which means $H_{RX}\cong H_{R'X}$ in $\mathrm{Set}^{\mathcal{D}^{\mathrm{op}}}$. Using Proposition~\ref{prop:unique}, we conclude that there exists an isomorphism $RX\cong R'X$. This construction is natural in $X$, so $R\cong R'$ as functors.
\end{proof}

Let us now unravel the Definition~\ref{def:adj}. Let $L$ and $R$ be an adjoint pair $L\dashv R$. This means that there exist some natural transformations $\psi$ and $\phi$ with components:
\[\begin{aligned}
\psi_{A,X}&:\mathcal{C}(LA,X)\to\mathcal{D}(A,RX)\\[0.8em]
\phi_{A,X}&:\mathcal{D}(A,RX)\to\mathcal{C}(LA,X)
\end{aligned}\qquad\text{ satisfying: }\qquad
\left\{\begin{aligned}
\psi_{A,X}\circ\phi_{A,X}&=1_{\mathcal{D}(A,RX)}\\[0.8em]
\phi_{A,X}\circ\psi_{A,X}&=1_{\mathcal{C}(LA,X)}
\end{aligned}\right.\]
To express the two naturality conditions, we fix some $p^{\mathrm{op}}\in\mathcal{D}^{\mathrm{op}}(A,B)$ and some $f\in\mathcal{C}(X,Y)$. Then, we have the following four commutative diagrams:
\begin{center}
\begin{tikzcd}
A&[1.2em]
\mathcal{C}(LA,X)\ar[d,"-\circ Lp"']\ar[r,"\psi_{A,X}","\cong"']&\mathcal{D}(A,RX)\ar[d,"-\circ p"]&[1.2em]
\mathcal{C}(LA,X)\ar[d,"-\circ Lp"']&\mathcal{D}(A,RX)\ar[l,"\phi_{A,X}"',"\cong"]\ar[d,"-\circ p"]\\
B\ar[u,"p"']&
\mathcal{C}(LB,X)\ar[r,"\psi_{B,X}","\cong"']&\mathcal{D}(B,RX)&
\mathcal{C}(LB,X)&\mathcal{D}(B,RX)\ar[l,"\phi_{B,X}"',"\cong"]\\
X\ar[d,"f"]&
\mathcal{C}(LA,X)\ar[d,"f\circ -"']\ar[r,"\psi_{A,X}","\cong"']&\mathcal{D}(A,RX)\ar[d,"Rf\circ -"]&
\mathcal{C}(LA,X)\ar[d,"f\circ -"']&\mathcal{D}(A,RX)\ar[l,"\phi_{A,X}"',"\cong"]\ar[d,"Rf\circ -"]\\
Y&
\mathcal{C}(LA,Y)\ar[r,"\psi_{A,Y}","\cong"']&\mathcal{D}(A,RY)&
\mathcal{C}(LA,Y)&\mathcal{D}(A,RY)\ar[l,"\phi_{A,Y}"',"\cong"]\\
\end{tikzcd}
\end{center}
i.e. for any $q\in\mathcal{D}(A,RX)$ and $g\in\mathcal{C}(LA,X)$, we have:
\begin{center}
\begin{minipage}{0.5\linewidth}
%\makeatletter\tagsleft@true\makeatother
\begin{align}
\psi_{B,X}(g\circ Lp)&=\psi_{A,X}(g)\circ p\label{psiL}\\[1em]
\psi_{A,Y}(f\circ g)&=Rf\circ \psi_{A,X}(g)\label{psiR}
\end{align}
\end{minipage}\begin{minipage}{0.5\linewidth}
\begin{align}
\phi_{B,X}(q\circ p)&=\phi_{A,X}(q)\circ Lp\label{phiL}\\[1em]
\phi_{A,Y}(Rf\circ q)&=f\circ \phi_{A,X}(q)\label{phiR}
\end{align}
\end{minipage}
\end{center}
Right now, we will focus on equations \eqref{phiL} and \eqref{psiR}, which we also write schematically as:
\begin{align*}
\phi_{B,X}\left(B\overset{p}{\longrightarrow}A\overset{q}{\longrightarrow}RX\right)&=LB\overset{Lp}{\longrightarrow}LA\overset{\phi_{A,X}(q)}{\longrightarrow}X\\[1em]
\psi_{A,Y}\left(LA\overset{g}{\longrightarrow}X\overset{f}{\longrightarrow}Y\right)&=A\overset{\psi_{A,X}(g)}{\longrightarrow}RX\overset{Rf}{\longrightarrow}RY
\end{align*}
Interestingly, this hints towards the following fact: The values of $\phi$ (resp. $\psi$) on any $q\circ p$ (resp. $f\circ g$) only depend on $L$ (resp. $R$) and $\phi_{A,X}(q)$ (resp. $\psi_{A,X}(g)$). So, by choosing $q$ (resp. $g$) as ``naturally'' as possible, we can describe what exactly $\phi$ (resp. $\psi$) does on every other input. We thus choose $A=RX$, $q=1_{RX}$ for the equation involving $\phi$ and $X=LA$, $g=1_{LA}$ for the equation involving $\psi$. This gives:
\begin{align*}
\phi_{B,X}\left(B\overset{p}{\longrightarrow}RX\right)&=\phi_{B,X}\left(B\overset{p}{\longrightarrow}RX\xrightarrow{1_{RX}}RX\right)=LB\overset{Lp}{\longrightarrow}LRX\xrightarrow{\phi_{RX,X}(1_{RX})}X\\[1em]
\psi_{A,Y}\left(\mask{B\overset{p}{\longrightarrow}RX}{LA\overset{f}{\longrightarrow}Y}\right)&=\psi_{A,Y}\left(\mask{B\overset{p}{\longrightarrow}RX\xrightarrow{1_{RX}}RX}{LA\xrightarrow{1_{LA}}LA\overset{f}{\longrightarrow}Y}\right)=\mask{LB\overset{Lp}{\longrightarrow}LRX\xrightarrow{\phi_{RX,X}(1_{RX})}X}{A\xrightarrow{\psi_{A,LA}(1_{LA})}RLA\overset{Rf}{\longrightarrow}RY}
\end{align*}
\begin{definition} Let $\mathcal{C},\mathcal{D}$ be two locally small categories and $L:\mathcal{D}\to\mathcal{C}$, $R:\mathcal{C}\to\mathcal{D}$ a pair of adjoint functors $L\dashv R$. Then the natural transformations
\[\varepsilon:LR\Rightarrow \mathbbm{1}_{\mathcal{C}}\qquad\text{ and }\qquad\eta:\mathbbm{1}_{\mathcal{D}}\Rightarrow RL\]
defined by:
\[\varepsilon_X:=\phi_{RX,X}(1_{RX})\qquad\text{ and }\qquad\eta_A:=\psi_{A,LA}(1_{LA})\]
are called the \emph{counit} and the \emph{unit} of the adjunction, respectively.
\end{definition}
\begin{remark} In order for the counit and the unit to be well defined, we need to prove that they are indeed natural transformations, i.e. that for any $f\in\mathcal{C}(X,Y)$ and $p\in\mathcal{D}(B,A)$ the following diagrams commute:
\begin{center}
\begin{tikzcd}
X\ar[d,"f"']&[-1em]LR(X)\ar[r,"\varepsilon_X"]\ar[d,"LR(f)"']&X\ar[d,"f"]&[1.5em]
B\ar[d,"p"']&[-1em]B\ar[r,"\eta_B"]\ar[d,"p"']&RL(B)\ar[d,"RL(p)"]\\
Y&LR(Y)\ar[r,"\varepsilon_Y"]&Y&
A&A\ar[r,"\eta_A"]&RL(A)
\end{tikzcd}
\end{center}
\end{remark}
\begin{proof}
This is a direct consequence of the naturality of $\phi,\psi$:
\begin{align*}
f\circ\varepsilon_X&\overset{\eqref{phiR}}{=}\phi_{RX,Y}(1_{RY}\circ Rf\circ 1_{RX})\overset{\eqref{phiL}}{=}\varepsilon_Y\circ LRf\\[0.5em]
\eta_A\circ p&\overset{\eqref{psiL}}{=}\psi_{B,LA}(1_{LA}\circ Lp\circ 1_{LB})\overset{\eqref{psiR}}{=}RLp\circ\eta_B\qedhere
\end{align*}
\end{proof}

So, our discussion leads to the following fact. Let $f\in\mathcal{C}(LA,X)$ and $p\in\mathcal{D}(A,RX)$, then, equations \eqref{phiL} and \eqref{psiR} give:
\begin{align}
\begin{split}
\phi_{A,X}(p)&=\varepsilon_X\circ Lp\\[0.8em]
\psi_{A,X}(f)&=Rf\circ\eta_A
\end{split}\label{iso_from_unit}
\end{align}

This shows that the isomorphism \eqref{adj} can be completely retrieved by its counit and unit. Our goal for the rest of this section is to give an alternative definition for $L\dashv R$, only based on the choice of a particular natural transformation $\varepsilon$ to play the role of the counit, from which we first will construct $\eta$ and then $\phi$ and $\psi$. The reason we want to do this, is because this ``second definition'' will give us a way to more explicitly construct the right adjoint $R$ of a given $L$, if $L$ has a right adjoint.

First of all, notice that \eqref{iso_from_unit} on its own suffices to make $\phi$ and $\psi$ natural, given that $\varepsilon$ and $\eta$ are:
\begin{proposition}\label{prop:nat_gives_nat} Let $\mathcal{C},\mathcal{D}$ be two locally small categories and $L:\mathcal{D}\to\mathcal{C}$, $R:\mathcal{C}\to\mathcal{D}$ any pair of functors. Moreover, let $\varepsilon:LR\Rightarrow\mathbbm{1}_{\mathcal{C}}$ and $\eta:\mathbbm{1}_{\mathcal{D}}\Rightarrow RL$ be any two natural transformations. If we define $\phi_{A,X}$ and $\psi_{A,X}$ using the equations \eqref{iso_from_unit}, then $\phi$ and $\psi$ are natural transformations in $(A,X)\in\mathcal{D}^{\mathrm{op}}\times\mathcal{C}$ and additionally the following equations are satisfied:
\begin{align}
\begin{split}
\varepsilon_X=\phi_{RX,X}(1_{RX})\\[0.8em]
\eta_A=\psi_{A,LA}(1_{LA})
\end{split}\label{unit_from_iso}
\end{align}
\end{proposition}
\begin{proof} Let us fix some $p^{\mathrm{op}}\in\mathcal{D}^{\mathrm{op}}(A,B)$ and some $f\in\mathcal{C}(X,Y)$. Then it is easy to see that the equations \eqref{psiL}-\eqref{phiR} hold:
\begin{align*}
\psi_{B,X}(g\circ Lp)&=Rg\circ RLp\circ\eta_B=Rg\circ\eta_A\circ p=\psi_{A,X}(g)\circ p,&\text{for any } g&\in\mathcal{C}(LA,X)\\
\psi_{A,Y}(f\circ g)&=Rf\circ Rg\circ\eta_A=Rf\circ\psi_{A,X}(g),&\text{for any } g&\in\mathcal{C}(LA,X)\\
\phi_{B,X}(q\circ p)&=\varepsilon_X\circ Lq\circ Lp=\phi_{A,X}(q)\circ p,&\text{for any }q&\in\mathcal{D}(A,RX)\\
\phi_{A,Y}(Rf\circ q)&=\varepsilon_Y\circ LRf\circ Lq=f\circ\varepsilon_X\circ Lq=f\circ\phi_{A,X}(q),&\text{for any }q&\in\mathcal{D}(A,RX)
\end{align*}
We only used the equations \eqref{iso_from_unit} and the naturality of $\varepsilon,\eta$ in the first and last case. Moreover, it is again very easy to see that:
\begin{align*}
\phi_{RX,X}(1_{RX})&=\varepsilon_X\circ L1_{RX}=\varepsilon_X\\
\psi_{A,LA}(1_{LA})&=R1_{LA}\circ\eta_A=\eta_A\qedhere
\end{align*}
\end{proof}

Our problem now is that if we choose some natural transformations $\varepsilon$ and $\eta$ randomly and construct $\phi$ and $\psi$ using \eqref{iso_from_unit}, it is not at all guaranteed that $\phi$ and $\psi$ are inverses of each other. It so happens that $\varepsilon$ and $\eta$ must satisfy a universal property in order to be the counit and unit of an adjunction. Let's start examining this at the following definition.
\begin{definition}[2.3.4] Let $\mathcal{A},\mathcal{B}$ be two locally small categories, $F:\mathcal{A}\to\mathcal{B}$ some functor and $P$ any object in $\mathcal{B}$. We then define the category $\commaCat{F}{P}$, as follows:
\begin{itemize}
\item $\mathrm{ob}(\commaCat{F}{P})$ is the class of all pairs $(A,f)$, where $A$ is an object of $\mathcal{A}$ and $f:FA\to P$ is a morphism in $\mathcal{B}$.
\item $(\commaCat{F}{P})\big((A_1,f_1),(A_2,f_2)\big)$ is the set of maps $a:A_1\to A_2$ in $\mathcal{A}$, such that $f_1=f_2\circ Fa$, i.e. making the following diagram commute:
\begin{center}
%\begin{tikzcd}
%&[-1.5em]\mathcal{B}&[-1.5em]&&\mathcal{A}\ar[lll,bend right=20, "F"']\\
%&&FA_1\ar[dd,"Fa"]\ar[dll,"f_1"']&&A_1\ar[dd,"a"]\\[-1em]
%P\\[-1em]
%&&FA_1\ar[ull,"f_2"]&&A_2\\
%\end{tikzcd}
\begin{tikzcd}
&FA_1\ar[dd,"Fa"]\ar[dl,"f_1"']& && &A_1\ar[dd,"a"]\\[-1em]
P& &\mathcal{B}&&\mathcal{A}\ar[ll,bend right=20, "F"']\\[-1em]
&FA_1\ar[ul,"f_2"]& && &A_2\\
\end{tikzcd}
\end{center}
\end{itemize}
\end{definition}
\begin{proposition}[co-2.3.5] Let $\mathcal{C},\mathcal{D}$ be locally small categories and $L:\mathcal{D}\to\mathcal{C}$, $R:\mathcal{C}\to\mathcal{D}$ a pair of adjoint functors $L\dashv R$. Then, the pair $(RX, \varepsilon_X)$ is a terminal object in $\commaCat{L}{X}$, where $\varepsilon$ is the counit of the adjunction.
\end{proposition}
\begin{proof} We need to show that for every object $A$ in $\mathcal{D}$ and every morphism $f:LA\to X$ in $\mathcal{C}$, there exists a unique morphism $p\in\mathcal{D}(A,RX)$ such that $f=\varepsilon_X\circ Lp$, i.e. the following diagram commutes:
\begin{center}
\begin{tikzcd}
&[-1.5em]\mathcal{C}\ar[rrr,bend right=20,"R"'{name=R}]&[-1.5em]&&\mathcal{D}\ar[lll,bend right=20, "L"'{name=L}]\ar[phantom, from=L, to=R, "\dashv" rotate=-90]\\
&&LA\ar[dd,"Lp"]\ar[dll,"f"']&&A\ar[dd,"p"]\\[-1em]
X\\[-1em]
&&LRX\ar[ull,"\varepsilon_X"]&&RX\\
\end{tikzcd}
\end{center}
Since $L\dashv R$, then there exist mutually inverse isomorphisms $\psi_{A,X}:\mathcal{C}(LA,X)\to\mathcal{D}(A,RX)$ and $\phi_{A,X}:\mathcal{D}(A,RX)\to\mathcal{C}(LA,X)$ like above. If we let $p=\psi_{A,X}(f)\in\mathcal{D}(A,RX)$, then $f=\phi_{A,X}(p)=\varepsilon_X\circ Lp$, since $\psi,\phi$ are inverses and \eqref{iso_from_unit} holds. In order to prove the uniqueness, let $p':A\to RX$ be any other morphism satisfying the same identity. Then, $\phi_{A,X}(p)=\phi_{A,X}(p')$ which gives $p=p'$, since $\phi_{A,X}$ is an isomorphism.
\end{proof}
We now want to prove that this is not only a property of the counit, but it fully characterizes it:
\begin{theorem}[co-2.3.6]\label{th:alternative_def} Let $\mathcal{C},\mathcal{D}$ be locally small categories and $L:\mathcal{D}\to\mathcal{C}$, $R:\mathcal{C}\to\mathcal{D}$ be a pair of functors. Moreover, let $\varepsilon:LR\Rightarrow\mathbbm{1}_{\mathcal{C}}$ be some natural transformation, such that for every object $X$ of $\mathcal{C}$, $(RX,\varepsilon_X)$ is a terminal object in $\commaCat{L}{X}$. Then $L\dashv R$ and $\varepsilon$ is the counit of this adjunction.
\end{theorem}
\begin{proof} First of all, let $A$ be any object of $\mathcal{D}$. Since $(RLA,\varepsilon_{LA})$ is a terminal object of $\commaCat{R}{LA}$, for $f=1_{LA}$, there exists a unique $\eta_A:A\to RLA$, such that $\varepsilon_{LA}\circ L\eta_A=1_{LA}$, i.e. making the following diagram commutative:
\begin{center}
\begin{tikzcd}
&[-1.5em]\mathcal{C}\ar[rrr,bend right=20,"R"'{name=R}]&[-1.5em]&&\mathcal{D}\ar[lll,bend right=20, "L"'{name=L}]\\
&&LA\ar[dd,"L\eta_A"]\ar[dll,"1_{LA}"']&&A\ar[dd,"\eta_A"]\\[-1em]
LA\\[-1em]
&&LRLA\ar[ull,"\varepsilon_{LA}"]&&RLA\\
\end{tikzcd}
\end{center}
Now the proof will have three steps. First we will prove that $\eta_A$ is natural in $A$. This will be enough to define some natural transformations $\phi$ and $\psi$. The last two steps will be to prove that these two natural transformations are also inverses of each other.
\begin{itemize}
\item In order to prove that this $\eta$ we just defined is a natural transformation, fix some $p\in\mathcal{D}(A,B)$. Notice that the five subareas of the following diagram commute:
\begin{center}
\begin{tikzcd}
A\ar[dd,"p"']&&&LA\ar[dl,"Lp"']\ar[d,"1_{LA}"]\ar[r,"L\eta_A"]&LRLA\ar[dl,"\varepsilon_{LA}"]\ar[d,"LRLp"]\\[1em]
&&LB\ar[d,"L\eta_B"']\ar[dr,"1_{LB}"']&LA\ar[l,"Lp"']\ar[d,"Lp"']&LRLB\ar[dl,"\varepsilon_{LB}"]\\[1em]
B&&LRLB\ar[r,"\varepsilon_{LB}"']&LB
\end{tikzcd}
\end{center}
Two triangles because of the definition of $\eta$, the other two trivially and the square because of the naturality of $\varepsilon$. This gives us that the outer hexagon also commutes, and:
\[\varepsilon_{LB}\circ L(\eta_B\circ p)=Lp=\varepsilon_{LB}\circ L(RLp\circ\eta_A)\]
So, if we let
\[q=\eta_B\circ p\in\mathcal{D}(A,RLB)\qquad\text{ and }\qquad q'=RLp\circ\eta_A\in\mathcal{D}(A,RLB)\]
then, both $q$ and $q'$ satisfy the following commutative diagram:
\begin{center}
\begin{tikzcd}
&LA\ar[dd,"Lq=Lq'"]\ar[dl,"Lp"']&&A\ar[dd,shift right,"q"']\ar[dd,shift left,"q'"]\\[-1em]
LB\\[-1em]
&LRLB\ar[ul,"\varepsilon_{LB}"]&&RLB
\end{tikzcd}
\end{center}
Because $(RLB,\varepsilon_{LB})$ is terminal in $\commaCat{L}{LB}$, there is a unique morphism $(A,Lp)\to(RLB,\varepsilon_{LB})$, which means $q=q'$, so $\eta_B\circ p=RLp\circ\eta_A$, i.e. the following diagram commutes, as we wanted:
\begin{center}
\begin{tikzcd}
A\ar[d,"p"']&&A\ar[r,"\eta_A"]\ar[d,"p"']&RLA\ar[d,"RLp"]\\
B&&B\ar[r,"\eta_B"]&RLB
\end{tikzcd}
\end{center}
Since $\varepsilon:LR\Rightarrow\mathbbm{1}_{\mathcal{C}}$ and $\eta:\mathbbm{1}_{\mathcal{D}}\Rightarrow RL$, Proposition~\ref{prop:nat_gives_nat} gives that $\phi,\psi$, defined by the equations \eqref{iso_from_unit} are natural transformations in $(A,X)\in\mathcal{D}^{\mathrm{op}}\times\mathcal{C}$.
\item Let $f\in\mathcal{C}(LA,X)$. Then, the naturality of $\varepsilon$ and the definition of $\eta$, give:
\[\phi_{A,X}\big(\psi_{A,X}(f)\big)=\phi_{A,X}(Rf\circ\eta_A)=\varepsilon_X\circ LRf\circ L\eta_A=f\circ\varepsilon_{LA}\circ L\eta_A=f\]
\item Let $p\in\mathcal{D}(A,RX)$. Then, the fact that
\begin{equation}\label{duel_eta_eps}
R\varepsilon_X\circ\eta_{RX}=1_{RX}
\end{equation}
and the naturality of $\eta$ give:
\[\psi_{A,X}\big(\phi_{A,X}(p)\big)=\psi_{A,X}(\varepsilon_X\circ Lp)=R\varepsilon_X\circ RLp\circ\eta_A=R\varepsilon_X\circ\eta_{RX}\circ p=p\]
It only now remains to show \eqref{duel_eta_eps}. Notice that the triangle in the following diagram commutes because of the definition of $\eta$ and the square commutes because of the naturality of $\varepsilon$:
\begin{center}
\begin{tikzcd}
LRX\ar[d,"L\eta_{RX}"']\ar[dr,"1_{LRX}"]\\
LRLRX\ar[r,"\varepsilon_{LRX}"]\ar[d,"LR\varepsilon_X"']&LRX\ar[d,"\varepsilon_X"]\\
LRX\ar[r,"\varepsilon_X"]&X
\end{tikzcd}
\end{center}
This means that the outer pentagon also commutes, i.e.:
\[\varepsilon_X\circ L(R\varepsilon_X\circ\eta_{RX})=\varepsilon_X=\varepsilon_X\circ L(1_{RX})\]
So, if we let
\[q=R\varepsilon_X\circ\eta_{RX}\in\mathcal{D}(RX,RX)\qquad\text{ and }\qquad q'=1_{RX}\circ\eta_A\in\mathcal{D}(RX,RX)\]
then, both $q$ and $q'$ satisfy the following commutative diagram:
\begin{center}
\begin{tikzcd}
&LRX\ar[dd,"Lq=Lq'"]\ar[dl,"\varepsilon_X"']&&RX\ar[dd,shift right,"q"']\ar[dd,shift left,"q'"]\\[-1em]
X\\[-1em]
&LRX\ar[ul,"\varepsilon_X"]&&RX
\end{tikzcd}
\end{center}
Because $(RX,\varepsilon_X)$ is terminal in $\commaCat{L}{X}$, there is a unique morphism $(RX,\varepsilon_X)\to(RX,\varepsilon_X)$, which means $q=q'$, so $R\varepsilon_X\circ\eta_{RX}=1_{RX}$, which was our goal.
\end{itemize}
\end{proof}
Finally, this gives us the following way to construct $R$, given $L$:
\begin{corollary}[co-2.3.7] Let $\mathcal{C},\mathcal{D}$ be locally small categories and $L:\mathcal{D}\to\mathcal{C}$ be some functor. If $\commaCat{L}{X}$ has a terminal object for every object $X$ of $\mathcal{C}$, say $(\tilde{X},\varepsilon_X)$, then $L$ has a right adjoint functor $R:\mathcal{C}\to\mathcal{D}$ defined by:
\begin{itemize}
\item $RX$ to be equal to $\tilde{X}$, for every object $X$ in $\mathcal{C}$
\item $Rf$ to be equal to the unique morphism in $\commaCat{L}{Y}$ from $(RX,f\circ\varepsilon_X)$ to $(RY,\varepsilon_Y)$, for every $f\in\mathcal{C}(X,Y)$.
\end{itemize}
\end{corollary}
\begin{proof} Given \ref{th:alternative_def}, we only have to show that $R$ is indeed a functor and that $\varepsilon$ is a natural transformation:
\begin{itemize}
\item Let $X$ be an object in $\mathcal{C}$. Then $R1_X$ is the unique morphism in $\commaCat{L}{Y}$ from $(RX,\varepsilon_X)$ to $(RX,\varepsilon_X)$. Notice that $1_X$ is such a morphism:
\begin{center}
\begin{tikzcd}
X\ar[d,"1_X"']&LRX\ar[l,"\varepsilon_X"']\ar[d,"L1_{RX}",dashed]&&RX\ar[d,"1_{RX}"]\\
X&LRX\ar[l,"\varepsilon_X"']&&RX
\end{tikzcd}
\end{center}
Thus, the uniqueness gives $R1_X=1_X$. Moreover, let $f\in\mathcal{C}(X,Y)$ and $g\in\mathcal{C}(Y,Z)$. Then, $R(g\circ f)$ is the unique morphism in $\commaCat{L}{Z}$, from $(RX,g\circ f\circ\varepsilon_X)$ to $(RZ,\varepsilon_Z)$. Notice that $Rg\circ Rf$ is such a morphism:
\begin{center}
\begin{tikzcd}
X\ar[d,"f"']&LRX\ar[l,"\varepsilon_X"']\ar[d,"LRf"]\ar[dd,bend left=60,"L(Rg\circ Rf)",dashed]&&&RX\ar[d,"Rf"]\ar[dd,bend left=60,"Rg\circ Rf"]\\
Y\ar[d,"g"']&LRY\ar[l,"\varepsilon_Y"']\ar[d,"LRg"]&&&RY\ar[d,"Rg"]\\
Z&LRZ\ar[l,"\varepsilon_Z"']&&&RZ
\end{tikzcd}
\end{center}
Thus, the uniqueness gives $R(g\circ f)=Rg\circ Rf$.
\item Let $f\in\mathcal{C}(X,Y)$. Then, by definition of $Rf$, the following square commutes:
\begin{center}
\begin{tikzcd}
X\ar[d,"f"']&LRX\ar[l,"\varepsilon_X"']\ar[d,"LRf"]\\
Y&LRY\ar[l,"\varepsilon_Y"']
\end{tikzcd}
\end{center}
which means exactly that $\varepsilon$ is a natural transformation.\qedhere
\end{itemize}
\end{proof}
\section{Compact-Open Topology}


%\section{CW Complexes}
%A CW structure on some space $X$ is usually defined recursively, as an inductive ``glueing'' of cells of some dimension $k$ to the previous, lower dimensional, skeleton of $X$, forming a new, $k$-dimensional, skeleton of $X$. A space $X$ may exhibit many different CW structures, but the existence of one suffices in order for $X$ to be characterized as CW complex. Here, we are going to use the following formal formulation of the above definition.
%
%\begin{definition} A topological space $X$ is a \ul{CW-complex}, if there exists some filtration
%\[\emptyset=X_{-1}\subseteq X_0\subseteq X_1\subseteq X_2\subseteq\cdots\subseteq X\]
%such that:
%\begin{b_item}
%\item $X=\varinjlim X_i$ with respect to all inclusion maps.
%\item For every $n\geq0$ there exists a pushout diagram in the category of topological spaces:
%\begin{center}
%\begin{tikzcd}
%\displaystyle\coprod_{e\in\pi_0(X_n\setminus{X_{n-1}})}S^{n-1}\ar[r,"\coprod_e\phi_e"]\ar[d,hook,"\coprod_ej_e"']\ar[dr,phantom, very near start,"\ulcorner"]&[4em]X_{n-1}\ar[d,hook,"i_n"]\\[2em]
%\displaystyle\coprod_{e\in\pi_0(X_n\setminus{X_{n-1}})}D^n\ar[r,"\coprod_e\Phi_e"']&X_n\\
%\end{tikzcd}
%\end{center}
%where $j_e:S^{n-1}\to D^n$ is the usual inclusion map and $i_n:X_{n-1}\to X_n$ is the inclusion map given by the filtration.
%\end{b_item}
%\end{definition}
%
%A filtration of a topological space $X$, making $X$ a CW-complex is called a \ul{CW-structure} of $X$. Moreover, given a filtration of $X$ like in the above definition, the sets $\Phi_e\large({(D^n)}^{\circ}\large)$ (resp. $\Phi_e(D^n)$) are called the $n$-dimensional \ul{open} (resp. \ul{closed}) \ul{cells} of this CW-structure. Recall the following known facts regarding the dependencies between CW-complexes, structures and cells.
%
%\begin{notes}
%\begin{i_enum}
%\item A CW-complex $X$ can have more than one CW-structures, even structures having different number of $n$-dimensional open cells each.
%\item For a particular CW-structure of $X$, the maps $\phi_e$ and $\Phi_e$ are not predetermined by the structure, which means that there can be more than one choices for them. For example, one could always precompose $\Phi_e$ with a disc homeomorphism.
%\item Even if the maps $\phi_e$ and $\Phi_e$ can vary, the open and closed cells of a CW-structure are part of the structure (i.e.\ independent of the choice of the maps)
%\end{i_enum} \end{notes}
%
%\begin{remarks}
%\begin{i_enum}
%\item The property $X=\varinjlim X_i$ is equivalent to $X=\bigcup_{i\geq-1}X_i$ as a set, equipped with the final topology, with respect to all inclusion maps. In particular, a set $A$ is open (closed) in $X$, iff $A\cap X_i$ is open (closed) in $X_i$ for all $i\geq-1$, or equivalently, $A\cap\sigma$ is open (closed) in $\sigma$ for every open cell $\sigma$ of the CW structure. This property is what we usually refer to as the ``weak topology'' of $X$ (the ``W'' part of the CW).
%\item
%\end{i_enum}
%\end{remarks}

%	\chapter{Fiber Bundles}
The Fiber Bundle is a topological object generalizing the notion of the product space. A fiber bundle $E$ can be thought like a ``twisted product'' of a base topological space $B$ and a fiber space $F$. This means that locally it always looks like a product, but globally it may (homeomorphically) change, while we move in $B$. This is called local triviallity and is the least we can demand of a fiber bundle. Most of the time though, people are interested in fiber bundles with additional structure imposed on the fiber space $F$, such as a group action on $F$. These fiber bundles are said to then be equipped with a ``structure group'' $G<\mathrm{Aut}(F)$. On Chapter~\ref{chap:vec_bundles} we examine the notion of ``vector bundles'', which require that $F$ has the structure of a vector space. Also, it is worth noting at this point that the definition of the ``characteristic classes'' one sees in~\ref{def:char_class} originated from the study of ``sphere bundles'', requiring the fiber to be a sphere.

\begin{definition}\label{def:fiber_bundle} Let $F$, $E$ and $B$ be some topological spaces. A continuous map $p:E\to B$ is called a \ul{fiber bundle with fiber $F$}, if for every $x\in B$ there exists an open neighborhood $x\in U\subseteq B$, and a function $f_U:p^{-1}(U)\to U\times F$ (where $p^{-1}(U)$ has the subspace topology and $U\times F$ the product topology), such that:
\begin{i_enum}
\item $f_U$ is a homeomorphism.
\item  $p|_{p^{-1}(U)}=\pi_1\circ f_U$, i.e. the following diagram commutes:
\begin{center}
\begin{tikzcd}
E\ar[r,"\supseteq",phantom]&[-1em]p^{-1}(U)\ar[d,"p|_{p^{-1}(U)}"']\ar[r,"f_U"]&U\times F\ar[dl,"\pi_1"]\\[2em]
B\ar[r,"\supseteq",phantom]&U
\end{tikzcd}
\end{center}
\end{i_enum}
$B$ is then called \ul{base space} and $E$ \ul{total space} of the fiber bundle. Moreover, for every $x\in B$, the space $p^{-1}(\{x\})$ is called \ul{the fiber over $x$} and is denoted by $F_x$. Notice, that, sometimes, we refer to the fiber bundle just by its total space, if the definition of $p$ is clear, or by the notation $F\to E\overset{p}{\to}B$.
\end{definition}

\begin{remark}\label{rem:fiber_subcover} Let $E\overset{p}{\to} B$ be a fiber bundle, $x\in B$ and $U\subseteq B$ an open neighborhood of $x$, $f_U:p^{-1}(U)\to U\times F$ a homeomorphism as in the Definition~\ref{def:fiber_bundle}, i.e. it makes the following diagram commute:
\begin{center}
\begin{tikzcd}
p^{-1}(U)\ar[d,"p|_{p^{-1}(U)}"']\ar[r,"f_U"]&U\times F\ar[dl,"\pi_1"]\\
U
\end{tikzcd}
\end{center}
Moreover, let $V\subseteq U$ be some other open neighborhood of $x$. Then, $f_V:=f_U|_{p^{-1}(V)}$ is also a homeomorphism between $p^{-1}(V)$ and $V\times F$ as in the Definition~\ref{def:fiber_bundle}, i.e. it makes its associated diagram commute. Indeed, since $V\subseteq U$, $p^{-1}(V)\subseteq p^{-1}(U)$, the restriction $f_U|_{p^{-1}(V)}$ is first of all well defined. Moreover, we will prove that $f_V(p^{-1}(V))=V\times F$:
\begin{itemize}
\item[($\subseteq$)] Let $u\in p^{-1}(V)$. Then, since $f_U$ makes the above diagram commute, $f_V(u)=f_U(u)=(p(u),a)$ for some $a\in F$. Since $p(u)\in V$, $f_V(u)\in V\times F$.
\item[($\supseteq$)] Let $(b,a)\in V\times F$. Then, $p\circ f_U^{-1}(b,a)=b\in V$, thus $f_U^{-1}(b,a)\in p^{-1}(V)$ and $f_V(f_U^{-1}(b,a))=(b,a)$.
\end{itemize}
So, the function $f_V:p^{-1}(V)\to V\times F$ is well defined and a homeomorphism as a restriction of the homeomorphism $f_U$. Also, for every $u\in p^{-1}(V)$, we trivially have $\pi_1\circ f_V(u)=\pi_1\circ f_U(u)=p(u)$, which means that the following diagram commutes:
\begin{center}
\begin{tikzcd}
p^{-1}(V)\ar[d,"p|_{p^{-1}(V)}"']\ar[r,"f_V"]&V\times F\ar[dl,"\pi_1"]\\
U
\end{tikzcd}
\end{center}
\end{remark}

\begin{proposition}\label{prop:same_fiber} Let $p:E\to B$ be a fiber bundle with fiber $F$, then $F_x\cong F$ for all $x\in B$.
\end{proposition}
\begin{proof}
Let $x\in B$. Then, there exists a neighborhood $U$ of $x$ inside $B$ and a homeomorphism $f_U:p^{-1}(U)\to U\times F$, such that
\[p|_{p^{-1}(U)}=\pi_1\circ f_U\]
Then, $f_U|_{F_x}:F_x\to f_U(F_x)$ is also a homeomorphism, as a restriction of a homeomorphism. Moreover, it is easy to see that $f_U(F_x)=\{x\}\times F$:
\begin{itemize}
\item Let $(x',a)\in f_U(F_x)$. This means that there exists some $v\in F_x$ with $f_U(v)=(x',a)$. Since $v\in F_x$, $p(v)=x$ by definition. Then, $x=p(v)=\pi_1\circ f_U(v)=\pi_1(x',a)=x'$, which means, $(x',a)=(x,a)\in\{x\}\times F$.
\item Let $(x,a)\in\{x\}\times F$. Since $f_U$ is surjective, there exists some $v\in p^{-1}(U)$ with $f_U(v)=(x,a)$. Then, $p(v)=\pi_1\circ f_U(v)=\pi_1(x,a)=x$ which means that $v\in p^{-1}(\{x\})=F_x$.
\end{itemize}
$\{x\}\times F$ is trivially homeomorphic to $F$, which proves the assertion.
\end{proof}

Now, one can see how bundles generalize products. Formally, one just needs to choose $U=B$ for every $x$ and $f_B=id$. In fact, products are in this setup the ``trivial'' example.
\begin{definition} Let $F, B$ be two topological spaces, then the projection
\[\pi_1:B\times F\to B\]
is called the \ul{trivial bundle over $B$ with fiber $F$}, where $\pi_1$ is the projection on the first coordinate, i.e.\ $\pi_1(x,a)=x$.
\end{definition}

Historically, the most famous non-trivial example of a fiber bundle has been the M\"obius strip:
\begin{example} Let $B=S^1$, $F=[0,1]$ and $E=\sfrac{[0,1]\times F}{(0,a)\sim(1,1-a)}$. Then $p:E\to B$, with:
\[p([(x,a)]):= x,\qquad\forall [(x,a)]\in E\]
is a fiber bundle.
\end{example}
Notice how this fiber bundle is not the ``same'' as the trivial bundle over the circle with the same fiber. We are going to soon define the notion of bundle maps and isomorphisms, but before going into that let us try the same M\"obius like constructions, i.e.\ let us fix $B=S^1$ and examine how the bundles look like for different choices of fibers $F$ and different equivalence relations $\sim$, which define $E=\sfrac{[0,1]\times F}{\sim}$. The bundle in these examples will always be $p:E\to B$ with $p([(x,a)])=x$.

\begin{examples}
\begin{i_enum}
\item For $F=\{0,1\}$, and $(0,a)\sim(1,1-a)$, $E$ is obviously the boundary of the M\"obius strip, which is topologically a circle. The image of $p$ then traces the circle with twice the speed of the input variable. Notice, that this is different from the trivial bundle with the same fiber over $B$, since the trivial bundle is a disjoint union of two circles.
\item $F=\mathbb{R}$ or $F=(0,1)$, and again $(0,a)\sim(1,1-a)$. This is the open version of the M\"obius strip.
\item $F=S^1\subseteq\mathbb{C}$, and $(0,z)\sim(1,\overline{z})$. This is maybe the second non-trivial example one usually sees, namely the Klein bottle.
\item $F=S^1$ again, but this time $(0,z)\sim(1,-z)$. Notice that this bundle contains the boundary of the M\"obius strip (and in fact the boundary of every rotation of the M\"obius strip around $B$). Moreover, notice that ``turning'' the circle $\{1\}\times S^1$ by $\pi$, before gluing it back to $\{0\}\times S^1$, just produces a total space, which is topologically a torus, i.e.\ $E$ is the ``same'' as the trivial bundle (although it contains a non-trivial one).
\end{i_enum}
\end{examples}

Notice how in the above discussion we first fixed a base space $B$, before starting to ask questions about similarity of bundles. This is so common, that the first definition of a maps between bundles assumes that both involved bundles have the same base space.
\begin{definition}
Let $p_1:E_1\to B$ and $p_2:E_2\to B$ be two fiber bundles with fibers $F_1$ and $F_2$ respectively. A continuous map $\phi:E_1\to E_2$ is a \ul{bundle map from $p_1$ to $p_2$ over $B$}, if $p_1=p_2\circ\phi$, i.e. if the following diagram commutes:
\begin{center}
\begin{tikzcd}
E_1\ar[rd,"p_1"']\ar[rr,"\phi"]&[-2em]&[-2em]E_2\ar[ld,"p_2"]\\
&B
\end{tikzcd}
\end{center}
\end{definition}

\begin{proposition}\label{prop:fiberwise}
Let $p_1:E_1\to B$ and $p_2:E_2\to B$ be two fiber bundles and $\phi:E_1\to E_2$ be any continuous map, then $\phi$ is a bundle map, if and only if
\[\phi\big({(F_1)}_x\big)\subseteq{(F_2)}_x,\qquad\forall x\in B\]
i.e.\ iff $\phi$ maps every fiber over $x$, of the left fiber bundle, to the fiber over the same $x$, of the right fiber bundle.
\end{proposition}
\begin{proof}
\begin{b_item}
\item[($\Rightarrow$)] Let $\phi:E_1\to E_2$ be a bundle map, $x\in B$ and $v\in {(F_1)}_x$. Then $(p_2\circ\phi)(v)=p_1(v)=x$, which proves that $\phi(v)\in{(F_2)}_x$.
\item[($\Leftarrow$)] Let $\phi:E_1\to E_2$ be a continuous map, with $\phi\big({(F_1)}_x\big)\subseteq{(F_2)}_x$ for every $x\in B$. Moreover, let $v\in E_1$. Then, trivially, $v\in {(F_1)}_{p_1(v)}$, which gives $\phi(v)\in {(F_2)}_{p_1(v)}$, which means exactly that $(p_2\circ\phi)(v)=p_1(v)$.\qedhere
\end{b_item}
\end{proof}

\begin{definition}
Let $p_1:E_1\to B$ and $p_2:E_2\to B$ be two fiber bundles with fibers $F_1$ and $F_2$ respectively and let $\phi:E_1\to E_2$ be a bundle map from $p_1$ to $p_2$ over $B$. The map $\phi$ is called a \ul{bundle isomorphism}, if $\phi$ is a homeomorphism.
\end{definition}

\begin{proposition}
Let $p_1:E_1\to B$ and $p_2:E_2\to B$ be two fiber bundles with fibers $F_1$ and $F_2$ respectively and let $\phi:E_1\to E_2$ be a bundle isomorphism map from $p_1$ to $p_2$ over $B$. Then, for every $x\in B$, the map
\[\phi_x:{\left(F_1\right)}_x\to{\left(F_2\right)}_x\]
with $\phi_x(v)=\phi(v)$ for every $v\in\left(F_1\right)_x$ is a homeomorphism with $(\phi_x)^{-1}=\big(\phi^{-1}\big)_x$.
\end{proposition}
\begin{proof} First of all, the map $\phi_x$ is well defined because of Proposition~\ref{prop:fiberwise}. Moreover, $\phi_x$ is surjective. Indeed, let $v_2\in (F_2)_x$. Since $\phi$ is surjective, there exists some $v_1\in E_1$ such that $\phi(v_1)=v_2$. Since $\phi$ is a bundle map, it is true that $p_1(v_1)=p_2\circ\phi(v_1)=p_2(v_2)=x$, which means that $v_1\in(F_1)_x$. This proves that
\[\phi_x=\phi|_{{(F_1)}_x}:{(F_1)}_x\to\phi\big({(F_1)}_x\big)\]
and thus is a homeomorphism as a restriction of one. The fact that $\phi_x^{-1}$ is the restriction of $\phi^{-1}$ on ${(F_2)}_x$ follows by symmetry.
\end{proof}

We want now to investigate if the opposite is true. Given two fiber bundles $p_1, p_2$ over some space $B$ that have isomorphic fibers over each $b\in B$, could we ``glue'' these isomorphisms to a global one between the fiber bundles themselves? We immediately see that the information we get only from the fibers is not enough to reconstruct the whole bundle, since it is clear from the above examples that there are non-isomorphic fiber bundles $E_1, E_2$ over a space $B$, which have the same fiber $F$. In such cases, we cannot even construct a map $\phi:E_1\to E_2$ making the following diagram commutative, let alone construct an isomorphism.
\begin{center}
\begin{tikzcd}
E_1\ar[rd,"p_1"']\ar[rr,"\phi"]&[-2em]&[-2em]E_2\ar[ld,"p_2"]\\
&B
\end{tikzcd}
\end{center}
In order to resolve this, we should at least require that there already exists some bundle map $\phi$, the slices of which are the fiber-wise isomorphisms and ask if this is enough for $\phi$ to be an isomorphism. Remarkably, this is in general not true, as we exhibit in the following example. It is based on an example given in \cite{counterexample} of an exponentiable space, whose set of homeomorphisms does not form a topological group, because the inverse map isn't continuous.

\begin{example}\label{ex:counterexample} Let $B\subseteq[0,1]$ be a converging sequence, together with its limit:
\[B=\left\{\frac{1}{n}:n\in\mathbb{N}\right\}\cup\left\{0\right\}\]
and $F\subseteq[0,1]$ be a Cantor set, without its leftmost point:
\[F=\left\{\sum_{i=1}^{\infty}\frac{\alpha_n}{3^n}:\alpha_n\in\{0,2\}\right\}\setminus\{0\}\]
both equipped with the subspace topology. Moreover let $E_1,E_2$ be the trivial bundle over $B$, with fiber $F$, i.e. $E_1=E_2=B\times F$. Define now the following function $f:E_1\to E_2$:
\begin{center}
\begin{tabular}{ccc}
\begin{tikzcd}
f\ar[r,phantom,":"]&[-.8cm]B\times F\ar[r]&B\times F\\[-1.2em]
&(b,x)\ar[r,mapsto]&(b,f_b(x))
\end{tikzcd}
&,&
$\displaystyle f_b=\left\{\begin{array}{ll}
1_F&,b=0\\[1.2em]
h_n&,b=\frac{1}{n}
\end{array}\right.$
\end{tabular}
\end{center}
where $h_n:F\to F$ is defined as follows:
\begin{center}
\begin{tabular}{cc}
$\displaystyle h_n(x)=\left\{\begin{array}{ll}
3x&,x\in\left(0,\frac{1}{3^{n+1}}\right]\cap F\\[1em]
x+1-\frac{1}{3^n}&,x\in\left[\frac{2}{3^{n+1}},\frac{1}{3^n}\right]\cap F\\[1em]
x&,x\in\left[\frac{2}{3^n},1-\frac{2}{3^n}\right]\cap F\\[1em]
\frac{1}{3}x+\frac{2}{3}-\frac{2}{3^{n+1}}&,x\in\left[1-\frac{1}{3^n},1\right]\cap F
\end{array}\right.$
&
\end{tabular}
\begin{tikzpicture}[baseline=0, decoration=Cantor set]
\def\d{5}
\node at (0,0) {$h_2$};
\coordinate (O) at (-\d/2,-\d/2);
%\draw[->,very thin,gray,dotted] (O) + (-.2,0) -- +(\d+0.2,0);
%\draw[->,very thin,gray,dotted] (O) + (0,-.2) -- +(0,\d+0.2);
\draw[thick] decorate{ decorate{ decorate{ decorate{ (O) -- +(\d,0) }}}};
\draw[thick] decorate{ decorate{ decorate{ decorate{ (O) -- +(0,\d) }}}};

\draw[red, thick] decorate{ decorate{ (O) -- +(\d/27,\d/9) }};
\draw[red, thick] decorate{ (O) +(2*\d/27,\d-\d/27) -- +(\d/9,\d) };
\draw[red, thick] decorate{ decorate{ (O) +(2*\d/9,2*\d/9) -- +(\d/3,\d/3) }};
\draw[red, thick] decorate{ decorate{ (O) +(\d-\d/3,\d-\d/3) -- +(\d-2*\d/9,\d-2*\d/9) }};
\draw[red, thick] decorate{ decorate{ (O) +(\d-\d/9,\d-\d/9) -- +(\d,\d-2*\d/27) }};

\draw[very thin, gray, dotted] (O) +(\d/27,0) -- +(\d/27,\d/9);
\draw[very thin, gray, dotted] (O) +(2*\d/27,0) -- +(2*\d/27,\d-\d/27);
\draw[very thin, gray, dotted] (O) +(\d/9,0) -- +(\d/9,\d);
\draw[very thin, gray, dotted] (O) +(2*\d/9,0) -- +(2*\d/9,2*\d/9);
\draw[very thin, gray, dotted] (O) +(\d-2*\d/9,0) -- +(\d-2*\d/9,\d-2*\d/9);
\draw[very thin, gray, dotted] (O) +(\d-\d/9,0) -- +(\d-\d/9,\d-\d/9);
\draw[very thin, gray, dotted] (O) +(\d,0) -- +(\d,\d-2*\d/27);

\draw[very thin, gray, dotted] (O) +(0,\d/9) -- +(\d/27,\d/9);
\draw[very thin, gray, dotted] (O) +(0,\d-\d/27) -- +(2*\d/27,\d-\d/27);
\draw[very thin, gray, dotted] (O) +(0,\d) -- +(\d/9,\d);
\draw[very thin, gray, dotted] (O) +(0,2*\d/9) -- +(2*\d/9,2*\d/9);
\draw[very thin, gray, dotted] (O) +(0,\d-2*\d/9) -- +(\d-2*\d/9,\d-2*\d/9);
\draw[very thin, gray, dotted] (O) +(0,\d-\d/9) -- +(\d-\d/9,\d-\d/9);
\draw[very thin, gray, dotted] (O) +(0,\d-2*\d/27) -- +(\d,\d-2*\d/27);
\end{tikzpicture}
\end{center}
We will prove that $f$ is a bundle map, but it isn't a homeomorphism, although $f_b$ is a homeomorphism for every $b\in B$.

First of all, $f_0$ is trivially a homeomorphism and it is easy to see that $h_n$ is always a homeomorphism, since it is bijective and linear on clopen subsets of $F$. Now, let us prove that $f$ is continuous. It then will immediately follow that $f$ is also a bundle map, since $\pi_1\circ f=\pi_1$.

We will first define the function $g_f:B\times F\to F$, with $g_f:=\pi_2\circ f$. Notice that $g_f(x,a)=f_x(a)$ for every $(x,a)\in B\times F$. Since $F$ is Hausdorff and locally compact, the space $Top(F,F)$, topologised with the compact-open topology has the following properties, as we proved in Proposition~\ref{prop:exponential_object}:
\begin{itemize}
\item The evaluation map $ev:\mathrm{Top}(F,F)\times F\to F$, defined by $(f,x)\mapsto f(x)$ is continuous.
\item For every topological space $T$ and any function $g:T\times F\to F$, there exists a unique function $g^{\#}:T\to\mathrm{Top}(F,F)$ such that $g=ev\circ(g^{\#}\times 1_F)$.
\item $g$ is continuous if and only if $g^{\#}$ is continuous.
\end{itemize}

In our example, if we let $\tilde{g}_f:B\to\mathrm{Top}(F,F)$ to be the function taking $b$ to $f_b$, we have the following commutative diagram:
\begin{center}
\begin{tikzcd}
B\times F\ar[r,"\tilde{g}_f\times 1_F"]\ar[d,"g_f"]&\mathrm{Top}(F,F)\times F\ar[dl,"ev"]\\
F
\end{tikzcd}
\end{center}
which proves that $g_f^{\#}=\tilde{g}_f$, due to the uniqueness property. Thus, $g_f$ is continuous, if and only if $\tilde{g}_f$ is continuous., which is equivalent to $\lim_{n\to\infty}h_n=1_F$ in $\mathrm{Top}(F,F)$.

In order to show this, we need to prove that every open neighborhood of $1_F$ contains every $h_n$ for some big enough $n$. In fact, it suffices to check this only for these neighborhoods of $1_F$ that are elements in a subbasis of the topology of $\mathrm{Top}(F,F)$, so we will only regard open sets in the following family:
\[\Big\{S(K,O):=\{f:K\to O\,|\,f(K)\subseteq O\}\ \Big|\ K,O\subseteq F,\ K\text{ compact},\ O\text{ open}\Big\}\]
Let $S(K,O)$ be such a set in the neighborhood of $1_F$. This means that $K\subseteq O$. First of all, notice, that there exists some $n_1\in\mathbb{N}$ such that $\frac{1}{3^n}<\min K$ for all $n\geq n_1$. We now examine the following two cases:
\begin{itemize}
\item[$1\not\in K$:] In this case, there exists also a $n_2\in\mathbb{N}$ such that $1-\frac{1}{3^n}>\max K$ for all $n\geq n_2$. This means that $K\subseteq[\frac{2}{3^n},1-\frac{2}{3^n}]$ for all $n\geq\max\{n_1,n_2\}$, which means that $h_n(K)=K\subseteq O$, i.e. $h_n\in S(K,O)$.
\item[$1\in K$] In this case, there exists some $n_2'\in\mathbb{N}$ such that $V_n:=[1-\frac{1}{3^n},1]\cap F\subseteq O$ for every $n\geq n_2'$. Then, $h_n(V_n\cap K)\subseteq h_n(V_n)\subseteq V_n\subseteq O$ for every $n\geq n_2'$ and also, like in the previous case, $h_n(K\setminus V_n)= h_n(K\cap[\frac{2}{3^n},1-\frac{2}{3^n}]\cap F)=K\cap[\frac{2}{3^n},1-\frac{2}{3^n}]\cap F\subseteq O$ for every $n\geq n_1$. Thus, $h_n\in S(K,O)$ for every $n\geq\max\{n_1,n_2'\}$.
\end{itemize}
This proves the continuity of $\tilde{g}_f$, which implies the continuity of $g_f$. So, by the universal property of the product, there exists a unique continuous $\left<\pi_1,g_f\right>:B\times F\to B\times F$ as in the following commutative diagram:
\begin{center}
\begin{tikzcd}
B&B\times F\ar[l,"\pi_1"']\ar[r,"\pi_2"]&F\\[1em]
&B\times F\ar[ul,"\pi_1"]\ar[u,dotted,"\left<\pi_1{,}g_f\right>" description,background color=none]\ar[ur,"g_f"']
\end{tikzcd}
\end{center}
Notice that $\left<\pi_1,g_f\right>(x,a)=(x,f_x(a))=f(x,a)$, which proves that $f$ is continuous.

Lastly, we will prove that $f^{-1}$ is not continuous using the same machinery. So, $f^{-1}$ is continuous if and only if $g_{f^{-1}}:B\times F\to F$, with $g_{f^{-1}}(x,a)=f^{-1}_x(a)$ is continuous. This, again, is the case if and only if $g_{f^{-1}}^{\#}=\tilde{g}_{f^{-1}}:B\to\mathrm{Top}(F,F)$, taking $b$ to $f_b^{-1}$ is continuous, which is equivalent to $\lim_{n\to\infty}h_n^{-1}=1_F^{-1}=1_F$ in $\mathrm{Top}(F,F)$. So, it remains to prove that this limit doesn't converge to $1_F$.

Let $V_n:=[1-\frac{1}{3^n},1]\cap F$. This is both open and compact, so $S(V_n,V_n)$ is a subbasic neighborhood of $1_F$. For every $n\in\mathbb{N}$ we have that $h_n^{-1}(1)=\frac{1}{3^n}<1-\frac{1}{3^n}=\min V_n$, which means that for every $n\in\mathbb{N}$, $h_n^{-1}(V_n)\not\subseteq V_n$, i.e. $h^{-1}_n\not\in S(V_n,V_n)$.
\end{example}

As we've seen in Theorem~\ref{thm:inverse}, we know some sufficient conditions in order to lift the pathogen of the inverse mapping not being continuous, so we will now prove that the above example wouldn't fail if $F$ was also locally connected in addition to being Hausdorff and locally compact:

%We will now prove the following theorem, which is based on Theorem~4 in \cite{top_group} which proves the continuity of the inverse operator on $\mathrm{Top}(F,F)$, provided that $F$ has some good properties

\begin{theorem} Let $F$ be a Hausdorff, locally compact, locally connected topological space and let $p_1: E_1\to B$ and $p_2: E_2\to B$ be two fiber bundles with fibers $F_1$ and $F_2$ respectively, both isomorphic to $F$. Moreover, let $\phi:E_1\to E_2$ be a bundle map such that its restriction
\[\phi_x:(F_1)_x\to (F_2)_x\]
with $\phi_x(v)=\phi(v)$ is a homeomorphism for every $x\in B$. Then $\phi$ is also a homeomorphism.
\end{theorem}
\begin{proof} First of all, it is easy to see that $\phi$ is a bijection:
\begin{itemize}
\item Let $u_1,u_2\in E_1$ and $\phi(u_1)=\phi(u_2)$. Since $\phi$ is a bundle map, we have:
\[p_1(u_1)=p_2(\phi(u_1))=p_2(\phi(u_2))=p_1(u_2)=:x_0\]
So, $u_1,u_2\in\left(F_1\right)_{x_0}$ and $\phi_{x_0}(u_1)=\phi(u_1)=\phi(u_2)=\phi_{x_0}(u_2)$. Since $\phi_{x_0}$ is injective, $u_1=u_2$, which proves that $\phi$ is injective as well.
\item Let $v\in E_2$. For $x_0:=p_2(v)$, we have that $v\in\left(F_2\right)_{x_0}$. Since $\phi_{x_0}$ is surjective, there exists some $u\in\left(F_1\right)_{x_0}\subseteq E_1$ with $\phi(u)=\phi_{x_0}(u)=v$, proving that $\phi$ is also surjective.
\end{itemize}
It now remains to prove that $\phi^{-1}$ is continuous as well. For every $x\in B$, there exist open neighborhoods ${(U_1)}_x,{(U_2)}_x$ of $x$, such that $p_1^{-1}({(U_1)}_x)$ is isomorphic to $U_1\times F_1$ and $p_2^{-1}({(U_2)}_x)$ is isomorphic to $U_2\times F_2$, satisfying the commutative diagrams of the Definition~\ref{def:fiber_bundle}. Let $U_x:=(U_1)_x\cap(U_2)_x$. Then, as seen in Remark~\ref{rem:fiber_subcover}, $U_x$ satisfies the requirements of the \ref{def:fiber_bundle}, which means there exist homeomorphisms $(f_1)_{U_x}:p_1^{-1}(U_x)\to U_x\times F_1$ and $(f_2)_{U_x}:p_2^{-1}(U_x)\to U_x\times F_2$ making the following diagrams commute:
\begin{center}
\begin{tikzcd}
p_1^{-1}(U_x)\ar[d,"p_1|_{p_1^{-1}(U_x)}"']\ar[r,"(f_1)_{U_x}"]&U_x\times F_1\ar[dl,"\pi_1"]&&
p_2^{-1}(U_x)\ar[d,"p_2|_{p_2^{-1}(U_x)}"']\ar[r,"(f_2)_{U_x}"]&U_x\times F_2\ar[dl,"\pi_1"]\\
U_x&&&U_x
\end{tikzcd}
\end{center}
Let $\mathcal{U}=\{U_x:x\in B\}$. Since $\{p_2^{-1}(U):U\in\mathcal{U}\}$ forms an open cover of $E_2$, it suffices to show that $\phi^{-1}|_{p_2^{-1}(U)}$ is continuous for every $U\in\mathcal{U}$ and then apply the glueing lemma. Let us fix such a $U\in\mathcal{U}$. Then, the above commutative diagrams let us define the map $\psi:U\times F\to U\times F$ as the following composition:
\vspace*{-1em}
\begin{center}
\begin{tikzcd}
U\times F\ar[r,"\cong"']\ar[drrr,"\pi_1"',near start]\ar[rrrrrr,bend left=10,dotted,"\psi"]&U\times F_1\ar[r,"\cong"',"(f_1)_U^{-1}"]\ar[drr,"\pi_1"',near start]&p_1^{-1}(U)\ar[rr,"\phi|_{p_1^{-1}(U)}"]\ar[dr,"p_1"',near start]&[-2em]&p_2^{-1}(U)\ar[r,"\cong"',"(f_2)_U"]\ar[dl,"p_2",near start]&U\times F_2\ar[r,"\cong"']\ar[dll,"\pi_1",near start]&U\times F\ar[dlll,"\pi_1",near start]\\[2em]
&&&U
\end{tikzcd}
\end{center}
This means that the map $\psi$ has the following properties:
\begin{itemize}
\item $\psi$ is continuous and a bijection.
\item For every $x\in U$ there exists some map $\psi_x:F\to F$ such that $\psi(x,a)=(x,\psi_x(a))$. These $\psi_x$ are homeomorphisms, since the following diagram commutes for every $x\in U$:
\vspace*{-1em}
\begin{center}
\begin{tikzcd}
F\ar[r,"\cong"']\ar[rrrrr,bend left=15,"\psi_x"]&\{x\}\times F_1\ar[r,"\cong"']&(F_1)_x\ar[r,"\cong"',"\phi_x"]&(F_2)_x\ar[r,"\cong"']&\{x\}\times F_2\ar[r,"\cong"']&F
\end{tikzcd}
\end{center}
\item $\phi^{-1}|_{p_2^{-1}(U)}=(\phi|_{p_1^{-1}(U)})^{-1}$ is continuous if and only if $\psi^{-1}$ is continuous.
\end{itemize}
Essentially, we have reduced the problem to the case of $E_1,E_2$ both being the trivial bundle $U\times F$ and $\psi:U\times F\to U\times F$.

Now, we follow the same arguments we used in the Example~\ref{ex:counterexample}. First, define the continuous $g_{\psi}:U\times F\to F$ with $g_{\psi}=\pi_2\circ\psi$, i.e. $g_{\psi}(x,a)=\psi_x(a)$. Since $F$ is Hausdorff and locally compact, the space $\mathrm{Top}(F,F)$, topologised with the compact-open topology, together with the map $ev:\mathrm{Top}(F,F)\times F\to F$ has the final property proved in Proposition~\ref{prop:exponential_object}. So, there exists a unique continuous $g_{\psi}^{\#}:U\to\mathrm{Top}(F,F)$ making the following diagram commute:
\begin{center}
\begin{tikzcd}
U\times F\ar[r,"g_{\psi}^{\#}\times 1_F"]\ar[d,"g_{\psi}"']&\mathrm{Top}(F,F)\times F\ar[dl,"ev"]\\
F
\end{tikzcd}
\end{center}
Notice that $g_{\psi}^{\#}(x)=\psi_x$, since this makes the above diagram commute and $g_{\psi}^{\#}$ is unique. This also means in particular that $g_{\psi}^{\#}(U)\subseteq\mathrm{Hom}(F)$. Since $F$ is additionally locally connected, the map $(-)^{-1}:\mathrm{Hom}(F)\to\mathrm{Hom}(F)$ is continuous, as we proved in Theorem~\ref{thm:inverse}, where $\mathrm{Hom}(F)\subseteq\mathrm{Top}(F,F)$ is equipped with the subspace topology. Thus, we have a continuous map $h_{\psi}:U\to\mathrm{Top}(F)$ defined by the following composition:
\vspace*{-1em}
\begin{center}
\begin{tikzcd}
U\ar[r,"g_{\psi}^{\#}"]\ar[rrr,bend left=15, dotted, "h_{\psi}"]&\mathrm{Hom}(F)\ar[r,"(-)^{-1}"]&\mathrm{Hom}(F)\ar[r,"i",hook]&\mathrm{Top}(F,F)
\end{tikzcd}
\end{center}
i.e. $h_{\psi}(x)=(g_{\psi}^{\#}(x))^{-1}=\psi^{-1}_x$. This lets us define next the continuous map $h_{\psi}^{\flat}:U\times F\to F$, as the following composition:
\begin{center}
\begin{tikzcd}
U\times F\ar[r,"h_{\psi}\times 1_F"]\ar[d,"h_{\psi}^{\flat}"']&\mathrm{Top}(F,F)\times F\ar[dl,"ev"]\\
F
\end{tikzcd}
\end{center}
Notice that $h_{\psi}^{\flat}(x,a)=ev(\psi^{-1}_x,a)=\psi^{-1}_x(a)$. By the universal property of the product, we now get a unique continuous map $\left<\pi_1,h_{\psi}^{\flat}\right>:U\times F\to U\times F$ as in the diagram:
\begin{center}
\begin{tikzcd}
U&U\times F\ar[l,"\pi_1"']\ar[r,"\pi_2"]&F\\[1em]
&U\times F\ar[ul,"\pi_1"]\ar[u,dotted,"\left<\pi_1{,}h_{\psi}^{\flat}\right>" description,background color=none]\ar[ur,"h_{\psi}^{\flat}"']
\end{tikzcd}
\end{center}
where $\left<\pi_1,h_{\psi}^{\flat}\right>(x,a)=\big(x,\psi_x^{-1}(a)\big)=\psi^{-1}(x,a)$, for every $(x,a)\in U\times F$, which proves finally that $\psi^{-1}$ is continuous.
\end{proof}





\begin{example} Define the topological spaces $B=(\mathbb{R}, \tau)$ and $B'=(\mathbb{R}, \tau')$, where $\tau$ is the usual topology on $\mathbb{R}$ generated by the open intervals and $\tau'$ is the topology on $\mathbb{R}$ generated by the semi-open intervals, i.e. of all intervals of the form $[a,b)$. Then examine the following diagram:
\begin{center}
\begin{tikzcd}
B'\ar[rd,"p_1\ =\ id_{\mathbb{R}}"']\ar[rr,"\phi\ =\ id_{\mathbb{R}}"]&[-1.5em]&[-1.5em]B\ar[ld,"p_2\ =\ id_{\mathbb{R}}"]\\
&B
\end{tikzcd}
\end{center}
\end{example}

Now that we have defined what an isomorphism of two fiber bundles is, let us examine what it means for a fiber bundle to be isomorphic to the trivial bundle.

\begin{remark}
Let $p:E\to B$ be a fiber bundle with fiber $F$. Then, $p$ is isomorphic to a trivial fiber bundle, if and only if there exists a map
\[\phi:F\times B\to E\]
such that, for every $x\in B$:
\begin{i_enum}
\item $p\circ\phi(a,x)=x$ (i.e. $\phi$ is a bundle map), and
\item The restriction map $p_x:F\times\{x\}\to p^{-1}(\{x\})$ is an homeomorphism. (i.e. $\phi$ is a bundle isomorphism).
\end{i_enum}
\end{remark}



%%%%TODO
%- pullback





%---------
%%%TODO: build proof environment for equivalences
%
%
%%%TODO: check if lex order is used correctly
%
%
%

	\end{myappendix}
\backmatter
	\bibliographystyle{alpha}
	\bibliography{bibliography/bibsample}
\end{document}
