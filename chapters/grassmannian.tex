%!TEX root = Cohomology of real Grassmannians.tex
\chapter{Grassmannians}\label{chap:grassmannians}
Since the topology of projective spaces has been thoroughly studied and characterized, a logical generalization is imposing a natural topology on the set of $k$-dimensional subspaces of a vector space for any $k\geq1$, which are called a \emph{Grassmannian}.

Recall the definition of the real projective spaces as topological spaces,
$P\mathbb{R}^n\cong\big(\mathbb{R}^{n+1}\setminus\{0\}\big)/\sim$,
where two vectors are equivalent, if they span the same line. Notice that in this definition a proper subset of the whole vector space is required, for the quotient to be well defined. Namely the set of all vectors, which span a line. In accordance to that, for the case of Grassmannians, the discussion starts at the space of all $k$-tuples of vectors in $\mathbb{R}^n$, spanning a $k$-dimensional vector space.

\section{Stiefel Manifolds}
\begin{definition} For $k,n\in\mathbb{N}$ such that $k\leq n$, the space
\[\St{k}{n}:=\big\{\left(v_1,\ldots,v_k\right)\in{\left(\mathbb{R}^n\right)}^k:\dim_{\mathbb{R}}\spans\{v_1,\ldots,v_k\}=k\big\}\]
equipped with the subspace topology is the \emph{non-compact Stiefel manifold}. Every element in $\St{k}{n}$ is called a \emph{$k$-frame of $\mathbb{R}^n$}.
\end{definition}
\begin{proposition}\label{prop:St_open} For $k,n\in\mathbb{N}$ such that $k\leq n$, $\St{k}{n}$ is open in ${\left(\mathbb{R}^n\right)}^k$. In particular, $\St{k}{n}$ is a real manifold of dimension $kn$.
\end{proposition}
\begin{proof} Define the homeomorphism $\phi:{\left(\mathbb{R}^n\right)}^k\to \mathbb{R}^{n\times k}$ with $\phi(v_1,\ldots,v_k)=(v_1|\cdots|v_k)$, where $\mathbb{R}^{n\times k}$ is equipped with the usual topology of $\mathbb{R}^{nk}$. Let $D:=\big\{A\in\mathbb{R}^{n\times k}:\exists I\in\binom{[n]}{k}\ \det(A_I)\neq0\big\}$, where $A_I\in\mathbb{R}^{k\times k}$ is the $k\times k$ submatrix of $A$ formed by the $k$ rows indexed by $I$ for any $I\in\binom{[n]}{k}$. Notice that $\St{k}{n}=\phi^{-1}(D)\cong D$ and $D$ is open in $\mathbb{R}^{n\times k}$, so is $\St{k}{n}$ in ${\left(\mathbb{R}^n\right)}^k$.
\end{proof}

\begin{definition} For $k,n\in\mathbb{N}$ such that $k\leq n$, the space $\StO{k}{n}:=\big\{\left(v_1,\ldots,v_k\right)\in{\left(\mathbb{R}^n\right)}^k:v_i^tv_j=\delta_{i,j}\big\}$ equipped with the subspace topology is the \emph{Stiefel manifold}. Every element in $\StO{k}{n}$ is called an \emph{orthonormal $k$-frame of $\mathbb{R}^n$}.
\end{definition}

\begin{proposition}\label{prop:StO_dim_closed} For $k,n\in\mathbb{N}$ such that $k\leq n$, $\StO{k}{n}$ is closed in ${\left(\mathbb{R}^n\right)}^k$. Moreover, $\StO{k}{k}$ is a real manifold of dimension $nk-\frac{k^2}{2}-\frac{k}{2}$.
\end{proposition}
\begin{proof} Let $\phi:{\left(\mathbb{R}^n\right)}^k\to\mathbb{R}^{n\times k}$ be the homeomorphism $\phi(v_1,\ldots,v_k)=(v_1|\cdots|v_k)$. Moreover, let $S:=\left\{A\in\mathbb{R}^{n\times k}:A^t A=I_k\right\}$ be the set of all $n\times k$ semi-orthogonal matrices. Notice that $S$ is closed in $\mathbb{R}^{n\times k}$ and thus $\StO{k}{n}=\Phi^{-1}(S)\cong S$ is also closed in ${\left(\mathbb{R}^n\right)}^k$.

The set of equations $E=\{v_i^tv_j=0:i,j\in[k]\text{ with }i<j\}\cup\{v_i^tv_i=1:i\in[k]\}$ on $nk$ variables is linearly independent, since for any subset $R$ of them, it is clear that there exist $k$-tuples in $\mathbb{R}^n$ satisfying $R$ and not $E\setminus R$. Each remaining restriction in the definition of $\StO{k}{n}$, i.e.\ that $v_i^tv_j=0$ for every $i,j\in[k]$ with $i>j$ is linearly dependent on the ones in $E$, since $u^tv=v^tu$. This means that $\StO{k}{n}$ is an $nk-\frac{k^2}{2}-\frac{k}{2}$ dimensional manifold.
\end{proof}

\begin{lemma}\label{lem:StO_compact} For $k,n\in\mathbb{N}$ such that $k\leq n$, $\StO{k}{n}$ is compact.
\end{lemma}
\begin{proof} Due to Proposition~\ref{prop:StO_dim_closed}, $\StO{k}{n}$ is a closed subset of ${\left(\mathbb{R}^n\right)}^k\cong\mathbb{R}^{kn}$. Moreover, notice that $\StO{k}{n}$ is bounded in ${\left(\mathbb{R}^n\right)}^k$, with the usual metric of $\mathbb{R}^{kn}$. Indeed, for every $f=(v_1,\ldots,v_k)\in\StO{k}{n}$, $\left\|f\right\|_2^2=\sum_{i=1}^k\left\|v_i\right\|_2^2=k$. Hence, $\StO{k}{n}$ is compact.
\end{proof}

\begin{definition} For $k,n\in\mathbb{N}$ such that $k<n$, the topological space $\Gr{k}{n}:=\big(\St{k}{n}{\mathbb{R}}\big)/\sim$, where $(v_1,\ldots,v_k)\sim(u_1,\ldots,u_k)$, if $\spans\{v_1,\ldots,v_k\}=\spans\{u_1,\ldots,u_k\}$, equipped with the quotient topology is the \emph{Grassmann space}. Every element $H\in\Gr{k}{n}$ is called an \emph{$k$-plane in $\mathbb{R}^n$}.
\end{definition}

\begin{lemma}\label{lem:q_open} For $k,n\in\mathbb{N}$ such that $k<n$, the quotient map $q:\St{k}{n}\to\Gr{k}{n}$ is open.
\end{lemma}
\begin{proof} Let $U\subseteq\St{k}{n}$ be an open set in $\St{k}{n}$, then $q(U)$ is open in $\Gr{k}{n}$ if and only if $q^{-1}(q(U))$ is open in $\St{k}{n}$, since $q$ is a quotient map. Moreover, $q^{-1}(q(U))$ is open in $\St{k}{n}$ if and only if it is open in $(\mathbb{R}^{n})^k$, since $\St{k}{n}$ is open in $(\mathbb{R}^{n})^k$. This means that it suffices to prove that $\tilde{U}:=q^{-1}(q(U))$ is open in $(\mathbb{R}^{n})^k$. In order to do this, we first construct a linear isomorphism $f_L:(\mathbb{R}^n)^k\to(\mathbb{R}^n)^k$ for every $L\in GL(k)$ and then we show that
$\tilde{U}=\bigcup_{L\in GL(k)}f_L(U)$.

For every $(x_1,\ldots,x_k)\in(\mathbb{R}^n)^k$ let $R_{(x_1,\ldots,x_k)}:\mathbb{R}^n\to\mathbb{R}^k$ and $S_{(x_1,\ldots,x_k)}:\mathbb{R}^k\to\mathbb{R}^n$ be the linear maps such that $R_{(x_1,\ldots,x_k)}(x_i)=e_i$ for every $i\in[k]$, $R_{(x_1,\ldots,x_k)}(v)=0$ for every $v\in\spans\{x_1,\ldots,x_k\}^{\perp}$ and $S_{(x_1,\ldots,x_k)}(e_i)=x_i$ for every $i\in[k]$, where $e_1,\ldots,e_k$ is the usual basis of $\mathbb{R}^k$. Then, for every $L\in GL(k)$, define
\begin{center}
\begin{tikzcd}
f_L\ar[r,phantom,":"]&[-3.5em](\mathbb{R}^n)^k\ar[r,"R_{(-)}"]&\mathbb{R}^k\ar[r,"L^{\oplus k}"]&\mathbb{R}^k\ar[r,"S_{(-)}"]&(\mathbb{R}^n)^k\\[-1.5em]
&(x_1,\ldots,x_k)\ar[r,mapsto]&(e_1,\ldots,e_k)\ar[r,mapsto]&(Le_1,\ldots,Le_k)\ar[r,mapsto]&\left(\sum_{j=1}^ke_i^t(Le_j)x_j\right)_{i\in[k]}
\end{tikzcd}
\end{center}
which clearly is linear for every fixed $L$. Moreover, notice that $\big(\sum_{j=1}^ke_i^t(Le_j)x_j\big)_{i\in[k]}=0$ can be written as $\big(x_1|\cdots|x_k\big)\cdot\big(e_i^t(Le_j)\big)=0$. Since $\big(e_i^t(Le_j)\big)_{i,j\in[k]}$ is the the matrix of the invertible $L$ on the basis $e_1,\ldots,e_k$, $x_1=\cdots=x_k=0$. So, $f_L$ is an injective linear endomorphism and thus a linear isomorphism.

Let $(u_1,\ldots,u_k)\in\tilde{U}$. Then there exists $(x_1,\ldots,x_k)\in U$ with $\spans\{x_1,\ldots,x_k\}=\spans\{u_1,\ldots,u_k\}$. Thus, there exist $\{\lambda_{i,j}\}_{i,j\in[k]}\in\mathbb{R}$ such that $u_i=\sum_{j=1}^k\lambda_{i,j}x_j$ for every $i\in[k]$. Define $L_0\in GL(k)$ by $L_0e_j:=\sum_{i=1}^k\lambda_{i,j}e_i$ for every $j\in[k]$. Notice that $f_{L_0}(x_1,\ldots,x_k)=\big(\sum_{j=1}^ke_i^t(L_0e_j)x_j\big)_{i\in[k]}=\big(\sum_{j=1}^k\lambda_{i,j}x_j\big)_{i\in[k]}=(u_1,\ldots,u_k)$, i.e.\ $(u_1,\ldots,u_k)\in f_{L_0}(U)$. Moreover, let $(u_1,\ldots,u_k)\in f_{L_0}(U)$ for some $L_0\in GL(k)$. Then there exists some $(x_1,\ldots,x_k)\in U$ with $(u_1,\ldots,u_k)=f_{L_0}(x_1,\ldots,x_k)=\big(\sum_{j=1}^ke_i^t(L_0e_j)x_j\big)_{i\in[k]}$. Define $\lambda_{i,j}:=e_i^tL_0e_j$ for every $i,j\in[k]$. This means that $u_i=\sum_{j=1}^k\lambda_{i,j}x_j$ for every $i$, and hence $\spans\{x_1,\ldots,x_k\}=\spans\{u_1,\ldots,u_k\}$, i.e.\ $(u_1,\ldots,u_k)\in\tilde{U}$. This proves that $\tilde{U}=\bigcup_{L\in GL(k)}f_L(U)$ and thus that it is open.
\end{proof}

Having the case of projective spaces in mind, it is natural to also provide an alternative definition of Grassmannians.
Specifically, the projective space of some dimension is also defined as the quotient over the unit sphere, rather than over the set of every nonzero vector.
In this case the Stiefel manifold $\StO{k}{n}$, is the analog needed.

\begin{proposition} For $k,n\in\mathbb{N}$ such that $k<n$, let $q_0:=q|_{\StO{k}{n}}:\StO{k}{n}\to\Gr{k}{n}$ be the restriction of the quotient map $q:\St{k}{n}\to\Gr{k}{n}$. Then $q_0$ is surjective and continuous. In other words, $q_0$ and $q$ induce the same quotient topology on the set of all $k$-planes in $\mathbb{R}^n$.
\end{proposition}
\begin{proof}
Let $i:\StO{k}{n}\to\St{k}{n}$ be the inclusion and $\mathfrak{gs}:\St{k}{n}\to\StO{k}{n}$ be the Gram-Schmidt process. Then, the diagram
\begin{center}
\begin{tikzcd}
\StO{k}{n}\ar[r,hook,"i"]\ar[rd,"q_0"']&\St{k}{n}\ar[r,"\mathfrak{gs}"]\ar[d,two heads,"q"']&\StO{k}{n}\ar[dl,"q_0"]\\
&\Gr{k}{n}
\end{tikzcd}
\end{center}
commutes. The left triangle implies $q_0=q\circ i$, and hence that $q_0$ is continuous. Moreover, from right triangle, $q=q_0\circ\mathfrak{gs}$, which results in two conclusions. First that $q_0$ is a surjective map, since $q$ is surjective and second that $q_0$ is an open map, since $\mathfrak{gs}$ is surjective and $q$ is open, as proved in \ref{lem:q_open}. Thus, the map $q_0$ is a quotient map.
\end{proof}

\section{Infinite Grassmannian}
So far, we presented the Grassmannians separately for each $k$ and $n$. In fact, there is a natural embedding of $\Gr{k}{n}$ into $\Gr{k}{n+1}$, since every $k$-plane in $\mathbb{R}^n$ is also a $k$-plane in $\mathbb{R}^{n+1}$. This naturally leads to the construction of a so called infinite Grassmannian for each $k$.

\begin{definition} Let $(X,\tau)$ be a topological space and $\{X_i\}_{i\in\mathbb{N}}$ a collection of subsets of $X$, such that $X_1\subseteq X_2\subseteq X_3\subseteq\cdots$. Then, $X$ is the \emph{direct limit} of $X_1,X_2,\ldots$ and $\tau$ the \emph{direct limit topology on $X$}, if $X=\bigcup_{i=1}^{\infty}X_i$ and $\tau$ is the final topology with respect to all inclusions. This is denoted by $X=\varinjlim X_i$.
\end{definition}
\begin{remark} $X=\varinjlim X_i$ means that for every $x\in X$ there exists some $i\in\mathbb{N}$ with $x\in X_i$ and also for any subset $A\subseteq X$, $A$ is open in $X$ if and only if $A\cap X_i$ is open in $X_i$ for every $i\in\mathbb{N}$.
\end{remark}

\begin{definition} Let $\mathbb{R}^1\subseteq\mathbb{R}^2\subseteq\cdots$ be the usual inclusions and let $\mathbb{R}^{\infty}:=\bigcup_{i=1}^{\infty}\mathbb{R}^i=\big\{(x_1,x_2,\ldots):\exists i\in\mathbb{N}\text{ s.t. }\forall n\geq i\ x_n=0\big\}$. Then $\mathbb{R}^{\infty}$ equipped with the direct limit topology and with the vector space structure inherited by the inclusions is the \emph{infinite coordinate space}.
\end{definition}
\begin{remark} Notice that $\mathbb{R}^{\infty}\cong\bigoplus_{i=1}^{\infty}\mathbb{R}\not\cong\prod_{i=1}^{\infty}\mathbb{R}\cong l_2$ as vector spaces, since $\mathbb{R}^{\infty}$ does not contain infinite sequences.
\end{remark}

\begin{lemma} For $k,n\in\mathbb{N}$ such that $k<n$, there exist embeddings $i_{k,n}:\St{k}{n}\hookrightarrow\St{k}{n+1}$ and $i^0_{k,n}:\StO{k}{n}\hookrightarrow\StO{k}{n+1}$ induced by $inc_{k,n}:(\mathbb{R}^n)^k\hookrightarrow(\mathbb{R}^{n+1})^k$ with $inc_{k,n}(u_1,\ldots,u_k)=(\bar{u}_1,\ldots,\bar{u}_k)$ and $\bar{u}_i=(u_i^t|0)^t$ for every $i\in[k]$.
\end{lemma}
\begin{proof} Let $(u_1,\ldots,u_k)\in\St{k}{n}$. Then $\dim_{\mathbb{R}}\spans\{\bar{u}_1,\ldots,\bar{u}_k\}=k$, which means $inc_{k,n}(u_1,\ldots,u_k)\in\St{k}{n}$, so $i_{k,n}=inc_{k,n}|_{\St{k}{n}}:\St{k}{n}\to\St{k}{n+1}$ is well defined. Let $A$ be some open subset of $\St{k}{n}$, i.e.\ there exists some open $B\subseteq(\mathbb{R}^n)^k$ such that $A=B\cap\St{k}{n}$. Since $inc_{k,n}$ is an embedding, there exists some open $C\subseteq(\mathbb{R}^{n+1})^k$ such that $B=inc_{k,n}^{-1}(C)$, i.e.\ $A=inc_{k,n}^{-1}(C)\cap\St{k}{n}=inc_{k,n}^{-1}(C\cap\St{k}{n+1})=i_{k,n}(C\cap\St{k}{n+1})$ which proves that $i_{k,n}$ is an embedding. The same arguments prove that $i_{k,n}^0$ is an embedding.
\end{proof}

\begin{proposition}\label{prop:gr_embedding} For $k,n\in\mathbb{N}$ such that $k<n$, there is an embedding $\iota_{k,n}:\Gr{k}{n}\hookrightarrow\Gr{k}{n+1}$ making the following diagram commute.
\begin{center}
\begin{tikzcd}
\St{k}{n}\ar[d,two heads,"q"]\ar[r,hook,"i_{k,n}"]&\St{k}{n+1}\ar[d,two heads,"q"]\\
\Gr{k}{n}\ar[r,hook,dotted,"\iota_{k,n}"]&\Gr{k}{n+1}
\end{tikzcd}
\end{center}
\end{proposition}
\begin{proof} Let $(u_1,\ldots,u_k),(v_1,\ldots,v_k)\in\St{k}{n}$ such that $\spans\{u_1,\ldots,u_k\}=\spans\{v_1,\ldots,v_k\}$. Then $\spans\{\bar{u}_1,\ldots,\bar{u}_k\}=\spans\{\bar{v}_1,\ldots,\bar{v}_k\}$, where $\bar{u}_i=(u^t|0)^t$. Since $q\circ i_{k,n}(u_1,\ldots,u_k)$ depends only on $q(u_1,\ldots,u_k)$, the universal property of the quotient ensures the existence of a unique $\iota_{k,n}:\Gr{k}{n}\to\Gr{k}{n+1}$ making the diagram commute. It remains to show that for every open $A\subseteq\Gr{k}{n}$ there exists some open $C\subseteq\Gr{k}{n+1}$ such that $A=\iota_{k,n}^{-1}(C)$. For any open $A\subseteq\Gr{k}{n}$, $q^{-1}(A)$ is an open subset of $\St{k}{n}$. Since $i_{k,n}$ is an embedding, there exists some open $B\subseteq\St{k}{n+1}$ with $q^{-1}(A)=i_{k,n}^{-1}(B)$. Since $q$ is an open map, as proved in Lemma~\ref{lem:q_open}, $q(B)$ is also open, which makes $\iota_{k,n}^{-1}(q(B))\subseteq\Gr{k}{n}$ open. It suffices now to show that $A=\iota_{k,n}^{-1}(q(B))$.

Let $H\in A$, then $H=q(u_1,\ldots,u_k)$ for some $(u_1,\ldots,u_k)\in q^{-1}(A)=i_{k,n}^{-1}(B)$. Which means that $(\bar{u}_1,\ldots,\bar{u}_k)\in B$, and by the commutativity, $\iota_{k,n}(H)=(\iota_{k,n}\circ q)(u_1,\ldots,u_k)=q(\bar{u}_1,\ldots,\bar{u}_k)\in q(B)$, i.e.\ $H\in\iota_{k,n}^{-1}(q(B))$. For the other direction, let $H\in\iota_{k,n}^{-1}(q(B))$. This means that there exists some $(a_1,\ldots,a_k)\in\St{k}{n+1}$ with $\iota_{k,n}(H)=\spans\{a_1,\ldots,a_k\}$. Since $q$ is a surjection, there exists some $(u_1,\ldots,u_k)\in\St{k}{n}$ such that $\spans\{u_1,\ldots,u_k\}=H$. Because of the commutativity of the diagram, $\spans\{a_1,\ldots,a_k\}=\spans\{\bar{u}_1,\ldots,\bar{u}_k\}$. Notice now that this can only be the case if the last coordinate of each $a_i$ is equal to $0$, i.e.\ if there exist some $(x_1,\ldots,x_k)\in\St{k}{n}$ with $\bar{x}_i=a_i$ for every $i$. Then $\spans\{\bar{u}_1,\ldots,\bar{u}_k\}=\spans\{\bar{x}_1,\ldots,\bar{x}_k\}$ which can only be the case if $\spans\{u_1,\ldots,u_k\}=\spans\{x_1,\ldots,x_k\}$. Also, because of the definition of $B$, $(x_1,\ldots,x_k)\in i_{k,n}^{-1}(B)=q^{-1}(A)$ and thus we finally get $H=\spans\{u_1,\ldots,u_k\}=\spans\{x_1,\ldots,x_k\}\in A$.
\end{proof}
\begin{remark} Whenever $\Gr{k}{n}\subseteq\Gr{k}{n+1}$ is written, the embedding $\iota_{k,n}$ is implied. As described in \ref{prop:gr_embedding}, $\iota_{k,n}$ takes a $k$-plane $H\subseteq\mathbb{R}^n$ to $H\subseteq\mathbb{R}^{n+1}$ by embedding $\mathbb{R}^n$ into $\mathbb{R}^{n+1}$ the usual way. Similarly for $\St{k}{n}\subseteq\St{k}{n+1}$ and $\StO{k}{n}\subseteq\StO{k}{n+1}$.
\end{remark}

\begin{definition} For $k\in\mathbb{N}$, $\Gr{k}:=\bigcup_{n=k+1}^{\infty}\Gr{k}{n}$ equipped with the direct limit topology is the \emph{infinite Grassmannian}.
\end{definition}

\begin{definition} For $k\in\mathbb{N}$, $\St{k}:=\bigcup_{n=k+1}^{\infty}\St{k}{n}$ equipped with the direct limit topology is the \emph{infinite non-compact Stiefel manifold}. Also, $\StO{k}:=\bigcup_{n=k+1}^{\infty}\StO{k}{n}$ equipped with the direct limit topology is the \emph{infinite Stiefel manifold}.
\end{definition}
\begin{remark}\label{rem:st_inclusion} Notice that $\StO{k}\subseteq\St{k}$, since for $(u_1,\ldots,u_k)\in\StO{k}$, there exists some $n\in\mathbb{N}$ such that $(u_1,\ldots,u_k)\in\StO{k}{n}\subseteq\St{k}{n}\subseteq\St{k}$.
\end{remark}

\begin{proposition} Let $k\in\mathbb{N}$ and also let $q:\St{k}\to\Gr{k}$ be the map induced by all the inclusions, i.e.\ $q(u_1,\ldots,u_k)=q|_{\St{k}{n}}(u_1,\ldots,u_k)$, for some $n$ sufficiently large. Moreover, let $q_0:\StO{k}\to\Gr{k}$ be the restriction $q|_{\StO{k}}$. Then, $\Gr{k}\cong{\St{k}/\sim}\cong{\StO{k}/\sim}$, where the equivalence relation is the same as in $\St{k}{n}$ for some large enough $n$.
\end{proposition}
\begin{proof} First of all, $q$ is a surjection. Indeed, for $H\in\Gr{k}{n}$, there exists some $(u_1,\ldots,u_k)\in\St{k}{n}\subseteq\St{k}$ with $q(u_1,\ldots,u_k)=H$. Next, in order to prove that $q$ is in fact a quotient map, let $U\subseteq\Gr{k}$ such that $q^{-1}(U)$ is open in $\St{k}$. We want to prove that $U$ is open in $\Gr{k}$. For that, we will use the definition of the direct limit topology and fix some $n\in\mathbb{N}$ with $n>k$. Then, since $\St{k}$ is equipped with the direct limit topology and $q^{-1}(U)$ is open, $q^{-1}(U)\cap\St{k}{n}=q^{-1}(U\cap\Gr{k}{n})$ is open as well. Since $q$ is a quotient map, this means that $U\cap\Gr{k}{n}$ is open as well. Hence, $U\cap\Gr{k}{n}$ is open for every $n>k$ and thus $U$ is open in $\Gr{k}$. For the case of $q_0:\StO{k}\to\Gr{k}$ notice that the map is well defined, since Remark~\ref{rem:st_inclusion} guaranties that $\StO{k}\subseteq\St{k}$. Then, the arguments are exactly the same as in the previous case.
\end{proof}
This proves that the infinite Grassmannian is the set of all $k$-planes inside $\mathbb{R}^{\infty}$. Notice that these planes never contain a vector with infinitely many non-zero entries, since such vectors do not exist inside $\mathbb{R}^{\infty}$.

\section{Grassmannians are Manifolds}
\begin{lemma} For $k,n\in\mathbb{N}$, such that $k<n$, $\Gr{k}{n}$ is a compact Hausdorff space.
\end{lemma}
\begin{proof} As we saw in Lemma~\ref{lem:StO_compact}, $\StO{k}{n}$ is compact and thus $\Gr{k}{n}=q_0\left(\StO{k}{n}\right)$ is also compact, since $q_0$ is continuous. In order to show that $\Gr{k}{n}$ is Hausdorff, it suffices to show that it is completely Hausdorff, i.e.\ that any two distinct points in $\Gr{k}{n}$ can be separated by a continuous function $\Gr{k}{n}\to\mathbb{R}$. For every $v\in\mathbb{R}^n$, we define $\phi_v:\StO{k}{n}\to\mathbb{R}$ such that $\phi_v(v_1,\ldots,v_k)=v^tv-\sum_{i=1}^k{\left(v^tv_i\right)}^2$. This is a continuous map, which depends only on the spanned $k$-plane. The universal property of the quotient map $q_0$ implies that there exists a unique continuous map $\psi_v:\Gr{k}{n}\to\mathbb{R}$ such that $\psi_v\circ q_0=\phi_v$. Notice that $\psi_v(H)=0$ if and only if $v\in H$, since $\phi_v(v_1,\ldots,v_k)=0$ if and only if $v\in \spans\{v_1,\ldots,v_k\}$. For $H_1,H_2\in\Gr{k}{n}$ with $H_1\neq H_2$, let $v\in H_1\setminus H_2$ and notice that $\psi_v(H_1)=0\neq\psi_v(H_2)$.
\end{proof}
\begin{proposition}\label{prop:gr_manifold} For $k,n\in\mathbb{N}$, such that $k<n$, $\Gr{k}{n}$ is locally homeomorphic to $\mathbb{R}^{k(n-k)}$.
\end{proposition}
\begin{proof} For every $H\in\Gr{k}{n}$, we define $U_H:=\left\{K\in\Gr{k}{n}:K\cap H^{\perp}=\{0\}\right\}\subseteq\mathbb{R}^n$. Notice that $U_H$ is an open neighborhood of $H$. Let $\{u_1,\ldots,u_k\}$ be an orthonormal basis of $H$ and $\{\bar{u}_1,\ldots,\bar{u}_{n-k}\}$ an orthonormal basis of $H^{\perp}$. Also, let $p_H:\mathbb{R}^n\cong H\oplus H^{\perp}\to H$ and $p_{H^{\perp}}:\mathbb{R}^n\cong H\oplus H^{\perp}\to H^{\perp}$ be the orthogonal projections. Notice that $U_H=\left\{K\in\Gr{k}{n}:p_H|_K:K\to H\text{ is isomorphism}\right\}$. Indeed, $\dim_{\mathbb{R}}(K\cap H^{\perp})>0$ if and only if $p_H|_K$ is not an isomorphism.

Denote with $\Hom_{\mathbb{R}}(H,H^{\perp})$ the set of all linear transformations from $H$ to $H^{\perp}$, topologised with the induced topology from the map $\Hom_{\mathbb{R}}(H,H^{\perp})\to\mathbb{R}^{n(n-k)}$ mapping $L$ to $(\bar{u}_j^tLu_i)_{(i,j)\in[k]\times[n-k]}$. Our goal is to define a homeomorphism $T:U_H\to\Hom_{\mathbb{R}}(H,H^{\perp})$. For that, we first define $\tilde{T}:q^{-1}(U_H)\to\Hom_{\mathbb{R}}\left(H,H^{\perp}\right)$ with $\tilde{T}(a_1,\ldots,a_k):=(p_{H^{\perp}}|_{\spans\{a_1,\ldots,a_k\}})\circ(p_H|_{\spans\{a_1,\ldots,a_k\}})^{-1}\in\Hom_{\mathbb{R}}(H,H^{\perp})$. Clearly $\tilde T$ is well defined, since $q^{-1}(U_H)$ is exactly the set of all $k$-frames, for which the map $p_H|^{-1}$ exists. Notice that the projection maps involve the vectors $(a_1,\ldots,a_k)$ only in inner products and linear combinations, which means that for every $i\in[k]$ and every $j\in[n-k]$ the map $\tilde{T}_{i,j}:q^{-1}(U_H)\to\mathbb{R}$ with $\tilde{T}_{i,j}(a_1,\ldots,a_k):=\bar{u}_j^t\tilde{T}(a_1,\ldots,a_k)u_i$ is continuous, i.e.\ $\tilde{T}$ is also continuous.

Since $\tilde{T}(a_1,\ldots,a_k)$ depends only on $\spans\{a_1,\ldots,a_k\}$, the universal property of the quotient map ensures the existence of a unique continuous map $T:U_H\to\Hom_{\mathbb{R}}(H,H^{\perp})$ such that $T\circ q|=\tilde T$, which means that $T(K)=(p_{H^{\perp}}|_K)\circ{(p_H|_K)}^{-1}$ is continuous. Notice that for every $K\in q^{-1}(U_H)$, the graph of $T(K)$ in $H\oplus H^{\perp}$ is exactly $K$, which means that $T$ is a bijection. Hence, there exists a homeomorphism $U_H\cong\mathbb{R}^{n(n-k)}$.
\end{proof}
In the next section we are going to further examine these spaces topologically and build the appropriate language in order to tackle problems regarding their Homology structure. Before we dive in into this topic though, it would be useful to notice a duality between these spaces, arising from the duality between a $k$-space and its $n-k$ complement inside $\mathbb{R}^n$.

\begin{lemma} \label{lem:gr_duality} For $k,n\in\mathbb{N}$ such that $k<n$, $\Gr{k}{n}\cong\Gr{n-k}{n}$, with the homeomorphism taking some $k$-space $H$ to $H^{\perp}$.
\end{lemma}

\begin{proof} The function $T:\Gr{k}{n}\to\Gr{n-k}{n}$ taking $H$ to $H^{\perp}$ is clearly bijective and its own inverse and thus, it suffices to show that it is continuous. Let $H\in\Gr{k}{n}$, $U_H=\{K\in\Gr{k}{n}:K\cap H^{\perp}=\{0\}\}$ be the open set defined in the proof of Lemma~\ref{prop:gr_manifold}, $\{\bar{u}_1,\ldots,\bar{u}_k\}$ a basis of $H^{\perp}$ and $p_{H^{\perp}}:\mathbb{R}^n\to H^{\perp}$ the orthogonal projection. Then, let $\tilde{T}_H:q_0^{-1}(U_H)\to\Gr{n-k}{n}$ such that $\tilde{T}_H(a_1,\ldots,a_k)=(q_0\circ \pi_{[k+1,n]}\circ\mathfrak{gs})(a_1,\ldots,a_k,\bar{u}_1,\ldots,\bar{u}_k)$, where $\mathfrak{gs}$ is the Gram-Schmidt process and $\pi_{[k+1,n]}$ the usual projection in the last $n-k$ coordinates. This map is well defined, since $\{a_1,\ldots,a_k,\bar{u}_1,\ldots,\bar{u}_{n-k}$ is a basis in $\mathbb{R}^n$, by the definition of $U_H$. Moreover, notice that $\tilde{T}_H(a_1,\ldots,a_k)\subseteq\spans\{a_1,\ldots,a_k\}^{\perp}$, since $\mathfrak{gs}$ only affects the last $n-k$ coordinates, since $a_1,\ldots,a_k$ are already orthonormal. Notice that $\tilde{T}_H(a_1,\ldots,a_k)$ only depends on $\spans\{a_1,\ldots,a_k\}$ and thus the universal property of the quotient map ensures that $T_H:U_H\to\Gr{n-k}{n}$ with $T_H(K)=K^{\perp}$ is continuous. For every $I\in\binom{[n]}{k}$, define $H_I:=\spans\{e_i:i\in I\}$ and notice that $\Gr{k}{n}=\bigcup_{I\in\binom{[n]}{k}}U_{H_I}$. This means that $\Gr{k}{n}$ is the union of finitely many open sets of the form $U_H$. Hence, pasting lemma ensures the existence of a continuous $T:\Gr{k}{n}\to\Gr{n-k}{n}$ such that $T|_{U_{H_i}}\equiv T_{H_i}$, which means that $T(K)=k^{\perp}$ is continuous.
\end{proof}

It is helpful at this point to mention what are the ``small'' examples of Grassmannians. Clearly $\Gr{1}{n}\cong\mathbb{P}^{n-1}$ and due to Lemma~\ref{lem:gr_duality}, $\Gr{n-1}{n}\cong\Gr{1}{n}\mathbb{P}^{n-1}$. This already covers the cases $n=2,3$. Hence, $\Gr{2}{4}$ is considered the smallest interesting example of a Grassmannian.

\section{Cell decomposition of the Grassmannian}
In this section we define a cell decomposition of each Grassmannian. Before laying out the formal definition, it is useful to mention the usual cell decomposition of the projective space $\mathbb{P}^{n-1}\cong\Gr{1}{n}$, consisting of the following $n$ cells:
\[\left\{l\subseteq\mathbb{R}^1\right\}\cong\mathbb{R}^0\ ,\ \left\{l\subseteq\mathbb{R}^2\setminus\mathbb{R}^1\right\}\cong\mathbb{R}^1\ ,\ \ldots\ ,\ \left\{l\subseteq\mathbb{R}^n\setminus\mathbb{R}^{n-1}\right\}\cong\mathbb{R}^{n-1}\]
The fact that this cell decomposition depends on our basis choice for $\mathbb{R}^n$, doesn't cause a problem, since for different base choices we get essentially the same decomposition. To illustrate this fact, we are going to define a cell decomposition for any flag of $\mathbb{R}^n$.

\begin{definition} A \emph{flag} $\mathbb{F}_{\bullet}$ of $\mathbb{R}^n$ is a sequence $0=\mathbb{F}_0\subset\mathbb{F}_1\subset\mathbb{F}_2\subset\cdots\subset\mathbb{F}_n=\mathbb{R}^n$, such that $\dim_k\mathbb{F}_i = i$ for every $i\in[n]_0$. An orthonormal basis $f_{\bullet}=(f_1,\ldots,f_n)$ of $\mathbb{R}^n$ is \emph{compatible with $\mathbb{F}_{\bullet}$}, if $f_i\in\mathbb{F}_i$ for every $i\in[n]$.
\end{definition}

Note that given a flag, one can always find a compatible basis with this flag. In fact, there always exist $2^n$ different compatible orthonormal bases and fixing one is like fixing an ``orientation'' of the flag.

\begin{definition} For $k,n\in\mathbb{N}$, such that $k<n$, a flag $\mathbb{F}_{\bullet}$ of $\mathbb{R}^n$ and $\mathbf{j}=\{1\leq j_1<\cdots<j_k\leq n\}\in\binom{[n]}{k}$,
\[\grcell{j}:=\big\{H\in\Gr{k}{n}:\ \dim\left(H\cap\mathbb{F}_i\right)=\max\{\ell\in{[k]}_0:j_{\ell}\leq i\}\ \ \forall i\in{[n]}_0\big\}\]
is the \emph{Schubert cell with symbol $\mathbf{j}$}, where $j_0$ is defined to be $0$.
\end{definition}

Let us now write down the Schubert cells of the first non-trivial example, $\Gr{2}{4}$, with respect to the standard flag of $\mathbb{R}^4$.
\begin{align*}
\mathcal{C}_{\{1,2\}}&=\big\{\mathbb{R}^2\big\}&
\mathcal{C}_{\{1,3\}}&=\big\{H:\ \mathbb{R}^1\subseteq H\subseteq\mathbb{R}^3,\ H\neq\mathbb{R}^2\big\}\\
\mathcal{C}_{\{1,4\}}&=\big\{H:\ \mathbb{R}^1\subseteq H\ H\not\subseteq\mathbb{R}^3\big\}&
\mathcal{C}_{\{2,3\}}&=\big\{H:\ H\subseteq\mathbb{R}^3,\ \mathbb{R}^1\not\subseteq H\big\}\\
\mathcal{C}_{\{2,4\}}&=\big\{H:\ \dim(H\cap\mathbb{R}^2)=1,\ \mathbb{R}^1\not\subseteq H,\ H\not\subseteq\mathbb{R}^3\big\}&
\mathcal{C}_{\{3,4\}}&=\big\{H:\ H\cap\mathbb{R}^2=\{0\}\big\}
\end{align*}
Notice that this cell structure doesn't feel as intuitive as the one defined for the projective spaces. This happens, because the set $\binom{[n]}{1}$ has a natural total order, whereas the more general set $\binom{[n]}{k}$ is naturally a poset, with respect to the coordinate-wise ordering on the set of $k$ element sequences ${[n]}^k$.

\begin{remark} The index $\mathbf{j}\in\binom{[n]}{k}$ of a Schubert cell is the set of \emph{jump points}, i.e. for $H\in\Gr{k}{n}$, $H\in\grcell{j}$ if and only if \[\mathbf{j}=\big\{\min\{i\in[n]:\dim(H\cap\mathbb{F}_i)=1\},\ldots,\min\{i\in[n]:\dim(H\cap\mathbb{F}_i)=k\}\big\}\]
\end{remark}

\begin{lemma}\label{lem:jump_pts} For $k,n\in\mathbb{N}$ such that $k<n$ and for every flag $\mathbb{F}_{\bullet}$ of $\mathbb{R}^n$, the collection  $\big\{\grcell{j}\big\}_{\mathbf{j}\in\binom{[n]}{k}}$ is a partition of $\Gr{k}{n}$.
\end{lemma}
\begin{proof} It is clear that any two cells are disjoint, since each set of $k$ elements in $[n]$ describes uniquely $k$ jump points of the dimensions of $\{H\cap\mathbb{F}_1,\ldots,H\cap\mathbb{F}_n\}$. Inversely, for any $H\in\Gr{k}{n}$, no two consecutive terms of $(H\cap\mathbb{F}_1,\ldots,H\cap\mathbb{F}_n)$ can differ by more than $1$. Indeed, the short exact sequence $0\to H\cap\mathbb{F}_{i-1}\to H\cap\mathbb{F}_i\to H\cap\mathbb{F}_i/H\cap\mathbb{F}_{i-1}\to0$ gives that $\dim_{\mathbb{R}}(H\cap\mathbb{F}_i)-\dim_{\mathbb{R}}(H\cap\mathbb{F}_{i-1})=\dim_{\mathbb{R}}(H\cap\mathbb{F}_i/H\cap\mathbb{F}_{i-1})$ for every $i\in[n]$. Next, using the second isomorphism theorem for vector spaces, we compute $H\cap\mathbb{F}_i/H\cap\mathbb{F}_{i-1}=H\cap\mathbb{F}_i/H\cap\mathbb{F}_i\cap\mathbb{F}_{i-1}\cong((H\cap\mathbb{F}_i)+\mathbb{F}_{i-1})/\mathbb{F}_{i-1}\cong(H+\mathbb{F}_{i-1})\cap(\mathbb{F}_i+\mathbb{F}_{i-1})/\mathbb{F}_{i-1}\cong((H+\mathbb{F}_{i-1})\cap\mathbb{F}_i)/\mathbb{F}_{i-1}$, which is clearly a subspace of $\mathbb{F}_i/\mathbb{F}_{i-1}$. Hence, $\dim_{\mathbb{R}}(H\cap\mathbb{F}_i/H\cap\mathbb{F}_{i-1})\leq\dim_{\mathbb{R}}(\mathbb{F}_i/\mathbb{F}_{i-1})=1$, which means that there exist exactly $k$ jump points in the sequence, putting $H$ in some Schubert cell.
\end{proof}

These cells are in fact homeomorphic to open balls of various dimensions. Their dimension corresponds to the ``degrees of freedom'' each cell provides. This becomes more clear in the next example.

\begin{example} For every plane $H$ in $\mathbb{R}^4$, there exists a unique $2\times 4$ matrix in reduced echelon form, the rows of which span $H$. Then, matrices with the same pivot points are in the same Schubert cell, since they have equal jump points. Notice that the dimension of each cell is the number of unspecified elements in each matrix.
\begin{center}
\begin{tikzcd}
\mathcal{C}_{1,2}\ar[r,leftrightarrow]&[-0.5em]\tworows{1}{0}{0}{0}{0}{1}{0}{0}&\mathcal{C}_{1,3}\ar[r,leftrightarrow]&[-0.5em]\tworows{1}{0}{0}{0}{0}{*}{1}{0}&\mathcal{C}_{1,4}\ar[r,leftrightarrow]&[-0.5em]\tworows{1}{0}{0}{0}{0}{*}{*}{1}\\[-2em]
\mathcal{C}_{2,3}\ar[r,leftrightarrow]&\tworows{*}{1}{0}{0}{*}{0}{1}{0}&\mathcal{C}_{2,4}\ar[r,leftrightarrow]&\tworows{*}{1}{0}{0}{*}{0}{*}{1}&\mathcal{C}_{3,4}\ar[r,leftrightarrow]&\tworows{*}{*}{1}{0}{*}{*}{0}{1}
\end{tikzcd}
\end{center}
\end{example}

\begin{remark} For $k,n\in\mathbb{N}$ such that $k<n$ and $\mathbf{j}=\{1\leq j_1<\cdots<j_k\leq n\}\in\binom{[n]}{k}$, let $\mathrm{d}(\mathbf{j})=(j_1-1)+(j_2-2)+\cdots+(j_k-k)$. Then, $\dim_{\mathbb{R}}\grcell{j}=\mathrm{d}(\mathbf{j})$. Indeed, defining the matrices as in the example, written in a basis compatible with the flag, the number of unspecified elements is equal to $\mathrm{d}(\mathbf{j})$.
\end{remark}

Since we are aiming to prove in Theorem~\ref{thm:gr_is_cw} that the Grassmannians have a CW structure, this matrix representation of the cells will not be helpful any more, as it is not clear how the boundary of these cells is glued to the cells of smaller dimension. So, the approach we employ here may seem more artificial but it is the canonical approach in the bibliography. First, in Definition~\ref{def:lift_of_Schubert_cells} we define appropriate sets $\tilde{\mathcal{C}}_{\mathbf{j}}(\mathbb{F}_{\bullet})$ living in the Stiefel manifold, each one ``above'' the matching Schubert cell. Then, in Lemma~\ref{lem:shub_dim} we prove that each $\tilde{\mathcal{C}}_{\mathbf{j}}{\left(\mathbb{F}_{\bullet}\right)}^-$ is topologically a closed disk of the right dimension. Finally, in Lemma~\ref{lem:cells_from_stiefel_to_gr} we prove that $q_0$ maps the interior of $\tilde{\mathcal{C}}_{\mathbf{j}}(\mathbb{F}_{\bullet})^-$ homeomorphically onto $\grcell{j}$ and its boundary inside cells of lower dimension.

\begin{definition}\label{def:lift_of_Schubert_cells} For $k,n\in\mathbb{N}$ such that $k<n$, a flag $\mathbb{F}_{\bullet}$ of $\mathbb{R}^n$, an orthonormal basis $f_{\bullet}$ of $\mathbb{R}^n$ compatible with $\mathbb{F}_{\bullet}$ and $\mathbf{j}=\{1\leq j_1<j_2<\cdots<j_k\leq n\}\in\binom{[n]}{k}$, define
\[\stcell{j}:=\left\{(v_1,v_2,\ldots,v_k)\in\StO{k}{n}:v_{\ell}\in\mathbb{H}_{j_{\ell}}\ \forall\ell\in[k]\right\},\]
where $\mathbb{H}_i:=\left\{v\in\mathbb{F}_i:v^tf_i>0\right\}$ is the \emph{positive open halfspace of $\mathbb{F}_{\bullet}$, with respect to $f_{\bullet}$} for every $i\in[n]$.
\end{definition}

Notice that for a fixed flag and a fixed basis compatible with the flag, these sets do not cover $\StO{k}{n}$, so they are not a cell decomposition, but if we regard all possible compatible bases and then only choose some of these sets for each basis, then we can create a cell decomposition of the Stiefel manifold. This however is not in the scope of this thesis.

\begin{lemma}\label{lem:closure_of_cells} For $k,n\in\mathbb{N}$ such that $k<n$, a flag $\mathbb{F}_{\bullet}$ of $\mathbb{R}^n$, an orthonormal basis $f_{\bullet}$ of $\mathbb{R}^n$, compatible with $\mathbb{F}_{\bullet}$ and a set of indices $\mathbf{j}\in\binom{[n]}{k}$, the closure of $\stcell{j}$ inside ${\left(\mathbb{R}^n\right)}^k$ is $\stcell{j}^-=\left\{(v_1,v_2,\ldots,v_k)\in\StO{k}{n}:v_{\ell}\in\mathbb{H}_{j_{\ell}}^-\ \forall\ell\in[k]\right\}$, where $\mathbb{H}_i^-:=\{v\in\mathbb{F}_i:v^tf_i\geq 0\}$.
\end{lemma}
\begin{proof} As we saw in Proposition~\ref{prop:StO_dim_closed}, $\StO{k}{n}$ is closed in ${\left(\mathbb{R}^n\right)}^k$. This means that $\stcell{j}^-={\big(\StO{k}{n}\cap\mathbb{H}_{j_1}\times\cdots\times\mathbb{H}_{j_k}\big)}^-\subseteq\StO{k}{n}\cap{(\mathbb{H}_{j_1}\times\cdots\times\mathbb{H}_{j_k})}^-=\StO{k}{n}\cap\mathbb{H}_{j_1}^-\times\cdots\times\mathbb{H}_{j_k}^-$. We show that this inclusion is actually an equality.

Let $(v_1,\ldots,v_k)\in\StO{k}{n}\cap(\mathbb{H}_{j_1}\times\cdots\times\mathbb{H}_{j_k})^-$. This means that there exists a sequence ${(v_1^m,\ldots,v_k^m)}_m\in\mathbb{H}_{j_1}\times\cdots\times\mathbb{H}_{j_k}$ converging to $(v_1,\ldots,v_k)\in\StO{k}{n}\subseteq\St{k}{n}$, inside ${\left(\mathbb{R}^n\right)}^k$. Since $\St{k}{n}$ is open, as proven in Proposition~\ref{prop:St_open}, there exists some $m_0\in\mathbb{N}$, such that ${(v_1^m,\ldots,v_k^m)}_m\in\St{k}{n}$ for all $m\geq m_0$. For each $m\geq m_0$, define now ${(w_1^m,\ldots,w_k^m)}_m:=\mathfrak{gs}({(v_1^m,\ldots,v_k^m)}_m)\in\StO{k}{n}$, where $\mathfrak{gs}$ is the Gram-Schmidt process. Since $\mathfrak{gs}:\St{k}{n}\to\StO{k}{n}$ is a continuous map, $\lim_{m\to\infty}(w_1^m,\ldots,w_k^m)_m=(v_1,\ldots,v_k)$, so it suffices to show that $\mathfrak{gs}(v_1',\ldots,v_k')\in\mathbb{H}_{j_1}\times\cdots\times\mathbb{H}_{j_k}$ for any $(v_1',\ldots,v_k')\in\mathbb{H}_{j_1}\times\cdots\times\mathbb{H}_{j_k}$.

Without loss of generality, we only show this for the last vector. Let $(w_1',\ldots,w_{k-1}')=\mathfrak{gs}(v_1',\ldots,v_{k-1}')$ and assume that $w_i'\in\mathbb{H}_{j_i}$ for every $i\in[k-1]$. Then $w_k'=\lambda\left(v_k'-\sum_{\ell=1}^{k-1}((w_{\ell}')^tv_k')w_{\ell}'\right)$, for some $\lambda>0$. Notice that $w_i'\in\mathbb{F}_{j_i}\subseteq\mathbb{F}_{j_k}$ for every $i\in[k-1]$ and that $v_k'\in\mathbb{F}_{j_k}$, which means that $w_k'\in\mathbb{F}_{j_k}$. Moreover, for every $v\in\mathbb{F}_{j_k-1}$ it is true that $v^tf_{j_k-1}=0$. Hence $(w_k')^tf_{j_k}=\lambda(v_k')^tf_{j_k}>0$, which proves that ${(w_1^m,\ldots,w_k^m)}_m\in\StO{k}{n}\cap\mathbb{H}_{j_1}\times\cdots\times\mathbb{H}_{j_k}$ and thus
$(v_1,\ldots,v_k)\in{\big(\StO{k}{n}\cap\mathbb{H}_{j_1}\times\cdots\times\mathbb{H}_{j_k}\big)}^-$.
\end{proof}

In order to prove Lemma~\ref{lem:shub_dim}, we use the notion of the fiber bundles and follow the proof in \cite{vec_bundles} (p.37). If the reader is not familiar with this notion, they may want to look into Definitions~\ref{def:fiber_bundle} and~\ref{def:trivial_fiber_bundle}, before proceeding to the next lemma.
\begin{lemma}\label{lem:trivial_fb} For $k,n\in\mathbb{N}$ such that $k<n$, a flag $\mathbb{F}_{\bullet}$ of $\mathbb{R}^n$, an orthonormal basis $f_{\bullet}$ of $\mathbb{R}^n$, compatible with $\mathbb{F}_{\bullet}$ and $\mathbf{j}=\{1\leq j_1<\cdots<j_k\leq n\}\in\binom{[n]}{k}$, the projection $\pi_1:\stcell{j}^-\to S^{n-1}\cap\mathbb{H}_{j_1}^-$ is a trivial fiber bundle.
\end{lemma}
\begin{proof} Due to Lemma~\ref{lem:closure_of_cells} the map $\pi_1$ is well defined, since it takes $(v_1,\ldots,v_k)$ to $v_1$, which is a unit vector and also an element of $\mathbb{H}_{j_1}^-$. Notice that since $f_{j_1}\in\mathbb{H}_{j_1}^-$, it suffices to prove that there exists a homeomorphism $\phi$ making the following diagram commute.
\begin{equation}\label{eq:phi_for_trivial}
\begin{tikzcd}
\stcell{j}^-\ar[r,"\phi","\cong"']\ar[d,"\pi_1"']&\big(S^{n-1}\cap\mathbb{H}_{j_1}^-\big)\times\pi_1^{-1}(f_{j_1})\ar[dl,"\pi_1"]\\
S^{n-1}\cap\mathbb{H}_{j_1}^-
\end{tikzcd}
\end{equation}
For every $v\in S^{n-1}\cap\mathbb{H}_{j_1}^-$ define the linear map $\rho_v:=I_n+2f_{j_1}v^t-\frac{(v+f_{j_1})(v+f_{j_1})^t}{v^tf_{j_1}+1}:\mathbb{R}^n\to\mathbb{R}^n$, where $I_n$ is the identity on $\mathbb{R}^n$ and $xy^t$ is the linear map taking $a\in\mathbb{R}^n$ to $(y^ta)x$. Notice that $\rho_v$ is the rotation on the plane $H:=\spans\{v,f_{j_1}\}$ taking $v$ to $f_{j_1}$, i.e. $\rho_v\in O(n)$, $\rho_v(H)=H$, $\rho_v|_{H^{\perp}}=I_n$ and $\rho_vv=f_{j_1}$.

Moreover, notice that $\rho_v$ depends continuously on $v$ and for $\rho_{f_{j_1}}:=I_n$, we get that $\lim_{v\to f_{j_1}}\rho_v=\rho_{f_{j_1}}$.This means that the map $\rho:S^{n-1}\cap\mathbb{H}_{j_1}^-\to O(n)$ with $\rho(v)=\rho_v$ is continuous. Since $O(n)$ acts continuously on $\mathbb{R}^n$, the map $ev^k:O(n)\times\stcell{j}^-\to(\mathbb{R}^n)^k$ with $ev^k(A,u_1,\ldots,u_k)=(Au_1,\ldots,Au_k)$ is also continuous. Thus, the composition that rotates a $k$-tuple by the rotation corresponding to the first vector, i.e. $ev^k\circ\left<\rho\circ\pi_1,id\right>:\stcell{j}^-\to(\mathbb{R}^n)^k$ with $ev^k\circ\left<\rho\circ\pi_1,id\right>(v_1,\ldots,v_k)=(\rho_{v_1}v_1,\ldots,\rho_{v_1}v_k)$ is continuous.

We now prove that the image of this map is contained in $\pi_1^{-1}(f_{j_1})$. Let $(v_1,\ldots,v_k)\in\stcell{j}^-$. Since $\rho_{v_1}\in O(n)$ and $(v_1,\ldots,v_k)\in\StO{k}{n}$, we have that $(\rho_{v_1}v_i)^t\rho_{v_1}v_j=v_i^tv_j=\delta_{i,j}$, which means that $(\rho_{v_1}v_1,\ldots,\rho_{v_1}v_k)\in\StO{k}{n}$. It remains to show that $\rho_vv_{\ell}\in\mathbb{H}_{j_{\ell}}$ for $v_{\ell}\in\mathbb{H}_{j_{\ell}}$. Decompose $v_{\ell}$ in two orthogonal parts $v_{\ell}=p_Hv_{\ell}+p_{H^{\perp}}v_{\ell}$, where $p_H$ and $p_{H^{\perp}}$ are the orthogonal projections on $H$ and $H^{\perp}$ respectively. Since $v_{\ell}\in\mathbb{F}_{j_{\ell}}$ and $H\subseteq\mathbb{F}_{j_{\ell}}$, we have that $p_{H^{\perp}}v_{\ell}\in H^{\perp}\cap\mathbb{F}_{j_{\ell}}$. Moreover, since $\rho_{v_1}$ is a rotation on $H$, $\rho_{v_1}p_Hv_{\ell}\in H$ and $\rho_{v_1}p_{H^{\perp}}v_{\ell}=p_{H^{\perp}}v_{\ell}$. Thus, $\rho_{v_1}v_{\ell}=\rho_{v_1}p_Hv_{\ell}+\rho_{v_1}p_{H^{\perp}}v_{\ell}=\rho_{v_1}p_Hv_{\ell}+p_{H^{\perp}}v_{\ell}\in H+\mathbb{F}_{j_{\ell}}=\mathbb{F}_{j_{\ell}}$. Next, notice that $(\rho_{v_1}v_{\ell})^tf_{j_{\ell}}=v_{\ell}^t\rho_{v_1}^tf_{j_{\ell}}=v_{\ell}^tf_{j_{\ell}}\geq0$. This proves that $(\rho_{v_1}v_1,\ldots,\rho_{v_1}v_k)\in\stcell{j}^-$. Also, since $\rho_{v_1}v_1=f_{j_1}$, it is also holds that $(\rho_{v_1}v_1,\ldots,\rho_{v_1}v_k)\in\pi_1^{-1}(f_{j_1})$.

Let us define $\phi:=\left<\pi_1,\mathrm{ev}^k\circ\left<\rho\circ\pi_1,id\right>\right>:\stcell{j}\to\big(S^{n-1}\cap\mathbb{H}_{j_1}^-\big)\times\pi_1^{-1}(f_{j_1})$ with $\phi(v_1,\ldots,v_k)=\big(v_1,(\rho_{v_1}v_1,\ldots,\rho_{v_1}v_k)\big)$. Notice that $\phi$ is well defined and continuous. Moreover, the restriction on each fiber $\phi|_{\pi_1^{-1}(v)}:\pi_1^{-1}(v)\to\pi^{-1}(f_{j_1})$ is a homeomorphism. Indeed, the map $\mathrm{ev}^k\circ\left<\rho'\circ\pi_1{,}id\right>:\pi_1^{-1}(f_{j_1})\to\pi_1^{-1}(v)$ taking $(f_{j_1},v_2,\ldots,v_k)$ to $(v,\rho_v^tv_2,\ldots,\rho_v^tv_k)$ is its inverse.

Since $\stcell{j}^-$ is a closed subset of the compact $\StO{k}{n}$, it is also compact. Moreover, the space $\big(S^{n-1}\cap\mathbb{H}_{j_1}^-\big)\times\pi_1^{-1}(f_{j_1})$ is clearly Hausdorff, and $\phi$ is also a homeomorphism, i.e.\ $\pi_1$ is a trivial fiber bundle due to Proposition~\ref{prop:extend_fb}.
\end{proof}

\begin{lemma}\label{lem:shub_dim} For any integers $0<k<n$, for any flag $\mathbb{F}_{\bullet}$ of $\mathbb{R}^n$, for any orthonormal basis $f_{\bullet}$ of $\mathbb{R}^n$, compatible with $\mathbb{F}_{\bullet}$ and for any set $\mathbf{j}\in\binom{[n]}{k}$, there exists a homeomorphism
$\tilde{\Phi}_{\mathbf{j}}:D^{\mathrm{d}(\mathbf{j})}\to\stcell{j}^-$,
such that $\tilde{\Phi}_{\mathbf{j}}\left({\left(D^{\mathrm{d}(\mathbf{j})}\right)}^{\circ}\right)\subseteq\stcell{j}$, and
$\tilde{\Phi}_{\mathbf{j}}|_{{\left(D^{\mathrm{d}(\mathbf{j})}\right)}^{\circ}}:{\left(D^{\mathrm{d}(\mathbf{j})}\right)}^{\circ}\to\stcell{j}$
is also a homeomorphism.
\end{lemma}
\begin{proof}
Notice that
\begin{equation}\label{eq:linear_proj_disc}
S^{n-1}\cap\mathbb{H}_{j_1}^-\cong D^{j_1-1}
\end{equation}
with a homeomorphism being the linear projection on $\mathbb{F}_{j_1-1}$. Moreover,
\begin{align}
\begin{split}
\pi_1^{-1}(f_{j_1})&=\big\{(f_{j_1},v_2,\ldots,v_k)\in\StO{k}{n}:v_{\ell}\in\mathbb{H}_{j_{\ell}}^-\ \forall\ell\in[k]\setminus\{1\}\big\}\\
&\cong\Big\{(v'_2,\ldots,v'_k)\in\StO{k-1}{n-1}
:v'_{\ell}\in\mathbb{H}\left(\faktor{\mathbb{F}_{j_{\ell}}}{\spans\{f_{j_1}\}},\pi_{\hat{j}_1}(f_{j_{\ell}})\right)^-\ \forall\ell\in[k]\setminus\{1\}\Big\}\\
&=\tilde{\mathcal{C}}_{\mathbf{j}'}{\left(\mathbb{F}'_{\bullet},f'_{\bullet}\right)}^-
\end{split}\label{eq:induction_step}
\end{align}
where $\pi_{\hat{j}_1}$ is the projection on the $n-1$ coordinates $\{1,\ldots,j_1-1,j_1+1,\ldots,n\}$ and
\begin{align*}
\mathbf{j}'&:=\{1\leq j_2-1<\cdots<j_k-1\leq n-1\}\\
\mathbb{F}'_i&:=\left\{\begin{array}{ll}\mathbb{F}_i&,0\leq i<j_1\\\faktor{\mathbb{F}_{i+1}}{\spans\{f_{j_1}\}}&,j_1\leq i<n\end{array}\right.\\
f'_i&:=\left\{\begin{array}{ll}\pi_{\hat{j}_1}(f_i)&,0\leq i<j_1\\\pi_{\hat{j}_1}(f_{i+1})&,j_1\leq i<n\end{array}\right.
\end{align*}
So, $\tilde{\Phi}_{\mathbf{j}}$ is constructed inductively, using \eqref{eq:trivial_fb}:
\begin{align*}
\stcell{j}^-&\cong D^{j_1-1}\times\tilde{\mathcal{C}}_{\mathbf{j}'}{\left(\mathbb{F}_{\bullet}',f_{\bullet}'\right)}^-\\
&\cong D^{j_1-1}\times D^{j_2-2}\times \tilde{\mathcal{C}}_{\mathbf{j}''}{\left(\mathbb{F}_{\bullet}'',f_{\bullet}''\right)}^-\\
&\ \,\vdots\\
&\cong D^{j_1-1}\times D^{j_2-2}\times\cdots\times D^{j_k-k}\\
&\cong D^{\mathrm{d}(\mathbf{j})}
\end{align*}
For the second claim, notice that
$S^{n-1}\cap\mathbb{H}_{j_1}\cong\big(D^{j_1-1}\big)^{\circ}$
through the same linear projection as in \eqref{eq:linear_proj_disc} and
$\pi_1|_{\stcell{j}}^{-1}(f_{j_1})\cong\tilde{\mathcal{C}}_{\mathbf{j}'}(\mathbb{F}_{\bullet}',f_{\bullet}')$
through the same map as in \eqref{eq:induction_step}. Using Remark~\ref{rem:trivial_fb_interior} we get
$\stcell{j}\cong\big(S^{n-1}\cap\mathbb{H}_{j_1}\big)\times\pi_1|_{\stcell{j}}^{-1}(f_{j_1})$
through the same map as in \eqref{eq:trivial_fb}.
So, the inductive construction above gives us that
$\stcell{j}\cong\big(D^{\mathrm{d}(\mathbf{j})}\big)^{\circ}$,
with the homeomorphism being a restriction of $\tilde{\Phi}_{\mathbf{j}}^{-1}$, which proves the assertion.
\end{proof}

In order to examine the CW structure of $\Gr{k}{n}$, we need to connect the sets $\stcell{j}$ defined to the cells $\grcell{j}$:
\begin{lemma}\label{lem:cells_from_stiefel_to_gr} Let $k,n\in\mathbb{N}$, with $0<k<n$. Moreover, let $\mathbb{F}_{\bullet}$ be a flag of $\mathbb{R}^n$ and $f_{\bullet}$ be an orthonormal basis of $\mathbb{R}^n$ compatible with $\mathbb{F}_{\bullet}$. Then, for each $\mathbf{j}\in\binom{[n]}{k}$ the following restriction of $q_0:\StO{k}{n}\to\Gr{k}{n}$ is a homeomorphism:
\begin{center}
\begin{tikzcd}
\stcell{j}\ar[r,"q_0"]&\grcell{j}\\[-1.5em]
(v_1,\ldots,v_k)\ar[r,mapsto]&\spans\{v_1,\ldots,v_k\}
\end{tikzcd}
\end{center}
\end{lemma}
\begin{proof} First notice that $q_0$ is well defined. Indeed, let $(v_1,\ldots,v_k)\in\stcell{j}$ and
$H:=q_0(v_1,\ldots,v_k)=\spans\{v_1,\ldots,v_k\spans\}\in\Gr{k}{n}$.
Fix some $i\in[n]$. For every $\ell\in[k]$ with $j_{\ell}\leq i$,  $v_{\ell}\in\mathbb{F}_{j_{\ell}}\subseteq\mathbb{F}_i$. On the other hand, if $j_{\ell}>i$, then $v_{\ell}\cdot f_{j_{\ell}}>0$, i.e.\ $v_{\ell}\not\in\mathbb{F}_i$ since $f_{j_{\ell}}\perp\mathbb{F}_i$. Hence,
$\dim(H\cap\mathbb{F}_i)=\dim\left(\spans\{\big\{v_{\ell}:j_{\ell}\leq i\big\}\}\right)=\max\{\ell\in[k]_0:j_{\ell}\leq i\}$
where $j_0=0$ as before, that proves that $H\in\grcell{j}$.

In order to prove that this is a homeomorphism, we construct its inverse map $r_0:\grcell{j}\to\tilde{\mathcal{C}}(\mathbb{F}_{\bullet},f_{\bullet})$ as follows: Let $H\in\grcell{j}$. Then, for any $\ell\in[k]$,
\begin{align*}
\dim\big(\left(H\cap\mathbb{F}_{j_{\ell}}\right)\cap\left(H\cap\mathbb{F}_{j_{\ell}-1}\right)^{\perp}\big)&=\dim\big(H\cap\mathbb{F}_{j_{\ell}}\big)+\dim\big(\left(H\cap\mathbb{F}_{j_{\ell}-1}\right)^{\perp}\big)-n\\
&=\dim\big(H\cap\mathbb{F}_{j_{\ell}}\big)-\dim\big(H\cap\mathbb{F}_{j_{\ell}-1}\big)=1
\end{align*}
so, there exist exactly two vectors $-a,a$ such that
\[-a,a\in H\cap\mathbb{F}_{j_{\ell}}\qquad\text{ and }\qquad-a,a\perp H\cap\mathbb{F}_{j_{\ell}-1}\qquad\text{ and }\qquad-a,a\in S^{n-1}\]
Notice that $a$ and $-a$ cannot be perpendicular to $f_{j_{\ell}}$: Indeed, if $a\in\mathbb{F}_{j_{\ell}}$ and $a^tf_{j_{\ell}}=0$, then $a\in\mathbb{F}_{j_{\ell}-1}$, which leads to a contradiction since $a\perp H\cap\mathbb{F}_{j_{\ell}-1}$. We now define $u_{\ell}\in\{a,-a\}$ to be the unique vector that satisfies all the above and $u_{\ell}^tf_{j_{\ell}}>0$ and then also define $r_0(H):=(u_1,\ldots,u_k)$.

First notice $r_0$ is well defined, since for $1\leq\ell_1<\ell_2\leq k$,
$u_{\ell_2}\perp H\cap\mathbb{F}_{j_{\ell}-1}\supseteq H\cap\mathbb{F}_{j_{\ell-1}}\ni u_{\ell_1}$
i.e.\ $u_{\ell_1}^tu_{\ell_2}=0$. Moreover, for $\ell\in[k]$, $u_{\ell}\in S^{n-1}$, i.e.\ $u_{\ell}^tu_{\ell}=1$. Also, $v_{\ell}\in\mathbb{H}_{j_{\ell}}$ by the construction of $v_{\ell}$. This proves that $r_0(H)=(u_1,\ldots,u_k)\in\stcell{j}$.

Then, notice that for $H\in\grcell{j}$,
$q_0(r_0(H))=q_0(u_1,\ldots,u_k)=\spans\{u_1,\ldots,u_k\}=H$.
Moreover, for $(v_1,\ldots,v_k)\in\stcell{j}$,
$v_{\ell}\in\spans\{v_1,\ldots,v_k\}\cap\mathbb{F}_{j_{\ell}}\text{ and } v_{\ell}\perp\spans\{v_1,\ldots,v_k\}\cap\mathbb{F}_{j_{\ell}-1}\text{ and }v_{\ell}\in S^{n-1}\text{ and }v_{\ell}^tf_{j_{\ell}}>0$
which means that
$r_0(q_0(v_1,\ldots,v_k))=r_0(\spans\{v_1,\ldots,v_k\})=(v_1,\ldots,v_k)$,
since there is a unique choice for $r_0$, as proven above.
\end{proof}

\begin{lemma}\label{lem:q0_on_bdr} Let $k,n\in\mathbb{N}$, with $0<k<n$. Moreover, let $\mathbb{F}_{\bullet}$ be a flag of $\mathbb{R}^n$ and $f_{\bullet}$ be an orthonormal basis of $\mathbb{R}^n$ compatible with $\mathbb{F}_{\bullet}$. Then, for each $\mathbf{j}\in\binom{[n]}{k}$ the restriction of $q_0:\StO{k}{n}\to\Gr{k}{n}$ has the following property:
\[q_0\big(\stcell{j}^-\setminus\stcell{j}\big)\subseteq\bigcup_{\substack{\mathbf{j}'\in\binom{[n]}{k}\\\mathrm{d}(\mathbf{j}')<\mathrm{d}(\mathbf{j})}}\grcell{j'}\]
\end{lemma}
\begin{proof} Let $(v_1,\ldots,v_k)\in\stcell{j}^-\setminus\stcell{j}$ and set $H=q_0(v_1,\ldots,v_k)=\spans\{v_1,\ldots,v_k\}$. Due to Lemma~\ref{lem:jump_pts} there exists some $\mathbf{j}'\in\binom{[n]}{k}$ such that $H\in\mathcal{C}_{\mathbf{j}'}(\mathbb{H}_{\bullet})$. For this $\mathbf{j}'$,
$j_{\ell}'=\min\{j\in[n]:\dim(H\cap\mathbb{F}_j)\geq\ell\}$.

Since $v_1,\ldots,v_{\ell}\in H\cap\mathbb{F}_{j_{\ell}}$ and are linearly independent, $j_{\ell}'\leq j_{\ell}$ for every $l\in[k]$. Moreover, since $(v_1,\ldots,v_k)\not\in\stcell{j}$, there exists some $\ell_0\in[k]$ such that $v_{\ell_0}\in\mathbb{H}_{j_{\ell_0}}^-\setminus\mathbb{H}_{j_{\ell_0}}$, i.e.\ $v_{\ell_0}\in\mathbb{F}_{j_{\ell_0}}$ and $v_{\ell_0}^tf_{j_{\ell_0}}=0$. This means that $v_1,\ldots,v_{\ell_0}\in\mathbb{F}_{j_{\ell_0}-1}<j_{\ell_0}$, i.e.\ $j_{\ell_0}'\leq j_{\ell_0}-1$. Combining these, we get
$\mathrm{d}(\mathbf{j}')=\sum_{\ell\in[k]}(j'_{\ell}-\ell)<\sum_{\ell\in[k]}(j_{\ell}-\ell)=\mathrm{d}(\mathbf{j})$.
\end{proof}

\begin{theorem}\label{thm:gr_is_cw} Let $k,n\in\mathbb{N}$ such that $0<k<n$ and $\mathbb{F}_{\bullet}$ be any flag of $\mathbb{R}^n$. Moreover for every integer $m\in[k(n-k)]_0\cup\{-1\}$ let \[X_m^{k,n}:=\bigcup_{\substack{\mathbf{j}\in\binom{[n]}{k}\\\mathrm{d}(\mathbf{j})\leq m}}\grcell{j}\]
Then, the filtration $\emptyset=X_{-1}^{k,n}\subseteq X_0^{k,n}\subseteq\cdots\subseteq X_{k(n-k)}^{k,n}=\Gr{k}{n}$ is a CW structure of $\Gr{k}{n}$. It is helpful for later to define $X_m^{k,n}:=X_{k(n-k)}^{k,n}$ for every $m>k(n-k)$.
\end{theorem}
\begin{proof}
Since the filtration is finite, $\Gr{k}{n}$ trivially has the direct limit topology with respect to the filtration, so we only need to prove the existence of a pushout square for every $m\in[k(n-k)]$:
\begin{center}
\begin{tikzcd}
\coprod_{\mathbf{j}\in I_m}S^{m-1}\ar[d,hook,"\amalg_{\mathbf{j}\in I_m}inc_{\mathbf{j}}"']\ar[r,"\amalg_{\mathbf{j}\in I_m}\phi_{\mathbf{j}}"]\ar[dr,phantom,near end,"\ulcorner"]&[3em]X_{m-1}^{k,n}\ar[d,hook,"inc_m"]\\[2em]
\coprod_{\mathbf{j}\in I_m}D^m\ar[r,"\amalg_{\mathbf{j}\in I_m}\Phi_{\mathbf{j}}"']&X_m^{k,n}
\end{tikzcd}
\end{center}
Let $I_m=\big\{\mathbf{j}\in\binom{[n]}{k}:\mathrm{d}(\mathbf{j})=m\big\}$ be the set of indices. Moreover, let $\Phi_{\mathbf{j}}=q_0\circ\tilde{\Phi}_{\mathbf{j}}:D^{\mathrm{d}(\mathbf{j})}\to \Gr{k}{n}$ be the characteristic maps, where $\tilde{\Phi}_{\mathbf{j}}:D^{\mathrm{d}(\mathbf{j})}\to\stcell{j}^-\subseteq\StO{k}{n}$ are the homeomorphisms defined in Lemma~\ref{lem:shub_dim} and $q_0:\StO{k}{n}\to\Gr{k}{n}$ is the usual quotient map. Also, let $\phi_{\mathbf{j}}=\Phi_{\mathbf{j}}|_{\partial D^{\mathrm{d}(\mathbf{j})}}$ just be the restriction on the boundary of the closed disk.
First, the maps in the diagram are well defined:
\begin{i_enum}
\item $\Phi_{\mathbf{j}}\big(D^{\mathrm{d}(\mathbf{j})}\big)\subseteq X_{\mathrm{d}(\mathbf{j})}^{k,n}$. Indeed, $\Phi_{\mathbf{j}}\big(D^{\mathrm{d}(\mathbf{j})}\big)=q_0\big(\tilde{\Phi}_{\mathrm{j}}(D^{\mathrm{d}(\mathbf{j})})\big)=q_0\big(\stcell{j}^-\big)$, since $\tilde{\Phi}_{\mathbf{j}}$ is a homeomorphism. Moreover, because of Lemmata~\ref{lem:cells_from_stiefel_to_gr} and~\ref{lem:q0_on_bdr}, we get that $q_0\big(\stcell{j}^-\big)=q_0\big(\stcell{j}^-\setminus\stcell{j}\big)\cup q_0\big(\stcell{j}\big)\subseteq\bigcup_{\substack{\mathbf{j}'\in\binom{[n]}{k}\\\mathrm{d}(\mathbf{j}')<\mathrm{d}(\mathbf{j})}}\grcell{j'}\cup\grcell{j}\subseteq X_{\mathrm{d}(\mathbf{j})}^{k,n}$.
\item $\phi_{\mathbf{j}}\big(S^{\mathrm{d}(\mathbf{j})-1}\big)\subseteq X_{\mathrm{d}(\mathbf{j})-1}^{k,n}$. Indeed, $\phi_{\mathbf{j}}\big(S^{\mathrm{d}(\mathbf{j})-1}\big)=q_0\big(\tilde{\Phi}_{\mathbf{j}}(\partial D^{\mathrm{d}(\mathbf{j})})\big)=q_0\big(\stcell{j}^-\setminus\stcell{j}\big)$, since $\tilde{\Phi}_{\mathbf{j}}$ is a homeomorphism. Moreover, because of Lemma~\ref{lem:q0_on_bdr}, we get that $q_0\big(\stcell{j}^-\setminus\stcell{j}\big)\subseteq\bigcup_{\substack{\mathbf{j}'\in\binom{[n]}{k}\\\mathrm{d}(\mathbf{j}')<\mathrm{d}(\mathbf{j})}}\grcell{j'}\subseteq X_{\mathrm{d}(\mathbf{j})-1}^{k,n}$.
\end{i_enum}
Next, the commutativity of the diagram is clear, since $\phi_{\mathbf{j}}$ is a restriction of $\Phi_{\mathbf{j}}$ and the two vertical maps are inclusions.
Last, using Lemma~\ref{lem:shub_dim} and Lemma~\ref{lem:cells_from_stiefel_to_gr}, we get that both of the functions involved in the following composition are homeomorphisms
\begin{center}
\begin{tikzcd}
\Phi_{\mathbf{j}}|_{(D^{\mathrm{d}(\mathbf{j})})^{\circ}}\ar[r,phantom,":"]&[-2.6em](D^{\mathrm{d}(\mathbf{j})})^{\circ}\ar[r,"\tilde{\Phi}_{\mathbf{j}}|_{(D^{\mathrm{d}(\mathbf{j})})^{\circ}}","\cong"']&[3em]\stcell{j}\ar[r,"q_0|_{\stcell{j}}","\cong"']&[3em]\grcell{j}
\end{tikzcd}
\end{center}
Also, $\Phi_{\mathbf{j}}((D^{\mathrm{d}(\mathbf{j})})^{\circ})=\grcell{j}\subseteq X_m\setminus X_{m-1}$, which proves that it is a pushout diagram.
\end{proof}

\begin{proposition}\label{prop:included_cell} For $k,n\in\mathbb{N}$, the inclusion $\iota_{k,n}:\Gr{k}{n}\hookrightarrow\Gr{k}{n+1}$ respects the cell structure, i.e.\ for every $\mathbf{j}\in\binom{[n]}{k}$, it is also true that $\mathbf{j}\in\binom{[n+1]}{k}$ and then $\iota_{k,n}(\grcell{j})=\grcell{j}\subseteq\Gr{k}{n+1}$.
\end{proposition}
\begin{proof} If $\mathbf{j}=\{1\leq j_1<j_2<\cdots<j_k\leq n\}$, then trivially $j_k\leq n+1$, i.e.\ $\{1\leq j_1<\cdots<j_k\leq n+1\}\in\binom{[n+1]}{k}$. Moreover, if $H\in\grcell{j}\subseteq\Gr{k}{n}$ then $\dim(H\cap\mathbb{F}_{n+1})=j_k=\max\{\ell\in[k]_0:j_{\ell}\leq i\}$, i.e.\ $\iota_{k,n}(H)=H\subseteq\mathbb{R}^{n+1}$ is an element of $\grcell{j}\subseteq\Gr{k}{n+1}$. Inversely, for $H\in\grcell{j}\subseteq\Gr{k}{n+1}$, then $H\in\grcell{j}\subseteq\Gr{k}{n}$ since $j_k\leq n$ and thus $\iota_{k,n}(H)=H$.
\end{proof}

\begin{lemma}\label{lem:stable_filtration} Let $k,m\in\mathbb{N}$, then there exists some $n_0\in\mathbb{N}$ such that
\[X_m^{k,k+1}\subseteq X_m^{k,k+2}\subseteq\cdots\subseteq X_m^{k,n_0}=X_m^{k,n_0+1}=X_m^{k,n_0+2}=\cdots\]
i.e.\ $X_m^{k,n}=X_m^{k,n_0}$ for every $n\geq n_0$.
\end{lemma}
\begin{proof} If $I_m^{k,n}=\{\mathbf{j}\in\binom{[n]}{k}:\mathrm{d}(\mathbf{j})\leq m\}$, then $X_m^{k,n}=\bigcup_{\mathbf{j}\in I_m^{k,n}}\grcell{j}$. Using Proposition~\ref{prop:included_cell}, we have $X_m^{k,n}=\bigcup_{I_m^{k,n}}\grcell{j}\subseteq\bigcup_{I_m^{k,n+1}}\grcell{j}=X_m^{k,n+1}$ for every $n\in\mathbb{N}$. Let $n>m+k$ and $\mathbf{j}\in I_m^{k,n}$. Notice then that $j_1-1+j_2-2+\cdots+j_k-k\leq m$ i.e.\ $j_k\leq m+k-\big((j_1-1)+\cdots+(j_{k-1}-k+1)\big)\leq m+k<n$, since for every $i\in[k-1]$ $j_i\geq i$. This means that $\mathbf{j}\in I_m^{k,n-1}$ and so $X_m^{k,n-1}\subseteq X_m^{k,n}$. This means that for $n_0:=m+k$ the assertion is true.
\end{proof}

\begin{theorem}\label{thm:gr_inf_cw} Let $k\in\mathbb{N}$ and $\mathbb{F}_{\bullet}$ be any flag of $\mathbb{R}^{\infty}$. Moreover, for every integer $m\geq-1$ let $X_m^k:=\bigcup_{n=k+1}^{\infty}X_m^{k,n}$. Then, the filtration
$\emptyset=X_{-1}^k\subseteq X_0^k\subseteq X_1^k\subseteq X_2^k\subseteq\cdots\subseteq\Gr{k}$
is a CW structure of $\Gr{k}$.
\end{theorem}
\begin{proof} Let $m\geq-1$ be any integer. Then, using Lemma~\ref{lem:stable_filtration} we get that there exists some $n_0$ such that $X_m^k=X_m^{k,n_0}$ and $X_{m-1}^k=X_{m-1}^{k,n_0}$. This means that there exists a pushout square involving $X_{m-1}^k$ and $X_m^k$, namely exactly the same we used in the proof of Theorem~\ref{thm:gr_is_cw}.

It remains to show that $\Gr{k}$ has the direct limit topology with respect to the filtration $X_m^k$. Notice that $\bigcup_{m=-1}^{\infty}X_m^k=\bigcup_{m=-1}^{\infty}\bigcup_{n=k+1}^{\infty}X_m^{k,n}=\bigcup_{n=k+1}^{\infty}\bigcup_{m=-1}^{\infty}X_m^{k,n}=\bigcup_{n=k+1}\Gr{k}{n}=\Gr{k}$. Moreover, let $U\in\Gr{k}$ be any open set, fix some integer $m\geq-1$ and notice that $X_m^k=X_m^{k,n_0}$ for some $n_0$. So, we get that $U\cap X_m^k=U\cap X_m^{k,n_0}=U\cap\Gr{k}{n_0}\cap X_m^{k,n_0}$ which is open in $X_m^{k,n_0}=X_m^k$, since $U\cap\Gr{k}{n_0}$ is open in $\Gr{k}{n_0}$ and $X_m^{k,n_0}\subseteq\Gr{k}{n_0}$. For the inverse direction, let $U\subseteq\Gr{k}$ be some set such that $U\cap X_m^k$ is open in $X_m^k$ for every integer $m\geq-1$. Then, we get that $U\cap\Gr{k}{n}=U\cap X_{k(n-k)}^{k,n-k}=U\cap X_{k(n-k)}^k\cap X_{k(n-k)}^{k,n-k}$ which is open in $X_{k(n-k)}^{k,n-k}=\Gr{k}{n}$, since $U\cap X_{k(n-k)}^k$ is open in $X_{k(n-k)}^k$ and $X_{k(n-k)}^{k,n-k}\subseteq X_{k(n-k)}^k$. Since this is true for every integer $n>0$, $U$ is open in $\Gr{k}$, because it is topologised with the direct limit topology with respect to the filtration $\Gr{k}{n}$.
\end{proof}

So, $\Gr{k}{n}$ is a CW complex and each cell is indexed by some $\mathbf{j}=\{1\leq j_1<\cdots<j_k\leq n\}\in\binom{[n]}{k}$, while $\Gr{k}$ is a CW complex and each cell is indexed by some $\mathbf{j}=\{1\leq j_1<j_2<\cdots\}\in\binom{\mathbb{N}}{k}$. In either case the dimension of the cell indexed by $\mathbf{j}$ is equal to $\mathrm{d}(\mathbf{j})=(j_1-1)+(j_2-2)+\cdots+(j_k-k)$.

\section{Enumerating the cells}
\begin{definition} For $m,r\in\mathbb{N}$, $\lambda=(\lambda_1,\ldots,\lambda_r)\in\mathbb{N}^r$ is a \emph{partition of $m$ of size $r$} if $\lambda_1\geq\lambda_2\geq\cdots\geq\lambda_r\geq1$ and $\lambda_1+\lambda_2+\cdots+\lambda_r=m$. In this case, we write $\lambda\vdash m$ and $|\lambda|=r$. For $m=0$, we define $P_0:=\{\lambda_{\emptyset}\}$, where $\lambda_{\emptyset}:=()$ is the \emph{empty partition} with $\lambda_{\emptyset}\vdash0$ and $|\lambda|=0$. Let $P_m$ be the set of all partitions of $m$ and $P:=\bigcup_{m=0}^{\infty}P_m$ the set of all partitions.
\end{definition}
\begin{remark}
In order to make the notation easier, any number of trailing zeros at the end of a partition are allowed, without this affecting the size of the partition, i.e.\ $\lambda=(5,4,4,2,0,0,0)=(5,4,4,2)$, with $\lambda\vdash15$ and $|\lambda|=4$.
\end{remark}
\begin{example} There are exactly $7$ partitions of $m=5$:
$(5)$, $(4,1)$, $(3,2)$, $(3,1,1)$, $(2,2,1)$, $(2,1,1,1)$, $(1,1,1,1,1)$
\end{example}

\begin{notation} For every partition $\lambda$ of $m$ in $r$ parts, we define the \emph{Young diagram} corresponding to $\lambda$ to be the shape consisting of $r$ left-justified rows of boxes, where row $i$ has $\lambda_i$ boxes.
\end{notation}
\begin{example} The young diagrams of the $7$ partitions of $m=5$ are:
\begin{center}
\ytableausetup{boxsize=0.7em,aligntableaux=top}\ydiagram{5}\qquad\ydiagram{4,1}\qquad\ydiagram{3,2}\qquad\ydiagram{3,1,1}\qquad\ydiagram{2,2,1}\qquad\ydiagram{2,1,1,1}\qquad\ydiagram{1,1,1,1,1}
\end{center}
\end{example}

\begin{proposition}\label{prop:poset_bijection} For $k,n\in\mathbb{N}$ such that $0<k<n$, the map
\begin{center}
\begin{tikzcd}
b_{k,n}\ar[r,phantom,":"]&[-7em]\binom{[n]}{k}\ar[r]&\big\{\lambda\in P:|\lambda|\leq k\text{ and }\lambda_1\leq n-k\big\}\\[-1.5em]
&\big\{1\leq j_1<\cdots<j_k\leq n\big\}\ar[r,mapsto]&\big(j_k-k\ ,\ j_{k-1}-(k-1)\ ,\ \ldots\ ,\ j_1-1\big)
\end{tikzcd}
\end{center}
is a bijection with $b_{k,n}(\mathbf{j})\vdash\mathrm{d}(\mathbf{j})$.
\end{proposition}
\begin{proof} First of all notice that for $\lambda=b_{k,n}(\mathbf{j})$ we have $|\lambda|\leq|\mathbf{j}|=k$, $\lambda_1=j_k-k\leq n-k$, $\lambda_k=j_1-1\geq 0$ and for every $i\in[k-1]$ $\lambda_i=j_{k-i+1}-(k-i+1)>j_{k-i}-(k-i)-1=\lambda_{i+1}-1$, i.e.\ $\lambda_i\geq\lambda_{i+1}$. These mean that $b_{k,n}$ is well defined.

In order to prove that $b_{k,n}$ is a bijection, we will prove that this is its inverse map:
\begin{center}
\begin{tikzcd}
c_{n,k}\ar[r,phantom,":"]&[-2.5em]\big\{\lambda\in P:|\lambda|\leq k\text{ and }\lambda_1\leq n-k\big\}\ar[r]&[-1em]\binom{[n]}{k}\\[-1.5em]
&(\lambda_1,\lambda_2,\ldots,\lambda_k)\ar[r,mapsto]&\{\lambda_k+1,\lambda_{k-1}+2,\ldots,\lambda_1+k\}
\end{tikzcd}
\end{center}
Notice that for $\mathbf{j}=c_{k,n}(\lambda)$ we have $|\mathbf{j}|=k$, $j_1=\lambda_k+1\geq1$, $j_k=\lambda_1+k\leq n$ and for ever $i\in[k-1]$ $j_i=\lambda_{k-i+1}+i\leq\lambda_{k-i}+(i+1)-1=j_{i+1}-1$, i.e.\ $j_i<j_{i+1}$. These mean that $c_{k,n}$ is well defined.

Moreover $c_{k,n}=b_{k,n}^{-1}$, since $c_{k,n}\big(b_{k,n}(\mathbf{j})\big)=c_{k,n}(j_k-k,\ldots,j_1-1)=\{j_1<\cdots<j_k\}=\mathbf{j}$ and $b_{k,n}\big(c_{k,n}(\lambda)\big)=b_{k,n}\big(\{\lambda_k+1<\cdots<\lambda_1+k\}\big)=(\lambda_1,\ldots,\lambda_k)=\lambda$.

Finally, we have that $b_{k,n}(\mathbf{j})\vdash (j_k-k)+(j_{k-1}-(k-1))+\cdots+(j_1-1)=\mathrm{d}(\mathbf{j})$.
\end{proof}
\begin{remark}\label{rem:inf_poset_bijection} Fix some $k\in\mathbb{N}$ and notice that if $\mathbf{j}\in\binom{[n]}{k}$ and $j_k\leq n-1$, then $b_{k,n}(\mathbf{j})=b_{k,n-1}(\mathbf{j})$, so for $n\geq n_0$, $b_{k,n}|_{\binom{[n_0]}{k}}=b_{k,n_0}$. This lets us define the map $b_k:\binom{\mathbb{N}}{k}\to\big\{\lambda\in P:|\lambda|\leq k\}$ on $\mathbf{j}$ to be $b_{k,n}(\mathbf{j})$ for some $n\geq j_k$. Then, $b_k$ is also a bijection with $b_k(\mathbf{j})\vdash\mathrm{d}(\mathbf{j})$.
\end{remark}

We can define the following partial order on the set of all partitions
\begin{definition} For $\lambda,\lambda'\in P$, we write $\lambda\leq\lambda'$ if $\lambda_i\leq\lambda_i'$ for every $i\in\mathbb{N}$.
\end{definition}
\begin{remarks}
\begin{i_enum}
\item This is a partial order. Indeed, it is easy to check that $\lambda\leq\lambda$, that $\lambda\leq\lambda'$ and $\lambda'\leq\lambda$ gives $\lambda=\lambda'$ and that $\lambda\leq\lambda'$ and $\lambda'\leq\lambda''$ gives $\lambda\leq\lambda''$.
\item In the language of Young diagrams, $\lambda\leq\lambda'$ if and only if the corresponding diagram of $\lambda$ is contained fully inside the corresponding diagram of $\lambda'$.
\end{i_enum}
\end{remarks}

\begin{proposition} $(P,\leq)$ is a graded poset, with grading $\rho(\lambda)=m$, where $\lambda\vdash m$.
\end{proposition}
\begin{proof} First of all, we have to prove that $\rho$ is order preserving. Let $\lambda,\lambda'\in P$ with $\lambda\leq\lambda'$. Then $\sum_{i=1}^{|\lambda|}\lambda_i\leq\sum_{i=1}^{|\lambda'|}\lambda_i'$ and thus $\rho(\lambda)\leq\rho(\lambda')$. Moreover, we have to show that for $\lambda,\lambda'\in P$ with $[\lambda,\lambda']=\{\lambda,\lambda'\}$, $\rho(\lambda)=\rho(\lambda')$. Let $\lambda,\lambda'\in P$ with $\lambda<\lambda'$, $\lambda\vdash m$ and $\lambda'\vdash m'$. Notice that if $m'-m\geq2$, then there exists at least one $\lambda''\in P$ with $\lambda<\lambda''<\lambda'$. Indeed, in the young diagram of $\lambda$ is missing at least $2$ boxes, compared to the one of $\lambda'$ and by attaching just one of them we create some $\lambda''$ strictly between $\lambda$ and $\lambda'$.
\end{proof}

The poset up to degree $6$ looks like this:
\begin{center}
\begin{tikzpicture}[scale=1.3, yscale=1.3]
\ytableausetup{boxsize=0.3em,aligntableaux=top}
\node[elt] (0) at (0,0) {};
\node[elt] (1) at (0,1) {\ydiagram{1}};
\node[elt] (2) at (-0.5,2) {\ydiagram{2}};
\node[elt] (11) at (0.5,2) {\ydiagram{1,1}};
\node[elt] (3) at (-1,3) {\ydiagram{3}};
\node[elt] (21) at (0,3) {\ydiagram{2,1}};
\node[elt] (111) at (1,3) {\ydiagram{1,1,1}};
\node[elt] (4) at (-2,4) {\ydiagram{4}};
\node[elt] (31) at (-1,4) {\ydiagram{3,1}};
\node[elt] (22) at (0,4) {\ydiagram{2,2}};
\node[elt] (211) at (1,4) {\ydiagram{2,1,1}};
\node[elt] (1111) at (2,4) {\ydiagram{1,1,1,1}};
\node[elt] (5) at (-3,5) {\ydiagram{5}};
\node[elt] (41) at (-2,5) {\ydiagram{4,1}};
\node[elt] (32) at (-1,5) {\ydiagram{3,2}};
\node[elt] (311) at (0,5) {\ydiagram{3,1,1}};
\node[elt] (221) at (1,5) {\ydiagram{2,2,1}};
\node[elt] (2111) at (2,5) {\ydiagram{2,1,1,1}};
\node[elt] (11111) at (3,5) {\ydiagram{1,1,1,1,1}};
\node[elt] (5) at (-3,5) {\ydiagram{5}};
\node[elt] (41) at (-2,5) {\ydiagram{4,1}};
\node[elt] (32) at (-1,5) {\ydiagram{3,2}};
\node[elt] (311) at (0,5) {\ydiagram{3,1,1}};
\node[elt] (221) at (1,5) {\ydiagram{2,2,1}};
\node[elt] (2111) at (2,5) {\ydiagram{2,1,1,1}};
\node[elt] (11111) at (3,5) {\ydiagram{1,1,1,1,1}};
\node[elt] (6) at (-5,6) {\ydiagram{6}};
\node[elt] (51) at (-4,6) {\ydiagram{5,1}};
\node[elt] (42) at (-3,6) {\ydiagram{4,2}};
\node[elt] (411) at (-2,6) {\ydiagram{4,1,1}};
\node[elt] (33) at (-1,6) {\ydiagram{3,3}};
\node[elt] (321) at (0,6) {\ydiagram{3,2,1}};
\node[elt] (222) at (1,6) {\ydiagram{2,2,2}};
\node[elt] (3111) at (2,6) {\ydiagram{3,1,1,1}};
\node[elt] (2211) at (3,6) {\ydiagram{2,2,1,1}};
\node[elt] (21111) at (4,6) {\ydiagram{2,1,1,1,1}};
\node[elt] (111111) at (5,6) {\ydiagram{1,1,1,1,1,1}};
\draw (111111)--(11111);
\draw (21111)--(11111);
\draw (21111)--(2111);
\draw (2211)--(2111);
\draw (2211)--(221);
\draw (222)--(221);
\draw (3111)--(2111);
\draw (3111)--(311);
\draw (321)--(221);
\draw (321)--(311);
\draw (321)--(32);
\draw (411)--(311);
\draw (411)--(41);
\draw (33)--(32);
\draw (42)--(32);
\draw (42)--(41);
\draw (51)--(41);
\draw (51)--(5);
\draw (6)--(5);
\draw (11111)--(1111);
\draw (2111)--(1111);
\draw (2111)--(211);
\draw (221)--(211);
\draw (221)--(22);
\draw (311)--(211);
\draw (311)--(31);
\draw (32)--(22);
\draw (32)--(31);
\draw (41)--(31);
\draw (41)--(4);
\draw (5)--(4);
\draw (1111)--(111);
\draw (211)--(111);
\draw (211)--(21);
\draw (22)--(21);
\draw (31)--(21);
\draw (31)--(3);
\draw (4)--(3);
\draw (111)--(11);
\draw (21)--(11);
\draw (21)--(2);
\draw (3)--(2);
\draw (11)--(1);
\draw (2)--(1);
\draw (1)--(0);
\end{tikzpicture}
\end{center}

\begin{proposition} For $k,n\in\mathbb{N}$, $\binom{[n]}{k}\cong[\lambda_{\emptyset},(n-k)^k]$ as graded posets, where $(n-k)^k=(n-k,\ldots,n-k)\in\mathbb{N}^k$ is the $k\times(n-k)$ young diagram. The partial order on $\binom{[n]}{k}$ is defined by $\mathbf{j},\mathbf{j}'\in\binom{[n]}{k}$, $\mathbf{j}\leq\mathbf{j}'$ if and only if $j_i\leq j_i'$ for all $i\in[k]$ and the grading is given by $\rho(\mathbf{j})=\mathrm{d}(\mathbf{j})$.
\end{proposition}
\begin{proof} Notice that $[\lambda_{\emptyset},(n-k)^k]=\{\lambda\in P:|\lambda|\leq k\text{ and }\lambda_1\leq n-k\}$, so we have already established a bijection $b_{k,n}$ in Proposition~\ref{prop:poset_bijection}, which is easy to see that is also an order preserving map. Moreover, since we also have that the grading agrees, since $\rho(b_{k,n}(\mathbf{j}))=\mathrm{d}(\mathbf{j})=\rho(\mathbf{j})$.
\end{proof}


