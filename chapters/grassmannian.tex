%!TEX root = Cohomology of real Grassmannians.tex
\chapter{Grassmannians}
Since the topology of projective spaces has been thoroughly studied and characterized, a logical generalization is imposing a natural topology on the set of $k$-dimensional subspaces of a vector space for any $k\geq1$, which are called a \ul{Grassmannian}.

This chapter aims to present basic properties of Grassmannians, and is used as a point of reference throughout this thesis.
Through this chapter the reader is able to get to know the combinatorial structure of a Grassmannian and do basic computations using the Schubert decomposition of these manifolds.

Recall the definition of the real projective spaces as topological spaces,
$P\mathbb{R}^n\cong\faktor{\mathbb{R}^{n+1}\setminus\{0\}}{\sim}$,
where two vectors are equivalent, if they span the same line. Notice that in this definition a proper subset of the whole vector space is required, in order for the quotient to be well defined. Namely the set of all vectors, which span a line. In accordance with that, for the case of the Grassmannians, the discussion starts at the space of all $k$-tuples of vectors in $\mathbb{R}^n$, spanning a $k$-dimensional vector space.

\section{Stiefel Manifolds}
\begin{definition} Let $0<k\leq n$ be some natural numbers. Then, define $\St{k}{n}{\mathbb{R}}=\St{k}{n}$ as
$\St{k}{n}{\mathbb{R}}:=\left\{\left(\vec{v}_1,\ldots,\vec{v}_k\right)\in{\left(\mathbb{R}^n\right)}^k:\dim\left<v_1,\ldots,v_k\right>=k\right\}$
equipped with the subspace topology. Every point in this set is called a \ul{$k$-frame} of $\mathbb{R}^n$.
\end{definition}
This is actually an open subspace of the space of all $k$-tuples of vectors in $\mathbb{R}^n$:
\begin{proposition}\label{prop:St_open} Let $0<k\leq n$ be some natural numbers. Then $\St{k}{n}$ is an open submanifold of ${\left(\mathbb{R}^n\right)}^k$.
\end{proposition}
\begin{proof} Define the map $\Phi$ taking a $k$-tuple of vectors in $\mathbb{R}^n$ to the matrix in $\mathbb{R}^{n\times k}$ having these vectors as columns:
\begin{align*}
\Phi:{\left(\mathbb{R}^n\right)}^k&\to \mathbb{R}^{n\times k}\\[1em]
(v_1,\ldots,v_k)&\mapsto \left(\begin{array}{ccc}
|&&|\\[-.3em]
v_1&\cdots&v_k\\[-.3em]
|&&|\\
\end{array}\right)
\end{align*}
Since the topology on both spaces is the product topology and $\Phi$ respects it, $\Phi$ is a homeomorphism.

Define now the subset $D$ of all $n\times k$ matrices having at least one non-zero $k\times k$ minor
$D:=\left\{A\in\mathbb{R}^{n\times k}:\exists I\in\binom{[n]}{k}\text{ s.t. }\det(A_I)\neq0\right\}$
where, given some $I\in\binom{[n]}{k}$, $A_I\in\mathbb{R}^{k\times k}$ is the matrix formed from the $k$ rows of $A$ indexed by $I$. Then, we have
$\St{k}{n}=\Phi^{-1}(D)\cong D$
and since $D$ is an open submanifold of $\mathbb{R}^{n\times k}$, we also get that $\St{k}{n}$ is an open submanifold of ${\left(\mathbb{R}^n\right)}^k$.
\end{proof}
\begin{remark} In particular, this proves that $\St{k}{n}$ is a real manifold of dimension $kn$.
\end{remark}

We now define the Stiefel manifold as the space of all \ul{orthonormal} $k$-frames:
\begin{definition} Let $0<k\leq n$ be some natural numbers. Then, define the \ul{Stiefel} space $\StO{k}{n}{\mathbb{R}}=\StO{k}{n}$ as
$\StO{k}{n}{\mathbb{R}}:=\left\{\left(\vec{v}_1,\ldots,\vec{v}_k\right)\in{\left(\mathbb{R}^n\right)}^k:\vec{v}_i^t\vec{v}_j=\delta_{i,j}\right\}$
equipped with the subspace topology. Every point in this set is called an \ul{orthonormal $k$-frame} of $\mathbb{R}^n$.
\end{definition}
Notice that we obviously have $\StO{k}{n}\subseteq\St{k}{n}$.

\begin{proposition}\label{prop:StO_dim_closed} Let $0<k\leq n$ be some natural numbers. Then $\StO{k}{n}$ is a closed submanifold of ${\left(\mathbb{R}^n\right)}^k$ of dimension $nk-\frac{k(k+1)}{2}$.
\end{proposition}
%TODO: Write this better
\begin{proof}
We need again the homeomorphism $\Phi:{\left(\mathbb{R}^n\right)}^k\to\mathbb{R}^{n\times k}$ taking a $k$-tuple in $\mathbb{R}^n$ to the matrix in $\mathbb{R}^{k\times n}$ having these vectors as columns. This time, define the subset $S$ of all $n\times k$ semi-orthogonal matrices
$S:=\left\{A\in\mathbb{R}^{n\times k}:A^t A=I_k\right\}$

Then, we have
$\StO{k}{n}=\Phi^{-1}(S)\cong S$,
where $S$ is a subset of $\mathbb{R}^{n\times k}$, defined by $\binom{k+1}{2}$ (linearly independent) equations. This makes $S$ a closed submanifold of $\mathbb{R}^{n\times k}$, of dimension $nk-\binom{k+1}{2}$. Thus, $\StO{k}{n}$ is a closed submanifold of ${\left(\mathbb{R}^n\right)}^k$ of the same dimension.
\end{proof}
\begin{remark} Let $0<k\leq n$ be some natural numbers. Then $\StO{k}{n}$ is a bounded subset of ${\left(\mathbb{R}^n\right)}^k$, with the metric induced by the isomorphism ${\left(\mathbb{R}^n\right)}^k\cong\mathbb{R}^{kn}$. Indeed, if $f=(v_1,\ldots,v_k)\in{\left(\mathbb{R}^n\right)}^k$ is an orthonormal $k$-frame of $\mathbb{R}^n$, then
$\left\|f\right\|_2^2=\sum_{i=1}^k\left\|v_i\right\|_2^2=k$
\end{remark}

\begin{lemma}\label{lem:StO_compact} Let $0<k\leq n$ be some natural numbers. Then $\StO{k}{n}$ is compact. Indeed the previous Proposition and Remark prove that $\StO{k}{n}$ is a closed and bounded subset of ${\left(\mathbb{R}^n\right)}^k\cong\mathbb{R}^{kn}$, which proves the assertion.
\end{lemma}

%TODO: change to: "Let k,n\in\mathbb N such that 0<k<n"
\begin{definition} Let $0<k<n$ be some natural numbers. Then, the real \ul{Grassmann space} $\Gr{k}{n}{\mathbb{R}}=\Gr{k}{n}$ is the set of all linear $k$-dimensional subspaces of $\mathbb{R}^n$, equipped with the quotient topology,
$\Gr{k}{n}{\mathbb{R}}:=\faktor{\St{k}{n}{\mathbb{R}}}{\sim}$,
where $(\vec{v}_1,\ldots,\vec{v}_k)\sim(\vec{u}_1,\ldots,\vec{u}_k)$, if $\left<\vec{v}_1,\ldots,\vec{v}_k\right>=\left<\vec{u}_1,\ldots,\vec{u}_k\right>$.
\end{definition}

\begin{lemma}\label{lem:q_open} Let $0<k<n$ be some natural numbers. Then the quotient map $q:\St{k}{n}\to\Gr{k}{n}$ is an open map.
\end{lemma}
\begin{proof} Let $U\subseteq\St{k}{n}$ be an open set in $\St{k}{n}$, then $q(U)$ is open in $\Gr{k}{n}$ if and only if $q^{-1}(q(U))$ is open in $\St{k}{n}$, since $q$ is a quotient map. Moreover, $q^{-1}(q(U))$ is open in $\St{k}{n}$ if and only if it is open in $(\mathbb{R}^{n})^k$, since $\St{k}{n}$ is open in $(\mathbb{R}^{n})^k$. This means that it suffices to prove that $\tilde{U}:=q^{-1}(q(U))$ is open in $(\mathbb{R}^{n})^k$. In order to do this, we first construct a linear isomorphism $f_L:(\mathbb{R}^n)^k\to(\mathbb{R}^n)^k$ for every $L\in GL(k)$, as follows: Let $(x_1,\ldots,x_k)\in(\mathbb{R}^n)^k$ and define the linear maps $R_{(x_1,\ldots,x_k)}:\mathbb{R}^n\to\mathbb{R}^k$ and $S_{(x_1,\ldots,x_k)}:\mathbb{R}^k\to\mathbb{R}^n$ such that $R_{(x_1,\ldots,x_k)}(x_i)=e_i$ for every $i\in[k]$, $R_{(x_1,\ldots,x_k)}(v)=0$ for every $v\in\mathrm{span}\{x_1,\ldots,x_k\}^{\perp}$ and $S_{(x_1,\ldots,x_k)}(e_i)=x_i$ for every $i\in[k]$, where $e_1,\ldots,e_k$ is the usual basis of $\mathbb{R}^k$. Then, define
\begin{center}
\begin{tikzcd}
f_L\ar[r,phantom,":"]&[-3.5em](\mathbb{R}^n)^k\ar[r,"R_{(-)}"]&\mathbb{R}^k\ar[r,"L^{\oplus k}"]&\mathbb{R}^k\ar[r,"S_{(-)}"]&(\mathbb{R}^n)^k\\[-1.5em]
&(x_1,\ldots,x_k)\ar[r,mapsto]&(e_1,\ldots,e_k)\ar[r,mapsto]&(Le_1,\ldots,Le_k)\ar[r,mapsto]&\left(\sum_{j=1}^ke_i^t(Le_j)x_j\right)_{i\in[k]}
\end{tikzcd}
\end{center}
and notice that for every fixed $L$, $f_L$ is linear: $f_L\big(\lambda(x_1,\ldots,x_k)+\mu(y_1,\ldots,y_k)\big)=\left(\sum_{j=1}^ke_i^t(Le_j)(\lambda x_j+\mu y_j)\right)_{i\in[k]}=\lambda\left(\sum_{j=1}^ke_i^t(Le_j)x_j\right)_{i\in[k]}+\mu\left(\sum_{j=1}^ke_i^t(Le_j)y_j\right)_{i\in[k]}=\lambda f_L(x_1,\ldots,x_k)+\mu f_L(y_1,\ldots,y_k)$. Also, $f_L$ is injective: indeed $f_L(x_1,\ldots,x_k)=0$ means $\left(\sum_{j=1}^ke_i^t(Le_j)x_j\right)_{i\in[k]}=0$ which can be written as $\big(x_1|\cdots|x_k\big)\cdot\big(e_i^t(Le_j)\big)=0$. Since the matrix $\big(e_i^t(Le_j)\big)$ is the the matrix of $L$ on the basis $e_1,\ldots,e_k$ which is invertible, this gives $x_1=\cdots=x_k=0$. So, $f_L$ is an injective linear endomorphism and thus a linear isomorphism. Next, we prove that
$\tilde{U}=\bigcup_{L\in GL(k)}f_L(U)$

For some $(u_1,\ldots,u_k)\in(\mathbb{R}^n)^k$, we have $(u_1,\ldots,u_k)\in\tilde{U}$ if and only if there exists $(x_1,\ldots,x_k)\in U$ with $\mathrm{span}(x_1,\ldots,x_k)=\mathrm{span}(u_1,\ldots,u_k)$. This is equivalent to the existence of $\{\lambda_{i,j}\}_{i,j\in[k]}\in\mathbb{R}$ such that $u_i=\sum_{j=1}^k\lambda_{i,j}x_j$ for every $i\in[k]$. On the one hand, if the $\lambda_{i,j}$ are defined, then we can also define $L_0\in GL(k)$ by $L_0e_j:=\sum_{i=1}^k\lambda_{i,j}e_i$ for every $j\in[k]$, which gives $f_{L_0}(x_1,\ldots,x_k)=\left(\sum_{j=1}^ke_i^t(L_0e_j)x_j\right)_{i\in[k]}=\left(\sum_{j=1}^k\lambda_{i,j}x_j\right)_{i\in[k]}=(u_i)_{i\in[k]}$, i.e. $(u_1,\ldots,u_k)\in f_{L_0}(U)$. On the other hand, if $(u_1,\ldots,u_k)\in f_{L_0}(U)$ for some $L_0\in GL(k)$, then $(u_1,\ldots,u_k)=f_{L_0}(x_1,\ldots,x_k)=\left(\sum_{j=1}^ke_i^t(L_0e_j)x_j\right)_{i\in[k]}$. Hence, we can define $\lambda_{i,j}:=e_i^t(L_0e_j)$ for every $i,j\in[k]$, which gives that $u_i=\sum_{j=1}^ke_i^t(L_0e_j)x_j$.
\end{proof}

Having the case of projective spaces in mind, it is natural to also provide an alternative definition of Grassmannians.
Specifically, the projective space of some dimension is also defined as the quotient over the unit sphere, rather than over the set of every nonzero vector.
In this case the Stiefel manifold $\StO{k}{n}$, is the analog needed.

\begin{proposition} Let $0<k<n$ be some natural numbers. Let $q:\St{k}{n}\to\Gr{k}{n}$ be the quotient map used in the definition of the Grassmannians and let $q_0:=q|_{\StO{k}{n}}:\StO{k}{n}\to\Gr{k}{n}$ be its restriction to the Stiefel manifold. Then $q_0$ is surjective and continuous. In other words, $q_0$ and $q$ induce the same quotient topology on the set of all $k$-planes in $\mathbb{R}^n$.
\end{proposition}
\begin{proof}
Let $i:\StO{k}{n}\to\St{k}{n}$ be the inclusion and $\mathfrak{gs}:\St{k}{n}\to\StO{k}{n}$ be the Gram-Schmidt process. Then, the following diagram commutes:
\begin{center}
\begin{tikzcd}
\StO{k}{n}\ar[r,hook,"i"]\ar[rd,"q_0"']&\St{k}{n}\ar[r,"\mathfrak{gs}"]\ar[d,two heads,"q"']&\StO{k}{n}\ar[dl,"q_0"]\\
&\Gr{k}{n}
\end{tikzcd}
\end{center}
The left part of this diagram implies $q_0=q\circ i$, and hence that $q_0$ is continuous. Moreover, the right part gives $q=q_0\circ\mathfrak{gs}$, which results in two conclusions: First that $q_0$ is a surjective map as well, since $q$ is surjective and second that $q_0$ is an open map, since $\mathfrak{gs}$ is surjective and $q$ is open, as proved in \ref{lem:q_open}.

Thus, the map $q_0$ is a quotient map. In other words, the maps $q$ and $q_0$ induce the same quotient topology on the set of all $k$-dim subspaces of $\mathbb{R}^n$.
\end{proof}

\section{Infinite Grassmannian}
So far, we presented the Grassmannians separately for each $k$ and $n$. In fact, there is a natural embedding of $\Gr{k}{n}$ into $\Gr{k}{n+1}$, since every $k$-plane in $\mathbb{R}^n$ is also a $k$-plane in $\mathbb{R}^{n+1}$. This naturally leads to the construction of a so called infinite Grassmannian for each $k$.

\begin{definition} Let $(X,\tau)$ be a topological space and $\{X_i\}_{i\in\mathbb{N}}$ a collection of subsets of $X$, such that $X_1\subseteq X_2\subseteq X_3\subseteq\cdots$. Then, $X$ is the \emph{direct limit} of $X_1,X_2,\ldots$ and $\tau$ is the \emph{direct limit topology on $\mathbb{X}$} if $X=\bigcup_{i=1}^{\infty}X_i$ and $\tau$ is the final topology with respect to all inclusions. In this case, we write $X=\varinjlim X_i$.
\end{definition}
\begin{remark} This means that
\begin{b_item}
\item for every $x\in X$ there exists some $i\in\mathbb{N}$ with $x\in X_i$ and
\item for any subset $A\subseteq X$, $A$ is open in $X$ if and only if $A\cap X_i$ is open in $X_i$ for every $i\in\mathbb{N}$.
\end{b_item}
\end{remark}

\begin{definition} Let $\mathbb{R}^1\subseteq\mathbb{R}^2\subseteq\cdots$ be the usual inclusions and let $\mathbb{R}^{\infty}:=\bigcup_{i=1}^{\infty}\mathbb{R}^i=\big\{(x_1,x_2,\ldots):\exists i\in\mathbb{N}\text{ s.t. }\forall n\geq i\ x_n=0\big\}$. Then $\mathbb{R}^{\infty}$ equipped with the direct limit topology and the vector space structure inherited by the inclusions is the \emph{infinite coordinate space}.
\end{definition}
\begin{remark} Notice that $\mathbb{R}^{\infty}$ does not contain arbitrary sequences of real numbers, i.e. that $\mathbb{R}^{\infty}\cong\bigoplus_{i=1}^{\infty}\mathbb{R}\not\cong\prod_{i=1}^{\infty}\mathbb{R}\cong l_2$ as vector spaces.
\end{remark}

\begin{lemma} Let $k,n\in\mathbb{N}$ such that $0<k<n$. Then, there exists a natural embedding $i_{k,n}:\St{k}{n}\hookrightarrow\St{k}{n+1}$ induced by $inc_{k,n}:(\mathbb{R}^n)^k\hookrightarrow(\mathbb{R}^{n+1})^k$ with $inc_{k,n}(u_1,\ldots,u_k)=(\bar{u}_1,\ldots,\bar{u}_k)$ and $\bar{u}_i=(u_i^t|0)^t$ for every $i\in[k]$.
\end{lemma}
\begin{proof} Let $(u_1,\ldots,u_k)\in\St{k}{n}$. Then $\dim\mathrm{span}\{\bar{u}_1,\ldots,\bar{u}_k\}=k$, which means that $inc_{k,n}(u_1,\ldots,u_k)\in\St{k}{n}$, so $i_{k,n}=inc_{k,n}|_{\St{k}{n}}:\St{k}{n}\to\St{k}{n+1}$ is well defined. Let $A$ be some open subset of $\St{k}{n}$, i.e. there exists some $B$, open subset of $(\mathbb{R}^n)^k$ such that $A=B\cap\St{k}{n}$. Since $inc_{k,n}$ is an embedding, there exists some $C$, open subset of $(\mathbb{R}^{n+1})^k$ such that $B=inc_{k,n}^{-1}(C)$, i.e. $A=inc_{k,n}^{-1}(C)\cap\St{k}{n}=inc_{k,n}^{-1}(C\cap\St{k}{n+1})=i_{k,n}(C\cap\St{k}{n+1})$ which proves that $i_{k,n}$ is an embedding.
\end{proof}
\begin{remark} The same arguments let us define a natural embedding $i_{k,n}:\StO{k}{n}\hookrightarrow\StO{k}{n+1}$.
\end{remark}

\begin{proposition}\label{prop:gr_embedding} Let $k,n\in\mathbb{N}$ such that $0<k<n$. Then, there exists a natural embedding $\iota_{k,n}:\Gr{k}{n}\hookrightarrow\Gr{k}{n+1}$ making the following diagram commute:
\begin{center}
\begin{tikzcd}
\St{k}{n}\ar[d,two heads,"q"]\ar[r,hook,"i_{k,n}"]&\St{k}{n+1}\ar[d,two heads,"q"]\\
\Gr{k}{n}\ar[r,hook,dotted,"\iota_{k,n}"]&\Gr{k}{n+1}
\end{tikzcd}
\end{center}
\end{proposition}
\begin{proof} Let $(u_1,\ldots,u_k),(u_1',\ldots,u_k')\in\St{k}{n}$ with $\mathrm{span}\{u_1,\ldots,u_k\}=\mathrm{span}\{u_1',\ldots,u_k'\}$, then it is also true that $\mathrm{span}\{\bar{u}_1,\ldots,\bar{u}_k\}=\mathrm{span}\{\bar{u}_1',\ldots,\bar{u}_k'\}$, where $\bar{u}_i=(u^t|0)^t$. Since $q\circ i_{k,n}(u_1,\ldots,u_k)$ depends only on $q(u_1,\ldots,u_k)$, the universal property of the quotient ensures the existence of a unique $\iota_{k,n}:\Gr{k}{n}\to\Gr{k}{n+1}$ making the diagram commute. Let $A$ be an open subset of $\Gr{k}{n}$, then $q^{-1}(A)$ is an open subset of $\St{k}{n}$. Since $i_{k,n}$ is an embedding, there exists some open $B\in\St{k}{n+1}$ with $q^{-1}(A)=i_{k,n}^{-1}(B)$. Since $q$ is an open map, as proved in Lemma~\ref{lem:q_open}, $q(B)$ is also open, which makes $\iota_{k,n}^{-1}(q(B))\subseteq\Gr{k}{n}$ open. It suffices now to show that $A=\iota_{k,n}^{-1}(q(B))$. Let $H\in A$, then $H=q(u_1,\ldots,u_k)$ for some $(u_1,\ldots,u_k)\in q^{-1}(A)=i_{k,n}^{-1}(B)$. Which means that $(\bar{u}_1,\ldots,\bar{u}_k)\in B$, and by the commutativity of the diagram, we get $\iota_{k,n}(H)=(\iota_{k,n}\circ q)(u_1,\ldots,u_k)=q(\bar{u}_1,\ldots,\bar{u}_k)\in q(B)$, i.e. $H\in\iota_{k,n}^{-1}(q(B))$.

For the other direction, let $H\in\iota_{k,n}^{-1}(q(B))$. This means that there exists some $(a_1,\ldots,a_k)\in\St{k}{n+1}$ with $\iota_{k,n}(H)=\mathrm{span}(a_1,\ldots,a_k)$. Since $q$ is a surjection, there exists some $(u_1,\ldots,u_k)\in\St{k}{n}$ such that $\mathrm{span}(u_1,\ldots,u_k)=H$. Because of the commutativity of the diagram, this gives $\mathrm{span}(a_1,\ldots,a_k)=\mathrm{span}(\bar{u}_1,\ldots,\bar{u}_k)$. Notice now that this can only be the case if the last coordinate of each $a_i$ is equal to $0$, i.e. if there exist some $(x_1,\ldots,x_k)\in\St{k}{n}$ with $\bar{x}_i=a_i$ for every $i$. This gives that $\mathrm{span}\{\bar{u}_1,\ldots,\bar{u}_k\}=\mathrm{span}\{\bar{x}_1,\ldots,\bar{x}_k\}$ which can only be the case if $\mathrm{span}\{u_1,\ldots,u_k\}=\mathrm{span}\{x_1,\ldots,x_k\}$. Also, because of the definition of $B$, $(x_1,\ldots,x_k)\in i_{k,n}^{-1}(B)=q^{-1}(A)$ and thus we finally get $H=\mathrm{span}(u_1,\ldots,u_k)=\mathrm{span}(x_1,\ldots,x_k)\in A$.
\end{proof}
\begin{remark}\label{rem:gr_inclusions} Topologically there is no difference between an embedding and a subspace, so from now on we are going to write $\Gr{k}{n}\subseteq\Gr{k}{n+1}$ and mean the embedding $\iota_{k,n}$ which, as described in \ref{prop:gr_embedding}, takes a $k$-plane $H\subseteq\mathbb{R}^n$ to $H\subseteq\mathbb{R}^{n+1}$ by embedding $\mathbb{R}^n$ into $\mathbb{R}^{n+1}$ the usual way. Similarly for $\St{k}{n}\subseteq\St{k}{n+1}$ and $\StO{k}{n}\subseteq\StO{k}{n+1}$.
\end{remark}

\begin{definition} Let $k\in\mathbb{N}$ and $\Gr{k}{k+1}\subseteq\Gr{k}{k+2}\subseteq\cdots$ be the inclusions as described in Remark~\ref{rem:gr_inclusions} and let $\Gr{k}:=\bigcup_{n=k+1}^{\infty}\Gr{k}{n}$. Then $\Gr{k}$ equipped with the direct limit topology is the \emph{infinite Grassmannian}.
\end{definition}

\begin{definition} Let $k\in\mathbb{N}$ and $\St{k}{k}\subseteq\St{k}{k+1}\subseteq\cdots$ be the inclusions as described in Remark~\ref{rem:gr_inclusions} and let $\St{k}:=\bigcup_{n=k}^{\infty}\St{k}{n}$. Then $\St{k}$ equipped with the direct limit topology is the \emph{infinite Stiefel manifold}. Analogously the \emph{infinite compact Stiefel manifold} $\StO{k}=\varinjlim\StO{k}{n}$ is defined.
\end{definition}
\begin{remark}\label{rem:st_inclusion} Notice that $\StO{k}\subseteq\St{k}$. Indeed, we define the inclusion map as follows: Let $(u_1,\ldots,u_k)\in\StO{k}$, then there exists some $n\in\mathbb{N}$ such that $(u_1,\ldots,u_k)\in\StO{k}{n}\subseteq\St{k}{n}\subseteq\St{k}$.
\end{remark}

\begin{proposition} Let $k\in\mathbb{N}$ and also let $q:\St{k}\to\Gr{k}$ be the map induced by all the inclusions, i.e. $q(u_1,\ldots,u_k)=q|_{\St{k}{n}}(u_1,\ldots,u_k)$, for some $n$ sufficiently large. Moreover, let $q_0:\StO{k}\to\Gr{k}$ be the restriction $q|_{\StO{k}}$. Then, $\Gr{k}\cong\faktor{\St{k}}{\sim}\cong\faktor{\StO{k}}{\sim}$, where the equivalence relation is the same as in $\St{k}{n}$ for some large enough $n$.
\end{proposition}
\begin{proof} First of all, $q$ is a surjection. Indeed, let $H\in\Gr{k}{n}$, then there exists some $(u_1,\ldots,u_k)\in\St{k}{n}\subseteq\St{k}$ with $q(u_1,\ldots,u_k)=H$. Next, in order to prove that $q$ is in fact a quotient map, let $U\subseteq\Gr{k}$ such that $q^{-1}(U)$ is open in $\St{k}$. We want to prove that $U$ is open in $\Gr{k}$. For that, we will use the definition of the direct limit topology and fix some $n\in\mathbb{N}$ with $n>k$. Then, since $\St{k}$ is equipped with the direct limit topology and $q^{-1}(U)$ is open, $q^{-1}(U)\cap\St{k}{n}=q^{-1}(U\cap\Gr{k}{n})$ is open as well. Since $q$ is a quotient map, this means that $U\cap\Gr{k}{n}$ is open as well. Hence, $U\cap\Gr{k}{n}$ is open for every $n>k$ and thus $U$ is open in $\Gr{k}$.

For the case of $q_0:\StO{k}\to\Gr{k}$ notice that the map is well defined, since Reark~\ref{rem:st_inclusion} guaranties that $\StO{k}\subseteq\St{k}$. Then, the arguments are exactly the same as in the previous case.
\end{proof}
\begin{remark} This proves that the infinite Grassmannian is the set of all $k$-planes inside $\mathbb{R}^{\infty}$. Notice that these planes never contain a vector with infinitely many non-zero entries, since such vectors do not exist inside $\mathbb{R}^{\infty}$.
\end{remark}

\section{Grassmannians are Manifolds}
In this section it is proven that Grassmannians are in fact compact topological manifolds.
\begin{lemma}\label{lem:gr_manifold} For each pair of natural numbers $k,n$, with $0<k<n$, the space $\Gr{k}{n}$ is compact, Hausdorff and locally homeomorphic to $\mathbb{R}^{k(n-k)}$.
\end{lemma}

\begin{proof} \begin{b_item}
\item The Stiefel manifold $\StO{k}{n}$ is a compact topological space, as proven in the Appendix in Lemma~\ref{lem:StO_compact} and $q_0$ is a continuous map. Thus, $\Gr{k}{n}=q_0\left(\StO{k}{n}\right)$ is also compact.
\item In order to show that $\Gr{k}{n}$ is Hausdorff, it suffices to show that it is completely Hausdorff, i.e.\ that any two distinct points in $\Gr{k}{n}$ can be separated by a continuous function $\Gr{k}{n}\to\mathbb{R}$. First, for every vector $v\in\mathbb{R}^n$, we define $\phi_v:\StO{k}{n}\to\mathbb{R}$ which maps each orthonormal $k$-frame $(v_1,\ldots,v_k)$ to the square of the distance between $v$ and the linear space spanned by $(v_1,\ldots,v_k)$, i.e.
$\phi_v(v_1,\ldots,v_k)=v^tv-\sum_{i=1}^k{\left(v^tv_i\right)}^2$.
This is a continuous map, which depends only on the spanned $k$-plane, i.e.\ if two $k$-frames have the same image under $q_0$, they also have the same image under $\phi_v$. The universal property of the quotient map $q_0$ implies that there exists a unique continuous map $\psi_v$ making the following diagram commute:

\begin{center}
\begin{tikzcd}
\StO{k}{n}\ar[d,"q_0"']\ar[dr,"\phi_v"]\\
\Gr{k}{n}\ar[r,"\psi_v"',dotted]&\mathbb{R}
\end{tikzcd}
\end{center}

Moreover, $\psi_v(H)=0$ iff $v\in H$, because $\phi_v(v_1,\ldots,v_k)=0$ iff $v\in \left<v_1,\ldots,v_k\right>$.
Let now $H_1,H_2\in\Gr{k}{n}$ be two distinct $k$-planes and $v\in H_1\setminus H_2$.
Since $\psi_v(H_1)=0\neq\psi_v(H_2)$, $\Gr{k}{n}$ is completely Hausdorff.

\item Next it is proven that for every $H\in\Gr{k}{n}$, the subset $\mathcal{U}_H:=\left\{K\in\Gr{k}{n}:K\cap H^{\perp}=\{0\}\right\}$ of $\mathbb{R}^n$ is a neighborhood of $H$, homeomorphic to $\mathbb{R}^{k(n-k)}$.

Firstly, one can fix an orthonormal basis $\{u_1,\ldots,u_k\}$ of $H$ and an orthonormal basis $\{\bar{u}_1,\ldots,\bar{u}_{n-k}\}$ of $H^{\perp}$. For the rest of this proof regard $\mathbb{R}^n$ as the direct sum $H\oplus H^{\perp}$. Define also the orthogonal projections $p_H,p_{H^{\perp}}$ from $H\oplus H^{\perp}$ to each component.
%Dialexe an tha xrhsimopoieis iff h if and only if kai kan to pantou.
Consider now $\mathcal{U}_H$ to be the set of all $k$-planes $K$ in $H\oplus H^{\perp}$ for which the map $\left.p_H\right|_{K}$ is a homeomorphism. Indeed, the subspace $K\cap H^{\perp}$ is at least one dimensional, iff $\left.p_H\right|_K$ is not injective, iff $\left.p_H\right|_K$ is not surjective.

Each $K\in\mathcal{U}_H$ can now be considered as the graph of some linear transformation $T_K$ inside $\Hom_{\mathbb{R}}\left(H,H^{\perp}\right)$ (which has the desired dimension as a vector space over $\mathbb{R}$). We procceed to define rigorously the function taking $K$ to $T_K$ and prove that this is bicontinuous. In particular, we prove that $T_K$ depends continuously on $K$ and vice versa, through a rather technical approach.

The construction of the desired function is based on the universal property of the quotient map $q:\St{k}{n}\to\Gr{k}{n}$. Moreover, since we need the topology of $\Hom_{\mathbb{R}}\left(H,H^{\perp}\right)$, we use the fact that its topology is also an induced topology by an isomorphism to $\mathbb{R}^{k(n-k)}$. Finally, this last space, being a direct product, is also topologized through the initial topology with respect to the orthogonal projections on each coordinate. These steps are summarized in the following commutative diagram:

\begin{center}
\begin{tikzcd}
&[-2em]&[-2em] (a_1,\ldots,a_k)\ar[rr,mapsto]&[-1em]&[-7em]\bar{u}_j^t\cdot p_{H^{\perp}}\circ{\left(\left.p_H\right|_{\left<a_1,\ldots,a_k\right>}\right)}^{-1}\left(u_i\right)&[-7em]\\[-2em]
{\left(\mathbb{R}^n\right)}^k\ar[r,phantom,"\supseteq"]&\St{k}{n}\ar[d,"q"]\ar[r,phantom,"\supseteq"]&q^{-1}\left(\mathcal{U}_H\right)\ar[d,"\left.q\right|"]\ar[dr,dotted,"\tilde{T}"']\ar[drrr,dotted,"f"']\ar[rrr,"f_{i,j}"']&&&\mathbb{R}\\[2em]
&\Gr{k}{n}\ar[r,phantom,"\supseteq"]&\mathcal{U}_H\ar[r,dotted,"T"]&\Hom_{\mathbb{R}}\left(H,H^{\perp}\right)\ar[rr,"\phi","\cong"']&&\mathbb{R}^{[k]\times[n-k]}\ar[u,"\pi_{i,j}"]\\[-2em]
&&K\ar[r,mapsto,dotted]&p_{H^{\perp}}\circ\left(\left.p_H\right|_K\right)^{-1}
\end{tikzcd}
\end{center}
Towards defining the map $T$, which takes each $k$-space $K$ to the function with graph $K$, we first define the map $\tilde T$ which is later going to be equal with $T\circ\left.q\right|$.
\[\tilde T(a_1,\ldots,a_k)=p_{H^{\perp}}\circ{\left(\left.p_H\right|_{\left<a_1,\ldots,a_k\right>}\right)}^{-1}\]
Notice that $\tilde T$ is well defined, since $q^{-1}\left(\mathcal{U}_H\right)$ is exactly the set of all $k$-frames, for which the map $\left.p_H\right|_{\left<a_1,\ldots,a_k\right>}$ is a homeomorphism. Also, notice that $\tilde T$ has the same value for any two $k$-frames that span the same $k$-space, since on input $(a_1,\ldots,a_k)$ the output only depends on $\left<a_1,\ldots,a_k\right>$.

In order to argue that $\tilde T$ is a continuous map, remember how $\Hom_{\mathbb{R}}\left(H,H^{\perp}\right)$ is topologized.
Having fixed the bases ${\{u_i\}}_{i\in[k]}$ and ${\{\bar{u}_j\}}_{j\in[n-k]}$, we define an isomorphism of vector spaces $\phi$, which takes a linear map $L$ to
$\phi(L)={\left(\bar{u}_j^t\cdot L(u_i)\right)}_{(i,j)\in[k]\times[n-k]}\in\mathbb{R}^{[k]\times[n-k]}$.
Then, the natural topology on $\Hom_{\mathbb{R}}\left(H,H^{\perp}\right)$ is the induced topology by $\phi$. This means that our $\tilde T$ is continuous iff $f:=\phi\circ\tilde T$ is continuous.

Finally, since $\mathbb{R}^{[k]\times[n-k]}=\prod_{(i,j)\in[k]\times[n-k]}\mathbb{R}$ is a categorical product, $\mathbb{R}^{[k]\times[n-k]}$ is equipped with the initial topology, with respect to all orthogonal projections ${\left(\pi_{i,j}\right)}_{(i,j)\in[k]\times[n-k]}$. Hence, the function $f$ that interests us is continuous iff every $f_{i,j}:=\pi_{i,j}\circ f$ is continuous. In order to show that every $f_{i,j}$ is a continuous map, we express it through linear algebra:

First, define $A$ to be the matrix of the linear function taking $u_i$ to $a_i$ for each $i\in[k]$, expressed in the bases $\{u_i\}$ and $\{\bar{u}_j\}$:
\[A=\left(\begin{array}{ccc}
a^t_1u_1&\cdots&a^t_ku_1\\
\vdots&\ddots&\vdots\\
a^t_1u_k&\cdots&a^t_k\cdot u_k\\
\midrule
a^t_1\bar{u}_1&\cdots&a^t_k\bar{u}_1\\
\vdots&\ddots&\vdots\\
a^t_1\bar{u}_{n-k}&\cdots&a^t_k\bar{u}_{n-k}\\
\end{array}\right)=:
\left(\begin{array}{c}A_H\\\midrule A_{H^{\perp}}\\
\end{array}\right)
\in\mathbb{R}^{n\times{k}}\]
The two blocks $A_H\in\mathbb{R}^{k\times k}$ and $A_{H^{\perp}}\in\mathbb{R}^{(n-k)\times k}$ of $A$ correspond to the maps taking $u_i$ to the projections of $a_i$ inside $H$ and inside $H^{\perp}$ respectively.

Notice that the matrix corresponding to the linear map $p_{H^{\perp}}$, with regard to the bases fixed is
\[\left(0_{(n-k)\times k}|I_{n-k}\right)\in\mathbb{R}^{(n-k)\times n}\]
and the matrix corresponding to the linear map ${\left(\left.p_H\right|_{\left<a_1,\ldots,a_k\right>}\right)}^{-1}$, with regard to the same bases is
\[A\cdot A_H^{-1}=\left(\begin{array}{c}
I_k\\
\midrule
A_{H^{\perp}}\cdot A_H^{-1}\\
\end{array}\right)
\in\mathbb{R}^{n\times{k}}\]
Indeed, since $\left<a_1,\ldots,a_k\right>\in\mathcal{U}_H$, $\left\{p_H(a_1),\ldots,p_H(a_k)\right\}$ is a basis of $K$, which means that $A_H$ is invertible and thus $A\cdot A_H^{-1}$ is well defined. Moreover, we can easily compute that the application of this matrix to any $p_H(a_r)$ gives us $a_r$ which is exactly what the map ${\left({\left.p_H\right|}_{\left<a_1,\ldots,a_k\right>}\right)}^{-1}$ does to the same basis, which proves our assertion.

This means, that the map $f_{ij}$ takes the element $\left(a_1,\ldots,a_k\right)$ to the real number
\[\begin{array}{rcl}f_{ij}\left(a_1,\ldots,a_k\right)
&=&\bar{u}_j^t\cdot p_{H^{\perp}}\circ{\left(\left.p_H\right|_{\left<a_1,\ldots,a_k\right>}\right)}^{-1}(u_i)\\
&=&\bar{u}_j^t\cdot\left(\left.0_{(n-k)\times k}\right|I_{n-k}\right)\cdot\left(\begin{array}{c}I_k\\\midrule A_{H^{\perp}}\cdot A_H^{-1}\\\end{array}\right)\cdot u_i\\
&=&\bar{u}_j^t\cdot A_{H^{\perp}}\cdot A_H^{-1}\cdot u_i\\
\end{array}\]
Since inner product, matrix multiplication and inversion are continuous, $f_{ij}$ is a continuous function for every $i,j\in[k]\times[n-k]$. This proves that $f$ is continuous as well, and hence $\tilde T$ is also continuous. Since $\tilde T$ is a continuous function which depends only on the $k$-plane spanned by its input, the universal property of the quotient spaces ensures the existence of a continuous map $T:\mathcal{U}_H\to\Hom_{\mathbb{R}}\left(H,H^{\perp}\right)$ such that $T\circ \left.q\right|=\tilde T$. The uniqueness condition of this property ensures that the map $T$ is indeed the one taking $K$ to $p_{H^{\perp}}\circ{\left(\left.p_H\right|_K\right)}^{-1}$.

Through viewing this function as taking $K$ to the linear function, whose graph is $K$, we see that this function is both one to one and onto, since two linear maps are different iff they have different graphs. This makes $T$ a homeomorphism, proving that there exists the homeomorphism
$\mathcal{U}_H\overset{\phi\circ T}{\cong}\mathbb{R}^{n(n-k)}$
which concludes the proof of this lemma.\qedhere
\end{b_item}
\end{proof}
In the next section we are going to further examine these spaces topologically and build the appropriate language in order to tackle problems regarding their Homology and Cohomology structures. Before we dive in into this topic though, it would be useful to notice a first duality between these spaces, arising from the duality between a $k$-space inside $\mathbb{R}^n$ and its $n-k$ complement.

In order to prove the following lemma the open sets $\mathcal{U}_H$ defined in the proof of Lemma~\ref{lem:gr_manifold} are used.

\begin{lemma} \label{lem:gr_duality}
	For each pair of natural numbers $k,n$, with $0<k<n$, the space $\Gr{k}{n}$ is homeomorphic to $\Gr{n-k}{n}$, with the homeomorphism taking some $k$-space to its orthogonal complement inside $\mathbb{R}^n$.
\end{lemma}

\begin{proof} The orthogonal-complement function $\perp:\Gr{k}{n}\to\Gr{n-k}{n}$ is obviously one to one and onto. Thus, it suffices to show that it is continuous, since ${\left(K^{\perp}\right)}^{\perp}=K$, for all spaces, continuity for every $0<k<n$ implies bicontinuity. We are going to prove first that for any $H\in\Gr{k}{n}$ the restriction of this function in $\mathcal{U}_H$ is continuous. For this proof the following commutative diagram comes in handy:
\begin{center}
\begin{tikzcd}
\left(\mathbb{R}^n\right)^k\ar[r,"\supseteq",phantom]&[-2em]q_0^{-1}\left(\mathcal{U}_H\right)\ar[d,"\left.q_0\right|"]\ar[r,"\mathfrak{gs}'"]\ar[rrd,dotted,"\left.\tilde\perp\right|"]&\StO{n}{n}\ar[r,"\pi_{[k+1,n]}"]&[5em]\StO{n-k}{n}\ar[d,"q_0"]\\[2em]
&\mathcal{U}_H\ar[rr,dotted,"\left.\perp\right|"]&&\Gr{n-k}{n}\\[-2em]
&K\ar[rr,mapsto]&&K^{\perp}
\end{tikzcd}
\end{center}
In this diagram, $\mathfrak{gs}'$ is the function that takes an orthonormal $k$-frame $(a_1,\ldots,a_k)$ to the orthonormal $n$-frame constructed after applying the Gram-Schmidt process to the $n$-basis $\left(a_1,\ldots,a_k,\bar{u}_1,\ldots,\bar{u}_{n-k}\right)$ where $\left(\bar{u}_j\right)$ is an orthonormal basis of $H^{\perp}$, just like in Lemma~\ref{lem:gr_manifold}. The next map $\pi_{[k+1,n]}$ is just the orthogonal projection in the last $n-k$ coordinates. Both of these maps are continuous and well defined, and thus we get a continuous map $\left.\tilde{\perp}\right|$, like in the diagram. This map depends only on the plane spanned by the input, and thus the universal property of the quotient spaces ensures the continuity of the map $\left.\perp\right|$.

After establishing the continuity of $\left.\perp\right|_{\mathcal{U}_H}$ for every $H$, notice that
$\Gr{k}{n} = \bigcup_{H\in\left\{\left<B\right>:B\in\binom{\{e_1,\ldots,e_n\}}{k}\right\}}\mathcal{U}_H$.
The union is over all $k$-planes spanned by any $k$ vectors among $\{e_1,\ldots,e_n\}$, where $e_1,\ldots,e_n$ is the standard basis of $\mathbb{R}^n$. This means that $\#\binom{\{e_1,\ldots,e_n\}}{k}=\binom{n}{k}<\infty$ sets are participating in the union and thus one can use the pasting lemma, proving that $\perp$ is continuous as a function from the whole space $\Gr{k}{n}$ to $\Gr{n-k}{n}$.
\end{proof}

It would be helpful at this point to mention what are the ``small'' examples of Grassmannians. It is known that $\Gr{1}{n}\cong\mathbb{P}^{n-1}$ and due to Lemma~\ref{lem:gr_duality} $\Gr{n-1}{n}\cong\mathbb{P}^{n-1}$ also holds. This already covers the cases $n=2,3$:
\[\Gr{1}{2}\cong\mathbb{P}^1\qquad\Gr{1}{3}\cong\Gr{2}{3}\cong\mathbb{P}^2\]
Hence, $\Gr{2}{4}$ is considered as the smallest non-trivial case further in this thesis.

\section{They are also CW-Complexes}
Now that the fact that Grassmannians are compact topological manifolds is proven we procceed in showing in this section that they are in fact finite CW complexes.
For this, we define a cell decomposition of each Grassmannian. Before laying out the formal definition, it is useful to also mention the analogous cell decomposition of the projective space $\mathbb{P}^{n-1}\cong\Gr{1}{n}$, consisting of the following $n$ cells:
\[\left\{l\subseteq\mathbb{R}^1\right\}\cong\mathbb{R}^0\ ,\ \left\{l\subseteq\mathbb{R}^2\setminus\mathbb{R}^1\right\}\cong\mathbb{R}^1\ ,\ \ldots\ ,\ \left\{l\subseteq\mathbb{R}^n\setminus\mathbb{R}^{n-1}\right\}\cong\mathbb{R}^{n-1}\]
This cell decomposition seems natural, but it depends heavily on our basis choice for $\mathbb{R}^n$. This however doesn't cause a problem, since for a different choice we get essentially the same decomposition, in terms of the homology classes we are going to eventually compute. This freedom of choice is going to play an important role towards the end of this chapter, when different decompositions are used (i.e.\ depending on different bases) in order to understand the multiplicative structure of the cohomology ring of the Grassmannians. The definition of flags in an $n$-dimensional vector space follows.

\begin{definition} Let $V$ be an $n$-dimensional vector space over a field $k$. A \ul{flag} $\mathbb{F}_{\bullet}$ for $V$ is a sequence $\left(\mathbb{F}_0,\mathbb{F}_1,\mathbb{F}_2,\ldots,\mathbb{F}_n\right)$, such that $\dim_k\mathbb{F}_i = i$ for all $i\in\{0,1,\ldots,n\}$ and
$0=\mathbb{F}_0\subset\mathbb{F}_1\subset\mathbb{F}_2\subset\cdots\subset\mathbb{F}_n=V$.
Given a flag $\mathbb{F}_{\bullet}$ of $V$, denote an orthonormal basis $f_{\bullet}=(f_1,\ldots,f_n)$ of $V$ to be \ul{compatible} with $\mathbb{F}_{\bullet}$, if
$f_i\in\mathbb{F}_i$
for every $i\in[n]$.
\end{definition}
%giati exei toso megalo keno sto pdf edw? (genika s ayth th selida)
For the rest of this thesis ${[n]}_0$ is used to denote the set $[n]\cup\{0\}=\{0,1,\ldots,n\}$.

An obvious example of flag is the one used above, namely the flag with $\mathbb{F}_i=\mathbb{R}_i=\left<e_1,\ldots,e_i\right>$. This is sometimes referred to as \ul{standard flag}, but since we avoid to fix some basis of $\mathbb{R}^n$ in this section, every flag is treated equally.

Noten that given a flag, one can always find a compatible basis with this flag. In fact, there always exist $2^n$ different compatible orthonormal bases and fixing one is like fixing an ``orientation'' of the flag.

Notice, that the role of the flags on the example above is to distinguish between all the different ways a line can intersect this flag. This is exactly the role a flag plays in the general definition of Schubert cells.

\begin{definition} Let $k,n\in\mathbb{N}$ with $0<k<n$. Moreover let $\mathbb{F}_{\bullet}$ be a flag of $\mathbb{R}^n$. Then, for each $k$-element subset $\mathbf{j}=\left\{j_1<j_2<\cdots<j_k\right\}$ of $[n]$ the \ul{Schubert cell} $\mathcal{C}_{\mathbf{j}}\left(\mathbb{F}_{\bullet}\right)$ is defined to be the following subset of $\Gr{k}{n}$:
\[\mathcal{C}_{\mathbf{j}}\left(\mathbb{F}_{\bullet}\right):=\big\{H\in\Gr{k}{n}:\ \dim\left(H\cap\mathbb{F}_i\right)=\max\{\ell\in{[k]}_0:j_{\ell}\leq i\}\ \ \forall i\in{[n]}_0\big\}\]
where $j_0$ is defined to be $0$.
\end{definition}

Before examining mathematically this definition, let us write down the Schubert cells of the first non-trivial example, $\Gr{2}{4}$, with respect to the standard flag of $\mathbb{R}^4$:
\[\begin{array}{rcl}
\mathcal{C}_{\{1,2\}}&=&\big\{H:\ \dim(H\cap\mathbb{R}^{0})=0,\ \dim(H\cap\mathbb{R}^{1})=1,\ \dim(H\cap\mathbb{R}^{2,3,4})=2\big\}\\
&=&\big\{\mathbb{R}^2\big\}\\[.6em]
\mathcal{C}_{\{1,3\}}&=&\big\{H:\ \dim(H\cap\mathbb{R}^{0})=0,\ \dim(H\cap\mathbb{R}^{1,2})=1,\ \dim(H\cap\mathbb{R}^{3,4})=2\big\}\\
&=&\big\{H:\ \mathbb{R}^1\subseteq H\subseteq\mathbb{R}^3,\ H\neq\mathbb{R}^2\big\}\\[.6em]
\mathcal{C}_{\{1,4\}}&=&\big\{H:\ \dim(H\cap\mathbb{R}^{0})=0,\ \dim(H\cap\mathbb{R}^{1,2,3})=1,\ \dim(H\cap\mathbb{R}^{4})=2\big\}\\
&=&\big\{H:\ \mathbb{R}^1\subseteq H,\ H\not\subseteq\mathbb{R}^3\big\}\\[.6em]
\mathcal{C}_{\{2,3\}}&=&\big\{H:\ \dim(H\cap\mathbb{R}^{0,1})=0,\ \dim(H\cap\mathbb{R}^{2})=1,\ \dim(H\cap\mathbb{R}^{3,4})=2\big\}\\
&=&\big\{H:\ H\subseteq\mathbb{R}^3,\ \mathbb{R}^1\not\subseteq H\big\}\\[.6em]
\mathcal{C}_{\{2,4\}}&=&\big\{H:\ \dim(H\cap\mathbb{R}^{0,1})=0,\ \dim(H\cap\mathbb{R}^{2,3})=1,\ \dim(H\cap\mathbb{R}^{4})=2\big\}\\
&=&\big\{H:\ \dim(H\cap\mathbb{R}^2)=1,\ \mathbb{R}^1\not\subseteq H,\ H\not\subseteq\mathbb{R}^3\big\}\\[.6em]
\mathcal{C}_{\{3,4\}}&=&\big\{H:\ \dim(H\cap\mathbb{R}^{0,1,2})=0,\ \dim(H\cap\mathbb{R}^{3})=1,\ \dim(H\cap\mathbb{R}^{4})=2\big\}\\
&=&\big\{H:\ H\cap\mathbb{R}^2=\{0\}\big\}\\[.6em]
\end{array}\]
Although there exists dimension restrictions in the definitions of the cells which can be omitted (for example that $H\cap\mathbb{R}^0=0$), the final form doesn't feel intuitive either. Let us take a step back for a moment and see what the Schubert cell decomposition of the (well-known) projective space is. Take for example $\Gr{1}{4}\cong\mathbb{P}^2$:
\[\begin{array}{rcl}
\mathcal{C}_{\{1\}}&=&\big\{l:\ \dim(l\cap\mathbb{R}^{0})=0,\ \dim(l\cap\mathbb{R}^{1,2,3,4})=1\big\}\\
&=&\big\{\mathbb{R}^1\big\}\\[.6em]
\mathcal{C}_{\{2\}}&=&\big\{l:\ \dim(l\cap\mathbb{R}^{0,1})=0,\ \dim(l\cap\mathbb{R}^{2,3,4})=1\big\}\\
&=&\big\{l:\ l\subseteq\mathbb{R}^2,\ l\neq\mathbb{R}^1\big\}\\[.6em]
\mathcal{C}_{\{3\}}&=&\big\{l:\ \dim(l\cap\mathbb{R}^{0,1,2})=0,\ \dim(l\cap\mathbb{R}^{3,4})=1\big\}\\
&=&\big\{l:\ l\subseteq\mathbb{R}^3,\ l\not\subseteq\mathbb{R}^2\big\}\\[.6em]
\mathcal{C}_{\{4\}}&=&\big\{l:\ \dim(l\cap\mathbb{R}^{0,1,2,3})=0,\ \dim(l\cap\mathbb{R}^{4})=1\big\}\\
&=&\big\{l:\ l\not\subseteq\mathbb{R}^3\big\}\\[.6em]
\end{array}\]
Notice that the general Schubert cell (with respect to some flag $\mathbb{F}_{\bullet}$) is the set of all lines contained in $\mathbb{F}_k\setminus\mathbb{F}_{k-1}$.
This gives a serial way to think of the cells of a particular projective space, which is the result of the total order that the set $\binom{[n]}{1}$ naturally has. Since $\binom{[n]}{k}$ is in general naturally a poset (inheriting the coordinate-wise ordering on the set of $k$ element sequences ${[n]}^k$), it is now of no surprise that the same poset structure lies behind the Schubert decomposition.
When examining the closure of these cells, we investigate more precisely this structure.

This decomposition of Grassmannians is meaningful since it is proven that it eventually makes every $\Gr{k}{n}$ into a CW complex. In order to do that we first prove that
${\left\{\mathcal{C}_{\mathbf{j}}(\mathbb{F}_{\bullet})\right\}}_{\mathbf{j}\in\binom{[n]}{k}}$ is indeed a decomposition. The following proof makes sense, if one conceptualizes a $k$-subset of $[n]$, as the $k$ ``jump points'' of the dimension of the intersections $H\cap\mathbb{F}_i$, for the various $i$.

\begin{lemma}\label{lem:jump_pts} For any integers $0<k<n$ and for every flag $\mathbb{F}_{\bullet}$ for $\mathbb{R}^n$, the set of all Schubert cells ${\left\{\mathcal{C}_{\mathbf{j}}(\mathbb{F}_{\bullet})\right\}}_{\mathbf{j}\in\binom{[n]}{k}}$ is a partition of the Grassmannian $\Gr{k}{n}$.
\end{lemma}
\begin{proof} It is rather obvious that two cells are disjoint, since each set of $k$ elements in $[n]$ describes uniquely $k$ jump points of the dimensions of $H\cap\mathbb{F}_0,H\cap\mathbb{F}_1,\ldots,H\cap\mathbb{F}_n$. Moreover, the dimensions of a $k$-plane $H$ in this sequence of intersections can increase at most by $1$ in each step. Indeed, because of the short exact sequence
\[0\to H\cap\mathbb{F}_{i-1}\to H\cap\mathbb{F}_i\to\faktor{H\cap\mathbb{F}_i}{H\cap\mathbb{F}_{i-1}}\to0\]
we get, for every $i\in[n]$,
$\dim_{\mathbb{R}}\left(H\cap\mathbb{F}_i\right)-\dim_{\mathbb{R}}\left(H\cap\mathbb{F}_{i-1}\right)=\dim_{\mathbb{R}}\faktor{H\cap\mathbb{F}_i}{H\cap\mathbb{F}_{i-1}}$.
Using the second isomorphism theorem for vector spaces, we get:
\[\begin{array}{>{\displaystyle}r>{\displaystyle}c>{\displaystyle}l}\faktor{H\cap\mathbb{F}_i}{H\cap\mathbb{F}_{i-1}}
&=&\faktor{H\cap\mathbb{F}_i}{H\cap\mathbb{F}_i\cap\mathbb{F}_{i-1}}\cong\faktor{(H\cap\mathbb{F}_i)+\mathbb{F}_{i-1}}{\mathbb{F}_{i-1}}\\[1.5em]
&\cong&\faktor{(H+\mathbb{F}_{i-1})\cap(\mathbb{F}_i+\mathbb{F}_{i-1})}{\mathbb{F}_{i-1}}\cong\faktor{(H+\mathbb{F}_{i-1})\cap\mathbb{F}_i}{\mathbb{F}_{i-1}}\\
\end{array}\]
with the last vector space being obviously a subspace of $\faktor{\mathbb{F}_i}{\mathbb{F}_{i-1}}$, which gives finally,
$\dim_{\mathbb{R}}\faktor{H\cap\mathbb{F}_i}{H\cap\mathbb{F}_{i-1}}\leq\dim_{\mathbb{R}}\faktor{\mathbb{F}_i}{\mathbb{F}_{i-1}}=1$.

This means that there exist exactly $k$ jump points in the sequence of the dimensions of $H\cap\mathbb{F}_0,\ldots,H\cap\mathbb{F}_n$, putting $H$ in some Schubert cell.

For every $H\in\Gr{k}{n}$ the sequence of its jump points $\mathbf{j}\in\binom{[n]}{k}$, i.e. the $\mathbf{j}$ for which $H\in\mathcal{C}_{\mathbf{j}}(\mathbb{F}_{\bullet})$ is given explicitly by
\[j_{\ell}=\min\{j\in[n]:\dim(H\cap\mathbb{F}_j)\geq\ell\}\qedhere\]
\end{proof}

We are now ready to prove that these cells are indeed useful building blocks, i.e.\ that they are in fact homeomorphic to open balls of various dimensions. Before writing down the formula for the dimension of a general Schubert cell, let us revisit the case of $\Gr{2}{4}$ and try to compute the dimension of the cells by hand-waving:

\begin{example} In $\Gr{2}{4}$ we have the Schubert decomposition we discussed earlier, w.r.t.\ the standard flag. Let us represent each plane $H$ in $\mathbb{R}^4$ by the unique corresponding $2\times 4$ matrix whose rows span $H$ and the matrix is in its reduced echelon form. Then, we get the following picture if we consider the form of the matrices for each cell. Remember, that the index $\mathbf{j}\subseteq\binom{[4]}{2}$ refers to the dimension jump points:
\begin{center}
\begin{tikzcd}
\mathcal{C}_{1,2}\ar[r,leftrightarrow]&\tworows{1}{0}{0}{0}{0}{1}{0}{0}&\mathcal{C}_{1,3}\ar[r,leftrightarrow]&\tworows{1}{0}{0}{0}{0}{*}{1}{0}\\
\mathcal{C}_{1,4}\ar[r,leftrightarrow]&\tworows{1}{0}{0}{0}{0}{*}{*}{1}&\mathcal{C}_{2,3}\ar[r,leftrightarrow]&\tworows{*}{1}{0}{0}{*}{0}{1}{0}\\
\mathcal{C}_{2,4}\ar[r,leftrightarrow]&\tworows{*}{1}{0}{0}{*}{0}{*}{1}&\mathcal{C}_{3,4}\ar[r,leftrightarrow]&\tworows{*}{*}{1}{0}{*}{*}{0}{1}\\
\end{tikzcd}
\end{center}
In fact, every matrix of a given form gives a unique plane in the according cell. Thus, we get a bijection and the dimension of the cells equals the number of the choices we have in each matrix, i.e.\ the number of stars. Notice how the jumps in the dimensions now correspond to pivot elements of the rows. Moreover, notice that the first row has always $j_1-1$ stars and the second $j_2-2$.
\end{example}

In order to compute the dimension in general, given a set $\mathbf{j}=\{j_1<\cdots<j_k\}\in\binom{[n]}{k}$, define the number
\[\mathrm{d}(\mathbf{j})=(j_1-1)+(j_2-2)+\cdots+(j_k-k)\]
Using the reduced echelon form for a matrix written in a suitable base of $\mathbb{R}^n$, depending on the flag $\mathbb{F}_{\bullet}$, one can easily argue that $\mathcal{C}_{\mathbf{j}}\left(\mathbb{F}_{\bullet}\right)$ is an open cell of dimension $\mathrm{d}(\mathbf{j})$, for any $\mathbf{j}\in\binom{[n]}{k}$, but our goal is to eventually prove that these open cells also give the Grassmannian a CW structure. For this it is required to find maps from the closed cells of the appropriate dimensions into the Grassmannian and unfortunately it is not easy to work with the closure of the matrices defined above. Thus, the approach in the bibliography may seem more artificial but it is also the one employed here:

The approach to prove Theorem~\ref{thm:cw_complex} is:
\begin{i_enum}
\item First to define appropriate sets $\tilde{\mathcal{C}}_{\mathbf{j}}(\mathbb{F}_{\bullet})$ living in the Stiefel manifold, each one ``above'' the matching Schubert cell (Definition~\ref{def:lift_of_Schubert_cells}).
\item Then to prove that each $\tilde{\mathcal{C}}_{\mathbf{j}}{\left(\mathbb{F}_{\bullet}\right)}^-$ is topologically a closed disk of the right dimension and to understand the interior and the boundary of this disc (Lemma~\ref{lem:shub_dim}).
\item Finally, to prove that $q_0$ maps the interior of $\tilde{\mathcal{C}}_{\mathbf{j}}(\mathbb{F}_{\bullet})^-$ homeomorphically onto $\mathcal{C}_{\mathbf{j}}(\mathbb{F}_{\bullet})$ and its boundary inside cells of lower dimension (Lemma~\ref{lem:cells_from_stiefel_to_gr}).
\end{i_enum}
Let us begin with the definition:

\begin{definition}\label{def:lift_of_Schubert_cells} Let $k,n\in\mathbb{N}$, with $0<k<n$. Moreover, let $\mathbb{F}_{\bullet}$ be a flag of $\mathbb{R}^n$ and $f_{\bullet}$ an orthonormal basis of $\mathbb{R}^n$, compatible with $\mathbb{F}_{\bullet}$. Then, for each $\mathbf{j}=\{j_1<j_2<\cdots<j_k\}\in\binom{[n]}{k}$ define $\tilde{\mathcal{C}}_{\mathbf{j}}(\mathbb{F}_{\bullet})=\tilde{\mathcal{C}}_{\mathbf{j}}(\mathbb{F}_{\bullet},f_{\bullet})$ to be the following subset of $\StO{k}{n}$:
\[\tilde{\mathcal{C}}_{\mathbf{j}}(\mathbb{F}_{\bullet},f_{\bullet}):=\left\{(v_1,v_2,\ldots,v_k)\in\StO{k}{n}:v_{\ell}\in\mathbb{H}(\mathbb{F}_{j_{\ell}},f_{j_{\ell}})\ \forall\ell\in[k]\right\}\]
where $\mathbb{H}(F,f)$ is defined to be the ``positive open halfspace'' of the vector space $F$, w.r.t.\ the orientation defined by $f\in F$:
\[\mathbb{H}(F,f):=\left\{v\in F:v^tf>0\right\}\]
for every $i\in[n]$.
\end{definition}

Although as we are going to prove the images of these subspaces of $\StO{k}{n}$ are the Schubert decomposition of $\Gr{k}{n}$, notice that for a fixed basis $f_{\bullet}$, they do not even cover $\StO{k}{n}$ and when regarding all possible compatible bases, they do cover the whole space, but we take most of these sets multiple times.

Let us prove now that these sets have the right dimension and that their closure (now much easier to handle than the matrices in echelon form) is homeomorphic to a closed cell of this dimension. In order to do so in Lemma~\ref{lem:shub_dim}, we first need the following trivial (but lengthy) assertion:

\begin{lemma}\label{lem:closure_of_cells} For any integers $0<k<n$, for any flag $\mathbb{F}_{\bullet}$ of $\mathbb{R}^n$, for any orthonormal basis $f_{\bullet}$ of $\mathbb{R}^n$, compatible with $\mathbb{F}_{\bullet}$ and for any set $\mathbf{j}\in\binom{[n]}{k}$ the closure of $\tilde{\mathcal{C}}_{\mathbf{j}}{\left(\mathbb{F}_{\bullet},f_{\bullet}\right)}$ inside ${\left(\mathbb{R}^n\right)}^k$ is:
\[{\tilde{\mathcal{C}}_{\mathbf{j}}{\left(\mathbb{F}_{\bullet},f_{\bullet}\right)}}^-=\left\{(v_1,v_2,\ldots,v_k)\in\StO{k}{n}:v_{\ell}\in{\mathbb{H}(\mathbb{F}_{j_{\ell}},f_{j_{\ell}})}^-\ \forall\ell\in[k]\right\}\]
where
${\mathbb{H}(F,f)}^-:=\left\{v\in F:v^tf\geq 0\right\}$
\end{lemma}
\begin{proof} Using simple point-set topology and the fact that $\StO{k}{n}$ is closed inside ${\left(\mathbb{R}^n\right)}^k$ as proven in Proposition~\ref{prop:StO_dim_closed}, we have:
\[\begin{array}{rcl}\tilde{\mathcal{C}}_{\mathbf{j}}{\left(\mathbb{F}_{\bullet},f_{\bullet}\right)}^-
&=&{\left\{(v_1,v_2,\ldots,v_k)\in\StO{k}{n}:v_{\ell}\in\mathbb{H}(\mathbb{F}_{j_{\ell}},f_{j_{\ell}})\ \forall\ell\in[k]\right\}}^-\\
&=&{\big(\StO{k}{n}\cap\mathbb{H}(\mathbb{F}_{j_1},f_{j_1})\times\cdots\times\mathbb{H}(\mathbb{F}_{j_k},f_{j_k})\big)}^-\\
&\subseteq&\StO{k}{n}\cap{\big(\mathbb{H}(\mathbb{F}_{j_1},f_{j_1})\times\cdots\times\mathbb{H}(\mathbb{F}_{j_k},f_{j_k})\big)}^-\\
&=&\StO{k}{n}\cap\mathbb{H}{\left(\mathbb{F}_{j_1},f_{j_1}\right)}^-\times\cdots\times\mathbb{H}{\left(\mathbb{F}_{j_k},f_{j_k}\right)}^-\\
&=&\left\{(v_1,v_2,\ldots,v_k)\in\StO{k}{n}:v_{\ell}\in{\mathbb{H}(\mathbb{F}_{j_{\ell}},f_{j_{\ell}})}^-\ \forall\ell\in[k]\right\}\\
\end{array}\]
The inclusion in the third line is actually an equality. Indeed, let $(v_1,\ldots,v_k)\in\StO{k}{n}\cap{\big(\mathbb{H}(\mathbb{F}_{j_1},f_{j_1})\times\cdots\times\mathbb{H}(\mathbb{F}_{j_k},f_{j_k})\big)}^-$. This means that there exists a sequence ${(v_1^m,\ldots,v_k^m)}_m\in\mathbb{H}(\mathbb{F}_{j_1},f_{j_1})\times\cdots\times\mathbb{H}(\mathbb{F}_{j_k},f_{j_k})$ converging to $(v_1,\ldots,v_k)\in\StO{k}{n}\subseteq\St{k}{n}$, inside ${\left(\mathbb{R}^n\right)}^k$. Since $\St{k}{n}$ is open, as proven in Proposition~\ref{prop:St_open}, there exists some $m_0\in\mathbb{N}$, such that ${(v_1^m,\ldots,v_k^m)}_m\in\St{k}{n}$ for all $m\geq m_0$. For each such tuple, define now ${(w_1^m,\ldots,w_k^m)}_m\in\StO{k}{n}$ to be the result of the Gram-Schmidt process on input ${(v_1^m,\ldots,v_k^m)}_m$ for every $m\geq m_0$,
${(w_1^m,\ldots,w_k^m)}_m:=\mathfrak{gs}\big({(v_1^m,\ldots,v_k^m)}_m\big)$

First of all, $\mathfrak{gs}:\St{k}{n}\to\StO{k}{n}$ is a continuous map, which means that
\[{(w_1^m,\ldots,w_k^m)}_m=\mathfrak{gs}\big({(v_1^m,\ldots,v_k^m)}_m\big)\overset{m\to\infty}{\to}\mathfrak{gs}(v_1,\ldots,v_k)=(v_1,\ldots,v_k)\]
It now suffices to show that $\mathfrak{gs}(v_1',\ldots,v_k')\in\mathbb{H}(\mathbb{F}_{j_1},f_{j_1})\times\cdots\times\mathbb{H}(\mathbb{F}_{j_k},f_{j_k})$ for any $(v_1',\ldots,v_k')\in\mathbb{H}(\mathbb{F}_{j_1},f_{j_1})\times\cdots\times\mathbb{H}(\mathbb{F}_{j_k},f_{j_k})$. We show this using the recursive nature of the Gram-Schmidt process. Assume that the assertion holds for every dimension between $1$ and $k-1$, let $(v_1',\ldots,v_k')\in\mathbb{H}(\mathbb{F}_{j_1},f_{j_1})\times\cdots\times\mathbb{H}(\mathbb{F}_{j_k},f_{j_k})$ and let also $(w_1',\ldots,w_{k-1}')=\mathfrak{gs}(v_1',\ldots,v_{k-1}')$. Then:
\[\mathfrak{gs}(v_1',\ldots,v_k')=\left(w_1',\ldots,w_{k-1}',\frac{1}{\left\|v_k'-\sum_{\ell=1}^{k-1}((w_{\ell}')^t v_k')w_{\ell}'\right\|}\left(v_k'-\sum_{\ell=1}^{k-1}((w_{\ell}')^tv_k')w_{\ell}'\right)\right)\]
Because of the inductive hypothesis, we only consider the last vector, which we denote as $w_k'$. Because $\mathbb{F}_{\bullet}$ is a flag, we have that $\mathbb{F}_{j_1}\subseteq\mathbb{F}_{j_2}\subseteq\cdots\subseteq\mathbb{F}_{j_k}$ and thus: $w_1',w_2',\ldots,w_{k-1}'\in\mathbb{F}_{j_k}$. Since $v_k'\in\mathbb{F}_{j_k}$ as well,
$w_k'=\frac{1}{\left\|v_k'-\sum_{\ell=1}^{k-1}((w_{\ell}')^tv_k')w_{\ell}'\right\|}\left(v_k'-\sum_{\ell=1}^{k-1}((w_{\ell}')^t v_k')w_{\ell}'\right)\in\mathbb{F}_{j_k}$

Moreover, it is true that $v^tf_i=0$, if $v\in\mathbb{F}_{i-1}$ for some $i\in[n]$. Indeed, $f_{\bullet}$ is compatible with $\mathbb{F}_{\bullet}$, which means that $v\perp f_i$ for every $v\in\mathbb{F}_{i-1}$. Since $w_1',\ldots,w_{k-1}'\in\mathbb{F}_{j_{k-1}}$,
\[(w_k')^tf_{j_k}=\frac{1}{\left\|v_k'-\sum_{\ell=1}^{k-1}((w_{\ell}')^tv_k')w_{\ell}'\right\|}(v_k')^tf_{j_k}>0\]
This proves that $w_k'\in\mathbb{H}(\mathbb{F}_{j_k},f_{j_k})$, which due to the induction assumption leads to
$\mathfrak{gs}(v_1',\ldots,v_k')\in\mathbb{H}(\mathbb{F}_{j_1},f_{j_1})\times\cdots\times\mathbb{H}(\mathbb{F}_{j_k},f_{j_k})$.
Thus the sequence ${(w_1^m,\ldots,w_k^m)}_m$ is inside $\StO{k}{n}\cap\mathbb{H}(\mathbb{F}_{j_1},f_{j_1})\times\cdots\times\mathbb{H}(\mathbb{F}_{j_k},f_{j_k})$ converging to $(v_1,\ldots,v_k)$, which proves the desired inclusion since
\[(v_1,\ldots,v_k)\in{\big(\StO{k}{n}\cap\mathbb{H}(\mathbb{F}_{j_1},f_{j_1})\times\cdots\times\mathbb{H}(\mathbb{F}_{j_k},f_{j_k})\big)}^-\qedhere\]
\end{proof}

The approach used to prove Lemma~\ref{lem:shub_dim} is the one of Hatcher~\cite{vec_bundles} (p.37), in order to already familiarize ourselves with the notion of a trivial fiber bundle. For this the following lemma is used.
\begin{lemma}\label{lem:trivial_fb} For any integers $0<k<n$, for any flag $\mathbb{F}_{\bullet}$ of $\mathbb{R}^n$, for any orthonormal basis $f_{\bullet}$ of $\mathbb{R}^n$, compatible with $\mathbb{F}_{\bullet}$ and for any set $\mathbf{j}\in\binom{[n]}{k}$, the projection to the first coordinate
\[\pi_1:\tilde{\mathcal{C}}_{\mathbf{j}}{\left(\mathbb{F}_{\bullet},f_{\bullet}\right)}^-\to S^{n-1}\cap\mathbb{H}{\left(\mathbb{F}_{j_1},f_{j_1}\right)}^-\]
is a trivial fiber bundle.
\end{lemma}
\begin{proof} Due to Lemma~\ref{lem:closure_of_cells} the map $\pi_1$ is well defined, since it takes $(v_1,\ldots,v_k)$ to $v_1$, which is a unit vector and also an element of $\mathbb{H}{\left(\mathbb{F}_{j_1},f_{j_1}\right)}^-$. Notice that since $f_{j_1}\in\mathbb{H}{\left(\mathbb{F}_{j_1},f_{j_1}\right)}^-$, it suffices to prove that there exists a homeomorphism $\phi$ making the following diagram commute:
\begin{equation}\label{eq:phi_for_trivial}
\begin{tikzcd}
\tilde{\mathcal{C}}_{\mathbf{j}}{\left(\mathbb{F}_{\bullet},f_{\bullet}\right)}^-\ar[r,"\phi","\cong"']\ar[d,"\pi_1"']&\big(S^{n-1}\cap\mathbb{H}{\left(\mathbb{F}_{j_1},f_{j_1}\right)}^-\big)\times\pi_1^{-1}(f_{j_1})\ar[dl,"\pi_1"]\\
S^{n-1}\cap\mathbb{H}{\left(\mathbb{F}_{j_1},f_{j_1}\right)}^-
\end{tikzcd}
\end{equation}
First, we define a transformation $\rho_{v}\in SO(n)$ for every $v\in S^{n-1}\cap\mathbb{H}{\left(\mathbb{F}_{j_1},f_{j_1}\right)}^-$, which takes $v$ to $f_{j_1}$ and is the identity in the orthogonal complement of the plane spanned by $v$ and $f_{j_1}$. Next we define $\phi$ and argue that it is continuous. We then use $\rho_{v}$ to map homeomorphically $\pi_1^{-1}(v)$ to $\pi_1^{-1}(f_{j_1})$ for each $v$. Finally we glue all of $\rho_v$ to a homeomorphism $\phi$.

As proven in Computation~\ref{com:rotation}, for any $v\in S^{n-1}\cap\mathbb{H}{\left(\mathbb{F}_{j_1},f_{j_1}\right)}^-\setminus\{f_{j_1}\}$, the rotation on the plane $\left<v,f_{j_1}\right>$ in $\mathbb{R}^n$ which takes $v$ to $f_{j_1}$ is given by:
\[\rho_v:=\rho_{v\to f_{j_1}}=I_n+2f_{j_1}v^t-\frac{(v+f_{j_1})(v+f_{j_1})^t}{v^tf_{j_1}+1}\in SO(n)\]
Notice that this depends continuously on $v$. Moreover, we define $\rho_{f_{j_1}}=I_n$, and hence $\lim_{v\to f_{j_1}}\rho_v=\rho_{f_{j_1}}$. These mean that the following map is continuous
\begin{center}
\begin{tikzcd}
S^{n-1}\cap\mathbb{H}{\left(\mathbb{F}_{j_1},f_{j_1}\right)}^-\ar[r,"\rho"]&[2em]SO(n)\\[-1.5em]
v\ar[r,mapsto]&\rho_v
\end{tikzcd}
\end{center}
Since $SO(n)$ acts continuously on $\mathbb{R}^n$, the following map is also continuous:
\begin{center}
\begin{tikzcd}
SO(n)\times \tilde{\mathcal{C}}_{\mathbf{j}}{\left(\mathbb{F}_{\bullet},f_{\bullet}\right)}^-\ar[r,hook]&SO(n)\times(\mathbb{R}^n)^k\ar[r,"\mathrm{ev}^k"]&(\mathbb{R}^n)^k\\[-1.5em]
(A,u_1,\ldots,u_k)\ar[r,mapsto]&(A,u_1,\ldots,u_k)\ar[r,mapsto]&(Au_1,\ldots,Au_k)
\end{tikzcd}
\end{center}
So, we define the following continuous map:
\begin{center}
\begin{tikzcd}
\tilde{\mathcal{C}}_{\mathbf{j}}{\left(\mathbb{F}_{\bullet},f_{\bullet}\right)}^-\ar[r, "\left<\rho\circ\pi_1{,}id\right>"]&SO(n)\times \tilde{\mathcal{C}}_{\mathbf{j}}{\left(\mathbb{F}_{\bullet},f_{\bullet}\right)}^-\ar[r,"\mathrm{ev}^k"]&(\mathbb{R}^n)^k\\[-1.5em]
(v_1,\ldots,v_k)\ar[r,mapsto]&(\rho_{v_1},v_1,\ldots,v_k)\ar[r,mapsto]&(\rho_{v_1}v_1,\ldots,\rho_{v_1}v_k)
\end{tikzcd}
\end{center}
We now prove that the image of this map is contained in $\pi_1^{-1}(f_{j_1})$. Let $(v_1,\ldots,v_k)\in\tilde{\mathcal{C}}_{\mathbf{j}}{\left(\mathbb{F}_{\bullet},f_{\bullet}\right)}^-$. Since $\rho_{v_1}\in SO(n)$ and $(v_1,\ldots,v_k)\in\StO{k}{n}$, we have that $(\rho_{v_1}v_i)^t\rho_{v_1}v_j=v_i^tv_j=\delta_{i,j}$, which means that $(\rho_{v_1}v_1,\ldots,\rho_{v_1}v_k)\in\StO{k}{n}$. Moreover, for any $\ell\in[k]$, $v_{\ell}\in\mathbb{F}_{j_{\ell}}$. Let us define $H:=\left<v_1,f_{j_1}\right>\subseteq\mathbb{F}_{j_1}\subseteq\mathbb{F}_{j_{\ell}}$ and then express it as the following decomposition $v_{\ell}=P_Hv_{\ell}+P_{H^{\perp}}v_{\ell}$, where $P_H$ and $P_{H^{\perp}}$ are the linear projections on $H$ and $H^{\perp}$ respectively. Since $v_{\ell}\in\mathbb{F}_{j_{\ell}}$ and $H\subseteq\mathbb{F}_{j_{\ell}}$, we have that $P_{H^{\perp}}v_{\ell}\in H^{\perp}\cap\mathbb{F}_{j_{\ell}}$. Moreover, as proved in Computation~\ref{com:rotation}, $\rho_{v_1}P_Hv_{\ell}\in H$ and $\rho_{v_1}P_{H^{\perp}}v_{\ell}=P_{H^{\perp}}v_{\ell}$. By putting all these together we get:
\[\rho_{v_1}v_{\ell}=\rho_{v_1}P_Hv_{\ell}+\rho_{v_1}P_{H^{\perp}}v_{\ell}=\rho_{v_1}P_Hv_{\ell}+P_{H^{\perp}}v_{\ell}\in H+\mathbb{F}_{j_{\ell}}=\mathbb{F}_{j_{\ell}}\]
Next, notice that in general $\rho_{x\to y}^t=\rho_{x\to y}^{-1}=\rho_{y\to x}$, so $\rho_{v_1}^ta=a$ for every $a\in H^{\perp}$. Thus,
$(\rho_{v_1}v_{\ell})^tf_{j_{\ell}}=v_{\ell}^t\rho_{v_1}^tf_{j_{\ell}}=v_{\ell}^tf_{j_{\ell}}\geq0$,
since $v_{\ell}\in\mathbb{H}{\left(\mathbb{F}_{j_{\ell}},f_{j_{\ell}}\right)}^-$. This proves that $(\rho_{v_1}v_1,\ldots,\rho_{v_1}v_k)\in\tilde{\mathcal{C}}_{\mathbf{j}}{\left(\mathbb{F}_{\bullet},f_{\bullet}\right)}^-$. Also, since $\rho_{v_1}v_1=f_{j_1}$, it is also holds that $\pi_1(\rho_{v_1}v_1,\ldots,\rho_{v_1}v_k)=f_{j_1}$, which proves the claim.

Let us now define $\phi=\left<\pi_1,\mathrm{ev}^k\circ\left<\rho\circ\pi_1,id\right>\right>$, i.e.:
\begin{equation}\label{eq:trivial_fb}
\begin{tikzcd}
\tilde{\mathcal{C}}_{\mathbf{j}}{\left(\mathbb{F}_{\bullet},f_{\bullet}\right)}^-\ar[r, "\left<\pi_1{,}\mathrm{ev}^k\circ\left<\rho\circ\pi_1{,}id\right>\right>"]&[4em]\big(S^{n-1}\cap\mathbb{H}{\left(\mathbb{F}_{j_1},f_{j_1}\right)}^-\big)\times\pi_1^{-1}(f_{j_1})\\[-1.5em]
(v_1,\ldots,v_k)\ar[r,mapsto]&\big(v_1,(\rho_{v_1}v_1,\ldots,\rho_{v_1}v_k)\big)
\end{tikzcd}
\end{equation}
This is well defined and continuous. We now prove that its restriction on each fiber $\pi_1^{-1}(v)$ is a homeomorphism, so we show that the following map is a homeomorphism for each $v\in S^{n-1}\cap\mathbb{H}{\left(\mathbb{F}_{j_1},f_{j_1}\right)}^-$:
\begin{center}
\begin{tikzcd}
\pi_1^{-1}(v)\ar[r, "\mathrm{ev}^k\circ\left<\rho\circ\pi_1{,}id\right>"]&[3em]\pi_1^{-1}(f_{j_1})\\[-1.5em]
(v,v_2,\ldots,v_k)\ar[r,mapsto]&(f_{j_1},\rho_vv_2,\ldots,\rho_vv_k)
\end{tikzcd}
\end{center}
We easily see that the above arguments let us also construct a continuous map
\begin{center}
\begin{tikzcd}
\pi_1^{-1}(f_{j_1})\ar[r, "\mathrm{ev}^k\circ\left<\rho'\circ\pi_1{,}id\right>"]&[3em]\pi_1^{-1}(v)\\[-1.5em]
(f_{j_1},v_2,\ldots,v_k)\ar[r,mapsto]&(v,\rho_v^tv_2,\ldots,\rho_v^tv_k)
\end{tikzcd}
\end{center}
and notice that these two are inverses of each other. Thus $\phi|_{\pi^{-1}(v)}$ is a homeomorphism for each $v$.

Since $\tilde{\mathcal{C}}_{\mathbf{j}}{\left(\mathbb{F}_{\bullet},f_{\bullet}\right)}^-$ is a closed subset of the compact $\StO{k}{n}$, it is also compact. The space $\big(S^{n-1}\cap\mathbb{H}{\left(\mathbb{F}_{j_1},f_{j_1}\right)}^-\big)\times\pi_1^{-1}(f_{j_1})$ is obviously Hausdorff, and $\phi$ is also a homeomorphism, i.e. $\pi_1$ is a trivial fiber bundle due to Proposition~\ref{prop:extend_fb}.
\end{proof}
\begin{remark}\label{rem:trivial_fb_interior} Following the same arguments as in Lemma~\ref{lem:trivial_fb}, we prove that the following restriction of $\pi_1$: \[\pi_1|_{\tilde{\mathcal{C}}_{\mathbf{j}}(\mathbb{F}_{\bullet},f_{\bullet})}:\tilde{\mathcal{C}}_{\mathbf{j}}(\mathbb{F}_{\bullet},f_{\bullet})\to S^{n-1}\cap\mathbb{H}(\mathbb{F}_{j_1},f_{j_1})\]
is also a trivial fiber bundle and the homeomorphism to the respective product is given by the restriction of the homeomorphism $\phi$ in the commutative diagram \eqref{eq:phi_for_trivial}:
\begin{center}
\begin{tikzcd}
\tilde{\mathcal{C}}_{\mathbf{j}}{\left(\mathbb{F}_{\bullet},f_{\bullet}\right)}\ar[r,"\phi|_{\tilde{\mathcal{C}}_{\mathbf{j}}(\mathbb{F}_{\bullet},f_{\bullet})}","\cong"']\ar[d,"\pi_1|_{\tilde{\mathcal{C}}_{\mathbf{j}}(\mathbb{F}_{\bullet},f_{\bullet})}"']&\big(S^{n-1}\cap\mathbb{H}{\left(\mathbb{F}_{j_1},f_{j_1}\right)}\big)\times\pi_1|_{\tilde{\mathcal{C}}_{\mathbf{j}}(\mathbb{F}_{\bullet},f_{\bullet})}^{-1}(f_{j_1})\ar[dl,"\pi_1"]\\
S^{n-1}\cap\mathbb{H}{\left(\mathbb{F}_{j_1},f_{j_1}\right)}
\end{tikzcd}
\end{center}
\end{remark}

\begin{computation}\label{com:rotation} Let $x,y\in S^{n-1}$ be any two unit vectors in $\mathbb{R}^n$ such that $x\neq y$ and $x\neq -y$. Let $\rho_{x\to y}$ be the rotation taking $x$ to $y$ and fixing the $n-2$ dimension subspace $\left<x,y\right>^{\perp}$. Then:
\[\rho_{x\to y}=I_n+2yx^t-\frac{(x+y)(x+y)^t}{x^ty+1}\in SO(n)\]
\end{computation}
\begin{proof} Let $\theta\in(0,\pi)$ be the angle between $x$ and $y$ and $H=\left<x,y\right>$ the plane spanned by $x,y$. It is easy to compute that the following is an orthonormal basis of $H$:
\[u=x,\qquad v=\frac{1}{\sin\theta}(y-\cos\theta x)\]
Let $P_H$ be the linear projection on $H$. Then for any $a\in\mathbb{R}^n$,
\begin{align*}
P_Ha&=(u^ta)u+(v^ta)v\\
&=(x^ta)x+\frac{1}{\sin^2\theta}\big((y^ta)-(x^ta)\cos\theta)\big)\big(y-\cos\theta x\big)\\
&=\Big((x^ta)-\frac{\cos\theta}{\sin^2\theta}(y^ta)+\frac{\cos^2\theta}{\sin^2\theta}(x^ta)\Big)x+\Big(\frac{1}{\sin^2\theta}(y^ta)-\frac{\cos\theta}{\sin^2\theta}(x^ta)\Big)y\\
&=\frac{(x^ta)-(y^ta)\cos\theta}{\sin^2\theta}x+\frac{(y^ta)-(x^ta)\cos\theta}{\sin^2\theta}y
\end{align*}
For the rotation $\rho_{x\to y}$ we have,
\[\rho_{x\to y}(a-P_Ha)=a-P_Ha,\qquad\rho_{x\to y}x=y,\qquad\rho_{x\to y}y=z\]
where we compute that $z=2\cos\theta y-x$, which give the desired formula,
\begingroup
\allowdisplaybreaks
\begin{align*}
\rho_{x\to y}a&=\rho_{x\to y}(a-P_Ha)+\rho_{x\to y}(P_Ha)\\
&=a-P_Ha+\frac{(x^ta)-(y^ta)\cos\theta}{\sin^2\theta}y+\frac{(y^ta)-(x^ta)\cos\theta}{\sin^2\theta}(2\cos\theta y-x)\\
&=a\begin{aligned}[t]&+\frac{(y^ta)\cos\theta-(x^ta)}{\sin^2\theta}x+\frac{(x^ta)\cos\theta-(y^ta)}{\sin^2\theta}y\\
&+\frac{(x^ta)-(y^ta)\cos\theta}{\sin^2\theta}y+\frac{(y^ta)-(x^ta)\cos\theta}{\sin^2\theta}(2\cos\theta y-x)\end{aligned}\\
&=a\begin{aligned}[t]&+\frac{(\cos\theta-1)(x^ta)+(\cos\theta-1)(y^ta)}{\sin^2\theta}x\\
&+\frac{(-2\cos^2\theta+\cos\theta+1)(x^ta)+(\cos\theta-1)(y^ta)}{\sin^2\theta}y\end{aligned}\\
&=a-\frac{1}{\cos\theta+1}\left((x^ta)x+(y^ta)x-\frac{2\cos^2\theta-\cos\theta-1}{\cos\theta-1}(x^ta)y+(y^ta)y\right)\\
&=a-\frac{1}{\cos\theta+1}\big(xx^t+xy^t-(2\cos\theta+1)yx^t+yy^t\big)a\\
&=\left(I_n+2yx^t-\frac{(x+y)(x+y)^t}{\cos\theta+1}\right)a
\end{align*}
\endgroup
At this point it is easy to do a sanity check and verify that indeed
\[\rho_{x\to y}x=y,\qquad\rho_{x\to y}y=2\cos\theta y-x,\qquad\forall a\perp H\ \ \rho_{x\to y}a=a\]
Moreover, we also verify that
\begin{align*}
\rho_{x\to y}\rho_{x\to y}^t&=\left(I_n+2yx^t-\frac{(x+y)(x+y)^t}{\cos\theta+1}\right)\left(I_n+2xy^t-\frac{(x+y)(x+y)^t}{\cos\theta+1}\right)\\
&=\begin{aligned}[t]&I_n+2xy^t-\frac{(x+y)(x+y)^t}{\cos\theta+1}+2yx^t+4yy^t-2\frac{y(1+\cos\theta)(x+y)^t}{\cos\theta+1}\\
&-\frac{(x+y)(x+y)^t}{\cos\theta+1}-2\frac{(x+y)(1+\cos\theta)y^t}{\cos\theta+1}+\frac{(x+y)(2+2\cos\theta)(x+y)^t}{(\cos\theta+1)^2}\end{aligned}\\
&=I_n
\end{align*}
which proves that $\rho_{x\to y}\in O(n)$. To compute the determinant of $\rho_{x\to y}$, first let $A=I_n-\frac{(x+y)(x+y)^t}{\cos\theta+1}$ and use the Matrix determinant lemma to get,
$\det(A)=\det\left(I_n-\frac{(x+y)(x+y)^t}{\cos\theta+1}\right)=1-\frac{(x+y)^t(x+y)}{\cos\theta+1}=-1$
and hence $A$ is invertible. Using the Sherman–Morrison formula we compute $A^{-1}$:
\begin{align*}
A^{-1}&=\left(I_n-\frac{(x+y)(x+y)^t}{\cos\theta+1}\right)^{-1}\\
&=I_n+\frac{1}{1-\frac{\mathrm{tr}\left((x+y)(x+y)^t\right)}{\cos\theta+1}}\frac{(x+y)(x+y)^t}{\cos\theta+1}\\
&=I_n-\frac{(x+y)(x+y)^t}{\cos\theta+1}
\end{align*}
Lastly, using the Matrix determinant lemma again, we have
\begin{align*}
\det(\rho_{x\to y})&=\det\left(I_n-\frac{(x+y)(x+y)^t}{\cos\theta+1}+2yx^t\right)\\
&=\left(1+2x^tA^{-1}y\right)\det(A)\\
&=-\left(1+2x^t\left(I_n-\frac{(x+y)(x+y)^t}{\cos\theta+1}\right)y\right)\\
&=-1-2\cos\theta+2\frac{(1+\cos\theta)(\cos\theta+1)}{\cos\theta+1}=1
\end{align*}
which proves that $\rho_{x\to y}\in SO(n)$.
\end{proof}

\begin{lemma}\label{lem:shub_dim} For any integers $0<k<n$, for any flag $\mathbb{F}_{\bullet}$ of $\mathbb{R}^n$, for any orthonormal basis $f_{\bullet}$ of $\mathbb{R}^n$, compatible with $\mathbb{F}_{\bullet}$ and for any set $\mathbf{j}\in\binom{[n]}{k}$, there exists a homeomorphism
\[\tilde{\Phi}_{\mathbf{j}}:D^{\mathrm{d}(\mathbf{j})}\to\tilde{\mathcal{C}}_{\mathbf{j}}{\left(\mathbb{F}_{\bullet},f_{\bullet}\right)}^-\]
such that $\tilde{\Phi}_{\mathbf{j}}\left({\left(D^{\mathrm{d}(\mathbf{j})}\right)}^{\circ}\right)\subseteq\tilde{\mathcal{C}}_{\mathbf{j}}(\mathbb{F}_{\bullet},f_{\bullet})$ and
\[\tilde{\Phi}_{\mathbf{j}}|_{{\left(D^{\mathrm{d}(\mathbf{j})}\right)}^{\circ}}:{\left(D^{\mathrm{d}(\mathbf{j})}\right)}^{\circ}\to\tilde{\mathcal{C}}_{\mathbf{j}}(\mathbb{F}_{\bullet},f_{\bullet})\]
is also a homeomorphism.
\end{lemma}
\begin{proof}
Notice that
\begin{equation}\label{eq:linear_proj_disc}
S^{n-1}\cap\mathbb{H}{\left(\mathbb{F}_{j_1},f_{j_1}\right)}^-\cong D^{j_1-1}
\end{equation}
with a homeomorphism being the linear projection on $\mathbb{F}_{j_1-1}$. Moreover
\begin{align}
\begin{split}
\pi_1^{-1}(f_{j_1})&=\big\{(f_{j_1},v_2,\ldots,v_k)\in\StO{k}{n}:v_{\ell}\in\mathbb{H}(\mathbb{F}_{j_{\ell}},f_{j_{\ell}})^-\ \forall\ell\in[k]\setminus\{1\}\big\}\\
&\cong\Big\{(v'_2,\ldots,v'_k)\in\StO{k-1}{n-1}\\
&\qquad\qquad\qquad\qquad\qquad:v'_{\ell}\in\mathbb{H}\left(\faktor{\mathbb{F}_{j_{\ell}}}{\left<f_{j_1}\right>},\pi_{\hat{j}_1}(f_{j_{\ell}})\right)^-\ \forall\ell\in[k]\setminus\{1\}\Big\}\\
&=\tilde{\mathcal{C}}_{\mathbf{j}'}{\left(\mathbb{F}'_{\bullet},f'_{\bullet}\right)}^-
\end{split}\label{eq:induction_step}
\end{align}
where $\pi_{\hat{j}_1}$ is the projection on the $n-1$ coordinates $\{1,\ldots,j_1-1,j_1+1,\ldots,n\}$ and
\begin{align*}
\mathbf{j}'&:=\{1\leq j_2-1<\cdots<j_k-1\leq n-1\}\\
\mathbb{F}'_i&:=\left\{\begin{array}{ll}\mathbb{F}_i&,0\leq i<j_1\\\faktor{\mathbb{F}_{i+1}}{\left<f_{j_1}\right>}&,j_1\leq i<n\end{array}\right.\\
f'_i&:=\left\{\begin{array}{ll}\pi_{\hat{j}_1}(f_i)&,0\leq i<j_1\\\pi_{\hat{j}_1}(f_{i+1})&,j_1\leq i<n\end{array}\right.
\end{align*}
So, $\tilde{\Phi}_{\mathbf{j}}$ is constructed inductively, using \eqref{eq:trivial_fb}:
\begin{align*}
\tilde{\mathcal{C}}_{\mathbf{j}}{\left(\mathbb{F}_{\bullet},f_{\bullet}\right)}^-&\cong D^{j_1-1}\times\tilde{\mathcal{C}}_{\mathbf{j}'}{\left(\mathbb{F}_{\bullet}',f_{\bullet}'\right)}^-\\
&\cong D^{j_1-1}\times D^{j_2-2}\times \tilde{\mathcal{C}}_{\mathbf{j}''}{\left(\mathbb{F}_{\bullet}'',f_{\bullet}''\right)}^-\\
&\ \,\vdots\\
&\cong D^{j_1-1}\times D^{j_2-2}\times\cdots\times D^{j_k-k}\\
&\cong D^{\mathrm{d}(\mathbf{j})}
\end{align*}
For the second claim, notice that
$S^{n-1}\cap\mathbb{H}(\mathbb{F}_{j_1},f_{j_1})\cong\big(D^{j_1-1}\big)^{\circ}$
through the same linear projection as in \eqref{eq:linear_proj_disc} and
$\pi_1|_{\tilde{\mathcal{C}}_{\mathbf{j}}(\mathbb{F}_{\bullet},f_{\bullet})}^{-1}(f_{j_1})\cong\tilde{\mathcal{C}}_{\mathbf{j}'}(\mathbb{F}_{\bullet}',f_{\bullet}')$
through the same map as in \eqref{eq:induction_step}. Using Remark~\ref{rem:trivial_fb_interior} we get
$\tilde{\mathcal{C}}_{\mathbf{j}}(\mathbb{F}_{\bullet},f_{\bullet})\cong\big(S^{n-1}\cap\mathbb{H}(\mathbb{F}_{j_1},f_{j_1})\big)\times\pi_1|_{\tilde{\mathcal{C}}_{\mathbf{j}}(\mathbb{F}_{\bullet},f_{\bullet})}^{-1}(f_{j_1})$
through the same map as in \eqref{eq:trivial_fb}.
So, the inductive construction above gives us that
\[\tilde{\mathcal{C}}_{\mathbf{j}}(\mathbb{F}_{\bullet},f_{\bullet})\cong\big(D^{\mathrm{d}(\mathbf{j})}\big)^{\circ}\]
with the homeomorphism being a restriction of $\tilde{\Phi}_{\mathbf{j}}^{-1}$, which proves the assertion.
\end{proof}

In order to examine the CW structure of $\Gr{k}{n}$, we need to connect the sets $\tilde{\mathcal{C}}_{\mathbf{j}}(\mathbb{F}_{\bullet},f_{\bullet})$ defined to the cells $\mathcal{C}_{\mathbf{j}}(\mathbb{F}_{\bullet})$:
\begin{lemma}\label{lem:cells_from_stiefel_to_gr} Let $k,n\in\mathbb{N}$, with $0<k<n$. Moreover, let $\mathbb{F}_{\bullet}$ be a flag of $\mathbb{R}^n$ and $f_{\bullet}$ be an orthonormal basis of $\mathbb{R}^n$ compatible with $\mathbb{F}_{\bullet}$. Then, for each $\mathbf{j}\in\binom{[n]}{k}$ the following restriction of $q_0:\StO{k}{n}\to\Gr{k}{n}$ is a homeomorphism:
\begin{center}
\begin{tikzcd}
\tilde{\mathcal{C}}_{\mathbf{j}}(\mathbb{F}_{\bullet},f_{\bullet})\ar[r,"q_0"]&\mathcal{C}_{\mathbf{j}}(\mathbb{F}_{\bullet})\\[-1.5em]
(v_1,\ldots,v_k)\ar[r,mapsto]&\left<v_1,\ldots,v_k\right>
\end{tikzcd}
\end{center}
\end{lemma}
\begin{proof} First notice that $q_0$ is well defined. Indeed, let $(v_1,\ldots,v_k)\in\tilde{\mathcal{C}}_{\mathbf{j}}(\mathbb{F}_{\bullet},f_{\bullet})$ and
\[H:=q_0(v_1,\ldots,v_k)=\left<v_1,\ldots,v_k\right>\in\Gr{k}{n}\]
Fix some $i\in[n]$. For every $\ell\in[k]$ with $j_{\ell}\leq i$,  $v_{\ell}\in\mathbb{F}_{j_{\ell}}\subseteq\mathbb{F}_i$. On the other hand, if $j_{\ell}>i$, then $v_{\ell}\cdot f_{j_{\ell}}>0$, i.e. $v_{\ell}\not\in\mathbb{F}_i$ since $f_{j_{\ell}}\perp\mathbb{F}_i$. Hence,
$\dim(H\cap\mathbb{F}_i)=\dim\left(\left<\big\{v_{\ell}:j_{\ell}\leq i\big\}\right>\right)=\max\{\ell\in[k]_0:j_{\ell}\leq i\}$
where $j_0=0$ as before, that proves that $H\in\mathcal{C}_{\mathbf{j}}(\mathbb{F}_{\bullet})$.

In order to prove that this is a homeomorphism, we construct its inverse map $r_0:\mathcal{C}_{\mathbf{j}}(\mathbb{F}_{\bullet})\to\tilde{\mathcal{C}}(\mathbb{F}_{\bullet},f_{\bullet})$ as follows: Let $H\in\mathcal{C}_{\mathbf{j}}(\mathbb{F}_{\bullet})$. Then, for any $\ell\in[k]$,
\begin{align*}
\dim\big(\left(H\cap\mathbb{F}_{j_{\ell}}\right)\cap\left(H\cap\mathbb{F}_{j_{\ell}-1}\right)^{\perp}\big)&=\dim\big(H\cap\mathbb{F}_{j_{\ell}}\big)+\dim\big(\left(H\cap\mathbb{F}_{j_{\ell}-1}\right)^{\perp}\big)-n\\
&=\dim\big(H\cap\mathbb{F}_{j_{\ell}}\big)-\dim\big(H\cap\mathbb{F}_{j_{\ell}-1}\big)=1
\end{align*}
so, there exist exactly two vectors $-a,a$ such that
\[-a,a\in H\cap\mathbb{F}_{j_{\ell}}\qquad\text{ and }\qquad-a,a\perp H\cap\mathbb{F}_{j_{\ell}-1}\qquad\text{ and }\qquad-a,a\in S^{n-1}\]
Notice that $a$ and $-a$ cannot be perpendicular to $f_{j_{\ell}}$: Indeed, if $a\in\mathbb{F}_{j_{\ell}}$ and $a^tf_{j_{\ell}}=0$, then $a\in\mathbb{F}_{j_{\ell}-1}$, which leads to a contradiction since $a\perp H\cap\mathbb{F}_{j_{\ell}-1}$. We now define $u_{\ell}\in\{a,-a\}$ to be the unique vector that satisfies all the above and $u_{\ell}^tf_{j_{\ell}}>0$ and then also define $r_0(H):=(u_1,\ldots,u_k)$.

First notice $r_0$ is well defined, since for $1\leq\ell_1<\ell_2\leq k$,
$u_{\ell_2}\perp H\cap\mathbb{F}_{j_{\ell}-1}\supseteq H\cap\mathbb{F}_{j_{\ell-1}}\ni u_{\ell_1}$
i.e. $u_{\ell_1}^tu_{\ell_2}=0$. Moreover, for $\ell\in[k]$, $u_{\ell}\in S^{n-1}$, i.e. $u_{\ell}^tu_{\ell}=1$. Also, $v_{\ell}\in\mathbb{H}(\mathbb{F}_{j_{\ell}},f_{j_{\ell}})$ by the construction of $v_{\ell}$. This proves that $r_0(H)=(u_1,\ldots,u_k)\in\tilde{\mathcal{C}}_{\mathbf{j}(\mathbb{F}_{\bullet},f_{\bullet})}$.

Then, notice that for $H\in\mathcal{C}_{\mathbf{j}}(\mathbb{F}_{\bullet})$,
\[q_0(r_0(H))=q_0(u_1,\ldots,u_k)=\left<u_1,\ldots,u_k\right>=H\]
Moreover, for $(v_1,\ldots,v_k)\in\tilde{\mathcal{C}}_{\mathbf{j}}(\mathbb{F}_{\bullet},f_{\bullet})$,
$v_{\ell}\in\left<v_1,\ldots,v_k\right>\cap\mathbb{F}_{j_{\ell}}\text{ and } v_{\ell}\perp\left<v_1,\ldots,v_k\right>\cap\mathbb{F}_{j_{\ell}-1}\text{ and }v_{\ell}\in S^{n-1}\text{ and }v_{\ell}^tf_{j_{\ell}}>0$
which means that
\[r_0(q_0(v_1,\ldots,v_k))=r_0(\left<v_1,\ldots,v_k\right>)=(v_1,\ldots,v_k)\]
since there is a unique choice for $r_0$, as proven above.
\end{proof}

\begin{lemma}\label{lem:q0_on_bdr} Let $k,n\in\mathbb{N}$, with $0<k<n$. Moreover, let $\mathbb{F}_{\bullet}$ be a flag of $\mathbb{R}^n$ and $f_{\bullet}$ be an orthonormal basis of $\mathbb{R}^n$ compatible with $\mathbb{F}_{\bullet}$. Then, for each $\mathbf{j}\in\binom{[n]}{k}$ the restriction of $q_0:\StO{k}{n}\to\Gr{k}{n}$ has the following property:
\[q_0\big(\tilde{\mathcal{C}}_{\mathbf{j}}(\mathbb{F}_{\bullet},f_{\bullet})^-\setminus\tilde{\mathcal{C}}_{\mathbf{j}}(\mathbb{F}_{\bullet},f_{\bullet})\big)\subseteq\bigcup_{\substack{\mathbf{j}'\in\binom{[n]}{k}\\\mathrm{d}(\mathbf{j}')<\mathrm{d}(\mathbf{j})}}\mathcal{C}_{\mathbf{j}'}(\mathbb{F}_{\bullet})\]
\end{lemma}
\begin{proof} Let $(v_1,\ldots,v_k)\in\tilde{\mathcal{C}}_{\mathbf{j}}(\mathbb{F}_{\bullet},f_{\bullet})^-\setminus\tilde{\mathcal{C}}_{\mathbf{j}}(\mathbb{F}_{\bullet},f_{\bullet})$ and set $H=q_0(v_1,\ldots,v_k)=\left<v_1,\ldots,v_k\right>$. Due to Lemma~\ref{lem:jump_pts} there exists some $\mathbf{j}'\in\binom{[n]}{k}$ such that $H\in\mathcal{C}_{\mathbf{j}'}(\mathbb{H}_{\bullet})$. For this $\mathbf{j}'$,
\[j_{\ell}'=\min\{j\in[n]:\dim(H\cap\mathbb{F}_j)\geq\ell\}\]
Since $v_1,\ldots,v_{\ell}\in H\cap\mathbb{F}_{j_{\ell}}$ and are linearly independent, $j_{\ell}'\leq j_{\ell}$ for every $l\in[k]$. Moreover, since $(v_1,\ldots,v_k)\not\in\tilde{\mathcal{C}}_{\mathbf{j}}(\mathbb{F}_{\bullet},f_{\bullet})$, there exists some $\ell_0\in[k]$ such that $v_{\ell_0}\in\mathbb{H}(\mathbb{F}_{j_{\ell_0}},f_{j_{\ell_0}})^-\setminus\mathbb{H}(\mathbb{F}_{j_{\ell_0}},f_{j_{\ell_0}})$, i.e. $v_{\ell_0}\in\mathbb{F}_{j_{\ell_0}}$ and $v_{\ell_0}^tf_{j_{\ell_0}}=0$. This means that $v_1,\ldots,v_{\ell_0}\in\mathbb{F}_{j_{\ell_0}-1}<j_{\ell_0}$, i.e. $j_{\ell_0}'\leq j_{\ell_0}-1$. Combining these, we get
\[\mathrm{d}(\mathbf{j}')=\sum_{\ell\in[k]}(j'_{\ell}-\ell)<\sum_{\ell\in[k]}(j_{\ell}-\ell)=\mathrm{d}(\mathbf{j})\qedhere\]
\end{proof}

\begin{theorem}\label{thm:gr_is_cw} Let $k,n\in\mathbb{N}$ such that $0<k<n$ and $\mathbb{F}_{\bullet}$ be any flag of $\mathbb{R}^n$. Moreover for every integer $m\in[k(n-k)]_0\cup\{-1\}$ let \[X_m^{k,n}:=\bigcup_{\substack{\mathbf{j}\in\binom{[n]}{k}\\\mathrm{d}(\mathbf{j})\leq m}}\mathcal{C}_{\mathbf{j}}(\mathbb{F}_{\bullet})\]
Then, the filtration $\emptyset=X_{-1}^{k,n}\subseteq X_0^{k,n}\subseteq\cdots\subseteq X_{k(n-k)}^{k,n}=\Gr{k}{n}$ is a CW structure of $\Gr{k}{n}$. It is helpful for later to define $X_m^{k,n}:=X_{k(n-k)}^{k,n}$ for every $m>k(n-k)$.
\end{theorem}
\begin{proof}
Since the filtration is finite, $\Gr{k}{n}$ trivially has the direct limit topology with respect to the filtration, so we only need to prove the existence of a pushout square for every $m\in[k(n-k)]$:
\begin{center}
\begin{tikzcd}
\coprod_{\mathbf{j}\in I_m}S^{m-1}\ar[d,hook,"\amalg_{\mathbf{j}\in I_m}inc_{\mathbf{j}}"']\ar[r,"\amalg_{\mathbf{j}\in I_m}\phi_{\mathbf{j}}"]\ar[dr,phantom,near end,"\ulcorner"]&[3em]X_{m-1}^{k,n}\ar[d,hook,"inc_m"]\\[2em]
\coprod_{\mathbf{j}\in I_m}D^m\ar[r,"\amalg_{\mathbf{j}\in I_m}\Phi_{\mathbf{j}}"']&X_m^{k,n}
\end{tikzcd}
\end{center}
Let $I_m=\big\{\mathbf{j}\in\binom{[n]}{k}:\mathrm{d}(\mathbf{j})=m\big\}$ be the set of indices. Moreover, let $\Phi_{\mathbf{j}}=q_0\circ\tilde{\Phi}_{\mathbf{j}}:D^{\mathrm{d}(\mathbf{j})}\to \Gr{k}{n}$ be the characteristic maps, where $\tilde{\Phi}_{\mathbf{j}}:D^{\mathrm{d}(\mathbf{j})}\to\tilde{\mathcal{C}}_{\mathbf{j}}(\mathbb{F}_{\bullet},f_{\bullet})^-\subseteq\StO{k}{n}$ are the homeomorphisms defined in Lemma~\ref{lem:shub_dim} and $q_0:\StO{k}{n}\to\Gr{k}{n}$ is the usual quotient map. Also, let $\phi_{\mathbf{j}}=\Phi_{\mathbf{j}}|_{\partial D^{\mathrm{d}(\mathbf{j})}}$ just be the restriction on the boundary of the closed disk.

First, the maps in the diagram are well defined:
\begin{i_enum}
\item $\Phi_{\mathbf{j}}\big(D^{\mathrm{d}(\mathbf{j})}\big)\subseteq X_{\mathrm{d}(\mathbf{j})}^{k,n}$. Indeed, $\Phi_{\mathbf{j}}\big(D^{\mathrm{d}(\mathbf{j})}\big)=q_0\big(\tilde{\Phi}_{\mathrm{j}}(D^{\mathrm{d}(\mathbf{j})})\big)=q_0\big(\tilde{\mathcal{C}}_{\mathbf{j}}(\mathbb{F}_{\bullet},f_{\bullet})^-\big)$, since $\tilde{\Phi}_{\mathbf{j}}$ is a homeomorphism. Moreover, because of Lemmata~\ref{lem:cells_from_stiefel_to_gr} and~\ref{lem:q0_on_bdr}, we get that $q_0\big(\tilde{\mathcal{C}}_{\mathbf{j}}(\mathbb{F}_{\bullet},f_{\bullet})^-\big)=q_0\big(\tilde{\mathcal{C}}_{\mathbf{j}}(\mathbb{F}_{\bullet},f_{\bullet})^-\setminus\tilde{\mathcal{C}}_{\mathbf{j}}(\mathbb{F}_{\bullet},f_{\bullet})\big)\cup q_0\big(\tilde{\mathcal{C}}_{\mathbf{j}}(\mathbb{F}_{\bullet},f_{\bullet})\big)\subseteq\bigcup_{\substack{\mathbf{j}'\in\binom{[n]}{k}\\\mathrm{d}(\mathbf{j}')<\mathrm{d}(\mathbf{j})}}\mathcal{C}_{\mathbf{j}'}(\mathbb{F}_{\bullet})\cup\mathcal{C}_{\mathbf{j}}(\mathbb{F}_{\bullet})\subseteq X_{\mathrm{d}(\mathbf{j})}^{k,n}$.
\item $\phi_{\mathbf{j}}\big(S^{\mathrm{d}(\mathbf{j})-1}\big)\subseteq X_{\mathrm{d}(\mathbf{j})-1}^{k,n}$. Indeed, $\phi_{\mathbf{j}}\big(S^{\mathrm{d}(\mathbf{j})-1}\big)=q_0\big(\tilde{\Phi}_{\mathbf{j}}(\partial D^{\mathrm{d}(\mathbf{j})})\big)=q_0\big(\tilde{\mathcal{C}}_{\mathbf{j}}(\mathbb{F}_{\bullet},f_{\bullet})^-\setminus\tilde{\mathcal{C}}_{\mathbf{j}}(\mathbb{F}_{\bullet},f_{\bullet})\big)$, since $\tilde{\Phi}_{\mathbf{j}}$ is a homeomorphism. Moreover, because of Lemma~\ref{lem:q0_on_bdr}, we get that $q_0\big(\tilde{\mathcal{C}}_{\mathbf{j}}(\mathbb{F}_{\bullet},f_{\bullet})^-\setminus\tilde{\mathcal{C}}_{\mathbf{j}}(\mathbb{F}_{\bullet},f_{\bullet})\big)\subseteq\bigcup_{\substack{\mathbf{j}'\in\binom{[n]}{k}\\\mathrm{d}(\mathbf{j}')<\mathrm{d}(\mathbf{j})}}\mathcal{C}_{\mathbf{j}'}(\mathbb{F}_{\bullet})\subseteq X_{\mathrm{d}(\mathbf{j})-1}^{k,n}$.
\end{i_enum}
Next, the commutativity of the diagram is clear, since $\phi_{\mathbf{j}}$ is a restriction of $\Phi_{\mathbf{j}}$ and the two vertical maps are inclusions.

Last, using Lemma~\ref{lem:shub_dim} and Lemma~\ref{lem:cells_from_stiefel_to_gr}, we get that both of the functions involved in the following composition are homeomorphisms
\begin{center}
\begin{tikzcd}
\Phi_{\mathbf{j}}|_{(D^{\mathrm{d}(\mathbf{j})})^{\circ}}\ar[r,phantom,":"]&[-2.6em](D^{\mathrm{d}(\mathbf{j})})^{\circ}\ar[r,"\tilde{\Phi}_{\mathbf{j}}|_{(D^{\mathrm{d}(\mathbf{j})})^{\circ}}","\cong"']&[3em]\tilde{\mathcal{C}}_{\mathbf{j}}(\mathbb{F}_{\bullet},f_{\bullet})\ar[r,"q_0|_{\tilde{\mathcal{C}}_{\mathbf{j}}(\mathbb{F}_{\bullet},f_{\bullet})}","\cong"']&[3em]\mathcal{C}_{\mathbf{j}}(\mathbb{F}_{\bullet})
\end{tikzcd}
\end{center}
Also, $\Phi_{\mathbf{j}}((D^{\mathrm{d}(\mathbf{j})})^{\circ})=\mathcal{C}_{\mathbf{j}}(\mathbb{F}_{\bullet})\subseteq X_m\setminus X_{m-1}$, which proves that it is a pushout diagram.
\end{proof}

\begin{proposition}\label{prop:included_cell} For $k,n\in\mathbb{N}$, the inclusion $\iota_{k,n}:\Gr{k}{n}\hookrightarrow\Gr{k}{n+1}$ respects the cell structure, i.e. for every $\mathbf{j}\in\binom{[n]}{k}$, it is also true that $\mathbf{j}\in\binom{[n+1]}{k}$ and then $\iota_{k,n}(\mathcal{C}_{\mathbf{j}}(\mathbb{F}_{\bullet}))=\mathcal{C}_{\mathbf{j}}(\mathbb{F}_{\bullet})\subseteq\Gr{k}{n+1}$.
\end{proposition}
\begin{proof} If $\mathbf{j}=\{1\leq j_1<j_2<\cdots<j_k\leq n\}$, then trivially $j_k\leq n+1$, i.e. $\{1\leq j_1<\cdots<j_k\leq n+1\}\in\binom{[n+1]}{k}$. Moreover, if $H\in\mathcal{C}_{\mathbf{j}}(\mathbb{F}_{\bullet})\subseteq\Gr{k}{n}$ then $\dim(H\cap\mathbb{F}_{n+1})=j_k=\max\{\ell\in[k]_0:j_{\ell}\leq i\}$, i.e. $\iota_{k,n}(H)=H\subseteq\mathbb{R}^{n+1}$ is an element of $\mathcal{C}_{\mathbf{j}}(\mathbb{F}_{\bullet})\subseteq\Gr{k}{n+1}$. Inversely, for $H\in\mathcal{C}_{\mathbf{j}}(\mathbb{F}_{\bullet})\subseteq\Gr{k}{n+1}$, then $H\in\mathcal{C}_{\mathbf{j}}(\mathbb{F}_{\bullet})\subseteq\Gr{k}{n}$ since $j_k\leq n$ and thus $\iota_{k,n}(H)=H$.
\end{proof}

\begin{lemma}\label{lem:stable_filtration} Let $k,m\in\mathbb{N}$, then there exists some $n_0\in\mathbb{N}$ such that
\[X_m^{k,k+1}\subseteq X_m^{k,k+2}\subseteq\cdots\subseteq X_m^{k,n_0}=X_m^{k,n_0+1}=X_m^{k,n_0+2}=\cdots\]
i.e. $X_m^{k,n}=X_m^{k,n_0}$ for every $n\geq n_0$.
\end{lemma}
\begin{proof} If $I_m^{k,n}=\{\mathbf{j}\in\binom{[n]}{k}:\mathrm{d}(\mathbf{j})\leq m\}$, then $X_m^{k,n}=\bigcup_{\mathbf{j}\in I_m^{k,n}}\mathcal{C}_{\mathbf{j}}(\mathbb{F}_{\bullet})$. Using Proposition~\ref{prop:included_cell}, we have $X_m^{k,n}=\bigcup_{I_m^{k,n}}\mathcal{C}_{\mathbf{j}}(\mathbb{F}_{\bullet})\subseteq\bigcup_{I_m^{k,n+1}}\mathcal{C}_{\mathbf{j}}(\mathbb{F}_{\bullet})=X_m^{k,n+1}$ for every $n\in\mathbb{N}$. Let $n>m+k$ and $\mathbf{j}\in I_m^{k,n}$. Notice then that $j_1-1+j_2-2+\cdots+j_k-k\leq m$ i.e. $j_k\leq m+k-\big((j_1-1)+\cdots+(j_{k-1}-k+1)\big)\leq m+k<n$, since for every $i\in[k-1]$ $j_i\geq i$. This means that $\mathbf{j}\in I_m^{k,n-1}$ and so $X_m^{k,n-1}\subseteq X_m^{k,n}$. This means that for $n_0:=m+k$ the assertion is true.
\end{proof}

\begin{theorem} Let $k\in\mathbb{N}$ and $\mathbb{F}_{\bullet}$ be any flag of $\mathbb{R}^{\infty}$. Moreover, for every integer $m\geq-1$ let $X_m^k:=\bigcup_{n=k+1}^{\infty}X_m^{k,n}$. Then, the filtration
\[\emptyset=X_{-1}^k\subseteq X_0^k\subseteq X_1^k\subseteq X_2^k\subseteq\cdots\subseteq\Gr{k}\]
is a CW structure of $\Gr{k}$.
\end{theorem}
\begin{proof} Let $m\geq-1$ be any integer. Then, using Lemma~\ref{lem:stable_filtration} we get that there exists some $n_0$ such that $X_m^k=X_m^{k,n_0}$ and $X_{m-1}^k=X_{m-1}^{k,n_0}$. This means that there exists a pushout square involving $X_{m-1}^k$ and $X_m^k$, namely exactly the same we used in the proof of Theorem~\ref{thm:gr_is_cw}.

It remains to show that $\Gr{k}$ has the direct limit topology with respect to the filtration $X_m^k$. Notice that $\bigcup_{m=-1}^{\infty}X_m^k=\bigcup_{m=-1}^{\infty}\bigcup_{n=k+1}^{\infty}X_m^{k,n}=\bigcup_{n=k+1}^{\infty}\bigcup_{m=-1}^{\infty}X_m^{k,n}=\bigcup_{n=k+1}\Gr{k}{n}=\Gr{k}$. Moreover, let $U\in\Gr{k}$ be any open set, fix some integer $m\geq-1$ and notice that $X_m^k=X_m^{k,n_0}$ for some $n_0$. So, we get that $U\cap X_m^k=U\cap X_m^{k,n_0}=U\cap\Gr{k}{n_0}\cap X_m^{k,n_0}$ which is open in $X_m^{k,n_0}=X_m^k$, since $U\cap\Gr{k}{n_0}$ is open in $\Gr{k}{n_0}$ and $X_m^{k,n_0}\subseteq\Gr{k}{n_0}$. For the inverse direction, let $U\subseteq\Gr{k}$ be some set such that $U\cap X_m^k$ is open in $X_m^k$ for every integer $m\geq-1$. Then, we get that $U\cap\Gr{k}{n}=U\cap X_{k(n-k)}^{k,n-k}=U\cap X_{k(n-k)}^k\cap X_{k(n-k)}^{k,n-k}$ which is open in $X_{k(n-k)}^{k,n-k}=\Gr{k}{n}$, since $U\cap X_{k(n-k)}^k$ is open in $X_{k(n-k)}^k$ and $X_{k(n-k)}^{k,n-k}\subseteq X_{k(n-k)}^k$. Since this is true for every integer $n>0$, $U$ is open in $\Gr{k}$, because it is topologised with the direct limit topology with respect to the filtration $\Gr{k}{n}$.
\end{proof}

So, $\Gr{k}{n}$ is a CW complex and each cell is indexed by some $\mathbf{j}=\{1\leq j_1<\cdots<j_k\leq n\}\in\binom{[n]}{k}$, while $\Gr{k}$ is a CW complex and each cell is indexed by some $\mathbf{j}=\{1\leq j_1<j_2<\cdots\}\in\binom{\mathbb{N}}{k}$. In either case the dimension of the cell indexed by $\mathbf{j}$ is equal to $\mathrm{d}(\mathbf{j})=(j_1-1)+(j_2-2)+\cdots+(j_k-k)$.

\section{Enumerating the cells}
\begin{definition} For $m,r\in\mathbb{N}$, $\lambda=(\lambda_1,\ldots,\lambda_r)\in\mathbb{N}^r$ is a \emph{partition of $m$ of size $r$} if $\lambda_1\geq\lambda_2\geq\cdots\geq\lambda_r\geq1$ and $\lambda_1+\lambda_2+\cdots+\lambda_r=m$. In this case, we write $\lambda\vdash m$ and $|\lambda|=r$. For $m=0$, we define $P_0:=\{\lambda_{\emptyset}\}$, where $\lambda_{\emptyset}:=()$ is the \emph{empty partition} with $\lambda_{\emptyset}\vdash0$ and $|\lambda|=0$. Let $P_m$ be the set of all partitions of $m$ and $P:=\bigcup_{m=0}^{\infty}P_m$ the set of all partitions.
\end{definition}
\begin{remark}
In order to make the notation easier, any number of trailing zeros at the end of a partition are allowed, without this affecting the size of the partition, i.e. $\lambda=(5,4,4,2,0,0,0)=(5,4,4,2)$, with $\lambda\vdash15$ and $|\lambda|=4$.
\end{remark}
\begin{example} There are exactly $7$ partitions of $m=5$:
\[(5),\ (4,1),\ (3,2),\ (3,1,1),\ (2,2,1),\ (2,1,1,1),\ (1,1,1,1,1)\]
\end{example}

\begin{notation} For every partition $\lambda$ of $m$ in $r$ parts, we define the \emph{Young diagram} corresponding to $\lambda$ to be the shape consisting of $r$ left-justified rows of boxes, where row $i$ has $\lambda_i$ boxes.
\end{notation}
\begin{example} The young diagrams of the $7$ partitions of $m=5$ are:
\begin{center}
\ytableausetup{boxsize=0.7em,aligntableaux=top}\ydiagram{5}\qquad\ydiagram{4,1}\qquad\ydiagram{3,2}\qquad\ydiagram{3,1,1}\qquad\ydiagram{2,2,1}\qquad\ydiagram{2,1,1,1}\qquad\ydiagram{1,1,1,1,1}
\end{center}
\end{example}

\begin{proposition}\label{prop:poset_bijection} For $k,n\in\mathbb{N}$ such that $0<k<n$, the map
\begin{center}
\begin{tikzcd}
b_{k,n}\ar[r,phantom,":"]&[-7em]\binom{[n]}{k}\ar[r]&\big\{\lambda\in P:|\lambda|\leq k\text{ and }\lambda_1\leq n-k\big\}\\[-1.5em]
&\big\{1\leq j_1<\cdots<j_k\leq n\big\}\ar[r,mapsto]&\big(j_k-k\ ,\ j_{k-1}-(k-1)\ ,\ \ldots\ ,\ j_1-1\big)
\end{tikzcd}
\end{center}
is a bijection with $b_{k,n}(\mathbf{j})\vdash\mathrm{d}(\mathbf{j})$.
\end{proposition}
\begin{proof} First of all notice that for $\lambda=b_{k,n}(\mathbf{j})$ we have $|\lambda|\leq|\mathbf{j}|=k$, $\lambda_1=j_k-k\leq n-k$, $\lambda_k=j_1-1\geq 0$ and for every $i\in[k-1]$ $\lambda_i=j_{k-i+1}-(k-i+1)>j_{k-i}-(k-i)-1=\lambda_{i+1}-1$, i.e. $\lambda_i\geq\lambda_{i+1}$. These mean that $b_{k,n}$ is well defined.

In order to prove that $b_{k,n}$ is a bijection, we will prove that this is its inverse map:
\begin{center}
\begin{tikzcd}
c_{n,k}\ar[r,phantom,":"]&[-2.5em]\big\{\lambda\in P:|\lambda|\leq k\text{ and }\lambda_1\leq n-k\big\}\ar[r]&[-1em]\binom{[n]}{k}\\[-1.5em]
&(\lambda_1,\lambda_2,\ldots,\lambda_k)\ar[r,mapsto]&\{\lambda_k+1,\lambda_{k-1}+2,\ldots,\lambda_1+k\}
\end{tikzcd}
\end{center}
Notice that for $\mathbf{j}=c_{k,n}(\lambda)$ we have $|\mathbf{j}|=k$, $j_1=\lambda_k+1\geq1$, $j_k=\lambda_1+k\leq n$ and for ever $i\in[k-1]$ $j_i=\lambda_{k-i+1}+i\leq\lambda_{k-i}+(i+1)-1=j_{i+1}-1$, i.e. $j_i<j_{i+1}$. These mean that $c_{k,n}$ is well defined.

Moreover $c_{k,n}=b_{k,n}^{-1}$, since $c_{k,n}\big(b_{k,n}(\mathbf{j})\big)=c_{k,n}(j_k-k,\ldots,j_1-1)=\{j_1<\cdots<j_k\}=\mathbf{j}$ and $b_{k,n}\big(c_{k,n}(\lambda)\big)=b_{k,n}\big(\{\lambda_k+1<\cdots<\lambda_1+k\}\big)=(\lambda_1,\ldots,\lambda_k)=\lambda$.

Finally, we have that $b_{k,n}(\mathbf{j})\vdash (j_k-k)+(j_{k-1}-(k-1))+\cdots+(j_1-1)=\mathrm{d}(\mathbf{j})$.
\end{proof}
\begin{remark} Fix some $k\in\mathbb{N}$ and notice that if $\mathbf{j}\in\binom{[n]}{k}$ and $j_k\leq n-1$, then $b_{k,n}(\mathbf{j})=b_{k,n-1}(\mathbf{j})$, so for $n\geq n_0$, $b_{k,n}|_{\binom{[n_0]}{k}}=b_{k,n_0}$. This lets us define the map $b_k:\binom{\mathbb{N}}{k}\to\big\{\lambda\in P:|\lambda|\leq k\}$ on $\mathbf{j}$ to be $b_{k,n}(\mathbf{j})$ for some $n\geq j_k$. Then, $b_k$ is also a bijection with $b_k(\mathbf{j})\vdash\mathrm{d}(\mathbf{j})$.
\end{remark}

We can define the following partial order on the set of all partitions
\begin{definition} For $\lambda,\lambda'\in P$, we write $\lambda\leq\lambda'$ if $\lambda_i\leq\lambda_i'$ for every $i\in\mathbb{N}$.
\end{definition}
\begin{remarks}
\begin{i_enum}
\item This is a partial order. Indeed, it is easy to check that $\lambda\leq\lambda$, that $\lambda\leq\lambda'$ and $\lambda'\leq\lambda$ gives $\lambda=\lambda'$ and that $\lambda\leq\lambda'$ and $\lambda'\leq\lambda''$ gives $\lambda\leq\lambda''$.
\item In the language of Young diagrams, $\lambda\leq\lambda'$ if and only if the corresponding diagram of $\lambda$ is contained fully inside the corresponding diagram of $\lambda'$.
\end{i_enum}
\end{remarks}

\begin{proposition} $(P,\leq)$ is a graded poset, with grading $\rho(\lambda)=m$, where $\lambda\vdash m$.
\end{proposition}
\begin{proof} First of all, we have to prove that $\rho$ is order preserving. Let $\lambda,\lambda'\in P$ with $\lambda\leq\lambda'$. Then $\sum_{i=1}^{|\lambda|}\lambda_i\leq\sum_{i=1}^{|\lambda'|}\lambda_i'$ and thus $\rho(\lambda)\leq\rho(\lambda')$. Moreover, we have to show that for $\lambda,\lambda'\in P$ with $[\lambda,\lambda']=\{\lambda,\lambda'\}$, $\rho(\lambda)=\rho(\lambda')$. Let $\lambda,\lambda'\in P$ with $\lambda<\lambda'$, $\lambda\vdash m$ and $\lambda'\vdash m'$. Notice that if $m'-m\geq2$, then there exists at least one $\lambda''\in P$ with $\lambda<\lambda''<\lambda'$. Indeed, in the young diagram of $\lambda$ is missing at least $2$ boxes, compared to the one of $\lambda'$ and by attaching just one of them we create some $\lambda''$ strictly between $\lambda$ and $\lambda'$.
\end{proof}

The poset up to degree $6$ looks like this:
\begin{center}
\begin{tikzpicture}[scale=1.3, yscale=1.3]
\ytableausetup{boxsize=0.3em,aligntableaux=top}
\node[elt] (0) at (0,0) {};
\node[elt] (1) at (0,1) {\ydiagram{1}};
\node[elt] (2) at (-0.5,2) {\ydiagram{2}};
\node[elt] (11) at (0.5,2) {\ydiagram{1,1}};
\node[elt] (3) at (-1,3) {\ydiagram{3}};
\node[elt] (21) at (0,3) {\ydiagram{2,1}};
\node[elt] (111) at (1,3) {\ydiagram{1,1,1}};
\node[elt] (4) at (-2,4) {\ydiagram{4}};
\node[elt] (31) at (-1,4) {\ydiagram{3,1}};
\node[elt] (22) at (0,4) {\ydiagram{2,2}};
\node[elt] (211) at (1,4) {\ydiagram{2,1,1}};
\node[elt] (1111) at (2,4) {\ydiagram{1,1,1,1}};
\node[elt] (5) at (-3,5) {\ydiagram{5}};
\node[elt] (41) at (-2,5) {\ydiagram{4,1}};
\node[elt] (32) at (-1,5) {\ydiagram{3,2}};
\node[elt] (311) at (0,5) {\ydiagram{3,1,1}};
\node[elt] (221) at (1,5) {\ydiagram{2,2,1}};
\node[elt] (2111) at (2,5) {\ydiagram{2,1,1,1}};
\node[elt] (11111) at (3,5) {\ydiagram{1,1,1,1,1}};
\node[elt] (5) at (-3,5) {\ydiagram{5}};
\node[elt] (41) at (-2,5) {\ydiagram{4,1}};
\node[elt] (32) at (-1,5) {\ydiagram{3,2}};
\node[elt] (311) at (0,5) {\ydiagram{3,1,1}};
\node[elt] (221) at (1,5) {\ydiagram{2,2,1}};
\node[elt] (2111) at (2,5) {\ydiagram{2,1,1,1}};
\node[elt] (11111) at (3,5) {\ydiagram{1,1,1,1,1}};
\node[elt] (6) at (-5,6) {\ydiagram{6}};
\node[elt] (51) at (-4,6) {\ydiagram{5,1}};
\node[elt] (42) at (-3,6) {\ydiagram{4,2}};
\node[elt] (411) at (-2,6) {\ydiagram{4,1,1}};
\node[elt] (33) at (-1,6) {\ydiagram{3,3}};
\node[elt] (321) at (0,6) {\ydiagram{3,2,1}};
\node[elt] (222) at (1,6) {\ydiagram{2,2,2}};
\node[elt] (3111) at (2,6) {\ydiagram{3,1,1,1}};
\node[elt] (2211) at (3,6) {\ydiagram{2,2,1,1}};
\node[elt] (21111) at (4,6) {\ydiagram{2,1,1,1,1}};
\node[elt] (111111) at (5,6) {\ydiagram{1,1,1,1,1,1}};
\draw (111111)--(11111);
\draw (21111)--(11111);
\draw (21111)--(2111);
\draw (2211)--(2111);
\draw (2211)--(221);
\draw (222)--(221);
\draw (3111)--(2111);
\draw (3111)--(311);
\draw (321)--(221);
\draw (321)--(311);
\draw (321)--(32);
\draw (411)--(311);
\draw (411)--(41);
\draw (33)--(32);
\draw (42)--(32);
\draw (42)--(41);
\draw (51)--(41);
\draw (51)--(5);
\draw (6)--(5);
\draw (11111)--(1111);
\draw (2111)--(1111);
\draw (2111)--(211);
\draw (221)--(211);
\draw (221)--(22);
\draw (311)--(211);
\draw (311)--(31);
\draw (32)--(22);
\draw (32)--(31);
\draw (41)--(31);
\draw (41)--(4);
\draw (5)--(4);
\draw (1111)--(111);
\draw (211)--(111);
\draw (211)--(21);
\draw (22)--(21);
\draw (31)--(21);
\draw (31)--(3);
\draw (4)--(3);
\draw (111)--(11);
\draw (21)--(11);
\draw (21)--(2);
\draw (3)--(2);
\draw (11)--(1);
\draw (2)--(1);
\draw (1)--(0);
\end{tikzpicture}
\end{center}

\begin{proposition} For $k,n\in\mathbb{N}$, $\binom{[n]}{k}\cong[\lambda_{\emptyset},(n-k)^k]$ as graded posets, where $(n-k)^k=(n-k,\ldots,n-k)\in\mathbb{N}^k$ is the $k\times(n-k)$ young diagram. The partial order on $\binom{[n]}{k}$ is defined by $\mathbf{j},\mathbf{j}'\in\binom{[n]}{k}$, $\mathbf{j}\leq\mathbf{j}'$ if and only if $j_i\leq j_i'$ for all $i\in[k]$ and the grading is given by $\rho(\mathbf{j})=\mathrm{d}(\mathbf{j})$.
\end{proposition}
\begin{proof} Notice that $[\lambda_{\emptyset},(n-k)^k]=\{\lambda\in P:|\lambda|\leq k\text{ and }\lambda_1\leq n-k\}$, so we have already established a bijection $b_{k,n}$ in Proposition~\ref{prop:poset_bijection}, which is easy to see that is also an order preserving map. Moreover, since we also have that the grading agrees, since $\rho(b_{k,n}(\mathbf{j}))=\mathrm{d}(\mathbf{j})=\rho(\mathbf{j})$.
\end{proof}


%TODO: remove $\vec{}$
%TODO: ake span uniform
%TODO: name the stiefel consistently
%TODO: make old proofs smaller and less categorical
%TODO: spellcheck - decide on the spelling of Gr
%TODO: Introduce paragraphs during big proofs?