\chapter{Cohomology computation}
This chapter starts by a direct combination of the Grassmannians discussed in Chapter~1 and the vector bundles discussed in Chapter~2. Namely, the tautological vector bundle is defined over the Grassmannian and additionaly it is proven that this is a universal bundle, in the sense that any other vector bundle over almost any topological space is a subbundle of the tautological bundle. Then, using the universal line bundles we define the Stiefel-Whitney class of any vector bundle to be a sequence of elements in the cohomology of the base space and prove that it is invariant under bundle isomorphisms. At the end of the chapter, we compute the Stiefel-Whitney classes of the tautological vector bundle and prove that it generates the Cohomology of the Grassmannian.

\begin{theorem}\label{thm:projective_spaces_cohomology} There exist the following isomorphisms of graded $\mathbb{Z}_2$-algebras:
\begin{b_item}
\item $H^*(\mathbb{R}P^{\infty};\mathbb{Z}_2)\cong\mathbb{Z}_2[z]$, where $\deg(z)=1$ and
\item $H^*(\mathbb{R}P^n;\mathbb{Z}_2)\cong\mathbb{Z}_2[z_n]/(z_n^{n+1})$, where $\deg(z_n)=1$, for every $n\in\mathbb{N}$.
\end{b_item}
Moreover, for every $n\in\mathbb{N}$, the inclusion $\iota_{1,n+1}:\mathbb{R}P^n\hookrightarrow\mathbb{R}P^{\infty}$ induces an epimorphism $\mathbb{Z}_2[z]\to\mathbb{Z}_2[z_n]/(z_n^{n+1})$ with $\iota_{1,n+1}^*z=z_n$.
\end{theorem}

\section{Tautological vector bundle}
\begin{proposition} Let $k,n\in\mathbb{N}$, with $0<k<n$. Moreover let $\xi:\big\{(H,v)\in\Gr{k}{n}\times\mathbb{R}^n:v\in H\}\to\Gr{k}{n}$ be the function with $\xi(H,v)=H$. Also, for $H_0\in\Gr{k}{n}$ identify $\xi^{-1}(H_0)=\{H_0\}\times H_0$ with $H_0$ as vector spaces. Then $\xi$ is a $k$-plane vector bundle.
\end{proposition}
\begin{proof} We only need to show the local triviality of $\xi$. Let $H_0\in\Gr{k}{n}$ and $U_{H_0}:=\{K\in\Gr{k}{n}:K\cap H_0^{\perp}=\{0\}\}$ as we also defined in the proof of \ref{lem:gr_manifold}. Notice that $\xi^{-1}(U_{H_0})=\{(K,v)\in\Gr{k}{n}\times\mathbb{R}^n:K\cap H_0^{\perp}=\{0\}\text{ and }v\in K\}$. Fix $v_1,\ldots,v_k\in H_0$ to be any basis of $H_0$. We define then the map $\phi:\xi^{-1}(U_{H_0})\to U_{H_0}\times\mathbb{R}^k$ as follows: $\phi(K,v):=(K,v^tv_1,v^tv_2,\ldots,v^tv_k)$.

This map is obviously a continuous bundle map. Also, over each fiber it is linear, since $\phi(K,\lambda u+\mu v)=(K,(\lambda u+\mu v)^tv_1,\ldots,(\lambda u+\mu v)^tv_1)=\lambda(K,u)+\mu(K,u)=\lambda\phi(K,u)+\mu\phi(K,v)$. Notice that since $K\cap H_0^{\perp}=\{0\}$, $v^tv_i=0$ for all $i\in[k]$ if and only if $v=0\in K$, so $\phi(K,v)=0$ if and only if $v=0$. This proves that the kernel of the linear map $\phi_K:=\phi(K,-)$ vanishes and since $\dim_{\mathbb{R}}K=k=\dim_{\mathbb{R}}\mathbb{R}^k$, $\phi_K$ is a linear isomorphism. Thus, Proposition~\ref{prop:local_to_global_iso_vector} gives us that $\phi$ is a vector bundle isomorphism, proving the assertion.
\end{proof}
\begin{remark} Similarly, for $\xi:\{(H,v)\in\Gr{k}\times\mathbb{R}^{\infty}\}\to\Gr{k}$ with $\xi(H,v)=H$, $\xi$ is a $k$-plane vector bundle over $\Gr{k}$. Indeed, for the local triviality we again define $U_{H_0}=\{K\in\Gr{k}:K\cap H_0^{\perp}=\{0\}\}$, since there is a well defined inner product on $\mathbb{R}^{\infty}$, as we saw in Example~\ref{ex:infinite_euclidean_space} and then, the arguments are exactly the same.
\end{remark}

\begin{definition} Let $k,n\in\mathbb{N}$, with $0<k<n$. Then the \emph{tautological} vector bundle on $\Gr{k}{n}$ is the map $\gamma_n^k:\big\{(H,v)\in\Gr{k}{n}\times\mathbb{R}^n:v\in H\big\}$ with $\gamma_n^k(H,v)=H$.
\end{definition}
\begin{remark} This is a generalization of the tautological line bundle over the projective space we defined at Example~\ref{ex:vector_bundles}~\!\ref{ex:tautological_line_bundle}.
\end{remark}

\begin{definition} For $k\in\mathbb{N}$, the \emph{tautological}, or \emph{universal} vector bundle on $\Gr{k}$ is the map $\gamma^k:\big\{(H,v)\in\Gr{k}\times\mathbb{R}^{\infty}:v\in H\big\}$ with $\gamma^k(H,v)=H$.
\end{definition}

\begin{proposition}\label{prop:restriction} For any $k,n\in\mathbb{N}$, with $0<k<n$ it is true that $\gamma_n^k\cong\iota^*_{k,n}\gamma^k$, where $\iota_{k,n}:\Gr{k}{n}\hookrightarrow\Gr{k}$.
\end{proposition}
\begin{proof} Notice that $E(\iota_{k,n}^*\gamma^k)=\big\{(H,H',v)\in\Gr{k}{n}\times\Gr{k}\times\mathbb{R}^{\infty}:v\in H'\text{ and }H=H'\big\}\cong\big\{(H,v)\in\Gr{k}{n}\times\mathbb{R}^{\infty}:v\in H\big\}\cong\big\{(H,v)\in\Gr{k}{n}\times\mathbb{R}^n:v\in H\big\}=E(\gamma_n^k)$, where the homeomorphism is clearly a bundle isomorphism and linear over each fiber.
\end{proof}

\begin{theorem}\label{thm:universal} Let $B$ be any paracompact topological space and $\xi:E(\xi)\to B$ any $k$-plane vector bundle. Then, there exists a continuous function $f:B\to\Gr{k}$ such that $\xi\cong f^*\gamma^k$.
\end{theorem}

\begin{lemma}\label{lem:injection_from_total_space} Let $\xi:E(\xi)\to B$ be a vector bundle and $\hat{f}:E(\xi)\to\mathbb{R}^{\infty}$ be a continuous map such that its restriction $\hat{f}_x:\xi^{-1}(x)\to\mathbb{R}^{\infty}$ is a linear injection for every $x\in B$. Then, for the function $f:B\to\Gr{k}$ with $f(x):=\hat{f}(\xi^{-1}(x))$ it is true that $\xi\cong f^*\gamma^k$. Conversely, given $f$ the function $\hat{f}$ can be uniquely retrieved.
\end{lemma}
\begin{proof} Let $\hat{f}:E(\xi)\to\mathbb{R}^{\infty}$, such that its restriction $\hat{f}_x:\xi^{-1}(x)\to\mathbb{R}^{\infty}$ is a linear injection for every $x\in B$. This means that $\dim_{\mathbb{R}}(\hat{f}(\xi^{-1}(x)))=k$, which lets us define the function $f:B\to\Gr{k}$ such that $f(x)=\hat{f}(\xi^{-1}(x))\in\Gr{k}$. In order to prove that this is a continuous function, we will factor $f$ through $q:V{k}\to\Gr{k}$ locally. For $x\in B$, there exists some open $U\subseteq B$ containing $x$, such that $\xi|U$ is a trivial vector bundle. Equivalently, there exist $k$ nowhere dependent sections $s_1,\ldots,s_k:U\to\xi^{-1}(U)$. Notice that $f|_U(x)=q\big(\hat{f}(s_1(x)),\hat{f}(s_2(x)),\ldots,\hat{f}(s_k(x))\big)$, which is continuous. This means that $f:B\to\Gr{k}$ is continuous.

Next, we define $\bar{f}:E(\xi)\to E(\gamma^k)=\big\{(H,v)\in\Gr{k}\times\mathbb{R}^{\infty}:v\in H\big\}$ to be the continuous function $\bar{f}(v):=\big(f(\xi(v)),\hat{f}(v)\big)$. It is well defined, since $\hat{f}(v)\in\hat{f}(\xi^{-1}(\xi(v)))=f(\xi(v))$. Moreover, since $(f\circ\xi)(v)=(\gamma^k\circ\bar{f})(v)$, it is true that $\bar{f}(\xi^{-1}(x))\subseteq(\gamma^k)^{-1}(f(x))$. Also, since $\hat{f}$ is fiber-wise linear, so is $\bar{f}$ as well. Now, define $\phi:E(\xi)\to E(f^*\gamma^k)=\big\{(x,H,v)\in B\times\Gr{k}\times\mathbb{R}^{\infty}:v\in H\text{ and }f(x)=H\big\}$ by $\phi(v):=(\xi(v),\bar{f}(v))=(\xi(v),f(\xi(v)),\hat{f}(v))$. This is clearly a vector bundle map. For the restriction $\phi_x:\xi^{-1}(x)\to\{x\}\times\{f(x)\}\times f(x)\cong f(x)\subseteq\mathbb{R}^{\infty}$, it is clear that $\phi_x(v)=(x,f(x),\hat{f}(v))$, which is a linear isomorphism, since $\hat{f}$ is injective and $\dim\xi^{-1}(x)=k=\dim f(x)$. Hence, Proposition~\ref{prop:local_to_global_iso_vector} ensures that $\phi$ is a vector bundle isomorphism.

For the converse, given $f:B\to\Gr{k}$ and a vector bundle isomorphism $\phi:E(\xi)\to E(f^*\gamma)$, we can always define the function $\bar{f}:E(f^*\gamma^k)\to E(\gamma_k)$ which completes the pushout square and define $\hat{f}:=\pi_2\circ\bar{f}\circ\phi:E(\xi)\to\mathbb{R}^{\infty}$. It is trivial to check that $\hat{f}$ is a linear injection on each fiber and that $f(x)=\hat{f}(\xi^{-1}(x))$.
\end{proof}

\begin{proof}[Proof of Theorem~\ref{thm:universal}] Because of Lemma~\ref{lem:injection_from_total_space}, it suffices to define a continuous, fiber-wise linear injection map $\hat{f}:E(\xi)\to\mathbb{R}^{\infty}$.

Since $B$ is paracompact, Lemma~\ref{lem:paracompact_countable_cover} gives us a locally finite countable open cover $\{U_i\}_{i\in\mathbb{N}}$ of $B$, such that $\xi|U_i$ is a trivial vector bundle for every $i\in\mathbb{N}$. Let $f_{U_i}:\xi^{-1}(U_i)\to U_i\times\mathbb{R}^k$ be a bundle isomorphism for every $i\in\mathbb{N}$. Also, Theorem~\ref{thm:paracompact_partition_of_unity} lets us find a subcover $\mathcal{V}$ with a subordinate partition of unity $\mathcal{F}=\{u_i\}_{i\in\mathbb{N}}$. Then, $\mathcal{V}$ is also a locally finite countable open cover making $\xi$ trivial, so without loss of generality we will assume that $\mathcal{F}$ is subordinate of $\{U_i\}_{i\in\mathbb{N}}$.

For every $i\in\mathbb{N}$ we define $h_i:\xi^{-1}(U_i)\to\mathbb{R}^k$ by
\begin{center}
\begin{tikzcd}
h_i\ar[r,phantom,":"]&[-2em]\xi^{-1}(U_i)\ar[r,"f_{U_i}","\cong"']&U_i\times\mathbb{R}^k\ar[r,"(u_i\circ\pi_1)\cdot(\pi_2)"]&\mathbb{R}^k\\[-1.5em]
&v\ar[r,mapsto]&(\xi(v),(\pi_2\circ f_{U_i})(v))\ar[r,mapsto]&u_i(\xi(v))\cdot(\pi_2\circ f_{U_i})(v)
\end{tikzcd}
\end{center}
This map is clearly continuous and linear over $\xi^{-1}(x)$ for $x\in U_i$. We extend $h_i$ to the whole $E(\xi)$ by setting $h_i(v)=0$ for every $v\in E(\xi)\setminus\xi^{-1}(U_i)$. We will abuse the notation and use $h_i$ for the extended function as well. Notice that $h_i$ remains continuous and linear over each fiber as a function $h_i:E(\xi)\to\mathbb{R}^k$.

Then, let $\hat{f}:E(\xi)\to\mathbb{R}^{\infty}$ be the function with $\hat{f}(v):=\big(h_1(v),h_2(v),\ldots\big)$ for every $v\in E(\xi)$. This is well defined, since $\{U_i\}_{i\in\mathbb{N}}$ is a locally finite cover of $B$, which means that for every $v\in E(\xi)$ there are only finitely many indices $i\in\mathbb{N}$ such that $h_i(\xi)\neq0$. This means that for every $v\in E(\xi)$ there exists some $n_0\in\mathbb{N}$ with $\hat{f}(v)\in\mathbb{R}^{kn_0}\subseteq\mathbb{R}^{\infty}$. Moreover, $\hat{f}$ is clearly continuous and its restriction $\hat{f}_x:\xi^{-1}(x)\to\mathbb{R}^{\infty}$ is linear.

Next, notice that $\hat{f}_x:\xi^{-1}(x)\to\mathbb{R}^{\infty}$ is injective. Indeed, let $v\in\ker\hat{f}_x$, i.e. $h_1(v)=h_2(v)=\ldots=0\in\mathbb{R}^k$. Since $\mathcal{F}$ is a partition of unity on $B$, there exists some $i_0\in\mathbb{N}$ with $u_{i_0}(\xi(v))>0$. Then $\xi(v)\in U_{i_0}$ and thus $h_{i_0}(v)=0$ means $f_{U_{i_0}}(v)=(\xi(v),0)$. Since the restriction of $f_{U_{i_0}}|:\xi^{-1}(\xi(v))\to\{\xi(v)\}\times\mathbb{R}^k$ is a homeomorphism, $v=0\in\xi^{-1}(\xi(v))$, which proves the assertion.
\end{proof}

\begin{theorem}\label{thm:universal_uniqueness} Let $B$ be any paracompact topological space and $f,g:B\to\Gr{k}$ any two continuous maps, such that $f^*\gamma^k\cong g^*\gamma^k$ as vector bundles. Then $f\simeq g$.
\end{theorem}

\begin{lemma}\label{lem:even_odd} Let $b_0,b_1:\mathbb{R}^{\infty}\to\mathbb{R}^{\infty}$ be the following linear maps
\begin{align*}
b_0(x_1,x_2,x_3,x_4,x_5,\ldots)&=(0,x_1,0,x_2,0,x_3,0,x_4,\ldots)\\
b_1(x_1,x_2,x_3,x_4,x_5,\ldots)&=(x_1,0,x_2,0,x_3,0,x_4,0,\ldots)
\end{align*}
Then, $b_0,b_1$ induce some $d_0,d_1:\Gr{k}\to\Gr{k}$, for which it is true that $d_i^*\gamma_k\cong\gamma$ and $d_i\simeq id_{\Gr{k}}$ for both $i\in\{0,1\}$.
\end{lemma}
\begin{proof} Let us fix an $i\in\{0,1\}$ for the whole proof. First, we define $d_i$ as follows: Let $b_i^k:\St{k}\to\St{k}$ be the induced product map, i.e. let $b_i^k(v_1,\ldots,v_k)=(b_i(v_1),\ldots,b_i(v_k))$. This map is well defined, since $v_1,\ldots,v_k$ are linearly independent if and only if $b_i(v_1),\ldots,b_i(v_k)$ are linearly independent. Moreover, notice that for any two $(v_1,\ldots,v_k),(v_1',\ldots,v_k')\in\St{k}$ with $\mathrm{span}\{v_1,\ldots,v_k\}=\mathrm{span}\{v_1',\ldots,v_k'\}$ we have $\mathrm{span}\{b_i(v_1),\ldots,b_i(v_k)\}=\mathrm{span}\{b_i(v_1'),\ldots,b_i(v_k')\}$, i.e. the map $q\circ b_i^k:\St{k}\to\Gr{k}$ induces a continuous map $d_i:\Gr{k}\to\Gr{k}$.

Next, notice that $E(d_i^*\gamma^k)=\big\{(H_1,H_2,v)\in\Gr{k}\times\Gr{k}\times\mathbb{R}^{\infty}:v\in H_2\text{ and }d_i(H_1)=H_2\big\}\cong\big\{(H,v)\in\Gr{k}\times\mathbb{R}^{\infty}:v\in d_i(H)\big\}$, with $(d_i^*\gamma^k)(H,v)=H$ and $\bar{d}_i(H,v)=(d_i(H),v)$. In order to show that $d_i^*\gamma^k\cong\gamma^k$ we construct the following map $\phi_i:E(\gamma^k)\to E(d_i^*\gamma^k)$ which takes $(H,v)$ to $(H,b_i(v))$. This is well defined, since $b_i(v)\in d_i(H)$. Moreover, it is clearly continuous and linear isomorphism over the fibers. Thus, Proposition~\ref{prop:local_to_global_iso_vector} gives us that $\phi_i$ is a vector bundle isomorphism.

It now remains to define a homotopy $h:\Gr{k}\times[0,1]\to\Gr{k}$, such that $h(H,0)=H$ and $h(H,1)=d_i(H)$. For the maps $d_i,id:\Gr{k}\to\Gr{k}$, we define $\hat{d}_i,\hat{id}:E(\gamma^k)\to \mathbb{R}^{\infty}$ using Lemma~\ref{lem:injection_from_total_space}, i.e. $\hat{id}(H,v)=v$ and
\begin{center}
\begin{tikzcd}
\hat{d}_i\ar[r,phantom,":"]&[-2em]E(\gamma^k)\ar[r,"\phi_i"]&E(d_i^*\gamma^k)\ar[r,"\bar{d}_i"]&E(\gamma^k)\ar[r,"\pi_2"]&\mathbb{R}^{\infty}\\[-1.5em]
&(H,v)\ar[r,mapsto]&(H,b_i(v))\ar[r,mapsto]&(d_i(H),b_i(v))\ar[r,mapsto]&b_i(v)
\end{tikzcd}
\end{center}
Next, let $\hat{h}:E(\gamma^k)\times[0,1]\to\mathbb{R}^{\infty}$ be the homotopy $\hat{h}(v,t)=(1-t)v+tb_i(v)$. Notice that each $\hat{h}_t:E(\gamma^k)\to\mathbb{R}^{\infty}$ is continuous and its restriction $(\hat{h}_t)_H:(\gamma^k)^{-1}(H)\to\mathbb{R}^{\infty}$ is linear. In fact $(\hat{h}_t)_H$ is injective. We already this is the case for $t=0$ and $t=1$ so fix a $t\in(0,1)$ and let $v\in(\gamma^k)^{-1}(H)$ such that $(1-t)v+tb_i(v)=0$, i.e. $b_i(v)=\frac{t-1}{t}v$. Since $v\in\mathbb{R}^{\infty}$, there exists some $n_0$ such that $v_n=0$ for every $n\geq n_0$, where $v_n$ is the $n$-th coordinate of $v$. Let us choose the minimum such $n_0$. This means that $b_i(v)_n=0$ for every $n\geq n_0$. For the case of $b_0$ this means that $v_n=0$ for $n\geq\left\lceil\frac{n_0}{2}\right\rceil$ but if $n_0>1$, then $n_0>\left\lceil\frac{n_0}{2}\right\rceil$ which contradicts the minimality of $n_0$. So we have that $n_0=1$ and $v=0$. For the case of $b_1$ this means that $v_n=0$ for $n\geq\left\lceil\frac{n_0+1}{2}\right\rceil$ but if $n_0>2$, then $n_0>\left\lceil\frac{n_0+1}{2}\right\rceil$ which contradicts the minimality of $n_0$. If $n_0\leq2$, then we have that $v_1=\frac{t-1}{t}v_1$, which again means $n_0=1$ and $v=0$.

Using Lemma~\ref{lem:injection_from_total_space} again, we can thus define continuous $h_t:\Gr{k}\to\Gr{k}$ by $h_t(H):=\hat{h}_t((\gamma^k)^{-1}(H))$ and finally, we define $h:\Gr{k}\times[0,1]\to\Gr{k}$ by $h(H,t)=\hat{h}(\xi^{-1}(H),t)$, which is a homotopy between $d_i$ and $id$.
\end{proof}

\begin{proof}[Proof of Theorem~\ref{thm:universal_uniqueness}] We will use the maps $d_0,d_1:\Gr{k}\to\Gr{k}$ defined in Lemma~\ref{lem:even_odd}. Since $d_i\simeq id$, it is true that $f\simeq d_0\circ f$ and $g\simeq d_1\circ g$. Moreover, since $d_i^*\gamma^k\cong\gamma^k$, it is also true that $(d_i\circ f)^*\gamma^k=f^*\gamma^k$ and similarly $(d_i\circ g)^*\gamma^k=g^*\gamma^k$. So, without loss of generality we will suppose that $f(x)\subseteq\{(x_1,x_2,\ldots)\in\mathbb{R}^{\infty}:\forall n\in\mathbb{N}\ x_{2n}=0\}$ and $g(x)\subseteq\{(x_1,x_2,\ldots)\in\mathbb{R}^{\infty}:\forall n\in\mathbb{B}\ x_{2n-1}=0\}$. Let $\xi\cong f^*\gamma^k\cong g^*\gamma^k$ and $\hat{f},\hat{g}:E(\xi)\to\mathbb{R}^{\infty}$ be the associated maps of $f,g:B\to\Gr{k}$, as defined in Lemma~\ref{lem:injection_from_total_space}.

Like in the proof of the Lemma~\ref{lem:even_odd}, we first define the homotopy $\hat{h}:E(\xi)\times[0,1]\to\mathbb{R}^{\infty}$ with $\hat{h}_t(v)=(1-t)\hat{f}+t\hat{g}$. This makes $\hat{h}$ continuous and its restriction $(\hat{h}_t)_x:\xi^{-1}(x)\to\mathbb{R}^{\infty}$ linear. In fact, $\hat{h}_t$ is a linear injection. Indeed, fix some $t\in[0,1]$ and let $v\in E(\xi)$ such that $\hat{h}_t(v)=(1-t)\hat{f}(v)+t\hat{g}(v)=0$. But, $\hat{h}_t(v)_n=(1-t)\hat{f}(v)_n+t\hat{g}(v)_n$ which equals $(1-t)\hat{f}(v)_n$ if $n$ is odd and $t\hat{g}(v)_n$ if $n$ is even. This means that $\hat{f}(v)=\hat{g}(v)=0$, i.e. $v=0$ since both $\hat{f}$ and $\hat{g}$ are injective.

Hence, using Lemma~\ref{lem:injection_from_total_space} again, we can now define a continuous $h_t:B\to\Gr{k}$ by $h_t(x)=\hat{h}_t(\xi^{-1}(x))$ and thus $h:B\times[0,1]\to\Gr{k}$ with $h(x,t)=\hat{h}(\xi^{-1}(x),t)$ is a homotopy between $f$ and $g$.
\end{proof}

\section{Stiefel-Whitney classes}
\begin{axioms} Let $B$ be a topological space and $\xi:E(\xi)\to B$ any $k$-plane vector bundle. Then, an element
\[w(\xi)=w_0(\xi)+w_1(\xi)+w_2(\xi)+\cdots\in H^{\prod}(B;\mathbb{Z}_2)\]
where $w_i(\xi)\in H^i(B;\mathbb{Z}_2)$ for every $i\in\mathbb{N}_0$ \emph{satisfies the Stiefel-Whitney Axioms} if it satisfies the following four Axioms:
\begin{b_item}
\item[(SW1)] $w_0(\xi)=1$ and $w_{k+1}(\xi)=w_{k+2}(\xi)=\cdots=0$, for every $k$-plane $\xi:E(\xi)\to B$.
\item[(SW2)] $w(f^*\xi)=f^*w(\xi)$, for every $f:B'\to B$ and every $\xi:E(\xi)\to B$ and\\[0.2em]
$w(\xi)=w(\eta)$, for every $\xi\cong\eta$.
\item[(SW3)] $w(\eta\oplus\xi)=w(\eta)w(\xi)$, for every $\eta:E(\eta)\to B$ and every $\xi:E(\xi)\to B$.
\item[(SW4)] $w(\gamma_1^1)\neq0$.
\end{b_item}
\end{axioms}

For our discussion about Stiefel-Whitney classes, we are heavily relying on ~\cite{husemoller}, Chapter~17, Sections~2-6.

\subsection{Basic properties of Stiefel Whitney classes}
In this subsection, we assume that there exists some $w(\xi)$ satisfying the Stiefel-Whitney axioms and examine the nice properties that it also satisfies.
\begin{proposition}\label{prop:trivial_sw} For all $k\in\mathbb{N}$, $w(\varepsilon_B^k)=1$, i.e. $w_i(\varepsilon_B^k)=0$ for every $i>0$.
\end{proposition}
\begin{proof} Let $c_B:B\to\{*\}$ be the map to the one-point set. Then $\varepsilon_B^n\cong c_B^*\varepsilon_{\{*\}}^n$. Indeed, $E(c_B^*\varepsilon_{\{*\}}^n)\cong B\times\mathbb{R}^n=E(\varepsilon_B^n)$. Let $i>0$, then using (SW1) we then get $w_i(\varepsilon_B^n)=c_B^*w_i(\varepsilon_{\{*\}}^n)=0$, since $w_i(\varepsilon_{\{*\}}^n)\in H^i(\{*\},\mathbb{Z}_2)=0$.
\end{proof}

\begin{proposition} Let $\xi:E(\xi)\to B$ and $n\in\mathbb{N}$, then $w(\xi\oplus\varepsilon_B^n)=w(\xi)$.
\end{proposition}
\begin{proof} Using (SW2) and Proposition~\ref{prop:trivial_sw} we get $w(\xi\oplus\varepsilon_B^n)=w(\xi)w(\varepsilon_B^n)=w(\xi)$.
\end{proof}

\begin{corollary} Let $\eta:E(\eta)\to B$, $\xi:E(\xi)\to B$ two vector bundles such that $\eta\oplus\varepsilon_B^k\cong\xi\oplus\varepsilon_B^n$ for some $k,n\in\mathbb{N}$. Then $w(\eta)=w(\eta\oplus\varepsilon_B^k)=w(\xi\oplus\varepsilon_B^n)=w(\xi)$.
\end{corollary}

\begin{proposition} Let $\xi:E(\xi)\to B$ be an $n$-plane euclidean vector bundle and $s_1,\ldots,s_k:B\to E(\xi)$ $k$ nowhere linearly dependent sections. Then $w_{n-k+1}(\xi)=w_{n-k+2}=\cdots=w_n(\xi)=0$.
\end{proposition}
\begin{proof} Given $k$ nowhere linearly dependent sections of $\xi$, we can define a $k$-plane trivial vector bundle $\eta$ with $\eta\leq\xi$. Indeed, let $E(\eta)=\big\{v\in E(\xi):v\in\mathrm{span}\{s_1(\xi(v)),\ldots,s_k(\xi(v))\}\subseteq E(\xi)$ and $\eta(v)=\xi(v)$. Then $\phi:E(\eta)\to E(\varepsilon_B^k)$ takes $v=a_1s_1(\xi(v))+\cdots+a_ks_k(\xi(v))$ to $(\xi(v),a_1,\ldots,a_k)$. Since $\xi$ is euclidean, Lemma~\ref{prop:vb_decomposition} gives us that $\xi\cong\eta\oplus\eta^{\perp_{\xi}}$. Thus, (SW2) and Proposition~\ref{prop:trivial_sw} give: $w(\xi)=w(\eta)w(\eta^{\perp_{\xi}})=w(\eta^{\perp_{\xi}})$ and (SW0) gives $w_i(\eta^{\perp_{\xi}})=0$ for every $i>n-k$, since $\eta^{\perp_{\xi}}$ is an $(n-k)$-plane vector bundle.
\end{proof}

\begin{lemma} For any topological space $B$, the space $\big\{1+a_1+a_2+\cdots\in H^{\prod}(B;\mathbb{Z}_2)\big\}$ is a group with respect to the cup product.
\end{lemma}
\begin{proof} We only have to check that every $1+a_1+a_2+\cdots\in H^{\prod}(B;\mathbb{Z}_2)$ is invertible. We can construct the inverse element $1+\bar{a}_1+\bar{a}_2+\cdots$ inductively by:
\[\bar{a}_n=-a_n-\sum_{i=k}^{n-1}a_k\bar{a}_{n-k}\qedhere\]
\end{proof}

\begin{definition} Let $\xi:E(\xi)\to B$ be a $k$-plane vector bundle. Then, we define the \emph{dual of the stiefel whitney classes of $\xi$} to be $\bar{w}(\xi):=(\bar{w}(\xi))^{-1}\in H^{\prod}(B;\mathbb{Z}_2)$.
\end{definition}

\begin{proposition} Let $M$ be a smooth manifold embedded in $\mathbb{R}^d$, $\tau_M:TM\to M$ be its tangent bundle and $\nu_{M,d}:N_dM\to M$ its normal bundle. Then $\bar{w}(\tau_M)=w(\nu_{M,d})$
\end{proposition}
\begin{proof} Proposition~\ref{prop:tangent_normal_vb} gives us that $\nu_{M,d}=\tau_M^{\perp_{\mathbb{R}^d}}$ and thus, because of Proposition~\ref{prop:vb_decomposition}, we have that $\varepsilon_M^d\cong\tau_M\oplus\nu_{M,d}$. Using (SW2) we get $w(\tau_M)w(\nu_{M,d})=1$, i.e. $\bar{w}(\tau_M)=w(\nu_{M,d})$.
\end{proof}

\subsection{Definition of Stiefel Whitney classes}
In this subsection we define the Stiefel-Whitney classes for every $k$-plane bundle over a paracompact space.
\begin{definition} A fiber bundle $p:E\to B$ is \emph{of finite type}, if there exists a finite open cover $\{U_i\}_{i\in[n]}$ such that $p|U_i$ is the trivial fiber bundle.
\end{definition}
\begin{theorem}[Leray-Hirsch]\label{thm:leray_hirsch} Let $B$ be a topological space and $p:E(p)\to B$ be a fiber bundle of finite type with fiber $F$. For each $x\in B$ fix some homeomorphism $j_x:F\to p^{-1}(x)$. Let $a_1\in H^{n_1}(E;\mathbb{Z}_2),\ldots,a_r\in H^{n_r}(E;\mathbb{Z}_2)$ such that for every $x\in B$
\begin{b_item}
\item the elements $j_x^*a_1,\ldots,j_x^*a_r$ are $\mathbb{Z}_2$-linearly independent in $H^*(F;\mathbb{Z}_2)$ and
\item $\mathbb{Z}_2\left<j_x^*a_1,\ldots,j_x^*a_r\right>\cong H^*(F;\mathbb{Z}_2)$ as $\mathbb{Z}_2$-modules.
\end{b_item}
Then,
\begin{b_item}
\item the elements $a_1,\ldots,a_r$ are $H^*(B;\mathbb{Z}_2)$-linearly independent in $H^*(E;\mathbb{Z}_2)$ and
\item $H^*(B;\mathbb{Z}_2)\left<a_1,\ldots,a_r\right>\cong H^*(E;\mathbb{Z}_2)$ as $H^*(B;\mathbb{Z}_2)$-modules,
\end{b_item}
where the $H^*(B;\mathbb{Z}_2)$-module structure on $H^*(E;\mathbb{Z}_2)$ is defined by $p^*:H^*(B;\mathbb{Z}_2)\to H^*(E;\mathbb{Z}_2)$. This means that the isomorphism is given by
\begin{center}
\begin{tikzcd}
H^*(B;\mathbb{Z}_2)\left<a_1,\ldots,a_r\right>\ar[r]&H^*(E;\mathbb{Z}_2)\\[-1.5em]
\displaystyle\sum_{i=0}^{k-1}x_i\cdot a_{\xi}^i\ar[r,mapsto]&p^*(x_i)a_{\xi}^i
\end{tikzcd}
\end{center}
In particular, $p^*$ is a monomorphism.
\end{theorem}
\begin{proof}
%TODO
\end{proof}

\begin{proposition}\label{prop:proj_bundle} Let $\xi:E(\xi)\to B$ be a $k$-plane vector bundle. Moreover, let $E_0:=\{v\in E:v\neq 0\}$ and $q_E:E_0\to\sfrac{E_0}{\sim}$, where $v\sim v'$ if and only if $\xi(v)=\xi(v')$ and $v=\lambda v'$ for some $\lambda\in\mathbb{R}\setminus\{0\}$. Also, let $p:E':=\sfrac{E_0}{\sim}\to B$ be the function $p([v])=\xi(v)$. Then $p$ is a fiber bundle with fiber $\mathbb{R}P^{k-1}$. Moreover, if for some $x\in B$ we fix a linear isomorphism $f_x:\xi^{-1}(x)\to\mathbb{R}^k$, then there exists an induced homeomorphism $g_x:p^{-1}(x)\to\mathbb{R}P^{k-1}$ such that $q\circ f_x(v)=g_x([v])$ for every $v\in E_0\cap\xi^{-1}(b)$, where $q:\mathbb{R}^k\to\mathbb{R}P^{k-1}$ is the usual quotient map.
\end{proposition}
\begin{proof} First of all $p$ is well defined and continuous, since $v\sim v'$ means in particular that $\xi(v)=\xi(v')$. So, it only remains to show that $p$ is locally trivial. Let $x\in B$, then since $\xi$ is locally trivial, there exists some open $U\subseteq B$ containing $x$ and a vector bundle isomorphism $f_U:\xi^{-1}(U)\to U\times\mathbb{R}^k$. Then, for the function $(id\times q)\circ f_U:E_0\cap\xi^{-1}(U)\to U\times\mathbb{R}P^{k-1}$ and for any $v\sim v'$ it is true that $((id\times q)\circ f_U)(v)=((id\times q)\circ f_U)(v')$, since $f_U$ is linear on each fiber and $q(a)=q(\lambda a)$. So there exists a unique continuous function $g_U:p^{-1}(U)\to U\times\mathbb{R}P^{k-1}$ such that $g_U([v])=((id\times q)\circ f_U)(v)$. The map $g_U$ is a homeomorphism, since its inverse is: $g_U^{-1}(x,[a]):=[f_U^{-1}(x,a)]$. Indeed, this is well defined and continuous, since for $q(a)=q(a')$ we have that $a=\lambda a'$ for some $\lambda\in\mathbb{R}\setminus\{0\}$ and $f_U^{-1}$ is linear over each fiber.

Let us fix some $x\in B$ and a linear isomorphism $f_x:\mathbb{R}^n\to\xi^{-1}(x)$. Then, using the same arguments as for $f_U$, the function $g_x:p^{-1}(x)\to\mathbb{R}P^{k-1}$ with $g_x([v]):=q\circ f_x(v)$ is well defined and continuous. To prove that this is a homeomorphism, we notice that its inverse $g_x^{-1}([a])=[f_x^{-1}(a)]$ is also continuous and well defined.
\end{proof}

\begin{definition} Let $\xi:E(\xi)\to B$ be a $k$-plane vector bundle. Then, the \emph{projective bundle associated with $\xi$} is the fiber bundle $P\xi:E(P\xi)\to B$ with fiber $\mathbb{R}P^{k-1}$, where $E(P\xi)=\sfrac{\{v\in E:v\neq0\}}{\sim}$, $v\sim v'$ if and only if $\xi(v)=\xi(v')$ and $v=\lambda v'$ for some $\lambda\in\mathbb{R}\setminus\{0\}$ and $(P\xi)([v])=\xi(v)$.
\end{definition}

\begin{definition}\label{def:lambda_xi} For any $k$-plane vector bundle $\xi:E(\xi)\to B$, we define the line bundle $\lambda_{\xi}:E(\lambda_\xi)\to E(P\xi)$ as follows. First, notice that we can write $E(P\xi)=\big\{L\subseteq E(\xi):\exists v\in E(\xi)\text{ such that }L=\mathrm{span}(v)\big\}$. Then, we construct the induced vector bundle $(P\xi)^*\xi:E((P\xi)^*\xi)\to E(P\xi)$. Recall that
\[E((P\xi)^*\xi)=\big\{(L,v)\in E(P\xi)\times E(\xi):(P\xi)(L)=\xi(v)\big\}\]
and define $\lambda_{\xi}\leq(P\xi)^*\xi$ with $E(\lambda_{\xi}):=\big\{(L,v)\in E(P\xi)\times E(\xi):(P\xi)(L)=\xi(v)\text{ and }v\in L\big\}\subseteq E((P\xi)^*\xi)$ and $\lambda_{\xi}(L,v)=\xi(v)$.
\begin{center}
\begin{tikzcd}
E(\lambda_{\xi})\ar[dr,"\lambda_{\xi}"']\ar[r,"inc",hook]&E((P\xi)^*\xi)\ar[d,"(P\xi)^*\xi"']\ar[r,"\pi_2"]\ar[dr,phantom,"\lrcorner",near start]&E(\xi)\ar[d,"\xi"]\\[1.7em]
&E(P\xi)\ar[r,"P\xi"]&B
\end{tikzcd}
\end{center}
\end{definition}
\begin{remark}\label{rem:lambda_for_line_bundle} If $\xi:E(\xi)\to B$ is a line bundle, then $\lambda_\xi\cong\xi$.
\end{remark}
\begin{proof} First, notice that for every $x\in B$ and $v_1,v_2\in\xi^{-1}(x)$ with $v_1,v_2\neq 0$ it is true that $[v_1]=[v_2]$. So $P\xi:E(\xi)\to \{*\}\times B$ is a bundle isomorphism and thus $(P\xi)^*\xi\cong(id_B)^*\xi\cong\xi$. Then, since $\lambda_{\xi}$ is also a line bundle with $\lambda_{\xi}\leq\xi$, $\lambda_{\xi}\cong\xi$.
\end{proof}

\begin{lemma}\label{lem:projective_paracompact} Let $B$ be a paracompact space and $\xi:E(\xi)\to B$ a vector bundle. Then $E(P\xi)$ is also paracompact.
\end{lemma}
\begin{proof} Since $B$ is paracompact, Proposition~\ref{lem:paracompact_countable_cover} gives us a locally finite, countable open cover $\mathcal{U}=\{U_i\}_{i\in\mathbb{N}}$ of $B$, such that $\xi|U_i$ is a trivial vector bundle. Let $f_{U_i}:\xi^{-1}(U_i)\to U_i\times\mathbb{R}^k$. Next, we use \ref{thm:paracompact_partition_of_unity} to find a subordinate partition of unity $\mathcal{F}=\{u_i\}_{i\in\mathbb{N}}$ of an open subcover of $\mathcal{U}$. Without loss of generality, we suppose that $\mathcal{F}$ is directly subordinate of $\mathcal{U}$, since it is already locally finite. In particular, if we let $V_i:=\mathrm{supp}(u_i)$, then $\mathcal{V}$ is a closed locally finite countable cover of $B$ such that $\xi|V_i$ is a trivial vector bundle and $\{V_i^{\circ}\}_{i\in\mathbb{N}}$ also covers $B$. This means that $\xi^{-1}(V_i)\cong V_i\times\mathbb{R}^k$. As we proved in Proposition~\ref{prop:proj_bundle}, this gives us a homeomorphism $p^{-1}(V_i)\cong V_i\times\mathbb{R}P^{k-1}$. Proposition~\ref{prop:paracompact_closed_subset} gives us that $V_i$ is paracompact and then, Proposition~\ref{prop:paracompact_times_compact} gives us that $V_i\times\mathbb{R}P^{k-1}\cong p^{-1}(V_i)$ is paracompact. If we write $E(P\xi)=\bigcup_{i\in\mathbb{N}}p^{-1}(V_i)$ and notice that $\{p^{-1}(V_i)^{\circ}\}_{i\in\mathbb{N}}$ also covers $E(P\xi)$, then we get that $E(P\xi)$ is paracompact.%TODO write this better
\end{proof}

\begin{remark} Let $B$ be paracompact, $\xi:E(\xi)\to B$ be any $k$-plane vector bundle and $\lambda_{\xi}:E(\lambda_{\xi})\to E(P\xi)$ the line bundle with $E(\lambda_{\xi})=\big\{(L,v)\in E(P\xi)\times E(\xi):(P\xi)(L)=\xi(v)\text{ and }v\in L\big\}$ as defined above. Since $B$ is paracompact, Lemma~\ref{lem:projective_paracompact} gives us that $E(P\xi)$ is paracompact as well. Then, using Theorem~\ref{thm:universal} we find a map $u_{\xi}:E(P\xi)\to\Gr{1}\cong\mathbb{R}P^{\infty}$ such that $\lambda_{\xi}\cong u_{\xi}^*\gamma^1$.
\begin{center}
\begin{tikzcd}
E(\lambda_{\xi})\ar[d,"\lambda_{\xi}\cong u_{\xi}^*\gamma^1"']\ar[r,"\bar{f}_{\xi}"]\ar[dr,phantom,"\lrcorner",near start]&E(\gamma^1)\ar[d,"\gamma^1"]\\[1.7em]
E(P\xi)\ar[r,"u_{\xi}"]&\mathbb{R}P^{\infty}
\end{tikzcd}
\end{center}
Define now $a_{\xi}:=u_{\xi}^*(z)\in H^*(E(P\xi);\mathbb{Z}_2)$, where $z$ is the generator of $H^*(\mathbb{RP}^{\infty};\mathbb{Z}_2)$, as defined in Theorem~\ref{thm:projective_spaces_cohomology}. Then Theorem~\ref{thm:universal_uniqueness} gives us that $u_{\xi}$ is unique, up to homotopy and thus $a_{\xi}\in H^*(E(P\xi);\mathbb{Z}_2)$ is well defined.
\end{remark}

\begin{proposition}\label{prop:basis_of_EP} Let $B$ be paracompact and $\xi:E(\xi)\to B$ any $k$-plane vector bundle and also let $a_{\xi}:=u_{\xi}^*z\in H^*(E(P\xi);\mathbb{Z}_2)$ as defined in the previous remark. Then the elements $1,a_{\xi},a_{\xi}^2,\ldots,a_{\xi}^{k-1}$ are a basis of the $H^*(B;\mathbb{Z}_2)$-module $H^*(E(P\xi);\mathbb{Z}_2)$, where the $H^*(B;\mathbb{Z}_2)$-module structure on $H^*(E(P\xi);\mathbb{Z}_2)$ is defined by the function $(P\xi)^*$. In particular, $(P\xi)^*$ is a monomorphism.
\end{proposition}
\begin{proof} The strategy is to use Leray-Hirsch Theorem (\ref{thm:leray_hirsch}) for the fiber bundle $P\xi:E(P\xi)\to B$, which has fiber $F=\mathbb{R}P^{k-1}$. Since $B$ is paracompact, so is $E(P\xi)$ as well, as we saw in Lemma~\ref{lem:projective_paracompact}. So, we need to construct for every $x\in B$ a homeomorphism $j_x:\mathbb{R}P^{k-1}\to(P\xi)^{-1}(x)$, such that $\{j_x^*(1),j_x^*(a_{\xi}),\ldots,j_x^*(a_{\xi}^{k-1})\}$ is a $\mathbb{Z}_2$-basis of $H^*(\mathbb{R}^{k-1};\mathbb{Z}_2)$.

Let $x\in B$. We fix a linear isomorphism $f_x:\xi^{-1}(x)\to\mathbb{R}^k$ and then we define the homeomorphism $j_x([a]):=g_x^{-1}([a])=[f_x^{-1}(a)]$, as in Proposition~\ref{prop:proj_bundle}. We examine now the line bundle $j_x^*u_{\xi}^*\gamma^1=j_x^*\lambda_{\xi}:E(j_x^*\lambda_{\xi})\to\mathbb{R}P^{k-1}$. Notice that $j_x(\mathbb{R}P^{k-1})=(P\xi)^{-1}(x)\subseteq E(P\xi)$ and thus $j_x^*\lambda_{\xi}=j_x^*(inc^*(\lambda_{\xi}))$:
\begin{center}
\begin{tikzcd}
E(j_x^*\lambda_{\xi})\ar[d,"j_x^*\lambda_{\xi}"']\ar[r,"\bar{j}_x"]\ar[dr,phantom,"\lrcorner",near start]&E(inc^*\lambda_{\xi})\ar[d,"inc^*\lambda_{\xi}"']\ar[r,"inc",hook]\ar[dr,phantom,"\lrcorner",near start]&E(\lambda_{\xi})\ar[d,"\lambda_{\xi}"']\ar[r,"\bar{u}_{
\xi}"]\ar[dr,phantom,"\lrcorner",near start]&E(\gamma^1)\ar[d,"\gamma^1"]\\[1.7em]
\mathbb{R}P^{k-1}\ar[r,"j_x"]&(P\xi)^{-1}(x)\ar[r,"inc",hook]&E(P\xi)\ar[r,"u_{\xi}"]&\mathbb{R}P^{\infty}
\end{tikzcd}
\end{center}
Notice that we can write $E(inc^*\lambda_{\xi})=\big\{(L,v)\in (P\xi)^{-1}\times\xi^{-1}:v\in L\big\}$ and thus $E(j^*\lambda_{\xi})=\big\{([a],L,v)\in\mathbb{R}P^{k-1}\times(P\xi)^{-1}(x)\times\xi^{-1}(x):v\in L\text{ and }j_x([a])=L\big\}\cong\big\{([a],v)\in\mathbb{R}P^{k-1}\times\xi^{-1}(x):v\in j_x([a])\big\}\cong\big\{([a],b)\in\mathbb{R}P^{k-1}\times\mathbb{R}^k:f_x^{-1}(b)\in j_x([a])\big\}$. But $f_x^{-1}(b)\in j_x([a])$ if and only if $[f_x^{-1}(b)]=j_x([a])$ and since $j_x$ is induced by $f_x$, this is equivallent to $j_x([b])=j_x([a])$, i.e. $[b]=[a]$ since $j_x$ is a homemorphism which is equivalent to $b\in[a]$. This proves that $j_x^*\lambda_{\xi}\cong\gamma_k^1$, i.e. $(u_{\xi}\circ j_x)^*(\gamma^1)\cong\gamma_k^1=\iota_{1,k}^*(\gamma^1)$, because of Proposition~\ref{prop:restriction}. So, using Theorem~\ref{thm:universal_uniqueness} $u_{\xi}\circ j_x\simeq\iota_{1,k}$ and thus $j_x^*u_{\xi}^*=\iota_{1,k}^*:H^*(\mathbb{R}P^{\infty};\mathbb{Z}_2)\to H^*(\mathbb{R}P^{k-1};\mathbb{Z}_2)$. Thus, due to the definition of $a_{\xi}$ and Theorem~\ref{thm:projective_spaces_cohomology}, we get $j_x^*(a_{\xi})=j_x^*(u_{\xi}^*(z))=\iota_{1,k}(z)=z_{k-1}\in H^1(\mathbb{R}P^{k-1};\mathbb{Z}_2)$ and hence $j_x^*(a_{\xi}^i)=j_x^*(a_{\xi})^i=z_{k-1}^i$. This means that $\{j_x^*(1),j_x^*(a_{\xi}),j_x^*(a_{\xi}^2),\ldots,j_x^*(a_{\xi}^{k-1})\}=\{1,z_{k-1},z_{k-1}^2,\ldots,z_{k-1}^{k-1}\}$ is a $\mathbb{Z}_2$-linear independent set that generates the $\mathbb{Z}_2$-module $H^*(\mathbb{R}P^{k-1})$.

Leray-Hirsch theorem gives us now that the set $\{1,a_{\xi},\ldots,a_{\xi}^{k-1}\}$ is a $H^*(B;\mathbb{Z}_2)$-linearly independent set in $H^*(E(P\xi);\mathbb{Z}_2)$ and that
\[H^*(B;\mathbb{Z}_2)\left<1,a_{\xi},a_{\xi}^2,\ldots,a_{\xi}^{k-1}\right>\cong H^*(E(P\xi);\mathbb{Z}_2)\]
as $H^*(B;\mathbb{Z}_2)$-modules, where the isomorphism is given by:
\[\sum_{i=0}^{k-1}x_i\cdot a_{\xi}^i\ \longmapsto\ \sum_{i=0}^{k-1}(P\xi)^*(x_i)a_{\xi}^i\qedhere\]
\end{proof}

\begin{definition}\label{def:SW} Let $B$ be paracompact and $\xi:E(\xi)\to B$ any $k$-plane vector bundle and also let $a_{\xi}:=u_{\xi}^*z\in H^*(E(P\xi);\mathbb{Z}_2)$ as before. Then, because of Proposition~\ref{prop:basis_of_EP}, for each $i\in[k]$ there exists a $w_i(\xi)\in H^i(B;\mathbb{Z}_2)$, such that
\[a_{\xi}^k=\sum_{i=0}^{k-1}\Big((P\xi)^*\big(w_{k-i}(\xi)\big)\Big)a_{\xi}^i\]
Then, $w_i(\xi)$ is called \emph{the $i$-th Stiefel-Whitney class of $\xi$} and $w(\xi):=1+w_1(\xi)+\ldots+w_k(\xi)$ is called the \emph{total Stiefel-Whitney class of $\xi$}.
\end{definition}

\subsection{Existence of Stiefel Whitney classes}
In this subsection we prove that the Stiefel-Whitney classes defined previously satisfy the Stiefel-Whitney axioms.
\begin{proposition} Let $B$ be paracompact and $\xi:E(\xi)\to B$ be any $k$-plane vector bundle. Then, $w_0(\xi)=1$ and $w_{k+1}(\xi)=w_{k+2}(\xi)=\cdots=0$.
\end{proposition}
\begin{proof} This is immediate from the definition.
\end{proof}

\begin{proposition} Let $B',B$ be two paracompact spaces, $f:B'\to B$ be any continuous function and $\xi:E(\xi)\to B$ be any $k$-plane vector bundle. Then, $w(f^*\xi)=f^*w(\xi)$.
\end{proposition}
\begin{proof} Write $E(f^*\xi)=\big\{(x,v)\in B'\times E(\xi):f(x)=\xi(v)\big\}$ and define $\bar{f}:E(f^*\xi)\to E(\xi)$ with $\bar{f}(x,v):=v$, completing the pullback square. Then, define $g:E(P(f^*\xi))\to E(P\xi)$ with $g([(x,v)]):=[\bar{f}(x,v)]=[v]$. This is continuous and well defined. Indeed, let $(x_1,v_1),(x_2,v_2)\in E(f^*\xi)$ with $v_1,v_2\neq 0$ and $[(x_1,v_1)]=[(x_2,v_2)]$ in $E(P(f^*\xi))$. Then $x_1=x_2$, i.e. $\xi(v_1)=f(x_1)=f(x_2)=\xi(v_2)$ and $v_1=\lambda v_2$ for some $\lambda\in\mathbb{R}\setminus\{0\}$. Thus, $[v_1]=[v_2]$ in $E(P\xi)$. So, we have the following two commutative diagrams:
\begin{center}
\begin{tikzcd}
&[-7em]&[-3em]&[-10em]\faktor{\big\{(x,v)\in B'\times E(\xi):f(b)=\xi(v)\text{ and }v\neq0\big\}}{\left\{\substack{(x_1,v_1)\sim(x_2,v_2)\text{ iff }\\[0.2em]x_1=x_2\text{ and }v_1=\lambda v_2}\right\}}\ar[dd,phantom,"\cong"{yshift=1em,rotate=90}]&[-10em]\\[-3.2em]
&\big\{(x,v)\in B'\times E(\xi):f(b)=\xi(v)\big\}\ar[d,phantom,"\cong"rotate=90]&\\[-1.2em]
&E(f^*\xi)\ar[r,"\bar{f}"]\ar[d,"f^*\xi"']&E(\xi)\ar[d,"\xi"]&[4em]E(P(f^*\xi))\ar[r,"g"]\ar[d,"P(f^*\xi)"']&E(P\xi)\ar[d,"P\xi"]\\[1.3em]
&B'\ar[r,"f"]&B&B'\ar[r,"f"]&B
\end{tikzcd}
\end{center}
Next, we will prove that $\lambda_{f^*\xi}\cong g^*\lambda_{\xi}$, i.e. we will define a map $\phi:E(\lambda_{f^*\xi})\to E(g^*\lambda_{\xi})$, where
\begin{align*}
E(\lambda_{f^*\xi})&=\big\{(L',v')\in E(P(f^*\xi))\times E(f^*\xi):(P(f^*\xi))(L')=f^*\xi(v')\text{ and }v'\in L'\big\},\\[0.5em]
E(g^*\lambda_{\xi})&=\big\{(L',L,v)\in E(P(f^*\xi))\times E(P\xi)\times E(\xi):(P\xi)(L)=\xi(v)\text{ and }v\in L\text{ and }g(L')=L\big\}\\
&\cong\big\{(L',v)\in E(P(f^*\xi))\times E(\xi):(P\xi)(g(L'))=\xi(v)\text{ and }v\in g(L')\big\}.
\end{align*}
Let $\phi(L',v'):=(L',\bar{f}(v'))$. Then, $\phi$ is well defined. Indeed, for $(L',v')\in E(\lambda_{f^*\xi})$ we first have $(P\xi)(g(L'))=f((P(f^*\xi))(L'))=f(f^*\xi(v'))=\xi(\bar{f}(v'))$ and if we write $v'=(x,a)$ and $L'=[(y,b)]$, then we also have that $v'\in L'$ means $[(x,a)]=[(y,b)]$ which gives that $[a]=[b]$. It is clear that $\phi$ is a fiber bundle map, which is a linear isomorphism fiber-wise, since $\bar{f}$ is a linear isomorphism fiber-wise. So, using \ref{prop:local_to_global_iso_vector}, $\phi$ is a vector space isomorphism. Hence, the following diagram is commutative and both squares are pullback squares:
\begin{center}
\begin{tikzcd}
E(\lambda_{f^*\xi})\ar[r]\ar[d,"\lambda_{f^*\xi}=g^*\lambda_{\xi}"']\ar[dr,phantom,"\lrcorner",near start]&[2em]E(\lambda_{\xi})\ar[r]\ar[d,"\lambda_{\xi}"']\ar[dr,phantom,"\lrcorner",near start]&[2em]E(\gamma^1)\ar[d,"\gamma^1"]\\[1.3em]
E(P(f^*\xi))\ar[r,"g"]\ar[rr,dotted,bend right=10,"u_{f^*\xi}"']&E(P\xi)\ar[r,"u_{\xi}"]&\mathbb{R}P^{\infty}
\end{tikzcd}
\end{center}
So, it is true that $\lambda_{f^*\xi}=g^*\lambda_{\xi}=g^*u_{\xi}^*\gamma^1=(u_{\xi}\circ g)^*\gamma^1$, but we also have that $\lambda_{f^*\xi}=u^*_{f^*\xi}\gamma^1$, so Theorem~\ref{thm:universal_uniqueness} gives us that $u_{\xi}\circ g\simeq u_{f^*\xi}$, which means that $g^*u^*_{\xi}=u^*_{f^*\xi}:H^*(\mathbb{R}P;\mathbb{Z}_2)\to H^*(E(P(f^*\xi));\mathbb{Z}_2)$. This gives: $a_{f^*\xi}=u_{f^*\xi}z=g^*u^*_{\xi}z=g^*a_{\xi}$ and thus:
\[g^*a^k_{\xi}=a^k_{f^*\xi}=\sum_{i=0}^{k-1}\Big((P(f^*\xi))^*w_{k-i}(f^*\xi)\Big)a_{f^*\xi}^i=\sum_{i=0}^{k-1}\Big((P(f^*\xi))^*w_{k-i}(f^*\xi)\Big)g^*a_{\xi}^i\]
On the other hand:
\[g^*a^k_{\xi}=g^*\left(\sum_{i=0}^{k-1}\Big((P\xi)^*w_{k-i}(\xi)\Big)a_{\xi}^i\right)=\sum_{i=0}^{k-1}\Big(g^*(P\xi)^*w_{k-i}(\xi)\Big)g^*a_{\xi}^i=\sum_{i=0}^{k-1}\Big((P(f^*\xi))^*f^*w_{k-i}(\xi)\Big)g^*a_{\xi}^i\]
which gives $w_i(f^*\xi)=f^*w_i(\xi)$ for all $i\in[k]$.
\end{proof}

\begin{proposition} Let $B$ be a paracompact space, $\xi:E(\xi)\to B$ be any $k$-plane vector bundle and $\eta:E(\eta)\to B$ be any $\ell$-plane vector bundle. Then, $w(\xi\oplus\eta)=w(\eta)w(\xi)$.
\end{proposition}
\begin{proof} Write $E(\xi\oplus\eta)=\big\{(a,b)\in E(\xi)\times E(\eta):\xi(a)=\eta(b)\big\}$ and define $f_{\xi}:E(\xi)\to E(\xi\oplus\eta)$ with $f_{\xi}(a)=(a,0_{\eta^{=1}(\xi(a))})=:(a,0)$ and $f_{\eta}:E(\eta)\to E(\xi\oplus\eta)$ with $f_{\eta}(b)=(0_{\xi^{-1}(\eta(b))},b)=:(0,b)$, where $0_V$ denotes the zero element in the vector space $V$, making $f_{\xi}$ and $f_{\eta}$ well defined. Notice that $f_{\xi}$ and $f_{\eta}$ are embeddings. Then, we define $g_{\xi}:E(P\xi)\to E(P(\xi\oplus\eta))$ with $g_{\xi}([a])=[f_{\xi}(a)]=[(a,0)]$ and $g_{\eta}:E(P\eta)\to E(P(\xi\oplus\eta))$ with $g_{\eta}([b])=[f_{\eta}(b)]$. Notice that $g_{\xi}$ and $g_{\eta}$ are continuous and well defined. Indeed, let $a_1,a_2\in E(\xi)$ with $a_1,a_2\neq0$ and $[a_1]=[a_2]$ in $E(P\xi)$. Then $\xi(a_1)=\xi(a_2)$, i.e. $\xi\oplus\eta(a_1,0)=\xi\oplus\eta(a_2,0)$ and $a_1=\lambda a_2$ for some $\lambda\in\mathbb{R}\setminus\{0\}$, i.e. $(a_1,0)=\lambda(a_2,0)$. Thus, $[(a_1,0)]=[(a_2,0)]$ in $E(P(\xi\oplus\eta))$ and the proof is exactly the same for $g_{\eta}$. Moreover, notice that the homeomorphism $E(\xi)\cong f_{\xi}(E(\xi))\subseteq E(\xi\oplus\eta)$ respects the equivalence relations in both spaces, so $g_{\xi}$ and $g_{\eta}$ are also embeddings. So, we have the following two commutative diagrams:
\begin{center}
\begin{tikzcd}
E(\xi)\ar[r,"f_{\xi}",hook]\ar[dr,"\xi"']&E(\xi\oplus\eta)\ar[d,"\xi\oplus\eta"',near start]&E(\eta)\ar[l,"f_{\eta}"',hook']\ar[dl,"\eta"]&[2em]E(P\xi)\ar[r,"g_{\xi}",hook]\ar[dr,"P\xi"']&E(P(\xi\oplus\eta))\ar[d,"P(\xi\oplus\eta)"',near start]&E(P\eta)\ar[l,"g_{\eta}"',hook']\ar[dl,"P\eta"]\\[1.3em]
&B&&&B
\end{tikzcd}
\end{center}
Next, we have $\lambda_{\xi}\cong g_{\xi}^*\lambda_{\xi\oplus\eta}$ and $\lambda_{\eta}\cong g_{\eta}^*\lambda_{\xi\oplus\eta}$. Indeed, we will define $\phi:E(\lambda_{\xi})\to E(g_{\xi}^*\lambda_{\xi\oplus\eta})$, where
\begin{align*}
E(\lambda_{\xi})&=\big\{(L,v)\in E(P\xi)\times E(\xi):(P\xi)(L)=\xi(v)\text{ and }v\in L\big\},\\[0.5em]
E(g_{\xi}^*\lambda_{\xi\oplus\eta})&=\big\{(L,L',v')\in E(P\xi)\times E(P(\xi\oplus\eta))\times E(\xi\oplus\eta):\\
&\qquad\qquad\qquad\qquad(P(\xi\oplus\eta))(L')=(\xi\oplus\eta)(v')\text{ and }v'\in L'\text{ and }g_{\xi}(L)=L'\big\}\\
&\cong\big\{(L,v')\in E(P\xi)\times E(\xi\oplus\eta):(P(\xi\oplus\eta))(g_{\xi}(L))=(\xi\oplus\eta)(v')\text{ and }v'\in g_{\xi}(L)\big\}\\
\end{align*}
and $\phi(L,v):=(L,f_{\xi}(v))$. Then, notice that $\phi$ is well defined. Indeed, for $(L,v)\in E(\lambda_{\xi})$ we first have $(P(\xi\oplus\eta))(g_{\xi}(L))=(P\xi)(L)=\xi(v)=(\xi\oplus\eta)(f_{\xi}(v))$ and if we write $L=[a]$, then we also have that $v\in L$ means $[v]=[a]$, which gives that $[f_{\xi}]=[(v,0)]=[(a,0)]=g_{\xi}(L)$. It is clear that $\phi$ is a fiber bundle map, which is a linear isomorphism fiber-wise, since $f_{\xi}$ is a linear monomorphism over each fiber and over each $L\in E(P\xi)$ the fibers are one dimensional in both spaces. So, using \ref{prop:local_to_global_iso_vector}, $\phi$ is a vector space isomorphism. The case of $\lambda_{\eta}\cong g^*_{\eta}\lambda_{\xi\oplus\eta}$ is identical. Hence, the following diagrams are commutative and all squares are pullback squares:
\begin{center}
\begin{tikzcd}
E(\lambda_{\xi})\ar[r]\ar[d,"\lambda_{\xi}=g_{\xi}^*\lambda_{\xi\oplus\eta}"']\ar[dr,phantom,"\lrcorner",near start]&E(\lambda_{\xi\oplus\eta})\ar[r]\ar[d,"\lambda_{\xi\oplus\eta}"']\ar[dr,phantom,"\lrcorner",near start]&E(\gamma^1)\ar[d,"\gamma^1"]&[2em]E(\lambda_{\eta})\ar[r]\ar[d,"\lambda_{\eta}=g_{\eta}^*\lambda_{\xi\oplus\eta}"']\ar[dr,phantom,"\lrcorner",near start]&E(\lambda_{\xi\oplus\eta})\ar[r]\ar[d,"\lambda_{\xi\oplus\eta}"']\ar[dr,phantom,"\lrcorner",near start]&E(\gamma^1)\ar[d,"\gamma^1"]\\[1.3em]
E(P\xi)\ar[r,"g_{\xi}"]\ar[rr,dotted,bend right=10,"u_{\xi}"']&E(P(\xi\oplus\eta))\ar[r,"u_{\xi\oplus\eta}"]&\mathbb{R}P^{\infty}&
E(P\eta)\ar[r,"g_{\eta}"]\ar[rr,dotted,bend right=10,"u_{\eta}"']&E(P(\xi\oplus\eta))\ar[r,"u_{\xi\oplus\eta}"]&\mathbb{R}P^{\infty}
\end{tikzcd}
\end{center}
So, it is true that $\lambda_{\xi}=g_{\xi}^*\lambda_{\xi\oplus\eta}=g_{\xi}^*u_{\xi\oplus\eta}^*\gamma^1=(u_{\xi\oplus\eta}\circ g)^*\gamma^1$, but we also have that $\lambda_{\xi}=u^*_{\xi}\gamma^1$, so Theorem~\ref{thm:universal_uniqueness} gives us that $u_{\xi\oplus\eta}\circ g_{\xi}\simeq u_{\xi}$, which means that $g_{\xi}^*u^*_{\xi\oplus\eta}=u^*_{\xi}:H^*(\mathbb{R}P;\mathbb{Z}_2)\to H^*(E(P(\xi));\mathbb{Z}_2)$. Also, for exactly the same reason we have $g_{\eta}^*u^*_{\xi\oplus\eta}=u^*_{\eta}:H^*(\mathbb{R}P;\mathbb{Z}_2)\to H^*(E(P(\eta));\mathbb{Z}_2)$. This gives: $a_{\xi}=u_{\xi}z=g_{\xi}^*u^*_{\xi\oplus}z=g_{\xi}^*a_{\xi\oplus\eta}$ and $a_{\eta}=g_{\eta}a_{\xi\oplus\eta}$.

For the rest of the proof, we will use the notation: $n:=k+\ell$, $P:=P(\xi\oplus\eta):E(P)\to B$, $a:=a_{\xi\oplus\eta}\in H^1(E(P);\mathbb{Z}_2)$ and $w_i:=w_i{\xi\oplus\eta}\in H^i(B;\mathbb{Z}_2)$. So, we write the definition of $w_i$ as:
\[a^n=\sum_{r=0}^{n-1}P^*w_{n-r}a^r\in H^n(E(P);\mathbb{Z}_2)\]
Now we define the elements
\begin{align}
x:=&\sum_{i=0}^{k-1}P^*w_{k-i}(\xi)a^i+a^k=\sum_{i=0}^kP^*w_{k-i}(\xi)a^i\in H^k(E(P);\mathbb{Z}_2),\\
y:=&\sum_{j=0}^{\ell-1}P^*w_{\ell-j}(\eta)a^i+a^{\ell}=\sum_{j=0}^{\ell}P^*w_{\ell-j}(\eta)a^i\in H^{\ell}(E(P);\mathbb{Z}_2)
\end{align}
Our goal now is to prove that $xy=0$. Notice that $g_{\xi}^*(x)=\sum_{i=0}^kg_{\xi}^*P^*w_{k-i}(\xi)g_{\xi}^*a^i=P_{\xi}^*w_{k-i}(\xi)a_{\xi}^i=0\in H^k(E(P\xi);\mathbb{Z}_2)$. Since $g_{\xi}$ is an embedding, $g_{\xi}^*$ is part of the long exact sequence of the nice pair $(E(P),E(P\xi))$ and thus there exists some $\bar{x}\in H^k(E(P),E(P\xi);\mathbb{Z}_2)$ with $i_{\xi}^*\bar{x}=x$, where $i_{\xi}:E(P)\hookrightarrow(E(P),E(P\xi))$ is the usual inclusion map. Similarly, since $g_{\eta}^*(y)=0\in H^{\ell}(E(P\eta);\mathbb{Z}_2)$, there exists some $\bar{y}\in H^{\ell}(E(P),E(P\eta);\mathbb{Z}_2)$, with $i_{\eta}^*\bar{y}=y$.

The naturality of the cup product, gives the following commutative diagram:
\begin{center}
\begin{tikzcd}
H^k(E(P),E(P\xi);\mathbb{Z}_2)\otimes H^{\ell}(E(P),E(P\eta);\mathbb{Z}_2)\ar[r,"\smile"]\ar[d,"i_{\xi}^*\otimes i_{\eta}^*"]&H^n(E(P),E(P\xi)\cup E(P\eta);\mathbb{Z}_2)\ar[d,"i^*"]\\[2em]
H^k(E(P);\mathbb{Z}_2)\otimes H^{\ell}(E(P);\mathbb{Z}_2)\ar[r,"\smile"]&H^n(E(P);\mathbb{Z}_2)
\end{tikzcd}
\end{center}
Notice now that $g_{\xi}(E(P\xi))\cap g_{\eta}(E(P\eta))=\emptyset$ and also that there exist deformation retractions $E(P)\setminus g_{\eta}(E(P\eta))\to g_{\xi}(E(P\xi))$ and $E(P)\setminus g_{\xi}(E(P\xi))\to g_{\eta}(E(P\eta))$, which means that we have the following homotopy: $(E(P),E(P\xi)\cup E(P\eta))\simeq\big(E(P),(E(P)\setminus E(P\eta))\cup(E(P)\setminus E(P\xi))\big)\simeq(E(P),E(P))$, i.e. that $H^n(E(P),E(P\xi)\cup E(P\eta);\mathbb{Z}_2)=0$. Thus, $xy=i^*(\bar{x}\bar{y})=0$, i.e. $\big(\sum_{i=0}^kP^*w_{k-i}(\xi)a^i\big)\big(\sum_{j=0}^{\ell}P^*w_{\ell-i}(\eta)a^j\big)=0$, which gives:
\[\sum_{r=0}^n\left(P^*\sum_{i=0}^rw_i(\xi)w_{r-i}(\eta)\right)a^{n-r}=0=\sum_{r=0}^nP^*w_ra^{n-r}\]
which exactly is $w(\xi)w(\eta)=w(\xi\oplus\eta)$.
\end{proof}

\begin{proposition} For the tautological line bundle $\gamma_1^1:\big\{(L,v)\in\mathbb{R}P^1\times\mathbb{R}^2:v\in L\big\}\to$ with $\gamma_1^1(L,v)=L$ it is true that $w(\gamma_1^1)\neq0$.
\end{proposition}
\begin{proof} First, using Remark~\ref{rem:lambda_for_line_bundle}, we get that the map $(P\gamma_1^1)^*:H^*(\mathbb{R}P^1;\mathbb{Z}_2)\to H^*(E(P\gamma_1^1);\mathbb{Z}_2)$ is an isomorphism and that $\lambda_{\gamma_1^1}\cong\gamma_1^1$. Then, using Proposition~\ref{prop:restriction}, we are able to express $\lambda_{\gamma_1^1}$ as a pullback: $\gamma_1^1=\iota_{1,2}^*\gamma^1$. By the uniqueness Thereom~\ref{thm:universal_uniqueness}, $u_{\gamma_1^1}\sim\iota_{1,2}$ and thus $u_{\gamma_1^1}^*=\iota_{1,2}^*:H^*(\mathbb{R}P^{\infty};\mathbb{Z}_2)\to H^*(\mathbb{R}P^1;\mathbb{Z}_2)$. So, using Theorem~\ref{thm:projective_spaces_cohomology}, we can explicitly compute $a_{\gamma_1^1}=u_{\gamma_1^1}z=\iota_{1,2}z=z_1$, which gives us $z_1=a_{\gamma_1^1}=\sum_{i=0}^0(P\gamma_1^1)^*(w_{1-i}(\gamma_1^1))a_{\gamma_1^1}^k=(P\gamma_1^1)^*w_1(\gamma_1^1)$, i.e. $w_1(\gamma_1^1)=z_1\in\mathbb{Z}_2[z_1]\cong H^*(E(P\gamma_1^1);\mathbb{Z}_2)$. In particular, $w_1(\gamma_1^1)\neq0$.
\end{proof}

\subsection{Uniqueness of Stiefel Whitney classes}
In this subsection we prove that if some $w(\xi)$ satisfies the the Stiefel-Whitney axioms, then this is unique.

\begin{definition} Let $B',B$ be any topological spaces and $\xi:E(\xi)\to B$ any $k$-plane vector bundle. Then, a \emph{splitting map of $\xi$} is a map $f:B'\to B$ such that $f^*\xi\cong\lambda_1\oplus\cdots\oplus\lambda_k$ for some line bundles $\lambda_1,\ldots,\lambda_k$ and $f^*:H^*(B;\mathbb{Z}_2)\to H^*(B';\mathbb{Z}_2)$ is a monomorphism.
\end{definition}

\begin{proposition}\label{prop:splitting_map} Let $B$ be a paracompact topological space and $\xi:E(\xi)\to B$ any $k$-plane vector bundles. Then, there exists a splitting map of $\xi$.
\end{proposition}
\begin{proof} For the proof we do induction on $k$. In the base case $k=1$, $\xi$ is already decomposed in a whitney sum of line bundles and since $id^*$ is a monomorphism, $id_B:B\to B$ is a splitting map.

For $k\geq2$, let $(P\xi):E(P\xi)\to B$ be the projective bundle associated with $\xi$. Then, $\lambda_{\xi}:E(\lambda_{\xi})\to E(P\xi)$, as defined in \ref{def:lambda_xi} is a line bundle with $\lambda_{\xi}\leq(P\xi)^*\xi$. Let $\sigma_{\xi}:=((P\xi)^*\xi)/\lambda_{\xi}:E((P\xi)^*\xi)\to E(P\xi)$ be the $(k-1)$-plane vector bundle over $E(P\xi)$, which is paracompact due to Lemma~\ref{lem:projective_paracompact}. Because of the induction hypothesis, there exists a splitting map of $\sigma_{\xi}$, i.e. some topological space $B'$ and a continuous function $f:B'\to E(P\xi)$ such that $f^*:H^*(E(P\xi);\mathbb{Z}_2)\to H^*(B';\mathbb{Z}_2)$ is a monomorphism and $f^*\sigma_{\xi}\cong\lambda_1\oplus\cdots\oplus\lambda_{k-1}$ for some line bundles $\lambda_1,\ldots,\lambda_{k-1}$.

Since $E(P\xi)$ is paracompact, Proposition~\ref{prop:paracompact_is_euclidean} gives that $\xi$ is a euclidean vector bundle and thus, because of Remark~\ref{rem:vb_decomposition}, we have that $(P\xi)^*\xi\cong\lambda_{\xi}\oplus\sigma_{\xi}$. Then, $(P\xi\circ f)^*\xi\cong f^*(\lambda_{\xi}\oplus\sigma_{\xi})\cong f^*(\lambda_{\xi})\oplus\lambda_1\oplus\cdots\oplus\lambda_{k-1}$, where we could to distribute $f^*$ to the whitney summands because of Lemma~\ref{lem:induced_sum}. Notice that $f^*\lambda_{\xi}$ is a line bundle as well and that $(P\xi)^*$ is a monomorphism, as it was proven in Proposition~\ref{prop:basis_of_EP}. Thus, $(P\xi\circ f)^*=f^*(P\xi)^*$ is also a monomorphism, which means that $P\xi\circ f:B'\to B$ is a splitting map of $\xi$.
\end{proof}

\begin{theorem} For a paracompact space $B$ let $w$ and $w'$ be two mappings of finite vector bundles in $H^{\prod}(B;\mathbb{Z}_2)$ that satisfy the Stiefel-Whitney axioms. Then $w(\xi)=w'(\xi)$ for every $k$-plane vector bundle $\xi:E(\xi)\to B$.
\end{theorem}
\begin{proof} First we prove the assertion for the line bundle $\gamma^1:E(\gamma^1)\to B$. Notice that Proposition~\ref{prop:restriction} gives $\iota_{1,2}^*\gamma^1=\gamma_1^1$. Then, $w_1(\gamma^1)=\iota^*w_1(\gamma_1^1)\neq0$ by using (SW2) and (SW4). This means that $\gamma^1\neq0$ as well and since $w_1(\gamma^1)\in H^1(\mathbb{R}P^{\infty})\cong\mathbb{Z}_2$, there is only one choice for $w_1(\gamma^1)$, namely $w_1(\gamma^1)=z$, where $H^*(\mathbb{R}P^{\infty},\mathbb{Z}_2)\cong\mathbb{Z}_2[z]$. Moreover, because of (SW1), $w_i(\gamma^1)=0$ for every $i>1$ and $w_0=1$. This means that $w(\gamma^1)=w'(\gamma^1)=1+z$.

Next, we prove the uniqueness for every line bundle $\lambda:E\to B$. Since $B$ is paracompact, we use Theorem~\ref{thm:universal} and Theorem~\ref{thm:universal_uniqueness} to write $\lambda=u_{\lambda}^*\gamma^1$ for some unique $u_{\lambda}:B\to\mathbb{R}P^{\infty}$, up to homotopy. So, using (SW2), $w(\lambda)=w(u_{\lambda}^*\gamma^1)=u_{\lambda}^*w(\gamma^1)$ and thus, $w(\lambda)=w'(\lambda)=u_{\lambda}^*(1+z)$.

And finally, we prove it for a general $k$-plane vector bundle $\xi:E\to B$. Since $B$ is paracompact, Proposition~\ref{prop:splitting_map} gives us a space $B'$ and a continuous map $f:B'\to B$ such that $f^*:H^*(B;\mathbb{Z}_2)\to H^*(B';\mathbb{Z}_2)$ is a monomorphism and $f^*\xi\cong\lambda_1\oplus\lambda_k$ for some line bundles $\lambda_1,\ldots,\lambda_k$. Then, (SW2) and (SW3) give us that $f^*w(\xi)=w(f^*\xi)=w(\lambda_1\oplus\cdots\oplus\lambda_k)=w(\lambda_1)\cdots w(\lambda_k)=w'(\lambda_1)\cdots w'(\lambda_k)=w'(\lambda_1\oplus\cdots\lambda_k)=w'(f^*\xi)=f^*w'(\xi)$. Since $f^*$ is a monomorphism, we get $w(\xi)=w'(\xi)$.
\end{proof}


\section{Computation of the Cohomology}
%* elementary schubert calculus?


\begin{definition} Let $k\in\mathbb{N}$. Define then the following graded commutative $\mathbb{Z}_2$-algebra:
\[A_k=\mathbb{Z}_2[w_1,\ldots,w_k]\]
with $\deg(w_i)=i$ for every $i\in[k]$.
\end{definition}
\begin{theorem} For any $k\in\mathbb{N}$, there exists a graded algebra isomorphism:
\[\phi_k:A_k\to H^*(\Gr{k};\mathbb{Z}_2)\]
with $\phi_k(w_i)=w_i(\gamma^k)$.
\end{theorem}
\begin{proof} TODO - Chapter 7 in \cite{char_class}.
\end{proof}
\begin{definition} Let $0<k<n$ be some natural numbers. Then, define the following graded commutative $\mathbb{Z}_2$-algebra:
\[A_{k,n}=\sfrac{\mathbb{Z}_2[w_1,\ldots,w_k,\bar{w}_1,\ldots,\bar{w}_{n-k}]}{I_{k,n}}\]
where $\deg(w_i)=i$ for every $i\in[k]$, $\deg(\bar{w}_j)=j$ for every $j\in[n-k]$ and
\[I_{k,n}=\big((1+w_1+\cdots+w_k)(1+\bar{w}_1+\cdots+\bar{w}_{n-k})+1\big)\]
\end{definition}
\begin{proposition} We can also think of $A_{k,n}$ as:
\[A_{k,n}=\sfrac{\mathbb{Z}_2[w_1,w_2,\ldots,w_k,\bar{w}_1,\bar{w}_2,\ldots]}{I_k+(\bar{w}_{n-k+1})+(\bar{w}_{n-k+2})+\cdots+(\bar{w}_{n})}\]
where
\[I_k=\big((1+w_1+w_2+\cdots+w_k)(1+\bar{w}_1+\bar{w}_2+\cdots)+1\]
\end{proposition}
\begin{proof} Notice that for fixed $k$, for any $j\in[n-k]$, the graded ideal $I_{k,n}$ gives:
\[\bar{w}_j+\bar{w}_{j-1}w_1+\cdots+\bar{w}_{j-k}w_k=0\]
i.e. if $k$ consecutive $\bar{w}_j$ are zero, then all $\bar{w}_j$ are zero from that point on.
\end{proof}
\begin{theorem} For any natural numbers $0<k<n$, there exists a graded algebra isomorphism:
\[\phi_{k,n}:A_{k,n}\to H^*(\Gr{k}{n};\mathbb{Z}_2)\]
with $\phi_{k,n}(w_i)=w_i(\gamma^k_n)$ and $\phi_{k,n}(\bar{w}_j)=w_j({\gamma^k_n}^{\perp})$.
\end{theorem}
\begin{proof}\begin{b_item}
\item $\phi_{k,n}$ is well defined, since the Stiefel-Whitney classes satisfy $I_{k,n}$.
\item $\phi_{k,n}$ is surjective. TODO Indeed, first of all for the Manifold $\Gr{k}{n}$, we have
\[|C_i|=|H_i|=|C^i|=|H^i|\]
for every $i\in[k(n-k)]$. Moreover, one can prove that the inclusion map
\[i_{k,n}:\Gr{k}{n}\to\Gr{k}\] is a cell map, taking a shubert cell to one with the same symbol. This makes the induced map on the cohomologies surjective:
\[i^*_{k,n}:H^*(\Gr{k})\to H^*(\Gr{k}{n})\] This inclusion map induces a bundle map $\gamma^k_n\to\gamma^k$ and because of naturality of the SW classes, we get $i^*(w_i)=w_i$, which means that $i^*_{k,n}$ factors through $\phi_{k,n}$.
\item $\phi_{k,n}$ is injective. For this, it suffices to show that
\[\dim_{\mathbb{Z}_2}(A_{k,n}^i)=\dim_{\mathbb{Z}_2}(H^i(\Gr{k}{n}))\] for every $i\geq0$, which will be proven in the next lemma.
\end{b_item}
\end{proof}
\begin{lemma} The Hilbert series of the graded algebras $A_{k,n}$ and $H^*(\Gr{k}{n})$ are equal.
\end{lemma}
\begin{proof} Let
\begin{align*}
p_{k,n}^i:=&\dim_{\mathbb{Z}_2}(H^i(\Gr{k}{n}))=|C_i|\\
=&\#\{\text{Young tableau of at most }k\text{ rows and at most }n-k\text{ columns.}\}
\end{align*}
Which makes its Hilbert series:
\[P_{k,n}(t):=\sum_{i\geq0}p_{k,n}^it^i=\binom{n}{k}_t\]
as proven in the Lemma \ref{lem:hilbert_series_of_young}. On the other hand, let
\[H_{k,n}(t)=\sum_{i\geq0}\dim_{\mathbb{Z}_2}(A_{k,n}^i)t^i\]
It suffices to prove that $H_{k,n}(t)$ satisfies the same recursive conditions as $P_{k,n}(t)$. First of all, we easily get $H_{1,n}(t)=P_{1,n}(t)$ and then, using the additivity of the following SES:
\begin{center}
\begin{tikzcd}
0\ar[r]& A_{k,n-1}^{[-k]}\ar[r,"\cdot w_k", "f"']&A_{k,n}\ar[r,"w_k\mapsto0","g"']&A_{k-1,n-1}\ar[r]&0
\end{tikzcd}
\end{center}
we get:
\[H_{k,n}(t)=t^kH_{k,n-1}(t)+H_{k-1,n-1}(t)\]
which is the same recursion that defines $P_{k,n}(t)$. The fact that this is indeed a SES is proven in the following lemma:
\end{proof}
\begin{lemma} The following is a SES of graded commutative $\mathbb{Z}_2$-algebras:
\begin{center}
\begin{tikzcd}
0\ar[r]& A_{k,n-1}^{[-k]}\ar[r,"\cdot w_k", "f"']&A_{k,n}\ar[r,"w_k\mapsto0","g"']&A_{k-1,n-1}\ar[r]&0
\end{tikzcd}
\end{center}
where ${A_{k,n}^{[-k]}}^i:=A_{k,n}^{i-k}$ is the proper shift to make $f$ a homomorphism of graded algebras.
\end{lemma}
\begin{proof}
Let
\[R_{k,n}:=\sfrac{\mathbb{Z}_2[w_1,w_2,\ldots,w_k,\bar{w}_1,\bar{w}_2,\ldots]}{I_k+(\bar{w}_{n-k+1})+(\bar{w}_{n-k+2})+\cdots+(\bar{w}_{n-1})}\]
then, the algebras can be written as:
\begin{center}
\begin{tikzcd}
0\ar[r]& \sfrac{R_{k,n}}{(\bar{w}_{n-k})}^{[-k]}\ar[r,"\cdot w_k", "f"']&\sfrac{R_{k,n}}{(\bar{w}_{n})}\ar[r,"w_k\mapsto0","g"']&\sfrac{R_{k,n}}{(w_k)}\ar[r]&0
\end{tikzcd}
\end{center}
\begin{b_item}
\item $f,g$ are well defined. Indeed, notice that
\[\bar{w}_{n}=\bar{w}_{n-1}w_1+\bar{w}_{n-2}w_2+\cdots+\bar{w}_{n-k+1}w_{k-1}+\bar{w}_{n-k}w_k=\bar{w}_{n-k}w_k,\text{ inside }R_{k,n}\]
\item $g$ is surjective and $\im(f)\subseteq\ker(g)$ are easy to prove.
\item $f$ is injective and $\im(f)\supseteq\ker(g)$ -- TODO!!
\end{b_item}
\end{proof}


\begin{definition} We write:
\[\binom{n}{k}_t:=\frac{[n]_t!}{[k]_t!\cdot[n-k]_t!}\]
where:
\[[x]_t:=\sum_{i=0}^{x-1}t^i=\frac{t^x-1}{t-1}\]
and:
\[[x]_t!=[x]_t\cdot[x-1]_t\cdots[1]_t\]
\end{definition}
