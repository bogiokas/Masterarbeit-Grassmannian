\chapter*{Introduction\markboth{Introduction}{Introduction}}
\addcontentsline{toc}{chapter}{Introduction}

In mathematics, the Grassmannian $Gr(k,V)$ is a space that parametrized all $k$-dimensional linear subspaces of the $n$-dimensional vector space $V$.
 
One of the tools used to study Grassmannians and in particular intersections in Grassmannians is Schubert calculus \cite{fulton2013intersection}.
Schubert calculus is also of independent interest as it was introduced in 1874 \cite{schubert1879kalkul} and it dealt with finding the number of points, lines, planes etc. satysfying certain geometric conditions (for a complete survey see \cite{kleiman1972schubert}).



The notion of a fibre bundle first arose in 1930s, with the first general definitions given by H. Whitney, through the fields of topology and geometry on manifolds and in particular the first international conference this subject appeared took place in Moscow in 1935 \cite{tucker1935topological}.
In particular one of the main motivations for fiber spaces to be defined was studying manifolds to find examples and invariants useful to settling Poincare's conjecture (whether a simply connected compact $n$-manifold having the same homology groups as $S^n$ is homeomorphic to $S^n$).
In the next years they became central in the study of algebraic topology and in 1949 the infuential seminar, Seminaire Henri Cartan in Paris \cite{cartan1955seminaire} was dedicated to fiber spaces.

Due to these applications, various research groups were working in on fibre bundles and hence lots of progress was made but also a variety of definitions of this notion was provided.
The first text that provided in a structured way these results came in 1951 by Steenrod \cite{steenrod1951topology}, which to this day is considered a classical text on this subject.
Among other subjects this book contained, such as homotopy groups and cohomology groups, it also contained the theory developed on homotopy classification of fiber bundles and the theory on characteristic classes, up that point by several mathematicians such as Chern, Whitney, Pontrajagin and Stiefel.

Since then, due to its applications in various fields like topology, lots of attention was drawn to this subject and interesting results arose (e.g. \cite{thom1952espaces},\cite{borel1955topology},\cite{hirzebruch1956}).
In particular until 1955 a construction of a universal fibre bundle for any topological group was given by Milnor \cite{milnor1956construction}\cite{milnor1956construction2}, and also Hirzebruch clarified the notion of characteristic classes.
A significant book on the characteristic classes of vector bundles was written by Milnor and Stasheff in 1974 \cite{char_class} which is also the book mainly followed in this thesis.

+Zisman Fibre bundles, fibre maps 1999 (origins of different definitions and relations among them)
\ \\
\ \\
\ \\
K-theory introduced by M.F.Atiyah and Hirzebruch

\ \\
\ \\
\ \\

%Vector bundles

%From "The topology of fiber bundles lecture notes"


In this thesis the  ultimate goal is to study the $\mathbb{F}_2$ Cohomology ring of the real Grassmannians $\Gr{k}{n}{\mathbb{R}}$ and to examine a fascinating connection between this algebraic structure and the set of all real vector bundles of $\Gr{k}{n}{\mathbb{R}}$, which is a purely geometric structure. We mainly follow the book \cite{char_class} but also try to present the theory in a more detailed and concise way. The thesis is divided in three chapters.
The first two chapters are intented to build the language we need. They are dedicated, respectively, to the two sides of the coin involved, namely the topology and the geometry. Each one of these two chapters is self-contained and thus a completely independent and sufficient read. This means in particular, that these two chapters do not aim at all towards the proof of the main theorem (this is stated and proved in the third chapter), but their goal is instead for the reader to build a stable entry point to the theory on Grassmannians and Vector bundles. Most importantly, this includes forming a sufficiently good intuition on the topics involved. Thus, the theory is enriched with plenty of examples and helpful comments. On the other side, we have tried to remove hand-waving as much as possible of our proofs. This has of course made some of the proofs more precise than in the literature and thus lengthier. The reader would be advised to skip some of them in the first read, since we always enclose the proofs inside more intuitive comments about the main arguments in the proof.
The reader can either choose to start at Chapter 1 or Chapter 2, in accordance with their interests and their level of comfort.

In the first chapter after the definition of Grassmannians it is immediately proved that they are indeed Manifolds. We also prove some first topological properties of them, in order to establish that they are "good" topological spaces which deserve our attention. Next, a very natural cell structure making them CW-manifolds is introduced. These cells are named \emph{Schubert cells}. This cell decomposition plays a very important role in the study of the (co)homology of the Grassmannians. Namely, one can define a very interesting product on the set of the Schubert classes, which -as you could already guess- are very closely related to Schubert cells. One then speaks for a whole area called \emph{Schubert calculus}. We get involved with Schubert calculus just enough to convince the reader that this is a really deep and interesting area, but our approach of this topic is very introductory and not even close to be characterized as complete.

In the second chapter we define the vector bundles in general and prove some basic first results, mainly, in order to understand how different bundles can get combined to form new ones. The highlight of this chapter is going to be the notion of the ``characteristic classes'', already giving the reader an idea how one can bridge the gap between the set of all vector bundles of a space and its topology.

In the last chapter we unleash the combined power gathered in the first two chapters of this thesis in order to prove an amazing result, fully characterising the cohomology algebra of any Grassmannian. At this point we first examine and fully prove the infinite case and then prove the general result involving the finite Grassmannians.

As far as we know, the results of the third chapter are well known and widely used in the community, but we have yet to see some written proof of the finite case, i.e. the last part of the thesis.

We hope that this over-analytic approach is going to help future students introduce themselves to such beautiful and deep mathematical concepts.
