\chapter*{Introduction\markboth{Introduction}{Introduction}}
\addcontentsline{toc}{chapter}{Introduction}
%like vector bundles \cite{guyot1985caracterisation}.

The Grassmannian $\Gr{k}{n}{F}$ is the set of all $k$-dimensional vector subspaces in $F^n$, where $F$ is usually $\mathbb{R}$ or $\mathbb{C}$. The study of \emph{Grassmannians} as we know it, started in the late 19th century by Hermann Schubert \cite{schubert}, when he defined the \emph{Schubert symbols} to be the elements of $\binom{[n]}{k}$, each of which indexes a Schubert variety. His motivation was to answer questions about number of intersection points of affine spaces. For a detailed discussion on the topic, refer to \cite{kleiman1972schubert} or \cite{fulton2013intersection}. Essentially, if we fix a flag $\mathbb{R}_1\subseteq\mathbb{R}_2\subseteq\cdots\subseteq\mathbb{R}^n$, two $k$-dimensional subspaces are in the same cell, if their intersection pattern with respect to this flag agrees. The intersection pattern can be thought as the \emph{jump points} of the dimension, when intersecting a $k$-subspace with the flag, i.e. for $H\in\Gr{k}{n}$ the sequence $j_1<j_2<\cdots<j_k$, where
\[\dim(\mathbb{R}^{j_i-1}\cap H)=i-1,\qquad\dim(\mathbb{R}^{j_i}\cap H)=i\]
for every $i\in[k]$.

In 1933, Ehresmann calculated in his thesis \cite{ehresmann} a basis of $H_*(\Gr{k}{n}{\mathbb{C}};\mathbb{Z})$ by using the schubert decomposition. He proved that every schubert cell is open of dimension $2(j_1-1)+2(j_2-2)+\cdots+2(j_k-k)$, and that the boundaries of these cells lie in lower dimensional cells and in particular inside cells that differ in their jump points at most by one. So, in the case of the complex Grassmannians, since there are no cells in odd dimensions, it can directly be deduced that its homology is a free $\mathbb{Z}$-module with the schubert varieties as basis. A few years later, in 1937, he also studied the homology of the real Grassmannian with $\mathbb{Z}_2$-coefficients \cite{ehresmann2}, this time computing the boundaries explicitly.

The full computation for the cohomology ring was done almost 10 years later by Chern \cite{chern}, by using the notion of \emph{sphere bundles}. At this point, the notion of cohomology was still pretty young. In fact, cohomology was introduced at a 1935 conference in Moscow \cite{tucker1935topological} by Andrey Kolmogorov and Alexander. A subject that played a central role in this conference was the topic of sphere bundles and in general fiber bundles. Around these years Stiefel associated to the sphere bundle of the tangent space of a manifold a homology class. At the same time, independently Whitney \cite{whitney} applied similar techniques to arbitrary sphere bundles $p:E\to B$, where the base space is locally finite simplicial complex. During these investigations, both of them studied the \emph{Stiefel manifold} and identified it with configuration spaces.

A few years after that, in 1937 Whitney used \cite{whitneyCohom} the newly defined notion of cohomology to describe the classes $w_i$ that are now known as \emph{Stiefel Whitney characteristic classes}. It took years to build the theory around the Stiefel Whitney classes and prove the axioms. Notably, the first published proof of the whitney sum formula, i.e. that
\[w_n(\eta\oplus\xi)=\sum_{i=0}^nw_i(\eta)\smile w_{n-i}(\xi)\]
was published by Chern in \cite{chern} and Wu Wen Ts\"un in \cite{wu} in the same exact volume of the journal. Wu Wen Ts\"un proved this in fact by working with vector bundles, instead of sphere bundles.

In the next years the characteristic classes and the study of fiber bundles became central in the study of algebraic topology and in 1949 the infuential seminar, Seminaire Henri Cartan in Paris \cite{cartan1955seminaire} was dedicated to fiber spaces. The first text that provided in a structured way these results came in 1951 by Steenrod \cite{steenrod1951topology}, which to this day is considered a classical text on this subject.  Since then, due to its applications in various fields, lots of attention was drawn to this subject and interesting results arose (e.g. \cite{thom1952espaces}, \cite{borel1955topology}, \cite{hirzebruch1956}). In particular until 1955 a construction of a universal fibre bundle for any topological group was given by Milnor \cite{milnor1956construction} \cite{milnor1956construction2}, and also Hirzebruch clarified the notion of characteristic classes. A significant book on the characteristic classes of vector bundles was written by Milnor and Stasheff in 1974 \cite{char_class} which is also the book mainly followed in this thesis.

The purpose of this thesis is to develop in detail all the theory needed to compute the Cohomology ring of the real Grassmannians with $\mathbb{Z}_2$-coefficients and to proceed with the computation at the end. The reader however is required to be familiar with some basic notions of topology, such as the notion of CW Complex, Homology and Cohomology.

Chapter~\ref{chap:grassmannians} is solely about the real Grassmannians as topological spaces. The $k,n$-real Grassmannian is the topological space of all $k$-planes in the $n$-th dimensional real vector space, topologized as the quotient of all possible $k$-bases in the $n$-th dimensional space. We also define the infinite $k$-Grassmannian, which is the topological space of all $k$-planes inside hte infinite dimensional real vector space. Towards the end of the Chapter, the focus lies on the definition of the Schubert cell decomposition, whis is a particularly nice cell decomposition of the Grassmannian. In the last section of the Chapter, we also establish a bijection between the cells and the young diagrams is established.

Chapter~\ref{chap:vector_bundles} mainly focuses on the notion of vector bundles. A vector bundle is a way to assign to each point in a base space a vector space in a continuous fashion. After the first definitions and results, we prove a local property of vector bundles that is not always true in fiber bundles, namely that if we manage to create a vector bundle map between two spaces and prove that this is a homeomorphism fiber-wise, then the map automatically is a vector bundle isomorphism. Then, some useful constructions of vector bundles are presented like the induced bundlde, which is a way to move the vector bundle structure from the codomain of a function on the base spaces to its domain. Also, we present the whitney sum and the quotient, which are bundle operations that are defined fiber-wise and are the usual operations on vector spaces. The notion of euclidean vector bundles play an important role in the discussion of the Chapter. We prove that if a vector bundle is euclidean, we can decompose this bundle in whitney sums of smaller ones, provided that we can find a subbundle. Next we prove some results regarding paracompact spaces, namely that vector bundles over them are euclidean and that in a paracompact space there always exists a countable open cover such that a given vector bundle is trivial over each set in the cover.

Chapter~\ref{chap:cohomology} combines the results in the previous two chapters. It starts with the definition of the tautological vector bundles over the grassmannians. These are particular vector bundles, one for each Grassmannian, which attach to an element in the $k,n$-Grassmannian the element itself as $k$-plane vector bundle. Next we prove that the tautological bundle over the infinite $k$-grassmannian is universal, in the sense that every other $k$ bundle over any other paracompact space can be described as being an induced bundle of the tautological of dimension $k$. Next, we define the Stiefel Whitney classes. In order to do so we state and prove a version of the Leray-Hirsch theorem and also prove the splitting principle, which states that every vector bundle over a paracompact space has an induced bundle that can be split and at the same time preserves the cohomology information of the original bundle. At the end of this Chapter, we use these Stiefel whitney classes and the theory of the vector bundles and the cell decomposition of the Grassmannian to prove that the Cohomology of the infinite $k$-Grassmannian is generated freely by $k$ elements of degree $1$ up to $k$.
