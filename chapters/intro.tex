\chapter*{Introduction\markboth{Introduction}{Introduction}}
\addcontentsline{toc}{chapter}{Introduction}
In this thesis our ultimate goal is to study the $\mathbb{F}_2$ Cohomology ring of the real Grassmannians $\Gr{k}{n}{\mathbb{R}}$ and to examine a fascinating connection between this algebraic structure and the set of all real vector bundles of $\Gr{k}{n}{\mathbb{R}}$, which is a purely geometric structure. We mainly follow the book \cite{char_class} but we try to present the theory in a more detailed and concise way. The thesis is divided in three chapters.
The first two chapters are going to build the language we need. They are dedicated, respectively, to the two sides of the coin, namely the topology and the geometry involved. Each one of these two chapters is going to be self-contained and thus a completely independent and sufficient read. This means in particular, that these two chapters do not aim at all towards the proof of the main theorem (this is stated and proved in the third chapter), but their goal is instead for the reader to build a stable entry point to the theory on Grassmannians and Vector bundles, respectively. Most importantly, this includes forming a sufficiently good intuition on the topics involved. Thus, we have enriched the theory with plenty of examples and helpful comments. On the other side, we have tried to remove hand-waving as much as possible of our proofs. This has of course made some of the proofs more precise than in the literature and thus lengthier. Our advice to the reader would be to skip some of them in the first read, since we always enclose the proofs inside more intuitive comments about the main arguments in the proof.
The reader can either choose to start at Chapter 1 or Chapter 2, in accordance with their interests and their level of comfort.

In the first chapter we define the Grassmannians and immediately prove that they are indeed Manifolds. We also prove some first topological properties of them, in order to establish that they are "good" topological spaces which deserve our attention. Next, we introduce a very natural cell structure making them CW-manifolds. These cells are going to be named ``Schubert cells''. This cell decomposition plays a very important role in the study of the (co)homology of the Grassmannians. Namely, one can define a very interesting product on the set of the Schubert classes, which -you could already guess- are very close related to Schubert cells. One then speaks for a whole area called ``Schubert calculus''. We get involved with Schubert calculus just enough to convince the reader that this is a really deep and interesting area, but our approach of this topic is very introductory and not even close to be characterized as complete.

In the second chapter we define the vector bundles in general and prove some basic first results, mainly, in order to understand how different bundles can get combined to form new ones. The highlight of this chapter is going to be the notion of the ``characteristic classes'', already giving the reader an idea how one can bridge the gap between the set of all vector bundles of a space and its topology.

In the last chapter we are going to unleash the combined power we have been gathering in the first two chapters of this thesis in order to prove an amazing result, fully characterising the cohomology algebra of any Grassmannian. At this point we first examine and fully prove the infinite case and then we prove the general result involving the finite Grassmannians.

As far as we know, the results of the third chapter are well known and widely used in the community, but we have yet to see some written proof of the finite case, i.e. the last part of the thesis.

We hope that this over-analytic approach is going to help future students introduce themselves to such beautiful and deep mathematical concepts.
