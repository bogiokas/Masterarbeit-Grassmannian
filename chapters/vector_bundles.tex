%!TEX root = Cohomology of real Grassmannians.tex
\chapter{Vector Bundles}\label{chap:vector_bundles}
The discussion in this chapter is based on the Chapters 2,3,4 of \cite{char_class} and its goal is first to investigate the notion of vector bundles and the additional structure they provide, when compared to the general fiber bundles (see Appendix~\ref{app:fiber_bundles}). At the end of this chapter, for each vector bundle the associated ``Stiefel-Whitney classes'' is defined to be some sequence of elements in the cohomology ring of the base space, one for each dimension. As they are defined to be invariant under bundle isomorphisms, they can be used to distinguish between non-isomorphic vector bundles.
%TODO: remove S-W from this intro

\section{Basic Notions}
Let us begin with the definition of a vector bundle. This is the same as Definition~\ref{def:fiber_bundle}, differing only on two aspects. First, the fibers here are vector spaces, and secondly the homeomorphisms making the total space locally trivial are here linear isomorphisms over each fiber:

\begin{definition}\label{def:vector_bundle} Let $E$ and $B$ be some topological spaces and $n\in\mathbb{N}$. A continuous map $\xi:E\to B$ is called an \ul{$\mathbb{R}^n$-vector bundle} or \ul{$n$-plane bundle}, if for every $x\in B$ the set $\xi^{-1}(x)$ is a vector space and also for every $x\in B$ there exists an open neighborhood $x\in U\subseteq B$, and a continuous map $f_U:\xi^{-1}(U)\to U\times\mathbb{R}^n$ (where $\xi^{-1}(U)$ has the subspace topology and $U\times\mathbb{R}^n$ the product topology), such that:
\begin{i_enum}
\item $f_U$ is a homeomorphism.
\item  $\xi|_{\xi^{-1}(U)}=\pi_1\circ f_U$, i.e. the following diagram commutes:
\begin{center}
\begin{tikzcd}
E\ar[r,"\supseteq",phantom]&[-1.5em]\xi^{-1}(U)\ar[d,"\xi|_{\xi^{-1}(U)}"']\ar[r,"f_U"]&U\times\mathbb{R}^n\ar[dl,"\pi_1"]\\[2em]
B\ar[r,"\supseteq",phantom]&U
\end{tikzcd}
\end{center}
\item $f_U|_{\xi^{-1}(x)}:\xi^{-1}(x)\to\{x\}\times\mathbb{R}^n$ is linear for every $x\in U$.
\end{i_enum}
$B=B(\xi)$ is then called \ul{base space} and $E=E(\xi)$ \ul{total space} of the vector bundle. Moreover, for every $x\in B$, the space $\xi^{-1}(x)$ is called \ul{the fiber over $x$}.
\end{definition}
\begin{remark} Since for every $x\in B$ $f_U:U\to U\times\mathbb{R}^n$ is a homeomorphism, Proposition~\ref{prop:same_fiber} gives us that $\xi^{-1}(x)\cong\mathbb{R}$ as topological spaces through $f_U|_{\xi^{-1}(x)}$ and since $f_U|_{\xi^{-1}(x)}$ is also linear, $\xi^{-1}(x)\cong\mathbb{R}^n$ as vector spaces. So, the total space $E$ of the fiber bundle $\xi$ can be visualized as a collection of $n$-planes, parametrized continuously by elements of $B$.
\end{remark}

\begin{remark}\label{rem:vector_subcover} Let $E\overset{\xi}{\to} B$ be a fiber bundle, $x\in B$ and $U\subseteq B$ an open neighborhood of $x$, $f_U:p^{-1}(U)\to U\times\mathbb{R}^n$ a homeomorphism as in the Definition~\ref{def:vector_bundle}, i.e. it makes the following diagram commute and is linear over each fiber:
\begin{center}
\begin{tikzcd}
\xi^{-1}(U)\ar[d,"\xi|_{\xi^{-1}(U)}"']\ar[r,"f_U"]&U\times\mathbb{R}^n\ar[dl,"\pi_1"]\\
U
\end{tikzcd}
\end{center}
Moreover, let $V\subseteq U$ be some other open neighborhood of $x$. Then, $f_V:=f_U|_{\xi^{-1}(V)}$ is a homeomorphism between $\xi^{-1}(V)$ and $V\times\mathbb{R}^n$ as in the Definition~\ref{def:vector_bundle}, i.e. it makes its associated diagram commute and is linear over each fiber. Indeed, using Remark~\ref{rem:fiber_subcover} we can prove that this makes the associated diagram commute. Since it is a restriction of $f_U$, which is linear over each fiber, $f_V$ is also linear over each fiber and thus satisfies the restrictions of Definition~\ref{def:vector_bundle}.
\end{remark}

Let us now list some examples of vector bundles:
\begin{examples}\label{ex:vector_bundles}
\begin{i_enum}
\item For any topological space $B$ and any $n\in\mathbb{N}$ we have the \emph{trivial} vector bundle
$\varepsilon_B^n:B\times\mathbb{R}^n\to B$,
which is the projection on the first coordinate.
\item Let $M$ be a smooth $n$-manifold then its \emph{tangent bundle} $\tau_M:TM\to M$ is the $n$-plane bundle constructed as follows: Let $(U_{\alpha})_{\alpha\in I}$ be an open cover of $M$ and $\mathcal{A}=\big\{(U_{\alpha},\phi_{\alpha}:U_{\alpha}\to\mathbb{R}^n)_{\alpha\in I}\big\}$ be a smooth atlas of $M$. Fix any point $x\in M$ and some $(U,\phi)\in\mathcal{A}$ such that $x\in U$. Then define the tangent space at $x$ to be the set of all possible \emph{tangent vectors} at $x$ inside $M$, i.e. the set of equivalence classes
$T_xM:=\faktor{\left\{\gamma:\mathbb{R}\to M\ |\ \gamma(0)=x\text{ and }\phi\circ\gamma\text{ smooth}\right\}}{\sim}$,
where $\gamma_1\sim\gamma_2$ iff $(\phi\circ\gamma_1)'(0)=(\phi\circ\gamma_2)'(0)$. This set can be given an $\mathbb{R}$-vector space structure as,
$\lambda_1\cdot[\gamma_1]+\lambda_2\cdot[\gamma_2]:=\left[\phi^{-1}\circ\big(\lambda_1(\phi\circ\gamma_1)+\lambda_2(\phi\circ\gamma_2)\big)\right]$.

Notice that $T_xM$ does not depend on the choice of $(U,\phi)\in\mathcal{A}$. Indeed, since the atlas is smooth and $U_{\alpha}$ are open, every transition map $\phi_{\alpha}\circ\phi_{\beta}:\mathbb{R}^n\to\mathbb{R}^n$ has an invertible Jacobi matrix and thus the equivalence relation is the same under any $\phi$. Then, we define the total space of the tangent bundle of $M$ to be the set of all possible tangent vectors at any point of $M$, i.e. the set
$TM:=\big\{(x,[\gamma]):x\in M\text{ and }[\gamma]\in T_xM\big\}$
which is now topologised as follows: Fix some $(U,\phi)\in\mathbb{A}$, then define the following map:
\begin{center}
\begin{tikzcd}
TM\ar[r,phantom,"\supseteq"]&[-2em]\pi_1^{-1}(U)\ar[r]&U\times\mathbb{R}^n\\[-1.5em]
&(x,[\gamma])\ar[r,mapsto]&(x,(\phi\circ\gamma)'(0))
\end{tikzcd}
\end{center}
This induces a topology on $\pi_1^{-1}(U)$ and since the transition maps are open maps, this topology is well defined on every $U_{\alpha}\cap U_{\beta}$. This defines a topology on $TM$. The vector bundle $\tau_{M}:TM\to M$ is exactly the function $\pi_1(x,[\gamma])=x$ we used above, so the local triviality is true by the definition of the topology.
\item Let $M$ be a smooth $n$-manifold embedded in $\mathbb{R}^d$. Then its \emph{normal bundle} $\nu_{M,d}:N_dM\to M$ is the $d-n$-plane bundle \emph{orthogonal} to $\tau_M$, i.e.
\[N_dM:=\big\{(x,v)\in M\times\mathbb{R}^d:x\in M\text{ and }\forall[\gamma]\in T_xM\ v\perp\gamma'(0)\in\mathbb{R}^d\big\}\]
and it is easy to see how the fibers are vector spaces and how the total space is topologised. The vector bundle $\nu_{M,d}$ is again the projection on the first coordinate, but notice that this vector bundle depends on the ambient space the $M$ lives in, while the tangent bundle defined above is an intrinsic construction, i.e. independent of any embedding.
\item\label{ex:tautological_line_bundle} There exists a natural $1$-plane bundle (also: \emph{line bundle}) over each $n$ dimensional real projective space: the \emph{tautological} line bundle
\begin{center}
\begin{tikzcd}
\gamma_n^1:\big\{(l,v)\in\mathbb{R}P^n\times\mathbb{R}^{n+1}:v\in l\big\}\ar[r]&\mathbb{R}P^n\\[-1.5em]
(l,v)\ar[r,mapsto]&l
\end{tikzcd}
\end{center}
The total space of this bundle is topologised as a subspace of $\mathbb{R}P^n\times\mathbb{R}^{n+1}$, where the $n$-dimensional projective space is defined to be the set of all lines through the origin in $\mathbb{R}^{n+1}$, topologized as the following quotient:
\[\mathbb{R}P^n=\{l\subseteq\mathbb{R}^{n+1}:l\text{ vector space with }\dim l=1\}\cong\faktor{S^n}{x\sim-x}\]
We need to show that the total space is locally homeomorphic to a trivial bundle with the homeomorphism being linear isomorphism on each fiber. We use the standard open cover of $\mathbb{R}P^n$ which makes it a manifold, the one also used in Lemma~\ref{lem:gr_manifold}: Let $l\in\mathbb{R}P^n$ and define $U:=\{l'\in\mathbb{R}P:l'\cap l^{\perp}=\{0\}\}$. Fix also some $v\in l\setminus\{0\}$. Then, define $f_U:(\gamma_n^1)^{-1}(U)\to\mathbb{R}P^n\times\mathbb{R}$ as
$f_U(l',v')=(l',v'^tv)$.
This is continuous and linear over each fiber and since $v'^tv\neq0$ for $v'\neq0$, $f_U$ is also a homeomorphism.
\end{i_enum}
\end{examples}

After seeing some examples for the objects we are interested in, it is time to define morphisms between these objects, in order to be able to differentiate them and construct new ones from old. Compared to the Definition~\ref{def:bundle_map} of the bundle map, we also need for it to respect the vector space structure over each fiber.

\begin{definition}
Let $\xi_1:E_1\to B$ be an $n$-plane vector bundle and $\xi_2:E_2\to B$ be an $m$-plane vector bundle. A continuous map $\phi:E_1\to E_2$ is a \ul{vector bundle map from $\xi_1$ to $\xi_2$ over $B$}, if
\begin{i_enum}
\item $\xi_1=\xi_2\circ\phi$, i.e. if the following diagram commutes:
\begin{center}
\begin{tikzcd}
E_1\ar[rd,"\xi_1"']\ar[rr,"\phi"]&[-2em]&[-2em]E_2\ar[ld,"\xi_2"]\\
&B
\end{tikzcd}
\end{center}
\item $\phi|_{\xi_1^{-1}(x)}:\xi_1^{-1}(x)\to\xi_2^{-1}(x)$ is linear for every $x\in B$.
\end{i_enum}
\end{definition}
\begin{remark} Due to Proposition~\ref{prop:fiberwise}, vector bundle maps can be characterized by their local behavior: Let $\phi:E_1\to E_2$ be a continuous map. Then, $\phi$ is a vector bundle map iff
\begin{i_enum}
\item $\phi|_{\xi_1^{-1}(x)}(\xi_1^{-1}(x))\subseteq\xi_2^{-1}(x)$ for every $x\in B$ and
\item $\phi|_{\xi_1^{-1}}:\xi_1^{-1}(x)\to\xi_2^{-1}(x)$ is linear for every $x\in B$.
\end{i_enum}
\end{remark}

\begin{definition} Let $\xi_1:E_1\to B$ be an $n$-plane bundle and $\xi_2:E_2\to B$ be an $m$-plane bundle and let $\phi:E_1\to E_2$ be a vector bundle map from $\xi_1$ to $\xi_2$ over $B$. The map $\phi$ is called a \emph{vector bundle isomorphism} if $\phi$ is a homeomorphism.
\end{definition}
\begin{remark}\label{rem:global_to_local_iso_vector} Due to Proposition~\ref{prop:global_to_local_iso} a vector bundle isomorphism $\phi$ induces a homeomorphism $\xi_1^{-1}(x)\cong\xi_2^{-1}(x)$ over each fiber. Since $\phi|_{\xi_1^{-1}(x)}$ is also linear, $\xi_1^{-1}(x)\cong\xi_2^{-1}(x)$ as vector spaces and thus $n=m$.
\end{remark}

Up to this point, our discussion about vector bundles mirrors the discussion on the fiber bundles of Appendix~\ref{app:fiber_bundles}. The fact that each fiber is a vector space is just an additional structure carried around through the definitions. The next proposition however is indicative of how much more restrictive the notion of a vector bundle is. In particular we directly prove that an inverse of Remark~\ref{rem:global_to_local_iso_vector} is always true. Compare this to counterexample~\ref{ex:counterexample} and Theorem~\ref{thm:local_to_global_iso}.

\begin{proposition}\label{prop:local_to_global_iso_vector} Let $\xi_1:E_1\to B$ and $\xi_2:E_2\to B$ be two $n$-plane vector bundles. Moreover, let $\phi:E_1\to E_2$ be a vector bundle map such that its restriction
$\phi_x:\xi_1^{-1}(x)\to\xi_2^{-1}(x)$
with $\phi_x(v)=\phi(v)$ is a vector space isomorphism for every $x\in B$. Then $\phi$ is also a homeomorphism and thus a vector bundle isomorphism.
\end{proposition}
\begin{proof} First of all, it is easy to see that $\phi$ is a bijection:
\begin{itemize}
\item Let $u_1,u_2\in E_1$ and $\phi(u_1)=\phi(u_2)$. Since $\phi$ is a vector bundle map,
$\xi_1(u_1)=\xi_2(\phi(u_1))=\xi_2(\phi(u_2))=\xi_1(u_2)=:x_0$.
So, $u_1,u_2\in\xi_1^{-1}(x_0)$ and $\phi_{x_0}(u_1)=\phi(u_1)=\phi(u_2)=\phi_{x_0}(u_2)$. Since $\phi_{x_0}$ is injective, $u_1=u_2$, which proves that $\phi$ is injective as well.
\item Let $v\in E_2$. For $x_0:=p_2(v)$, $v\in\xi)2^{-1}(x_0)$. Since $\phi_{x_0}$ is surjective, there exists some $u\in\xi_1^{-1}(x_0)\subseteq E_1$ with $\phi(u)=\phi_{x_0}(u)=v$, proving that $\phi$ is also surjective.
\end{itemize}
It now remains to prove that $\phi^{-1}$ is continuous as well. For every $x\in B$, there exist open neighborhoods $U_1,U_2$ of $x$, such that $\xi_1^{-1}(U_1)$ is isomorphic to $U_1\times\mathbb{R}^n$ and $\xi_2^{-1}(U_2)$ is isomorphic to $U_2\times\mathbb{R}^n$, satisfying the commutative diagrams of the Definition~\ref{def:vector_bundle}. Let $U_x:=U_1\cap U_2$. Then, as seen in Remark~\ref{rem:vector_subcover}, $U$ satisfies the requirements of the \ref{def:vector_bundle}. This means that there exist homeomorphisms $(f_1)_{U_x}:\xi_1^{-1}(U_x)\to U_x\times\mathbb{R}^n$ and $(f_2)_{U_x}:\xi_2^{-1}(U_x)\to U_x\times\mathbb{R}^n$ making the following diagrams commute and being linear over each fiber:
\begin{center}
\begin{tikzcd}
\xi_1^{-1}(U_x)\ar[d,"\xi_1|_{\xi_1^{-1}(U_x)}"']\ar[r,"(f_1)_{U_x}"]&U_x\times\mathbb{R}^n\ar[dl,"\pi_1"]&&
\xi_2^{-1}(U_x)\ar[d,"\xi_2|_{\xi_2^{-1}(U_x)}"']\ar[r,"(f_2)_{U_x}"]&U_x\times\mathbb{R}^n\ar[dl,"\pi_1"]\\
U&&&U
\end{tikzcd}
\end{center}
Let $\mathcal{U}=\{U_x:x\in B\}$. Since $\{\xi_2^{-1}(U):U\in\mathcal{U}\}$ forms an open cover of $E_2$, it suffices to show that $\phi^{-1}|_{\xi_2^{-1}(U)}$ is continuous for every $U\in\mathcal{U}$ and then apply the glueing lemma. Let us fix such a $U\in\mathcal{U}$. Then, the above commutative diagrams let us define the map $\psi:U\times\mathbb{R}\to U\times\mathbb{R}$ as the following composition:
\vspace*{-1em}
\begin{center}
\begin{tikzcd}
U\times\mathbb{R}^n\ar[r,"\cong"',"(f_1)_U^{-1}"]\ar[drr,"\pi_1"',near start]\ar[rrrr,bend left=15,dotted,"\psi"]&[2em]\xi_1^{-1}(U)\ar[rr,"\phi|_{\xi_1^{-1}(U)}"]\ar[dr,"\xi_1"',near start]&[-1em]&[-1em]\xi_2^{-1}(U)\ar[r,"\cong"',"(f_2)_U"]\ar[dl,"\xi_2",near start]&[2em]U\times\mathbb{R}^n\ar[dll,"\pi_1",near start]\\[2em]
&&U
\end{tikzcd}
\end{center}
Hence the map $\psi$ is continuous and a bijection. Moreover for every $x\in U$ there exists some linear map $\psi_x:\mathbb{R}^n\to\mathbb{R}^n$ such that $\psi(x,v)=(x,\psi_x(v))$, and $\psi_x\in GL(n)$, since the following diagram commutes for every $x\in U$:
\vspace*{-1em}
\begin{center}
\begin{tikzcd}
\mathbb{R}^n\ar[r,"\cong"']\ar[rrrrr,bend left=15,"\psi_x"]&\{x\}\times\mathbb{R}^n\ar[r,"\cong"']&\xi_1^{-1}(x)\ar[r,"\cong"',"\phi_x"]&\xi_2^{-1}(x)\ar[r,"\cong"']&\{x\}\times\mathbb{R}^n\ar[r,"\cong"']&\mathbb{R}^n
\end{tikzcd}
\end{center}
Also, $\phi^{-1}|_{\xi_2^{-1}(U)}=(\phi|_{\xi_1^{-1}(U)})^{-1}$ is continuous iff $\psi^{-1}$ is continuous.

Essentially, the problem is reduced to the case of $E_1,E_2$ both being the trivial vector bundle $U\times\mathbb{R}^n$ and $\psi:U\times\mathbb{R}^n\to U\times\mathbb{R}^n$.

We now prove that the function $U\to GL(n)$ with $x\mapsto\psi_x$ is continuous. Let us fix a basis $e_1,\ldots,e_n$ of $\mathbb{R}^n$ and write $\psi_x=(a_{i,j}(x))_{i,j\in[n]}\in GL(n)$ as a matrix in that basis. Then, it suffices to prove that $a_{i,j}:U\to\mathbb{R}$ is continuous for every $i,j\in[n]$, since $GL(n)$ is topologised as a subset of $\mathbb{R}^{n^2}$. This holds, since $a_{i,j}$ can be written as the following composition:
\begin{center}
\begin{tikzcd}
U\ar[r,hook]&U\times\mathbb{R}^n\ar[r,"\psi"]&U\times\mathbb{R}^n\ar[r,"\pi_2"]&\mathbb{R}^n\ar[r,two heads,"\pi_i"]&\mathbb{R}\\[-1.5em]
x\ar[r,mapsto]&(x,e_j)\ar[r,mapsto]&(x,\psi_xe_j)\ar[r,mapsto]&\psi_xe_j\ar[r,mapsto]&e_i^t\psi_xe_j\ar[r,phantom,"="]&[-2em]a_{i,j}(x)
\end{tikzcd}
\end{center}

Now, notice that $GL(n)$ is a topological group, so the inverse function is continuous, i.e. the function $U\to GL(n)$ with $x\mapsto\psi_x^{-1}$ is continuous. This finally means that $\psi^{-1}$ is also continuous, as it is the following composition:
\begin{center}
\begin{tikzcd}
U\times\mathbb{R}^n\ar[r]&U\times GL(n)\times\mathbb{R}^n\ar[r]&U\times\mathbb{R}^n\\[-1.5em]
(x,v)\ar[r,mapsto]&(x,\psi_x^{-1},v)\ar[r,mapsto]&(x,\psi_x^{-1}v)\ar[r,phantom,"="]&[-2em]\psi^{-1}(x,v)
\end{tikzcd}
\end{center}
Thus, $\phi^{-1}$ is also continuous, which completes this proof.
\end{proof}

This means that in order to distinguish between non-isomorphic vector bundles we need to examine them globally and a natural way to start doing this is to introduce the notion of a section. This is a continuous choice of an element inside every fiber over all points of the base space:

\begin{definition} Let $\xi:E\to B$ be a vector bundle. A \emph{cross section} or just \emph{section} $s:B\to E$ is a continuous right inverse of $\xi$, i.e. a continuous function taking each $x\in B$ to an element in $\xi^{-1}(x)$.
\end{definition}

Notice that the same definition of a section applies generally for fiber bundles, but in the case of vector bundles we use the vector structure to measure how many independent sections can be found in a space:

\begin{definition} Let $\xi:E\to B$ be a vector bundle and $s_1,\ldots,s_k:B\to E$ $k$ sections of $\xi$. Then $s_1,\ldots,s_k$ are \emph{nowhere dependent sections} if for each $x\in B$ the vectors $s_1(x),\ldots,s_k(x)$ are linearly independent. In particular, a nowhere dependent section $s$ is also called \emph{nowhere zero section}.
\end{definition}

Obviously, an $n$-plane vector bundle cannot have more than $n$ nowhere dependent sections and the next lemma asserts that the interesting cases arise when there exist less than $n$ nowhere dependent sections.

\begin{lemma}\label{lem:n_sections_makes_trivial} Let $\xi:E\to B$ be an $n$-plane vector bundle. Then $\xi\cong\varepsilon_B^n$ if and only if there exist $n$ nowhere dependent sections of $\xi$.
\end{lemma}
\begin{proof} First of all, it is easy to construct $n$ nowhere dependent sections of $\varepsilon_B^n=B\times\mathbb{R}^n$. Indeed, fix some basis $(e_1,\ldots,e_n)$ of $\mathbb{R}^n$ and let $s_i(x):=(x,e_i)$ for every $x\in B$ and every $i\in[n]$.

On the other hand, let $s_1,\ldots,s_n$ be $n$ nowhere dependent sections of $\xi$. Then, define $\phi:B\times\mathbb{R}^n\to\varepsilon_B^n$ as follows: Fix $e_1,\ldots,e_n$ to be a base of $\mathbb{R}^n$, define $\phi(x,e_i):=s_i(x)\in\xi^{-1}(x)$. Then extend $\phi$ linearly over each $x\in B$,
$\phi\left(x,\sum_{i\in[n]}a_ie_i\right)=\sum_{i\in[n]}a_i\phi(x,e_i)=\sum_{i\in[n]}a_is_i(x)\in\xi^{-1}(x)$
which is by definition linear over each fiber and continuous, since every section is continuous, i.e. $\phi$ is a vector bundle map. Each restriction $\phi_x:\{x\}\times\mathbb{R}^n\to\xi^{-1}(x)$ is a linear isomorphism, since $s_1(x),\ldots,s_n(x)$ are linearly independent and $\xi^{-1}(x)$ is $n$-dimensional. Thus, using Proposition~\ref{prop:local_to_global_iso_vector} we get that $\phi$ is a vector bundle isomorphism.
\end{proof}

Let us now use sections to prove that the tautological bundle of a projective space is not trivial.

\begin{proposition} Let $n\geq1$ be any natural number, then $\gamma_n^1\not\cong\varepsilon_{\mathbb{R}P^n}^1$.
\end{proposition}
\begin{proof} It suffices to prove that there does not exist any section $s$ of $\gamma_n^1$ that is nowhere zero. We prove this by contradiction. Suppose there exists some continuous map
\begin{center}
\begin{tikzcd}
\mathbb{R}P^n\ar[r,"s"]&\mathbb{R}P^n\times\mathbb{R}^{n+1}\\[-1.5em]
l\ar[r,mapsto]&(l,\tilde{s}(l))
\end{tikzcd}
\end{center}
where $\tilde{s}:=\pi_2\circ s$ and $\tilde{s}(l)\in l\setminus\{0\}$ for every $l\in\mathbb{R}P^n$. Let us now precompose $\tilde{s}$ with the quotient map $q:S^n\to\mathbb{R}P^n$ which sends $v\in S^n$ to $\mathrm{span}\{v\}\in\mathbb{R}P^n$ to get $\hat{s}=\tilde{s}\circ q:S^n\to\mathbb{R}^{n+1}$. Notice that $\hat{s}(v)$ is a non-zero vector in the line spanned by $v$, which means in particular that its inner product with $v$ is not zero. This lets us finally define the following continuous map $t:S^n\to\mathbb{R}\setminus\{0\}$:
\begin{center}
\begin{tikzcd}
t\ar[r,phantom,":"]&[-2em]S^n\ar[r,"\left<\hat{s}{,}id\right>"]&\mathbb{R}^{n+1}\times S^n\ar[r,"-^t-"]&\mathbb{R}\setminus\{0\}\\[-1.5em]
&v\ar[r,mapsto]&\left(\hat{s}(v),v\right)\ar[r,mapsto]&\hat{s}(v)^tv
\end{tikzcd}
\end{center}
Notice now that for any $v\in\mathbb{R}^n$,
$t(-v)=\hat{s}(-v)^t(-v)=-\tilde{s}(\mathrm{span}\{-v\})^tv=-\tilde{s}(\mathrm{span}\{v\})^tv=-\hat{s}(v)^tv=-t(v)$,
i.e. $t:S^n\to\mathbb{R}\setminus\{0\}$ is an antipodal map. Since $n\geq1$, $S^n$ is connected and the existence of such a continuous $t$ leads to a contradiction due to the intermediate value theorem. Notice that this is the easiest case of the (BU1b)-version of the Borsuk Ulam theorem as stated in Theorem~2.1.1 in \cite{BU_Matousek}.
\end{proof}

\section{Creating new bundles}
The goal of this section is to describe some ways to create new bundles by combining the data of already known ones. First new bundles are constructed based on continuous maps between the base spaces, disregarding the vector space structure on the fibers and then we construct new bundles by fixing the base space and doing operations on the vector spaces fiber-wise.

\subsection{By combining the base spaces}
In order to briefly ignore the vector space structure of the fibers we describe the next few constructions in terms of fiber bundles, in general.

\begin{proposition} Let $p:E\to B$ be a fiber bundle with fiber $F$, let $B'$ be any topological space and let $g:B'\to B$ be any continuous map. Moreover, let $E':=\{(x,v)\in B'\times E:g(x)=p(v)\}$, $g^*p(x,v)=x$ and $\bar{g}(x,v)=v$ be the space and the maps making the following a pullback square of topological spaces:
\begin{center}
\begin{tikzcd}
E'\ar[r,dotted,"\bar{f}"]\ar[d,dotted,"g^*p"']\ar[rd,phantom,"\lrcorner",near start]&E\ar[d,"p"]\\
B'\ar[r,"g"]&B
\end{tikzcd}
\end{center}
Then $g^*p:E'\to B'$ is a fiber bundle with fiber $F$.
\end{proposition}
\begin{proof} We have to show that $g^*p$ as defined above is locally trivial. Let $x\in B'$, then, since $p$ is a fiber bundle, there exists some open $U$ with $g(x)\in U\subseteq B$ and a homeomorphism $f_U:p^{-1}(U)\to U\times F$ such that $p=\pi_1\circ f_U$. Let $U':=g^{-1}(U)$. This contains $x$ and is open, since $g$ is continuous. Notice that
$(g^*p)^{-1}(U')=\{(x,v)\in U'\times E: g(x)=p(v)\}
=\{(x,v)\in U'\times p^{-1}(U):g(x)=p(v)\}$.
So, we define $f'_{U'}:= id\times(\pi_2\circ f_U):(g^*p)^{-1}(U')\to U'\times F$. This is a continuous map and obviously satisfies the equation $g^*p(x,a)=x=\pi_1\circ f'_{U'}$. In order to prove that it is also a homeomorphism, we prove that it has a continuous inverse map. Define $h_{U'}:U'\times F\to (g^*p)^{-1}(U')$ as follows:
\begin{center}
\begin{tikzcd}
h_{U'}\ar[r,phantom,":"]&[-2em]U'\times F\ar[r,"\left<id_{U'}{,}g\right>\times id_F"]&[3em]U'\times U\times F\ar[r,"id_{U'}\times f_U^{-1}"]&[3em](g^*p)^{-1}(U')\\[-1.5em]
&(x,a)\ar[r,mapsto]&(x,g(x),a)\ar[r,mapsto]&(x,f_U^{-1}(g(x),a))
\end{tikzcd}
\end{center}
This map is continuous well defined. Indeed, $p\big(f_U^{-1}(g(x),a)\big)=\pi_1(g(x),a)=g(x)$, so $(x,f_U^{-1}(g(x),a))\in (g^*p)^{-1}(U')$ for every $(x,a)\in U'\times F$. Moreover, notice that $(f'_{U'}\circ h_{U'})(x,a)=f'_{U'}(x,f_U^{-1}(g(x),a))=(x,\pi_2\circ f_U\circ f_U^{-1}(g(x),a))=(x,a)$ for every $(x,a)\in U'\times F$. Also, for every $(x,v)\in (g^*p)^{-1}(U')$,
\begin{align*}
(h_{U'}\circ f'_{U'})(x,v)&=h_{U'}(x,\pi_2\circ f_U(v))=\big(x,f_U^{-1}(g(x),\pi_2\circ f_U(v))\big)\\
&=\big(x,f_U^{-1}(p(v),\pi_2\circ f_U(v))\big)=\big(x,f_U^{-1}(\pi_1\circ f_U(v),\pi_2\circ f_U(v))\big)\\
&=\big(x,f_U^{-1}\circ f_U(v)\big)=(x,v)
\end{align*}
which completes the proof that $(f'_{U'})^{-1}=h_{U'}$
\end{proof}

\begin{definition} Let $p:E\to B$ be a fiber bundle with fiber $F$ and $g:B'\to B$ any continuous map as in the above proposition. The \emph{induced} fiber bundle of $p$ over $g$ is $g^*p:E'\to B'$, where $E':=\{(x,a)\in B'\times E:g(x)=p(a)\}$ and $g^*p(x,a)=x$.
\end{definition}
The way to imagine the pullback fiber bundle is that we attach to a point $x\in B'$ a copy of $F$ ``in the same way'' it is attached to $g(x)$.

\begin{remark} In the special case where $g:B'\to B$ is injective, then $E'\cong p^{-1}(g(B'))\subseteq E$ and $(g^*p)(v)=g^{-1}(p(v))$ for every $v\in p^{-1}(g(B'))$.
\end{remark}
\begin{proof} The function $g$ being injective means that it has a left inverse, which we denote by $g^{-1}$ here. Hence,
$E'=\{(x,v)\in B'\times E:g(x)=p(v)\}=\{(x,v)\in B'\times E:x=(g^{-1}\circ p)(v)\}
\cong\{v\in E:(g^{-1}\circ p)(v)\in B'\}=p^{-1}(g(B))$.
Factoring $g^*p$ through this homeomorphism we exactly get $(g^*p)(v)=x=g^{-1}(p(v))$, which proves the assertion.
\end{proof}

\begin{definition} Let $p:E\to B$ be a fiber bundle and $A\subseteq B$ any subset of $B$. Then, the \emph{restriction bundle of $p$ on $A$} is $p|A:E'\to A$, where $E'=p^{-1}(A)$ and $g^*p(v)=p|_{p^{-1}(A)}(v)$.
\end{definition}
Notice that $p|A$ has fiber $F$, which means that we restricted $B$ but not $F$.

\begin{proposition} Let $p_1:E_1\to B_1$ and $p_2:E_2\to B_2$ be two fiber bundles with fibers $F_1$ and $F_2$ respectively. Then $p_1\times p_2:E_1\times E_2\to B_1\times B_2$ is a fiber bundle with fiber $F_1\times F_2$.
\end{proposition}
\begin{proof} We have to show that $p_1\times p_2$ is locally trivial. Let $(x_1,x_2)\in B_1\times B_2$. Then, since $p_1$ and $p_2$ are fiber bundles there exist some open sets $U_1\subseteq E_1$ and $U_2\subseteq E_2$ with $x_1\in U_1$ and $x_2\in U_2$ and homeomorphisms $f_{U_1}:p_1^{-1}(U_1)\to U_1\times F_1$ and $g_{U_2}:p_2^{-1}(U_2)\to U_2\times F_2$ such that $p_1=\pi_1\circ f_{U_1}$ and $p_2=\pi_1\circ g_{U_2}$. Let $U=U_1\times U_2$. This contains $(x_1,x_2)$ and is open. Notice that $(p_1\times p_2)^{-1}(U)=p_1^{-1}(U_1)\times p_2^{-1}(U_2)$. So, we define $h_U:=\tau_{2,3}\circ(f_{U_1}\times g_{U_2}):(p_1\times p_2)^{-1}(U)\to U_1\times U_2\times F_1\times F_2$, where $\tau_{2,3}:U_1\times F_1\times U_2\times F_2\to U_1\times U_2\times F_1\times F_2$ is the homeomorphism swapping coordinates $2$ and $3$. This makes $h_U$ a homeomorphism and it is easy to check that $\pi_1\circ h_U(v_1,v_2)=(p_1\times p_2)(v_1,v_2)\in B_1\times B_2$, where $\pi_1$ is the projection on the first two coordinates.
\end{proof}

\begin{definition} Let $p_1:E_1\to B_1$ and $p_2:E_2\to B_2$ be two fiber bundles with fibers $F_1$ and $F_2$ respectively. Then, the \emph{product fiber bundle of $p_1$ and $p_2$} is the usual product map $p_1\times p_2:E_1\times E_2\to B_1\times B_2$.
\end{definition}

\begin{proposition} Let $p_1:E_1\to B_1$ and $p_2:E_2\to B_2$ be two fiber bundles both with fiber $F$ and $F$ respectively. Then $p_1\amalg p_2:E_1\amalg E_2\to B_1\amalg B_2$ is a fiber bundle with fiber $F$.
\end{proposition}
\begin{proof} We have to show that $p_1\amalg p_2:E_1\amalg E_2\to B_1\amalg B_2$ is locally trivial. Let $x\in B_1\amalg B_2$. Then $x\in B_i$ for some $i\in\{1,2\}$ and there exists some open $U\subseteq B_i$ containing $x$ and a homeomorphism $f_U:p_i^{-1}(U)\to U\times F$. Notice that $(p_1\amalg p_2)^{-1}(U)=p_i^{-1}(U)$, so $f_U$ itself is a local homeomorphism for $p_1\amalg p_2$.
\end{proof}

\begin{definition} Let $p_1:E_1\to B_1$ and $p_2:E_2\to B_2$ be two fiber bundles both with fiber $F$. Then, the \emph{coproduct fiber bundle of $p_1$ and $p_2$} is the usual coproduct map $p_1\amalg p_2:E_1\amalg E_2\to B_1\amalg B_2$.
\end{definition}
Notice that the product of the base spaces always defines a product of fiber bundles, but the disjoint union of two base spaces defines a fiber bundle only if the fibers of the two components were already isomorphic.

\subsection{By combining the fibers}
Aiming to exploit the vector space structure of the fibers, we fix some base space $B$ and define new bundles over $B$ using fiber-wise operations. In this section we construct some vector bundle isomorphisms, by constructing a bundle map that is linear on the fibers, proving that the desired isomorphism is true fiberwise and then using proposition~\ref{prop:local_to_global_iso_vector} to argue that this bundle map is in fact a bundle isomorphism.

\begin{definition} Let $\eta:E(\eta)\to B$ be an $n$-plane vector bundle and $\xi:E(\xi)\to B$ a $k$-plane vector bundle, where $E(\xi)\subseteq E(\eta)$. Then $\xi$ is a \emph{subbundle of $\eta$} and we write $\xi\leq\eta$, if $\xi^{-1}(x)<\eta^{-1}(x)$ as vector spaces for every $x\in B$. Notice that in this case $k\leq n$.
\end{definition}

\begin{definition} Let $\xi_1:E(\xi_1)\to B$ be an $n_1$-plane vector bundle, $\xi_2:E(\xi_2)\to B$ an $n_2$-plane vector bundle and $\eta:E(\eta)\to B$ an $n$-plane vector bundle, where $E(\xi_1),E(\xi_2)\subseteq E(\eta)$. Then $\eta$ is the \emph{whitney sum of $\xi_1$ and $\xi_2$} and we write $\eta=\xi_1\oplus\xi_2$, if $\eta^{-1}(x)=\xi_1^{-1}(x)\oplus\xi_2^{-1}(x)$ as vector spaces, for every $x\in B$. Notice that in this case $n_1+n_2=n$.
\end{definition}

Next we examine how to explicitly describe the whitney sum of two given bundles:
\begin{proposition} Let $\xi_1:E(\xi_1)\to B$ be an $n_1$-plane vector bundle and $\xi_2:E(\xi_2)\to B$ an $n_2$-plane vector bundle. Then, $\xi_1\oplus\xi_2\cong d^*(\xi_1\times\xi_2)$, where $d:B\to B\times B$ is the diagonal map taking $x\in B$ to $(x,x)\in B\times B$.
\end{proposition}
\begin{proof} Notice that $E(d^*(\xi_1\times\xi_2)=\big\{(x,v_1,v_2)\in B\times E(\xi_1)\times E(\xi_2):(x,x)=(\xi_1(v_1),\xi_2(v_2))\big\}$ and define the following map:
\begin{center}
\begin{tikzcd}
E(d^*(\xi_1\times\xi_2))\ar[r,"\phi"]&[5em]E(\xi_1\oplus\xi_2)\\[-1.5em]
(x,v_1,v_2)\ar[r,mapsto]&v_1+v_2
\end{tikzcd}
\end{center}
It is easy to see that this is a bundle map and linear over each fiber. Next, notice that its restriction on the fibers $\phi_x=\phi|:\big(d^*(\xi_1\times\xi_2)\big)^{-1}(x)\to(\xi_1\oplus\xi_2)^{-1}(x)$ is a linear isomorphism. Indeed, for any $x\in B$, its inverse is the map $\psi_x$ taking some $v=v_1+v_2\in(\xi_1\oplus\xi_2)^{-1}(x)=\xi_1^{-1}(x)\oplus\xi_2^{-1}(x)$ to $(x,v_1,v_2)\in E(d^*(\xi_1\times\xi_2))$. Due to Proposition~\ref{prop:local_to_global_iso_vector}, $\phi$ is a vector bundle isomorphism.
\end{proof}
\begin{remark} If we define $\zeta:\big\{(v_1,v_2)\in E(\xi_1)\times E(\xi_2):\xi_1(v_1)=\xi_2(v_2)\big\}\to B$ to be the vector bundle with $\zeta(v_1,v_2)=\xi_1(v_1)$, then $\xi_1\oplus\xi_2\cong\zeta$.
\end{remark}
\begin{proof} Notice that
$E(d^*(\xi_1\times\xi_2))=\big\{(x,v_1,v_2)\in B\times E(\xi_1)\times E(\xi_2):(x,x)=(\xi_1(v_1),\xi_2(v_2))\big\}
=\big\{(\xi_1(v_1),v_1,v_2)\in B\times E(\xi_1)\times E(\xi_2):\xi_1(v_1)=\xi_2(v_2)\big\}
\cong\big\{(v_1,v_2)\in E(\xi_1)\times E(\xi_2):\xi_1(v_1)=\xi_2(v_2)\big\}=E(\zeta)$,
where the homeomorphism is a vector bundle isomorphism.
\end{proof}
This is the description used in the remainder of the thesis.
\begin{lemma}\label{lem:induced_sum} Let $\eta,\xi$ be vector bundles over $B$ and $f:B'\to B$ be any continuous function. Then $f^*(\eta\oplus\xi)\cong(f^*\eta)\oplus(f^*\xi)$.
\end{lemma}
\begin{proof} Notice that $E\big((f^*\eta)\oplus(f^*\xi)\big)=\big\{(x,a,y,b)\in B'\times E(\eta)\times B'\times E(\xi):f(x)=\eta(a)\text{ and }f(y)=\xi(b)\text{ and }x=y\big\}\cong\big\{(x,a,b)\in B'\times E(\eta)\times E(\xi):\eta(a)=\xi(b)=x\big\}=E\big(f^*(\eta\oplus\xi)\big)$, where the homeomorphism is a bundle map and a linear isomorphism fiber-wise. So, using Proposition~\ref{prop:local_to_global_iso_vector} it is a vector bundle isomorphism.
\end{proof}

\begin{proposition} Let $\xi_1,\xi_2,\eta_1,\eta_2$ be vector bundles over $B$ and $\phi_1:E(\xi_1)\to E(\eta_1)$, $\phi_2:E(\xi_2)\to E(\eta_2)$ vector bundle maps. Moreover, let $\phi_1\oplus\phi_2:E(\xi_1\oplus\xi_2)\to E(\eta_1\oplus\eta_2)$ be the function taking $(v_1,v_2)$ to $(\phi_1(v_1),\phi_2(v_2))$. Then $\phi_1\oplus\phi_2$ is a vector bundle map.
\end{proposition}
\begin{proof} First of all, $\phi_1\oplus\phi_2$ is well defined, since $(\eta_1\circ\phi_1)(v_1)=\xi_1(v_1)=\xi_2(v_2)=(\eta_2\circ\phi_2)(v_2)$. Also it is a bundle map, since $\big((\eta_1\oplus\eta_2)\circ(\phi_1\oplus\phi_2)\big)(v_1,v_2)=(\eta_1\circ\phi_1)(v_1)=\xi_1(v_1)=(\xi_1\oplus\xi_2)(v_1,v_2)$. It is also trivial to check that $\phi_1\oplus\phi_2$ is linear on each fiber.
\end{proof}
\begin{definition}\label{def:wh_sum_map} Let $\xi_1,\xi_2,\eta_1,\eta_2$ be vector bundles over $B$ and $\phi_1:E(\xi_1)\to E(\eta_1)$, $\phi_2:E(\xi_2)\to E(\eta_2)$ vector bundle maps. Then the \emph{whitney sum of $\phi_1,\phi_2$} is $\phi_1\oplus\phi_2:E(\xi_1\oplus\xi_2)\to E(\eta_1\oplus\eta_2)$, where $\phi_1\oplus\phi_2(v_1,v_2)=(\phi_1(v_1),\phi_2(v_2))$.
\end{definition}

\begin{definition} Let $\eta:E(\eta)\to B$ be an $n$-plane vector bundle, $\xi:E(\xi)\to B$ a $k$-plane vector bundle and $\theta:E(\theta)\to B$ an $m$-plane vector bundle, where $E(\xi)\subseteq E(\eta)$. Then $\theta$ is the \emph{quotient vector bundle of $\eta$ and $\xi$} and we write $\theta=\sfrac{\eta}{\xi}$, if $\theta^{-1}(x)=\sfrac{\eta^{-1}(x)}{\xi^{-1}(x)}$ as vector spaces, for every $x\in B$. Notice that in this case $n-k=m$.
\end{definition}
Next we examine how to explicitly define the quotient bundles given a bundle and a subbundle.
\begin{proposition}\label{prop:quotient_bundle} Let $\eta:E(\eta)\to B$ be an $n$-plane vector bundle and $\xi:E(\xi)\to B$ a $k$-plane vector subbundle, i.e. $\xi\leq\eta$. Moreover, let $\zeta:\sfrac{E(\eta)}{\sim}\to B$, where $v_1\sim v_2$ if and only if $\eta(v_1)=\eta(v_2)$ and $v_1-v_2\in\xi^{-1}(\eta(v))$ be the vector bundle with $\zeta([v])=\xi(v)$. Then, $\sfrac{\eta}{\xi}\cong\zeta$.
\end{proposition}
\begin{proof} First of all notice that $\zeta$ is well defined, since $\xi(v)$ is the same for every choice of class representative $v$. Next, define the following map:
\begin{center}
\begin{tikzcd}
\sfrac{E(\eta)}{\sim}\ar[r,"\phi"]&E(\sfrac{\eta}{\xi})\\[-1.5em]
[v]\ar[r,mapsto]&v+\xi^{-1}(\eta(v))
\end{tikzcd}
\end{center}
Notice that this is well defined, by the definition of the equivalence relation. Moreover it is easy to see that it is a bundle map and also linear over each fiber. Next, we prove that it is in fact a vector isomorphism over each fiber. To do this, we define for every $x\in B$ a continuous map $\psi_x:(\sfrac{\eta}{\xi})^{-1}(x)\to\zeta^{-1}(x)$ and prove that it is the inverse of $\phi_x$. Let $v+\xi^{-1}(x)\in\sfrac{\eta^{-1}(x)}{\xi^{-1}(x)}$, then we define $\psi_x(v+\xi^{-1}(x))=[v]$. This is again well defined and is easy to see that it is the desired inverse. Using remark~\ref{rem:vector_subcover} we get that $\phi$ is a vector bundle isomorphism.
\end{proof}

At this point it is natural to wonder if given two bundles $\xi\leq\eta$ it is always possible to decompose $\eta$ in the whitney sum of $\xi$ and $\sfrac{\eta}{\xi}$. Trying to build a vector bundle isomorphism from $\xi\sfrac{\eta}{\xi}$ to $\eta$, we see however that we need to be able to choose an inner product on each fiber that changes continuously. So the next question arises naturally:

\section{Do we always have a dot product?}
\begin{definition} Let $V$ be an $\mathbb{R}$ vector space. Then $V$ is an \emph{Euclidean vector space}, if there exists a symmetric, bilinear function $\beta:V\times V\to\mathbb{R}$ such that $\beta(v,v)\geq0$ and $\beta(v,v)=0$ only if $v=0$. In this case $\beta$ is called \emph{inner product} and $\beta(u,v)$ denoted by $u\cdot v$.
\end{definition}

\begin{examples}\label{ex:infinite_euclidean_space}
\begin{i_enum}\item For every $n\in\mathbb{N}$, the usual inner product on $\mathbb{R}^n$ is defined by $\beta_n(u,v)=u^tv$.
\item Notice that the inclusion $\mathbb{R}^n\subseteq\mathbb{R}^{n+1}$ taking $v$ to $\bar{v}:=(v^t|0)^t$ respects the usual inner product, i.e. $\beta_n(u,v)=\beta_{n+1}(\bar{u},\bar{v})$. This means, that for every $u,v\in\mathbb{R}^{\infty}$, we can define the inner product $u^tv=\beta(u,v):=\beta_n(u,v)$ for some $n$ big enough such that $u,v\in\mathbb{R}^n\subseteq\mathbb{R}^{\infty}$. This makes $\mathbb{R}^{\infty}$ a euclidean space.
\end{i_enum}
\end{examples}

\begin{definition} Let $\xi:E\to B$ be a vector bundle. Then $\xi$ is a \emph{Euclidean vector bundle}, if there exists a continuous function $\beta:E(\xi\oplus\xi)\to\mathbb{R}$ the restriction of which on $(\xi\oplus\xi)^{-1}(x)$ makes $\xi^{-1}(x)$ a euclidean vector space for each $x\in B$.
\end{definition}

\begin{example} Let $B$ be any topological space and $\varepsilon_B^n:B\times\mathbb{R}^n\to B$ the trivial vector bundle over $B$. Then the usual inner product of $\mathbb{R}$ makes $\varepsilon_B^n$ a euclidean vector bundle: Let $(x,a),(x,b)\in\{x\}\times\mathbb{R}^n$. Then $(x,a)\cdot(x,b)=a^tb\in\mathbb{R}$.
\end{example}

\begin{definition} Let $\xi:E\to B$ be a euclidean vector bundle and $s_1,\ldots,s_k:B\to E$ be $k$ sections of $\xi$. Then $s_1,\ldots,s_k$ are \emph{orthonormal sections} if for each $x\in B$ the vectors $s_1(x),\ldots,s_k(x)$ are orthonormal, i.e. $s_i(x)\cdot s_j(x)=\delta_{i,j}$.
\end{definition}

If $V$ is a finitely dimensional euclidean vector space, then we can always transform a basis of $V$ into an orthonormal basis of $V$ using the Gram-Schmidt process. This also holds for euclidean vector bundles:
\begin{lemma}\label{lem:orthonormal_sections} Let $\xi:E\to B$ be a euclidean $n$-plane vector bundle which has $k$ nowhere dependent sections. Then, there exist $k$ orthonormal sections of $\xi$.
\end{lemma}
\begin{proof} Let $s_1,\ldots,s_k:B\to E$ be any $k$ nowhere dependent sections of $\xi$. Then, we define $t_1,\ldots,t_k:B\to E$ inductively by using the Gram-Schmidt process:
\[t_l(x):=\frac{1}{\left\|s_l(x)-\sum_{i=1}^{l-1}(t_i(x)\cdot s_i(x))t_i(x)\right\|}\left(s_l(x)-\sum_{i=1}^{l-1}(t_i(x)\cdot s_i(x))t_i(x)\right)\]
where the linear operations are well defined, since $s_i(x),t_i(x)\in\xi^{-1}(x)$. Since $t_i:B\to E$ is continuous for every $i$, $t_1,\ldots,t_k$ are $k$ sections for which $t_i(x)\cdot t_j(x)=\delta_{i,j}$ by the properties of the Gram-Schmidt process, which proves the assertion.
\end{proof}
\begin{remark}\label{rem:trivial_euclidian} In particular, using Lemma~\ref{lem:n_sections_makes_trivial}, we get that an $n$-plane euclidean vector bundle $\xi$ is trivial, i.e. $\xi\cong\varepsilon_B^n$ iff there exist $n$ orthonormal sections of $\xi$.
\end{remark}

In the particular case of euclidean vector bundles, we can always define a natural complementary bundle of a given subbundle:
\begin{proposition} Let $\eta:E(\eta)\to B$ be an $n$-plane euclidean vector bundle and $\xi:E(\xi)\to B$ a $k$-plane euclidean vector subbundle of $\eta$. Moreover, let $E:=\{v\in E(\eta):\forall u\in\xi^{-1}(\eta(v))\ v\cdot u=0\}$ and $\zeta:E\to B$ the map taking $v\in E$ to $\eta(v)$. Then $\zeta$ is a $(n-k)$-plane vector bundle.
\end{proposition}
\begin{proof} We need to show that $\zeta$ is locally trivial. Let us fix some $x_0\in B$. Then, there exist open sets $U_1,U_2\subseteq B$ both containing $x_0$ such that $\eta|U_1$ is an $n$-plane trivial vector bundle and $\xi|U_2$ is a $k$-plane trivial vector bundle. Using Remark~\ref{rem:vector_subcover} we restrict them further on $U:=U_1\cap U_2$ and they remain trivial. For $\eta$ we use the definition of local triviality, which gives a vector bundle isomorphism $f_U:\eta^{-1}(U)\to U\times\mathbb{R}^n$, whereas for $\xi$ we use Remark~\ref{rem:trivial_euclidian}, which gives $k$ orthonormal sections of $\xi$, namely $s_1,\ldots,s_k:U\to\xi^{-1}(U)$. This means that for each $x\in U$ the vectors $v_1(x):=(\pi_2\circ f_U\circ s_1)(x),\ldots,v_k(x):=(\pi_2\circ f_U\circ s_k)(x)$ are $k$ linearly independent vectors in $\mathbb{R}^n$. We extend this $k$-frame to a basis, specifically at the point $x=x_0$, i.e. we choose any $n-k$ vectors $v_{k+1},\ldots,v_n\in\mathbb{R}^n$ such that
$\det\big(v_1(x_0),\ldots,v_k(x_0),v_{k+1},\ldots,v_n\big)>0$.

Define $D:U\to\mathbb{R}$ to be $D(x):=\det\big(v_1(x),\ldots,v_k(x),v_{k+1},\ldots,v_n\big)$ for all $x\in U$. Since the determinant, the $f_U$ and the sections are continuous, $D$ is also continuous and thus there exists some open neighborhood $V\subseteq U$ of $x_0$ such that $D(x)>0$ for every $x\in V$, i.e. $v_1(x),\ldots,v_k(x),v_{k+1},\ldots,v_n$ are linearly independent for every $x\in B$. Define now $\bar{s}_i(x):=f_U^{-1}(x,v_i)\in\eta^{-1}(x)$ for every $x\in B$ and every $i\in\{k+1,\ldots,n\}$. Since $f_U$ is a linear isomorphism on every fiber, $s_1(x),\ldots,s_k(x),\bar{s}_{k+1}(x),\ldots,\bar{s}_n(x)$ are $n$ nowhere dependent sections. Using the Gramm-Schmidt process, as in Lemma~\ref{lem:orthonormal_sections}, we construct $n-k$ sections $s_{k+1},\ldots,s_n$ such that $s_1,\ldots,s_n$ are $n$ orthonormal sections. This means that $s_{k+1},\ldots,s_n$ are $n-k$ orthonormal sections of $\zeta|V$, i.e. $\zeta|V$ is isomorphic to the trivial $(n-k)$-plane vector bundle.
\end{proof}

\begin{definition} Let $\eta:E(\eta)\to B$ be an $n$-plane euclidean vector bundle and $\xi:E(\xi)\to B$ a $k$-plane euclidean vector subbundle of $\eta$. Then, the \emph{orthogonal complement of $\xi$ in $\eta$} is $\xi^{\perp_{\eta}}:E\to B$ (or just $\xi^{\perp}$, when $\eta$ is clear) where $E:=\{v\in E(\eta):\forall u\in\xi^{-1}(\eta(v))\ v\cdot u=0\}$ and $\xi^{\perp_{\eta}}(v)=\eta(v)$.
\end{definition}

\begin{proposition}\label{prop:tangent_normal_vb} Let $M$ be a smooth manifold, embedded in $\mathbb{R}^d$, $\tau_M:TM\to M$ its tangent bundle and $\nu_{M,d}:N_dM\to M$ its normal bundle. Then $\nu_{M,d}\cong\tau_M^{\perp_{\varepsilon_M^d}}$
\end{proposition}
\begin{proof}
%TODO
\end{proof}

\begin{lemma} Let $\eta:E(\eta)\to B$ be an $n$-plane euclidean vector bundle and $\xi:E(\xi)\to B$ a $k$-plane euclidean vector subbundle of $\eta$. Then $\xi^{\perp_{\eta}}\cong\sfrac{\eta}{\xi}$.
\end{lemma}
\begin{proof} Using Proposition~\ref{prop:quotient_bundle}, we write $E(\sfrac{\eta}{\xi})=\sfrac{E(\eta)}{\sim}$ for $v_1\sim v_2$ iff $\eta(v_1)=\eta(v_2)$ and $v_1-v_2\in\xi^{-1}(\eta(v_1))$. Moreover, we also write $\sfrac{eta}{\xi}([v])=\eta(v)$. Let us then define the following map:
\begin{center}
\begin{tikzcd}
E(\xi^{\perp_{\eta}})\ar[r,"\phi"]&\faktor{E}{\sim}\\[-1.5em]
v\ar[r,mapsto]&\left[v\right]
\end{tikzcd}
\end{center}
This is a bundle map, linear on each fiber. In order to prove that this is in fact a linear isomorphism fiberwise, notice that $\phi_x(v)=0$ means that $v\in\xi^{-1}(x)$, but since $v\in E(\xi^{\perp_{\eta}})$, we also have that $v\cdot u=0$ for every $u\in\xi^{-1}(x)$. Thus, $u=0$. Hence, $\ker\phi_x=\{0\}$ and since $\dim_{\mathbb{R}}(\xi^{\perp_{\eta}})^{-1}(x)=\dim_{\mathbb{R}}(\sfrac{\eta}{\xi})^{-1}(x)=n-k$, $\phi_x$ is a linear isomorphism. Due to Proposition~\ref{prop:local_to_global_iso_vector} $\phi$ is a vector bundle isomorphism.
\end{proof}

This lets us decompose some euclidean vector bundle to a whitney sum, provided we find some vector subbundle of it:
\begin{proposition}\label{prop:vb_decomposition} Let $\eta:E(\eta)\to B$ be an $n$-plane euclidean vector bundle and $\xi:E(\xi)\to B$ a $k$-plane euclidean vector subbundle of $\eta$. Then $\eta\cong\xi\oplus\xi^{\perp_{\eta}}$.
\end{proposition}
\begin{proof} Let us define the following map:
\begin{center}
\begin{tikzcd}
E(\xi\oplus\xi^{\perp_{\eta}})\ar[r,"\phi"]&E(\eta)\\[-1.5em]
(v_1,v_2)\ar[r,mapsto]&v_1+v_2
\end{tikzcd}
\end{center}
It is easy to see that this is a bundle map which is also a linear isomorphism on each fiber. Thus, due to Proposition~\ref{prop:local_to_global_iso_vector}, $\phi$ is a vector bundle isomorphism.
\end{proof}
\begin{remark}\label{rem:vb_decomposition} For every euclidean vector bundle $\eta$ and for any $\xi\leq\eta$ we have the following decomposition $\eta\cong\xi\oplus\sfrac{\eta}{\xi}$.
\end{remark}

At this point the question about decomposing a vector bundle is positively answered, provided that the vector bundle in question is a euclidean vector bundle. A very natural question now is how restrictive the notion of a euclidean vector is. As it is shown at the end of the next section, a vector bundle admits a euclidean vector bundle structure at least when the base space is paracompact.

\section{Paracompactness}
\begin{definition} Let $X$ be a topological space and $\mathcal{U}=\{U_i\}_{i\in I}$ be a collection of subsets of $X$. Then $\mathcal{U}$ is \emph{locally finite} if for every $x\in X$ there exists some open $V\subseteq X$ containing $x$ such that $U_i\cap V\neq\emptyset$ only for finitely many $i$.
\end{definition}
\begin{definition} Let $X$ be a topological space and $\mathcal{U},\mathcal{V}$ be two collection of subsets of $X$. Then $\mathcal{V}$ is a \emph{refinement of $\mathcal{U}$} if for every $V\in\mathcal{V}$ there exists some $U\in\mathcal{U}$ such that $V\subseteq U$.
\end{definition}
\begin{definition} A topological space $X$ is \emph{paracompact} if $X$ is Hausdorff and for every open cover $\mathcal{U}$ of $X$ there exists some locally finite open cover $\mathcal{V}$ of $X$ that is a refinement of $\mathcal{U}$.
\end{definition}

\begin{proposition}\label{prop:paracompact_closed_subset} Let $X$ be a paracompact space and $A\subset X$ any closed subset. Then $A$ is paracompact.
\end{proposition}
\begin{proof} Indeed, let $\{U_i\}_{i\in I}$ be an open cover of $A$, i.e. there exists some open collection $\{V_i\}_i$ of subsets of $X$, such that $U_i=A\cap V_i$. Let $W_i:=V_i\cup(X\setminus A)$. Then $\{W_i\}_{i\in I}$ is an open cover of $X$. Since $X$ is paracompact, there exists some locally finite open cover $\{Z_j\}_{j\in J}$ of $X$, that is a refinement of $\{W_i\}_{i\in I}$. Then $\{Z_j\cap A\}_{j\in J}$ is a locally finite open cover of $A$, that is a refinement of $\{U_i\}_{i\in I}$.
\end{proof}

\begin{proposition}\label{prop:paracompact_times_compact} Let $X$ be a paracompact space and $K$ a compact space. Then $X\times K$ is a paracompact space.
\end{proposition}
\begin{proof} Let $\{U_i\}_{i\in I}$ be an open cover of $X\times K$. Then, for each $(x,a)\in X\times K$ there exists some open $U_i\ni(x,k)$. So, there exists some open $A_{x,k}\subseteq X$ and open $B_{x,k}\subseteq K$, such that $A_{x,k}\times B_{x,k}\subseteq U_i$. Notice that $\{A_{x,k}\times B_{x,k}\}_{(x,k)\in X\times K}$ is an open cover of $X\times K$ that is a refinement of $\{U_i\}_{i\in I}$. Let us now fix some $x_0\in X$ and notice that the family $\{B_{x_0,k}\}_{k\in K}$ is an open cover of $K$. Since $K$ is compact, there exist only finitely many $k_1(x_0),\ldots,k_{r(x_0)}(x_0)\in K$ such that $\{B_{x_0,k}\}_{k\in\{k_1(x_0),\ldots,k_{r(x_0)}(x_0)\}}$ is an open cover of $K$. Define now $A_{x_0}:=\bigcap_{k\in\{k_1(x_0),\ldots,k_{r(x_0)}(x_0)\}}A_{x_0,k}\subseteq X$. This is a finite intersection of open sets, which contains $x_0$, so it is an open set. So, we get an open cover $\{A_x\}_{x\in X}$ of $X$. Since $X$ is paracompact, there exists some locally finite open cover $\{L_j\}_{j\in J}$ of $X$ that is a refinement of $\{A_x\}_{x\in X}$, i.e. for every $j\in J$ there exists some $x=x(j)\in X$ such that $L_j\subseteq A_{x(j)}$. So, we now define $\big\{L_j\times B_{x(j),k}\big\}_{j\in J, k\in\{k_1(x(j)),\ldots,k_{r(x(j))}(x(j))\}}$, which is a locally finite open cover of $X\times K$ and a refinement of $\{A_{x,k}\times B_{x,k}\}_{(x,k)\in X\times K}$.
\end{proof}

Paracompact spaces were first defined in \cite{paracompactness}, where it was also proven that they are $T_4$:
\begin{theorem}[Dieudonn\'e] Let $X$ be a paracompact space, then $X$ is normal.
\end{theorem}

Using the fact that paracompact spaces are normal together with Urysohn's lemma, it can be proven that any given cover not only has a finite open subcover, but there exists one such subcover that also has a subordinate partition of unity. For the proof of this, the reader can refer to Theorem~12.8 of \cite{bredon}.
\begin{definition} Let $X$ be a topological space and $\mathcal{F}=\{f_i:X\to[0,1]\}_{i\in I}$ be a collection of continuous maps. Then, $\mathcal{F}$ is a \emph{partition of unity of $X$} if for every $x\in X$ there exists some open $U\subseteq X$ containing $x$ such that $f_i|_U\equiv0$ for all but finitely many $i\in I$ and $\sum_{i\in I}f_i(x)=1$.
\end{definition}
\begin{remark} For every $f\in\mathcal{F}$ we are usually interested in the \emph{support of $f$}, which is defined as $\mathrm{supp}(f):=\left\{x\in X:f(x)\neq0\right\}^-$.
\end{remark}
\begin{definition} Let $X$ be a topological space, $\mathcal{U}=\{U_i\}_{i\in I}$ be a collection of subsets of $X$ and $\mathcal{F}=\{f_i:X\to[0,1]\}_{i\in I}$ a partition of unity of $X$ on the same indices. Then $\mathcal{F}$ is \emph{subordinate of $\mathcal{U}$}, if $\mathrm{supp}(f_i)\subseteq U_i$ for every $i\in I$.
\end{definition}
\begin{theorem}\label{thm:paracompact_partition_of_unity} Let $X$ be a paracompact topological space and let $\mathcal{U}$ be an open cover of $X$, then there exists some locally finite open cover $\mathcal{V}$ of $X$ that is a refinement of $\mathcal{U}$ and a partition of unity $\mathcal{F}$ that is subordinate of $\mathcal{V}$.
\end{theorem}

\begin{proposition}\label{prop:paracompact_is_euclidean} Let $B$ be a paracompact topological space and $\xi:E(\xi)\to B$ an $n$-plane vector bundle. Then, $\xi$ is a euclidean vector bundle.
\end{proposition}
\begin{proof} Our goal is to define a continuous $\beta:E(\xi\oplus\xi)\to\mathbb{R}$, the restriction of which on $(\xi\oplus\xi)^{-1}(x)$ is an inner product in $\xi^{-1}(x)$. For each $x\in B$ there exists some open $U_x\subseteq B$ containing $x$ and a vector bundle isomorphism $f_{U_x}:\xi^{-1}(U_x)\to U_x\times\mathbb{R}^n$. Notice that $\{U_x\}_{x\in B}$ is an open cover of $X$. Since $B$ is paracompact, Theorem~\ref{thm:paracompact_partition_of_unity} gives us a locally finite open cover $\mathcal{V}=\{V_i\}_{i\in I}$ of $X$ that is a refinement of $\{U_x\}_{x\in B}$ and a partition of unity $\mathcal{F}=\{h_i\}_{i\in I}$ that is subordinate of $\mathcal{V}$. Since $\mathcal{V}$ is a refinement, every $V_i\in\mathcal{V}$ is contained inside some $U_x$ and thus, using Remark~\ref{rem:vector_subcover} the restriction $f_{V_i}:\xi^{-1}(V_i)\to V_i\times\mathbb{R}^n$ of $f_{U_x}$ is a vector bundle isomorphism. Recall that we can write $E(\xi\oplus\xi)=\left\{(v_1,v_2)\in E(\xi)\times E(\xi):\xi(v_1)=\xi(v_2)\right\}$ and $(\xi\oplus\xi)(v_1,v_2)=\xi(v_1)$. Then, notice that $(\xi\oplus\xi)^{-1}(V_i)=E(\xi|V_i\oplus\xi|V_i)$ and let $f_{V_i}\oplus f_{V_i}:E(\xi|V_i\oplus\xi|V_i)\to V_i\times\mathbb{R}^n\times\mathbb{R}^n$ as in Definition~\ref{def:wh_sum_map}, taking $(v_1,v_2)$ to $(\xi_1(v_1),(\pi_2\circ f_{V_i})(v_1),(\pi_2\circ f_{V_i})(v_2))$. Moreover, let $\gamma:\mathbb{R}^n\times\mathbb{R}^n\to\mathbb{R}$ be the usual dot product of $\mathbb{R}^n$, i.e. $\gamma(u_1,u_2)=u_1^tu_2$. Then, we can define the following continuous map:
\begin{center}
\begin{tikzcd}
\beta_i\ar[r,phantom,":"]&[-2.5em](\xi\oplus\xi)^{-1}(V_i)\ar[r,"f_{V_i}\oplus f_{V_i}"]&V_i\times\mathbb{R}^n\times\mathbb{R}^n\ar[r,"(h_i\circ\pi_1)\cdot(\gamma\circ\pi_{[2,3]})"]&\mathbb{R}\\[-1.5em]
&(v_1,v_2)\ar[rr,mapsto]&&h_i(\xi_1(v_1))\cdot(\pi_2\circ f_{V_i})(v_1)^t(\pi_2\circ f_{V_i})(v_2)
\end{tikzcd}
\end{center}
If we fix some $x\in V_i$, $\beta_i$ restricted on $\xi^{-1}(x)\times\xi^{-1}(x)$ is bilinear and symmetric, since $h_i(\xi_1(v_1))$ is constant. Moreover, $\beta_i(v,v)\geq0$, since $h_i(x)\geq0$. Now, we extend $\beta_i$ to be $\equiv0$ for every $v\in E(\xi\oplus\xi)\setminus\big((\xi\oplus\xi)^{-1}(V_i)\big)$. Since $\mathrm{supp}(h_i)\subseteq V_i$, this is a continuous extension, which we also call $\beta_i$. Now, we define $\beta:=\sum_{i\in I}\beta_i:E(\xi\oplus\xi)\to\mathbb{R}$, where the sum is finite for each input, since $\mathcal{V}$ is locally finite and $\mathcal{F}$ is subordinate of $\mathcal{V}$. It is also continuous, non-negative and over each fiber symmetric and bilinear. It only remains to show that $\beta(v,v)\neq0$ for every $v\neq0$. Let us fix a $v\in E(\xi)$ and an open $U\subseteq B$ containing $\xi(v)$ such that $U\cap V_i\neq\emptyset$ only for finitely many $i\in I$. Let $V_{i_1},\ldots,V_{i_k}\in\mathcal{V}$ be the only sets non-trivially intersecting $U$. Then,  $\beta(v,v)=\beta_{i_1}(v,v)+\cdots+\beta_{i_k}(v,v)=\sum_{j=1}^kh_{i_j}(\xi_1(v))\cdot\big\|(\pi_2\circ f_{V_{i_j}})(v)\big\|^2$.

Define now $a=\min_{j\in[k]}\left\{\big\|(\pi_2\circ f_{V_{i_j}})(v)\big\|\right\}$ and notice that $a>0$, since $\pi_2\circ f_{V_i}$ is a linear isomorphism for every $i\in I$ and $v\neq0$. Hence, $\beta(v,v)\geq a^2\sum_{j=1}^kh_{i_j}(\xi(v))=a^2\sum_{i\in I}h_i(\xi(v))=a^2>0$, since $\mathcal{F}$ is a partition of unity.
\end{proof}

\begin{lemma}\label{lem:paracompact_countable_cover} Let $B$ be a paracompact space and $\xi:E(\xi)\to B$ any vector bundle. Then, there exists some locally finite countable open cover $\{U_k\}_{k\in\mathbb{N}}$ of $B$ such that $\xi|U_k$ is a trivial vector bundle for every $k\in\mathbb{N}$.
\end{lemma}
\begin{proof} Since $\xi$ is a vector bundle, there exists some open cover $\{U_x\}_{x\in B}$ such that $\xi|U_x$ is trivial for every $x\in B$. Using theorem~\ref{thm:paracompact_partition_of_unity}, we can find a locally finite open cover $\mathcal{V}=\{V_{\alpha}\}_{\alpha\in A}$ of $B$ that is a refinement of $\{U_x\}_{x\in B}$ and a partition of unity $\mathcal{F}=\{u_{\alpha}:B\to[0,1]\}$, that is subordinate of $\mathcal{V}$. Notice that Remark~\ref{rem:vector_subcover} gives us that $\xi|V$ is a trivial vector bundle for every $V\in\mathcal{V}$, since it is a subcover of $\{U_x\}_{x\in B}$. If $A$ is finite, $\mathcal{V}$ is a bundle with the desired properties. Otherwise, for every finite $S\subseteq A$ we define now
\[U(S):=\big\{x\in B: \forall\alpha\in S\ \forall\beta\in A\setminus S\ u_{\alpha}(x)>u_{\beta}(x)\big\}\]
Notice that $U(S)$ is open. Indeed, for $x\in U(S)$, since $\mathcal{F}$ is a partition of unity, there exists an open $X\subseteq B$ and only finitely many indices $\alpha\in A$ such that $u_{\alpha}|_X>0$, say $\{\alpha_1,\ldots,\alpha_k\}$. Hence, $S\subseteq\{\alpha_1,\ldots,\alpha_k\}$ and if we set $T=\{\alpha_1,\ldots,\alpha_k\}\setminus S$, we get that $x\in X\cap\bigcap_{\alpha\in S,\beta\in T}(u_{\alpha}|_X-u_{\beta}|_X)^{-1}\big((0,1]\big)\subseteq U(S)$ and since $S,T$ are finite, this is an open set. Since this is true for every $x\in U(S)$, $U(S)$ is open.

Moreover, notice that $\{U(S)\}_{S\in\mathcal{S}}$ covers $B$, where $\mathcal{S}:=\{S\subseteq A:|S|\in\mathbb{N}\}$ is the set of all finite subsets of $A$. Indeed, for every $x\in B$ there exist only finitely many indices $\{\alpha_1,\ldots,\alpha_k\}\subseteq A$ with $u_{\alpha}(x)>0$, as we argued before and thus $x\in U(\{\alpha_1,\ldots,\alpha_k\})$. Also, it is true that $U(S)\subseteq V_{\alpha}$ for every $\alpha\in S$. Indeed, if $\alpha\in S$ and $x\in U(S)$, then it should be true that $u_{\alpha}(x)>0$, because otherwise we would have $S=A$, but $A$ is not finite. So, $\{U_S\}_{S\in\mathcal{S}}$ is an open subcover of $\mathcal{V}$ and thus, it is a locally finite open cover of $B$ and again due to Remark~\ref{rem:vector_subcover} $\xi|_{U(S)}$ is a trivial vector bundle for every $S\in\mathcal{S}$.

Last, for every $k\in\mathbb{N}$ we define $U_k:=\bigcup_{S\in\mathcal{S}_k}U(S)$, where $\mathcal{S}_k=\{S\in\mathcal{S}:|S|=k\}$. Then, $\{U_k\}_{k\in\mathbb{N}}$ is clearly a cover of $B$ with $U_k$ open for every $k\in\mathbb{N}$. Moreover, it is true that $U_k$ is the disjoint union of all $U(S)$ for $S\in\mathcal{S}_k$. Indeed, for any two $S,S'\in\mathcal{S}_k$ with $S\neq S'$, there exist $\alpha\in S\setminus S'$ and $\beta\in S'\setminus S$. If there was some $x\in U(S)\cap U(S')$, then for this $x$ we would have simultaneously that $u_{\alpha}(x)>u_{\beta}(x)$ and $u_{\beta}(x)>u_{\alpha}(x)$, which is impossible. So, $\xi|U_k\cong\amalg_{S\in\mathcal{S}_k}\xi|_{U(S)}$ and thus is a trivial vector bundle. Also, for every $x\in B$ there exists some $X\subseteq B$ containing $x$, such that $X\cap U(S)\neq\emptyset$ only for finitely many $S\in\mathcal{S}$, since $\{U(S)\}_{S\in\mathcal{S}}$. This means that $X\cap U_k\neq\emptyset$ only for finitely many $k\in\mathbb{N}$. Hence $\{U_k\}_{k\in\mathbb{N}}$ is a locally finite, countable, open cover of $B$ making $\xi$ locally trivial.
\end{proof}
\begin{remark}\label{lem:paracompact_countable_cover_fb} At no point in this proof did we need that $\xi$ is linear over each fiber, so exactly the same proof can be done for a fiber bundle $p:E(p)\to B$. Namely, if $B$ is paracompact, then there exists some locally finite countable open cover $\{U_k\}_{k\in\mathbb{N}}$ of $B$ such that $p|U_k$ is a trivial fiber bundle for every $k\in\mathbb{N}$.
\end{remark}

%TODO: (SWi) changed
%TODO: pushout-> pullback
%TODO: base/basis uniform
%TODO: obviously/trivially -> clearly
%TODO: linear injection -> linear monomorphism
%TODO_MAYBE: tangent and normal vb
%TODO_MAYBE: define parallelizable spaces
%TODO_MAYBE: counterexample of vb over non-paracompact space w/o inner product
%TODO_MAYBE: refer to thms for generality of paracompact spaces
