\chapter{Grassmann Spaces}
The goal of this chapter is to present the basic properties of the Grassmann spaces. This chapter will be self-contained and used as a point of reference for the rest of the thesis. Our aim is for the reader to get to know the combinatorial structure of a Grassmannian and be able to do basic computations using the Schubert decomposition of these manifolds.
\section{Definition of the Grassmann Manifold}
Since the topology of projective spaces has been thoroughly studied and fully characterized, the natural question is if we can impose a natural topology on the set of $k$-dimensional subspaces of a vector space for some $k>1$ and what nice properties this space could have.

\begin{definition} Let $\mathbb{F}$ be a field and $0<k<n$ be some natural numbers. Then, the Grassmann topological space $Gr_{\mathbb{F}}(k,n)$ is the set of all linear $k$-dimensional subspaces of $\mathbb{F}^n$. This is to be topologized as the following quotient space.\end{definition}



